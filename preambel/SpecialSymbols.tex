%  Sondersymbole:
%
\newcommand{\Ast}{\ensuremath{\mathord{\ast}}}
\newcommand{\Sim}{\ensuremath{\mathord{\sim}}}
\newcommand{\Cdot}{\ensuremath{\mathord{\,\cdot\,}}}
\newcommand{\Tr}{\ensuremath{\mathsf{T}}}
\newcommand{\const}{\ensuremath{\mathord{\mathrm{const}}}}
\DeclareMathOperator{\supp}{supp}
\DeclareMathOperator{\rect}{rect}
\DeclareMathOperator{\ld}{ld}
\DeclareMathOperator{\SO}{SO}
%\DeclareMathOperator{\E}{E}
\DeclareMathOperator{\Var}{Var}
\DeclareMathOperator{\Cov}{Cov}
\DeclareMathOperator{\vol}{vol}
\DeclareMathOperator{\tr}{tr}
%\DeclareMathOperator*{\argmin}{arg\,min}
%\DeclareMathOperator*{\argmax}{arg\,max}
\DeclareMathOperator{\grad}{grad}
\DeclareMathOperator{\Arg}{Arg}
\DeclareMathOperator{\col}{col}
\DeclareMathOperator{\spn}{span}
\DeclareMathOperator{\aff}{aff}
% http://tex.stackexchange.com/questions/84302/what-is-the-difference-of-mathop-operatorname-and-declaremathoperator
\newcommand{\diff}{\mathop{}\!\mathrm{d}}
\newcommand{\ceq}{\mathrel{\mathop:}=}
%
%% Menge natürlicher Zahlen
\newcommand{\NatNum}{\mathbb{N}}
%
%-------------------------------------------------------------------------------
% Zustandsraum und Messraum:
%-------------------------------------------------------------------------------
\newcommand{\statespace}{\symbfcal{X}}
\newcommand{\measurementspace}{\symbfcal{Z}}
%-------------------------------------------------------------------------------
%
%-------------------------------------------------------------------------------
%Erwartungswert:
%
\newcommand{\E}{\mathbb{E}}
\newcommand{\Erw}[1]{ \E[#1] }
\newcommand{\ERW}[1]{ \E\bigl[#1\bigr] } %Mit großen Klammern
%
%
%-------------------------------------------------------------------------------
%
%Gauss-Verteilung an einer Stelle:
\newcommand{\GaussDist}{\mathcal{N}}
\newcommand{\GaussDistValue}[3]{\mathcal{N}({#1}; {#2}, {#3})}
%
%-------------------------------------------------------------------------------
%
%
%
%% Matrizen und Vektoren
%-------------------------------------------------------------------------------
%Rotation matrix
\newcommand{\RotMat}{\symbf{R}}
%Translation vector
\newcommand{\transVec}{\symbf{t}}
%
%
% Multidimensionale Zufallsvariablen
\newcommand{\mdrv}[1]{\symbfit{#1}}
%
\newcommand{\uVec}{\symbf{u}} % Control vector (vector with control parameters)
\newcommand{\wVec}{\symbf{w}} % 
\newcommand{\vVec}{\symbf{v}} % 
%-------------------------------------------------------------------------------
\newcommand{\Imat}{\symbf{I}} % Einheitsmatrix
\newcommand{\ZeroMat}{\symbf{0}} % zero matrix
\newcommand{\Fmat}{\symbf{F}} % system matrix of the Kalman Filter
\newcommand{\Gmat}{\symbf{G}} % control matrix of the Kalman Filter
\newcommand{\Hmat}{\symbf{H}} % measurement matrix of the Kalman Filter
\newcommand{\Qmat}{\symbf{Q}} % system noise covariance matrix of the Kalman Filter
\newcommand{\Rmat}{\symbf{R}} % measurement noise covariance matrix of the Kalman Filter
\newcommand{\Wmat}{\symbf{W}} % Jacobian matrix in the Extended Kalman Filter
\newcommand{\Vmat}{\symbf{V}} % Jacobian matrix in the Extended Kalman Filter
\newcommand{\KG}{\symbf{K}} % Kalman Gain
%-------------------------------------------------------------------------------
\newcommand{\uk}{\symbf{u}_k} % Control vector (vector with control parameters) at time $k$
\newcommand{\Qk}{\symbf{Q}_k} % System noise covariance matrix at time $k$
\newcommand{\Rk}{\symbf{R}_k} % Measurement noise covariance matrix at time $k$
\newcommand{\KGk}{\symbf{K}_{k}} % Kalman gain at time $k$ at time $k$
%
%
%
% point in image coordinates
\newcommand{\pkt}{\symbf{p}}
\newcommand{\pImage}{\pkt}
\newcommand{\pImageUntransf}{\symbf{p'}}
% points in camera coordinates
\newcommand{\PCam}{\pkt_{\textnormal{C}}}
\newcommand{\pLeft}{\pkt_{\textnormal{L}}}
\newcommand{\pRight}{\pkt_{\textnormal{R}}}
% point in world coordinates
\newcommand{\PWorld}{\pkt_{\textnormal{W}}}
% homogenious coordinates
\newcommand{\pHomogen}{\check{\pkt}}
