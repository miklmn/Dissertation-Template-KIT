%Tiefe des Inhaltsverzeichnisses setzen:
% bei Report: 3 geht bis subsection
\newcommand{\mytocdepth}{2}
%Tiefe des Inhaltsverzeichnisses in PDF-Lesezeichen setzen (evtl. höher als die des toc)
%\newcommand{\mybookmarkdepth}{\mytocdepth}
\newcommand{\mybookmarkdepth}{3}
%
% Hier Breite von marginnotes setzen
\newlength{\mymarginparwidth}
%\setlength{\mymarginparwidth}{0.15\paperwidth}
\setlength{\mymarginparwidth}{10mm}
%\setlength{\mymarginparwidth}{0pt}
% Abstand zwischen Seitennotizen und Text setzen
\newlength{\mymarginparsep}
%\setlength{\mymarginparsep}{}
\setlength{\mymarginparsep}{3mm}
%\setlength{\mymarginparsep}{5mm}
%\setlength{\mymarginparsep}{0pt}

% Abstand zwischen den Paragraphen
\newlength{\myparskip}
%\setlength{\myparskip}{0.5\baselineskip plus 0.5\baselineskip} % das wäre wohl default
%\setlength{\myparskip}{0.5\baselineskip} %ohne Pluswerte (= keine Dehnung)
%\setlength{\myparskip}{0.6\baselineskip} %ohne Pluswerte (= keine Dehnung)
\setlength{\myparskip}{0.6\baselineskip plus 1pt minus 1pt} %mit Pluswerten (= Dehnung) und Minuswerten (= Stauchung)

\newlength{\mychapterbeforeskip}
\newlength{\mychapterafterskip}
\newlength{\mysectionbeforeskip}
\newlength{\mysectionafterskip}
\newlength{\mysubsectionbeforeskip}
\newlength{\mysubsectionafterskip}
\newlength{\mysubsubsectionbeforeskip}
\newlength{\mysubsubsectionafterskip}
\newlength{\myparagraphbeforeskip}
\newlength{\myparagraphafterskip}

\setlength{\mychapterbeforeskip}{8\myparskip}
\setlength{\mychapterafterskip}{2.5\myparskip}
%
\setlength{\mysectionbeforeskip}{2\myparskip}
\setlength{\mysectionafterskip}{1.5\myparskip}
%
\setlength{\mysubsectionbeforeskip}{1.5\myparskip}
\setlength{\mysubsectionafterskip}{1\myparskip}
%
\setlength{\mysubsubsectionbeforeskip}{1.5\myparskip}
\setlength{\mysubsubsectionafterskip}{1\myparskip}
%
\setlength{\myparagraphbeforeskip}{1.5\myparskip}
\setlength{\myparagraphafterskip}{1\myparskip}

%Abstand zwischen der Abbilung/Tabelle und der Überschrift:
%funktioniert aus irgendeinem Grund nicht!
\newlength{\myaboveskip}
%\setlength{\myaboveskip}{10pt}
\setlength{\myaboveskip}{0pt}
%Abstand nach der Überschrift einer Abbilung/Tabelle:
\newlength{\mybelowskip}
%\setlength{\mybelowskip}{10pt}
\setlength{\mybelowskip}{0pt}
%Abstand zwischen der Abbilung/Tabelle und der Überschrift:
\newlength{\mysubcaptionskip}
\setlength{\mysubcaptionskip}{6pt}
%Abstand zwischen der Abbilung/Tabelle und der Überschrift:
\newlength{\mycaptionskip}
\setlength{\mycaptionskip}{5pt}
%Abstand zwischen zwei Subfloats (muss manuell hinzugefügt werden)
\newlength{\mysubfloatvskip}
\setlength{\mysubfloatvskip}{10pt} % (default 12pt plus 2pt minus 2pt)

%%% Vertikaler Abstand zwischen zwei Gleitobjekten
\newlength{\myfloatsep}
%\setlength{\myfloatsep}{\baselineskip} % (default 12pt plus 2pt minus 2pt)
\setlength{\myfloatsep}{\baselineskip}
%\setlength{\myfloatsep}{12pt}

%%% Vertikaler Abstand zwischen der Top- oder Bottom-Bereich und Text-Bereich
\newlength{\mytextfloatsep}
%\setlength{\mytextfloatsep}{\baselineskip} % (default 20pt plus 2pt minus 4pt)
%\setlength{\mytextfloatsep}{\myfloatsep}
\setlength{\mytextfloatsep}{1.5\baselineskip}
%\setlength{\mytextfloatsep}{12pt}

%% Abstand zwischen Inline-Gleitobjekten, die mit "here" platziert worden sind,
%% und dem darüber und darunter angeordneten Fließtext fest
%% Beispiel: \intextsep5mm plus3mm minus2mm
%\setlength{\intextsep}{0.5\baselineskip} % Platz ober- und unterhalb des Bildes
\newlength{\myintextsep}
\setlength{\myintextsep}{\baselineskip} % (default 12pt plus 2pt minus 2pt) 
%\setlength{\myintextsep}{10pt} % (default 12pt plus 2pt minus 2pt) 

%% Weitere Angaben:
%Vorlage KIT-Verlag:
%usepackage[a5paper,headheight=1.5\baselineskip,top=25mm,lines=31,heightrounded=true,bindingoffset=15mm,textwidth=106mm]{geometry}
%\usepackage[a4paper,headheight=1.5\baselineskip,top=25mm,lines=46,heightrounded=true,bindingoffset=15mm,textwidth=160mm]{geometry}
% footskip: Abstand zwischen Textunterkante und Seitenzahl
% KIT-Verlag: mindestens 10mm bzw. 3 Zeilen
% Koma-Standard 3.5\baselineskip
% IES-Vorlage: 2\baselineskip
% Abstand zwischen Textkörper und Unterkante Fußzeile (Seitenzahlen)
\newlength{\myfootskip}
\setlength{\myfootskip}{11mm}
% Abstand zwischen Textkörper und Linie in der Kopfzeile
\newlength{\myheadsep}
%\setlength{\myheadsep}{5mm}
\setlength{\myheadsep}{7mm} %Da eine kleinere Schriftgröße verwendet wird

\newlength{\myheadheight}
\setlength{\myheadheight}{1.5\baselineskip}

\newlength{\mytop}
\setlength{\mytop}{25mm}

\newlength{\mybindingoffset}
\setlength{\mybindingoffset}{8mm}

\newlength{\myinner}
\setlength{\myinner}{15mm}

\newlength{\myouter}
\setlength{\myouter}{15mm}

\newlength{\mytextwidth}
\setlength{\mytextwidth}{106mm}

\newboolean{SetFloatsVerticallyCentered}
\setboolean{SetFloatsVerticallyCentered}{false}