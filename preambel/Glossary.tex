%%Examlpe of a glossary entry:
%
%\newglossaryentry{BicycleModel}{%
%name=bicycle model,
%plural=bycicle models
%sort=bicycle model, % need a sort key if the name contains a command
%description=Einspurmodell
%}
%
%define own command for easier glossary term definition (without plural etc. if not needed)
\newcommand{\myglossaryentry}[3]{%
	%\AtBeginDocument{%
		\newglossaryentry{#1}{name=#2,description={#3}}%
	%}%
	%% Falls die Begriffe im Text nicht referenziert werden:
	%% alle Begriffe am Anfang des Dokuments zur Liste hinzufügen
	%% Problem: erzeuge eine leere Seite am Anfang des Dokumentes
	%\AtBeginDocument{%
		%\glsadd{#1}%
	%}%
}
%
\myglossaryentry{gls:BicycleModel}{bicycle model}{Einspurmodell}
\myglossaryentry{gls:DepthOfField}{depth of field}{Schärfebereich}
\myglossaryentry{gls:FieldOfView}{field of view}{Sichtfeld}
\myglossaryentry{gls:InterceptTheorem}{intercept theorem}{Strahlensatz}
\myglossaryentry{gls:InstantaneousCenterOfRotation}{instantaneous center of rotation}{Momentanpol}
\myglossaryentry{gls:LeastSquares}{least squares method}{Methode kleinster Fehlerquadrate}
\myglossaryentry{gls:ProbabilityDensityFunction}{probability density function}{Wahrscheinlichkeitsdichte}
\myglossaryentry{gls:ProbabilityMassFunction}{probability mass function}{Zähldichte}
\myglossaryentry{gls:DegreeOfFreedom}{degree of freedom}{Freiheitsgrad}

\newacronym[%
  shortplural={PCAs},%
  longplural={Principal Component Analyses}%
] {pca}{PCA}{Principal Component Analysis}

\newacronym[] {pdf}{PDF}{Portable Document Format}

\newacronym[] {eps}{EPS}{Embedded Postscript}

\newacronym[] {wysiwyg}{WYSIWYG}{What you see is what you get}

\longnewglossaryentry{pkg}{%
  name={Paket},%
  plural={Pakete}}%
{%
Ein LaTeX-Paket besteht aus einer oder mehrerer Dateien, die entweder vorhandene
Kernfunktionen von LaTeX umdefinieren und so das Verhalten derselbigen bzw.
das Erscheinungsbild des fertigen Dokuments verändern oder die zusätzliche
Befehle zur Verfügung stellen.}

\longnewglossaryentry{utf8}{%
  name={UTF-8}}%
{%
Ein Schema zur Kodierung von Zeichen in computerverarbeitbarer Form, die Zeichen
aus allen Sprachen umfasst.}

\longnewglossaryentry{tikz}{%
  name={TikZ}}%
{%
Eine Sammlung von LaTeX-Paketen, die ein direktes Erzeugen von (technischen)
Zeichnungen, Diagrammen, etc. in LaTeX erlaubt.}

\longnewglossaryentry{pgfplots}{%
  name={PGFplots}}%
{%
Eine Sammlung von TikZ-Paketen, die ein direktes Erzeugen von Diagrammen aller
Art (inkl. 3D-Diagramme) direkt aus LaTeX heraus ermöglicht.}

\longnewglossaryentry{biblatex}{%
  name={BibLaTex}}%
{%
Der Nachfolger von BibTex zum Erzeugen von Literaturverzeichnissen in LaTeX. Es
zeichnet sich vor allem durch deutlich bessere Flexibilität bei der Gestaltung
des Literaturverzeichnisses und der Art und Weise wie Zitatmarken gesetzt werden
aus. Darüber hinaus ist es vollständig UTF-8-kompatibel.
}

\longnewglossaryentry{bibtex}{%
  name={BibTeX}}%
{%
Der Vorgänger von BibLaTex.}

\longnewglossaryentry{postscript}{%
  name={PostScript}}%
{%
Eine von Adobe 1984 entwickelte Seitenbeschreibungssprache.}

\longnewglossaryentry{java}{%
  name={Java}}%
{%
Eine von Sun Microsystems 1995 veröffentlichte, objektorientierte Programmiersprache.}

\longnewglossaryentry{latex}{%
  name={LaTeX}}%
{%
Eine von Leslie Lamport 1980 entwickelter Satz von Makros zur Erweiterung von TeX.}


\longnewglossaryentry{umgebung}{%
  name={Umgebung},%
  plural={Umgebungen}}%
{%
Ein Bereich im LaTeX-Code der mit \texttt{begin} eingeleitet und mit \texttt{end}
beendet wird. Umgebungen können auch verschachtelt sein.}
