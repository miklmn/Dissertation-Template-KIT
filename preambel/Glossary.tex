%% ----------------------------------------------------------------------------------------
%% In dieser Datei werden die verwendeten Glossarbegriffe definiert
%% ----------------------------------------------------------------------------------------
%%
%%
%% ----------------------------------------------------------------------------------------
%% Definition eines eigenes Macro \myglossaryentry[]{}{}{} zur Erstellung von Glossareinträgen
%% ----------------------------------------------------------------------------------------
\newcommand{\myglossaryentry}[4][]{%
	%\AtBeginDocument{%
		\ifx\relax#1\relax%
			\newglossaryentry{#2}{name={#3},description={#4}}%
		\else%
			\newglossaryentry{#2}{name={#3},description={#4},#1}%
		\fi%
	%}%
}% ----------------------------------------------------------------------------------------
%%
%% ----------------------------------------------------------------------------------------
%% Beschreibung:
%% ----------------------------------------------------------------------------------------
%% \myglossaryentry[]{}{}{} nimmt 4 Argumente
%% Im ersten, optionalen Argument (in Eckigen Klammern), kann die Pluralform definiert werden
%% Außerdem kann hier der Sortierschlüssel definiert werden, falls der Akronym LaTeX-Befehle beinhaltet
%% Zweites Argument ist das Schlüsselwort. Konvontion: zur Markierung des Schlüsselwortes als solches
%%    und zur Unterscheidung dieses von der Kurzform sollte dem Schlüsselwort ein "ac:" vorangestellt werden.
%% Drittes Argument ist das eigentliche Akronym.
%% Viertes Argument ist die Langform.
%% ----------------------------------------------------------------------------------------
%%
%% ----------------------------------------------------------------------------------------
%% Beispiel:
%% ----------------------------------------------------------------------------------------
%\myglossaryentry[plural={Umgebungen}]%
%                {gls:umgebung}{Umgebung}{Ein Bereich im LaTeX-Code der mit
%								\texttt{begin} eingeleitet und mit \texttt{end} beendet wird.
%								Umgebungen können auch verschachtelt sein.}
%% ----------------------------------------------------------------------------------------
%%
%% ----------------------------------------------------------------------------------------
%%
%% A
%%
%% ----------------------------------------------------------------------------------------
%
%% ----------------------------------------------------------------------------------------
%%
%% B
%%
%% ----------------------------------------------------------------------------------------
\myglossaryentry{gls:biblatex}{BibLaTex}{Der Nachfolger von BibTex zum Erzeugen
                von Literaturverzeichnissen in LaTeX. Es zeichnet sich vor allem
                durch deutlich bessere Flexibilität bei der Gestaltung des
                Literaturverzeichnisses und der Art und Weise wie Zitatmarken
                gesetzt werden aus. Darüber hinaus ist es vollständig UTF-8-kompatibel.}
\myglossaryentry{gls:bibtex}{BibTeX}{Der Vorgänger von BibLaTex}
\myglossaryentry{gls:BicycleModel}{bicycle model}{Einspurmodell}
%% ----------------------------------------------------------------------------------------
%%
%% C
%%
%% ----------------------------------------------------------------------------------------
%
%% ----------------------------------------------------------------------------------------
%%
%% D
%%
%% ----------------------------------------------------------------------------------------
\myglossaryentry[plural={degrees of freedom}]%
                {gls:DegreeOfFreedom}{degree of freedom}{Freiheitsgrad}
\myglossaryentry{gls:DepthOfField}{depth of field}{Schärfebereich}
%% ----------------------------------------------------------------------------------------
%%
%% E
%%
%% ----------------------------------------------------------------------------------------
%
%% ----------------------------------------------------------------------------------------
%%
%% F
%%
%% ----------------------------------------------------------------------------------------
\myglossaryentry[plural={fields of view}]%
                {gls:FieldOfView}{field of view}{Sichtfeld}
%% ----------------------------------------------------------------------------------------
%%
%% G
%%
%% ----------------------------------------------------------------------------------------
%
%% ----------------------------------------------------------------------------------------
%%
%% H
%%
%% ----------------------------------------------------------------------------------------
%
%% ----------------------------------------------------------------------------------------
%%
%% I
%%
%% ----------------------------------------------------------------------------------------
\myglossaryentry{gls:InterceptTheorem}{intercept theorem}{Strahlensatz}
\myglossaryentry{gls:InstantaneousCenterOfRotation}{instantaneous center of rotation}{Momentanpol}
%% ----------------------------------------------------------------------------------------
%%
%% J
%%
%% ----------------------------------------------------------------------------------------
\myglossaryentry{gls:java}{Java}{Eine von Sun Microsystems 1995 veröffentlichte, objektorientierte Programmiersprache.}
%% ----------------------------------------------------------------------------------------
%%
%% K
%%
%% ----------------------------------------------------------------------------------------
%
%% ----------------------------------------------------------------------------------------
%%
%% L
%%
%% ----------------------------------------------------------------------------------------
\myglossaryentry{gls:latex}{LaTeX}{Eine von Leslie Lamport 1980 entwickelter Satz von Makros
                zur Erweiterung von TeX.}
\myglossaryentry{gls:LeastSquares}{least squares method}{Methode kleinster Fehlerquadrate}
%% ----------------------------------------------------------------------------------------
%%
%% M
%%
%% ----------------------------------------------------------------------------------------
%
%% ----------------------------------------------------------------------------------------
%%
%% N
%%
%% ----------------------------------------------------------------------------------------
%
%% ----------------------------------------------------------------------------------------
%%
%% O
%%
%% ----------------------------------------------------------------------------------------
%
%% ----------------------------------------------------------------------------------------
%%
%% P
%%
%% ----------------------------------------------------------------------------------------
\myglossaryentry{gls:ProbabilityDensityFunction}{probability density function}{Wahrscheinlichkeitsdichte}
\myglossaryentry{gls:ProbabilityMassFunction}{probability mass function}{Zähldichte}
\myglossaryentry{gls:pgfplots}{PGFplots}{Eine Sammlung von TikZ-Paketen, die ein direktes Erzeugen
                von Diagrammen aller Art (inkl. 3D-Diagramme) direkt aus LaTeX heraus ermöglicht.}
\myglossaryentry[plural={Pakete}]%
								{gls:pkg}{Paket}{Ein LaTeX-Paket besteht aus einer oder mehrerer Dateien,
								die entweder vorhandene Kernfunktionen von LaTeX umdefinieren und so das Verhalten
								derselbigen bzw. das Erscheinungsbild des fertigen Dokuments verändern
								oder die zusätzliche Befehle zur Verfügung stellen.}
\myglossaryentry{gls:postscript}{PostScript}{Eine von Adobe 1984 entwickelte Seitenbeschreibungssprache.}
%% ----------------------------------------------------------------------------------------
%%
%% R
%%
%% ----------------------------------------------------------------------------------------

%% ----------------------------------------------------------------------------------------
%%
%% S
%%
%% ----------------------------------------------------------------------------------------

%% ----------------------------------------------------------------------------------------
%%
%% T
%%
%% ----------------------------------------------------------------------------------------
\myglossaryentry{gls:tikz}{TikZ}{Eine Sammlung von LaTeX-Paketen, die ein direktes Erzeugen
                von (technischen) Zeichnungen, Diagrammen, etc. in LaTeX erlaubt.}
%% ----------------------------------------------------------------------------------------
%%
%% U
%%
%% ----------------------------------------------------------------------------------------
\myglossaryentry[plural={Umgebungen}]%
                {gls:umgebung}{Umgebung}{Ein Bereich im LaTeX-Code der mit
								\texttt{begin} eingeleitet und mit \texttt{end} beendet wird.
								Umgebungen können auch verschachtelt sein.}
\myglossaryentry{gls:utf8}{UTF-8}{Ein Schema zur Kodierung von Zeichen in computerverarbeitbarer Form,
                             die Zeichen aus allen Sprachen umfasst.}
%% ----------------------------------------------------------------------------------------
%%
%% V
%%
%% ----------------------------------------------------------------------------------------

%% ----------------------------------------------------------------------------------------
%%
%% W
%%
%% ----------------------------------------------------------------------------------------

%% ----------------------------------------------------------------------------------------
%%
%% X
%%
%% ----------------------------------------------------------------------------------------
%
%% ----------------------------------------------------------------------------------------
\newglossaryentry{pi}
{
  name={\ensuremath{\pi}},
  description={ratio of circumference of circle to its
               diameter},
  sort=pi,
	type=symbolslist
}
%%
%% Achtung: Falls die Begriffe im Text nicht mit \gls{label} referenziert werden:
%% Alle Begriffe am Anfang des Dokuments zur Liste hinzufügen
%\AtBeginDocument{%
	%\glsadd{#1}%
%}%
%% Problem: erzeugt eine leere Seite am Anfang des Dokumentes