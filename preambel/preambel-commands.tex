%%% Internal Commands: ----------------------------------------------
\makeatletter
%
\providecommand{\IfPackageLoaded}[2]{\@ifpackageloaded{#1}{#2}{}}
\providecommand{\IfPackageNotLoaded}[2]{\@ifpackageloaded{#1}{}{#2}}
\providecommand{\IfElsePackageLoaded}[3]{\@ifpackageloaded{#1}{#2}{#3}}
%
\newboolean{partavailable}%
\newboolean{chapteravailable}%
\setboolean{partavailable}{false}%
\setboolean{chapteravailable}{false}%

\ifcsname part\endcsname
  \setboolean{partavailable}{true}%
\else
  \setboolean{partavailable}{false}%
\fi

\ifcsname chapter\endcsname
  \setboolean{chapteravailable}{true}%
\else
  \setboolean{chapteravailable}{false}%
\fi


\providecommand{\IfPartDefined}[1]{\ifthenelse{\boolean{partavailable}}{#1}{}}%
\providecommand{\IfChapterDefined}[1]{\ifthenelse{\boolean{chapteravailable}}{#1}{}}%
\providecommand{\IfElsePartDefined}[2]{\ifthenelse{\boolean{partavailable}}{#1}{#2}}%
\providecommand{\IfElseChapterDefined}[2]{\ifthenelse{\boolean{chapteravailable}}{#1}{#2}}%

\providecommand{\IfDefined}[2]{%
\ifcsname #1\endcsname
   #2 %
\else
     % do nothing
\fi
}

\providecommand{\IfElseDefined}[3]{%
\ifcsname #1\endcsname
   #2 %
\else
   #3 %
\fi
}

\providecommand{\IfElseUnDefined}[3]{%
\ifcsname #1\endcsname
   #3 %
\else
   #2 %
\fi
}


%
% Check for 'draft' mode - commands.
\newcommand{\IfNotDraft}[1]{\ifx\@draft\@undefined #1 \fi}
\newcommand{\IfNotDraftElse}[2]{\ifx\@draft\@undefined #1 \else #2 \fi}
\newcommand{\IfDraft}[1]{\ifx\@draft\@undefined \else #1 \fi}
%

% Define frontmatter, mainmatter and backmatter if not defined
\@ifundefined{prefrontmatter}{%
   \newcommand*{\prefrontmatter}{%
      %In lateinischen Kleinbuchstaben nummerieren (a, b, c)
      %\pagenumbering{roman}
			\hypersetup{pageanchor=false}
			\cleardoubleoddpage  %% M. Kohm sagt, das sollte man vor jedem Pagenumbering-Wechsel tun
			\pagenumbering{alph}%
			%\renewcommand*\thepage{\texorpdfstring{\arabic{page}}{prefrontP.\arabic{page}}}%
			%\renewcommand*{\theHpage}{prefront.\thepage} %statt front.\thepage ginge auch \arabic{chapter}.\thepage. Hauptsache eindeutig: http://de.authex.info/1132586-pdflatex-und-hyperref-mit-plainpages 
			% http://tex.stackexchange.com/questions/65182/cross-references-linking-to-wrong-equations-using-hyperref
			% oder auch: http://tex.stackexchange.com/questions/6098/wrong-hyper-references-after-resetting-chapter-counter
			%\renewcommand*\theHchapter{prefrontC.\arabic{chapter}}
    }
}{}
\@ifundefined{frontmatter}{%
   \newcommand*{\frontmatter}{%
      %In Römischen Grossbuchstaben nummerieren (I, II, III)
      %\pagenumbering{Roman}
			\cleardoubleoddpage  %% M. Kohm sagt, das sollte man vor jedem Pagenumbering-Wechsel tun
			\pagenumbering{Roman}%
			\hypersetup{pageanchor=false}
			%\renewcommand*\thepage{\texorpdfstring{\arabic{page}}{frontP.\arabic{page}}}%
			%%\renewcommand*{\theHpage}{front.\thepage} %statt front.\thepage ginge auch \arabic{chapter}.\thepage. Hauptsache eindeutig: http://de.authex.info/1132586-pdflatex-und-hyperref-mit-plainpages 
			%% http://tex.stackexchange.com/questions/65182/cross-references-linking-to-wrong-equations-using-hyperref
			%% oder auch: http://tex.stackexchange.com/questions/6098/wrong-hyper-references-after-resetting-chapter-counter
			%\renewcommand*\theHchapter{frontC.\arabic{chapter}}
    }
}{}
\@ifundefined{mainmatter}{%
   % scrpage2 benötigt den folgenden switch
   % wenn \mainmatter definiert ist.
   \newif\if@mainmatter\@mainmattertrue
   \newcommand*{\mainmatter}{%
      % -- Seitennummerierung auf Arabische Zahlen zurücksetzen (1,2,3)
			\cleardoubleoddpage  %% M. Kohm sagt, das sollte man vor jedem Pagenumbering-Wechsel tun
      \pagenumbering{arabic}%
      %\setcounter{page}{1}%
			\hypersetup{pageanchor=true}
			%\renewcommand*\thepage{\texorpdfstring{\arabic{page}}{mainP.\arabic{page}}}%
			%%\renewcommand*{\theHpage}{main.\thepage}
			%\renewcommand\theHchapter{mainC.\arabic{chapter}}
			%\renewcommand{\theHequation}{\theHsection.\equationgrouping\arabic{equation}}
   }
}{}
\@ifundefined{backmatter}{%
   \newcommand*{\backmatter}{
      %In Römischen Kleinbuchstaben nummerieren (i, ii, iii)
			\cleardoubleoddpage  %% M. Kohm sagt, das sollte man vor jedem Pagenumbering-Wechsel tun
			%\pagenumbering{Roman}%
     \pagenumbering{roman}
			%\renewcommand*\thepage{\texorpdfstring{\arabic{page}}{backP.\arabic{page}}}%
			%%\renewcommand*{\theHpage}{back.\thepage}
			%\renewcommand\theHchapter{backC.\arabic{chapter}}
   }
}{}

% Pakete speichern die später geladen werden sollen
\newcommand{\LoadPackagesNow}{}
\newcommand{\LoadPackageLater}[1]{%
   \g@addto@macro{\LoadPackagesNow}{%
      \usepackage{#1}%
   }%
}


\makeatother


%for conditional text inclusions and invisible comments
\usepackage[nogroup]{versions}

%Hilfsmakros:

% Definition von Hilfsmacros \showif und \hideif, die
% das Verstecken bzw. Einblenden von Textteilen ermöglichen werden:
% Es wird eine Boolean-Variable mit dem angegebenen Namen angelegt
% und entsprechend auf true (showif) oder false (hideif) gesetzt.
% Außerdem werden entsprechende includeversion/excludeversion-Macros aufgerufen.
\newcommand{\showif}[1]{%
	\ifcsname#1\endcsname%
		%falls ein Makro mit diesem Namen bereits existiert: Fehlermeldung ausgeben
		\GenericError{}{A macro with the name #1 is already defined somewhere else! Please use another name}{}{}%
	\else%
		\newboolean{#1}%
	\fi%
	\setboolean{#1}{true}%
	\includeversion{#1}%
}
\newcommand{\hideif}[1]{%
	\ifcsname#1\endcsname%
		%falls ein Makro mit diesem namen bereits existiert: Fehlermeldung ausgeben
		\GenericError{}{A macro with the name #1 is already defined somewhere else! Please use another name}{}{}%
	\else%
		\newboolean{#1}%
	\fi%
	\setboolean{#1}{false}%
	\excludeversion{#1}%
}

% Definition von Hilfsmacros \setUserDefinedBool, die eine Boolean-Variable erstellen und sie sogleich auf den Wert true oder false
% Es wird eine Boolean-Variable mit dem angegebenen Namen angelegt
% und entsprechend auf true oder false gesetzt.
\newcommand{\setUserDefinedBoolean}[2]{%
	\ifcsname#1\endcsname%
		%falls ein Makro mit diesem Namen bereits existiert: Fehlermeldung ausgeben
		\GenericError{}{A macro with the name #1 is already defined somewhere else! Please use another name}{}{}%
	\else%
		\newboolean{#1}%
		\setboolean{#1}{#2}%
		\ifthenelse{\boolean{#2}}{\includeversion{#1}}{\excludeversion{#1}}
	\fi%
}

%%% ----------------------------------------------------------------