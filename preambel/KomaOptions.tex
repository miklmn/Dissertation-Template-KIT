%%% === Textbody ==============================================================
\KOMAoptions{%
		%% Wird nicht verwendet, da typearea benutzt wird
   %DIV=14,% (Size of Text Body, higher values = greater textbody)  %DIV 14 für A5, DIV 12 für A4
%   DIV=11,% Alt (MG: Mein Hack um größere Margin-Fläche zu errechen??)
    DIV=15,%
    %DIV=calc, % (also areaset/classic/current/default/last) 
   % -> after setting of spacing necessary!   
   %BCOR=10mm% (Bindekorrektur) % 8 gibt einen Rand innen von 1,5 cm bei DIV16. Laut Frau Mehl entspricht 7 1,6 cm und sollte 1,8 bis 2,0 cm sein --> Jetzt 10.
   %BCOR=5mm% (Bindekorrektur)
   %BCOR=15mm
	BCOR=\mybindingoffset
}%
%\areaset[BCOR]{Breite}{Höhe}
%A4 hat 210mm x 297mm
%\areaset[15mm]{172mm}{267mm}
%\areaset[15mm]{100mm}{200mm}
%%% === Headings ==============================================================
\KOMAoptions{%
   %%%% headings
	% headings=small,  % Small Font Size, thin spacing above and below
   headings=normal, % Medium Font Size, medium spacing above and below
   %headings=big, % Big Font Size, large spacing above and below
   %
   %headings=noappendixprefix, % chapter in appendix as in body text
	%headings=nochapterprefix,  % no prefix at chapters
   % headings=appendixprefix,   % inverse of 'noappendixprefix'
   % headings=chapterprefix,    % inverse of 'nochapterprefix'
   % headings=openany,   % Chapters start at any side
   % headings=openleft,  % Chapters start at left side
   headings=openright, % Chapters start at right side      
   %%% Add/Dont/Auto Dot behind section numbers 
   %%% (see DUDEN as reference)
   % numbers=autoenddot
   % numbers=enddot
   numbers=noenddot
   % secnumdepth=3 % depth of sections numbering (???)
}%
%\setcounter{secnumdepth}{3} % Überschriften nur bis subsection-ebene nummerieren (subsubsections nicht mehr nummerieren)
\setcounter{secnumdepth}{4} % auch subsubsections nummerieren
%%% === Page Layout ===========================================================
\KOMAoptions{% (most options are for package typearea)
   twoside=true, % two side layout (alternating margins, standard in books)
   % twoside=false, % single side layout 
   % twoside=semi,  % two side layout (non alternating margins!)
   %
   twocolumn=false, % (true)
   %
   headinclude=true,%
   footinclude=true,%
   mpinclude=false,%      
   %
   % Die Option headlines setzt die Anzahl der Kopfzeilen.
   % Normalerweise arbeitet das typearea-Paket mit 1,25 Kopfzeilen
   %headlines=1.25,%
   headlines=1,
   %headlines=2.1,%
   % headheight=2em,%
   %% Die Option footlines setzt die Anzahl der Fußzeilen.
   %% Normalerweise arbeitet das typearea-Paket mit 1,25 Fußzeilen
   %footlines=1.25,%
   footlines=1,
   %footlines=1.6,%
   % footheight=2em,%
   headsepline=true,% %bedingt headinclude
   footsepline=false,% %bedingt footinclude
   %% wenn Head und Foot in Seitenspiegel inkludiert werden sollen:
   %headinclude=true,%
   %footinclude=true,%  %ändert die komisch tiefen Seitenzahlen
	%Vakatseiten werden mit dem Style gesetzt
   cleardoublepage=empty %plain, headings
}%
%%% === Paragraph Separation ==================================================
\KOMAoptions{%
	 %%% The first two require the TikZ workaround
	 %parskip=relative, % change indentation according to fontsize (recommended)
   parskip=absolute, % do not change indentation according to fontsize
   %%% The following doesn't need the TikZ Workaround
   % parskip=false    % indentation of 1em
   % parskip=true   % parksip of 1 line - with free space in last line of 1em
   % parskip=full-  % parksip of 1 line - no adjustment
   % parskip=full+  % parksip of 1 line - with free space in last line of 1/4
   % parskip=full*  % parksip of 1 line - with free space in last line of 1/3    %% TeX Grouping Capacity Fehler wenn TikZ-Bilder kommen. Seltsam.
   %parskip=half   % parksip of 1/2 line - with free space in last line of 1em
   parskip=half-  % parksip of 1/2 line - no adjustment
   % parskip=half+  % parksip of 1/2 line - with free space in last line of 1/3
   % parskip=half*  % parksip of 1/2 line - with free space in last line of 1em
}%
%%% === Table of Contents =====================================================
%Inhaltesverzeichnis mit größerer Tiefe: bei Report: 3 geht bis subsection
\setcounter{tocdepth}{\mytocdepth} % Depth of TOC Display
\KOMAoptions{%
   %%% Setting of 'Style' and 'Content' of TOC
   % toc=left, %
   toc=indented,%
   %
   toc=bib,
   % toc=nobib,
   % toc=bibnumbered,
   %
	% toc=index,%
   toc=noindex,
	 %
   toc=chapterentrydotfill, % Bei den Kapiteleinträgen sollen Text und Seitenzahl ebenfalls
	                       % durch eine punktierte Linie miteinander verbunden werden
	 %toc=sectionentrywithdots, bei Abschnittseinträgen der Klasse scrartcl
   % funktioniert nicht! Warum?
	 %
   % toc=listof,
   toc=nolistof
   % toc=listofnumbered,
   %   
}%  
%%% === Lists of figures, tables etc. =========================================
\KOMAoptions{%
   %%% Setting of 'Style' and 'Content' of Lists 
   %%% (figures, tables etc)
	% --- General List Style ---
   listof=left, % tabular styles
   %listof=indented, % hierarchical style
   % --- chapter highlighting ---
   % listof=chapterentry, % ??? Chapter starts are marked in figure/table
   % listof=chaptergapline, % New chapter starts are marked by a gap 
      		  			   	 % of a single line
	%listof=chaptergapsmall, % New chapter starts are marked by a gap 
   	    					   % of a smallsingle line
   % listof=nochaptergap, % No Gap between chapters
   %
   % listof=leveldown, % lists are moved one level down ???
   % --- Appearance of Lists in TOC
   % listof=notoc, % Lists are not part of the TOC
   listof=totoc % add Lists to TOC without number
   % listof=totocnumbered, % add Lists to TOC with number
}%  
%%% === Bibliography ==========================================================
%% Setting of 'Style' and 'Content' of Bibliography
\KOMAoptions{%
   %bibliography=oldstyle,% "Klassischer" Stil des Literaturverzeichnisses: jeder Eintrag abgesetzt, hängend
   bibliography=openstyle,% "Moderner" Stil: keine vertikale Absetzung, dafür horizontaler Zusatzeinzug
   % bibliography=nottotoc, % Bibliography is not part of the TOC
   % bibliography=totocnumbered, % add Bibliography to TOC with number
   bibliography=totoc % add Bibliography to TOC without number
}%
%%% === Index =================================================================
%% Setting of 'Style' and 'Content' of Index in TOC
\KOMAoptions{%
   index=nottotoc, % index is not part of the TOC
   % index=totoc, % add index to TOC without number
   %
   chapterentrydots=true % Verbindung der Kapiteleinträge im Inhaltverzeichnis mit der Seitenzahl durch Punkte
   % funktioniert nicht! Warum?
	 %
}%
%%% === Titlepage =============================================================
\KOMAoptions{%
   titlepage=true %
   %titlepage=false %
}%
%%% === Miscellaneous =========================================================
\KOMAoptions{% 	
   footnotes=multiple% nomultiple
   %open=any,%
   %open=left,%
   %open=right,%
   %chapterprefix=false,%
   %appendixprefix=false,%
   %chapteratlists=10pt,% entry
}%