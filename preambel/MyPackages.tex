%% Define commands that don't eat spaces.
\usepackage{xspace}

%% IfThenElse: muss frueher eingebunden werden wegen Abfragen zur Sprache etc.
%\usepackage{ifthen}

%for compact lists
\usepackage{mdwlist}

% Für verzierte Überschriften
\usepackage{lettrine}

%% Einstellungen für die Aufzählungen
%% Kein Abstand zwischen den Item-Einträgen
%\setlist{noitemsep}
%%Verkleinerter Abstand zwischen den Item-Einträgen
\setlist{%
		%before=\RaggedRight ,%Flattersatz in itemize- und enumerate-Umgebungen erzwingen
		%funktioniert nicht so wie gewünscht, s.
		%https://tex.stackexchange.com/questions/104088/problem-with-enumitem-and-raggedright
		%Workaround weiter unten mit \AtBeginEnvironment{itemize}{\preto\item{\RaggedRight}}
		%Ränder setzen:
		%leftmargin=2em, %linkes Rand explizit setzen
		%rightmargin=1em, %rechtes Rand vergrößern
		rightmargin=2mm,
		itemsep=0.2\baselineskip,% 
		parsep=0.2\baselineskip%
		%topsep=0.2\baselineskip,
		%partopsep=0.2\baselineskip,%
}
%% Flattersatz in itemize- und enumerate-Umgebungen erzwingen
%% Forderung des KSP-Verlages: keine Worttrennung in Aufzählungen, daher \raggedright statt \RaggedRight
%\AtBeginEnvironment{itemize}{\preto\item{\RaggedRight}}
\AtBeginEnvironment{itemize}{\preto\item{\raggedright}}
\AtBeginEnvironment{enumerate}{\preto\item{\raggedright}}
%\AtBeginEnvironment{description}{\preto\item{\raggedright}}
\AtBeginEnvironment{itemize*}{\preto\item{\raggedright}}
\AtBeginEnvironment{enumerate*}{\preto\item{\raggedright}}
%\AtBeginEnvironment{description*}{\preto\item{\raggedright}}
%


\usepackage[super]{nth}

\usepackage{tikz}
\usetikzlibrary{%
				automata,%
				arrows,%
				backgrounds,%
				calc,%
				chains,%
				decorations,%
				decorations.markings,%
				decorations.pathmorphing,%
				external,%
				fadings,%
				fit,%
				matrix,%
				positioning,%
				scopes,%
				shadows,%
				shapes.geometric,%
				shapes.misc,%
				shapes.multipart,%
				through,%
				topaths}
\usepackage{tikz-3dplot}

%Einschliessen von Matlab-Grafiken:
\usepackage{pgfplots}
%Externe Generierung von plots, s. https://tex.stackexchange.com/questions/7953/how-to-expand-texs-main-memory-size-pgfplots-memory-overload
%% Braucht -shell-escape Kompilieroption!!!

\pgfplotsset{
  compat=1.14,%
  plot coordinates/math parser=false,%
  tick label style={font=\footnotesize},%
  label style={font=\small},%
	scale only axis,%
	axis lines=center,%
	axis on top,%
  every axis legend/.append style={cells={anchor=west},draw=none,font=\small},%
  every axis plot/.append style={semithick},%
	KIT scatter plot A/.style={%
    draw=none,only marks,mark=*,mark options={draw=KITblue,fill=KITblue}},%
	KIT scatter plot B/.style={%
    draw=none,only marks,mark=square*,mark options={draw=KITred,fill=KITred}},%
	KIT scatter plot C/.style={%
    draw=none,only marks,mark=diamond*,mark options={draw=KITorange,fill=KITorange}},%
	KIT scatter plot explicit/.style={%
    scatter,%
    scatter/classes={%
      a={mark=*,KITblue},%
      b={mark=square*,KITred},%
      c={mark=diamond*,KITorange}},%
    only marks,%
    scatter src=explicit symbolic,%
    z buffer=sort},%
  KIT ybar plot A/.style={%
    ybar,fill=KITblue,draw=none},%
  KIT ybar plot B/.style={%
    ybar,fill=KITred,draw=none},%
  KIT ybar plot C/.style={%
    ybar,fill=KITorange,draw=none},%
  KIT xbar plot A/.style={%
    ybar,fill=KITblue,draw=none},%
  KIT xbar plot B/.style={%
    ybar,fill=KITred,draw=none},%
  KIT xbar plot C/.style={%
    ybar,fill=KITorange,draw=none},%
  KIT line plot A/.style={%
    KITblue,semithick},%
  KIT line plot B/.style={%
    KITred,semithick},%
  KIT line plot C/.style={%
    KITorange,semithick},%
  KIT line plot D/.style={%
    KITlilac,semithick},%
  KIT line plot E/.style={%
    KITbrown,semithick},%
  KIT line plot F/.style={%
    KITblue,semithick,dashed},%
  KIT line plot G/.style={%
    KITred,semithick,dashed},%
  KIT line plot H/.style={%
    KITorange,semithick,dashed},%
  KIT line plot I/.style={%
    KITlilac,semithick,dashed},%
  KIT line plot J/.style={%
    KITbrown,semithick,dashed},%
	KIT smooth plot A/.style={%
    KITblue,semithick,smooth},%
  KIT smooth plot B/.style={%
    KITred,semithick,smooth},%
  KIT smooth plot C/.style={%
    KITorange,semithick,smooth},%
  KIT smooth plot D/.style={%
    KITlilac,semithick,smooth},%
  KIT smooth plot E/.style={%
    KITbrown,semithick,smooth},%
  KIT smooth plot F/.style={%
    KITblue,semithick,dashed,smooth},%
  KIT smooth plot G/.style={%
    KITred,semithick,dashed,smooth},%
  KIT smooth plot H/.style={%
    KITorange,semithick,dashed,smooth},%
  KIT smooth plot I/.style={%
    KITlilac,semithick,dashed,smooth},%
  KIT smooth plot J/.style={%
    KITbrown,semithick,dashed,smooth},%
}

%\usepgfplotslibrary{external} 
%\tikzexternalize
%\usepgfplotslibrary{ternary}
%\tikzsetexternalprefix{./figures-compiled/}
%\tikzset{external/aux in dpth={false},external/disable dependency files}
%\tikzexternalize[shell escape=-enable-write18]

%\usepackage{listings} \lstset{numbers=left, numberstyle=\tiny, numbersep=5pt} %\lstset{language=Perl}

%Achtung! Verwendung von Glossaries bzw. des Befehls \makeglossaries benötigt Perl!!!!
%Unter Linux ist Perl vorhanden, unter Windows muss es installiert werden, s.
%http://www.mrunix.de/forums/showthread.php?68892-Tip-Glossaries-Paket-was-oft-falsch-l%E4uft
%bzw.
%http://latex-community.org/know-how/latex/55-latex-general/263-glossaries-nomenclature-lists-of-symbols-and-acronyms#texniccenter
%\usepackage[ngerman]{translator}
\usepackage[%
xindy={language=german-din,codepage=utf8},
%automake=immediate, %
%automake=true, %
translate=babel, % verwende Babel zur Übersetzung der Überschriften
%translate=false, % keine Übersetzung der Überschriften da ansonsten Zusatzpakete geladen werden müssen
nonumberlist,    % keine Seitenzahlen anzeigen
nopostdot,       % den Punkt am Ende jeder Beschreibung deaktivieren
acronym,         % ein Abkürzungsverzeichnis erstellen
shortcuts,       % Shortcuts aus dem Acronym-Package verwenden, z.B. as \ac, \acs, \acl, \acf, etc.
symbols,         % ein Symbolverzeichnis erstellen
toc,             % Einträge im Inhaltsverzeichnis erzeugen
section=chapter, % Einträge im Inhaltsverzeichnis auf chapter-Ebene erscheinen lassen
%numberedsection=autolabel,%Kapitel nicht nummerieren  => auskommentiert lassen!
%nohypertypes={glossary}, %nur den ersten Eintrag als Hyperlink anzeigen
nohypertypes={acronym,notation}, % keine Hyperlinks für Acronyme und Notation
%style=list      % Es wird später ein eigener Stil definiert mylongglosstyle!
%style=long
%style=super
]%
{glossaries}

%% Zusatzpakete für Glossar-Stile "longragged" und "longbooktabs"
%% Hack: stattdessen eigenes Stil "mylongglossstyle" definieren (s. Datei GlossaryOptions.tex)
%\usepackage{glossary-longragged} %zusätzlich laden um Glossareinträge linksbündig zu setzen (Forderung des KSP-Verlages)
%%\usepackage{glossary-longbooktabs}

