%% Define commands that don't eat spaces.
\usepackage{xspace}

%% IfThenElse: muss frueher eingebunden werden wegen Abfragen zur Sprache etc.
%\usepackage{ifthen}

%% for adding invisible comments
%% not needed anymore since included in package "versions"
%% Braucht ein \write. Vielleicht eins zuviel, Asymptote braucht auch welche, es gibt ing. nur 16.
%\usepackage{comment}
%\includecomment{showcomment}
%\excludecomment{hidecomment}

%for conditional text inclusions
\usepackage[nogroup]{versions}

%for compact lists
\usepackage{mdwlist}

%% Einstellungen für die Aufzählungen
%% Kein Abstand zwischen den Item-Einträgen
%\setlist{noitemsep}
%%Verkleinerter Abstand zwischen den Item-Einträgen
\setlist{%
		%before=\RaggedRight ,%Flattersatz in itemize- und enumerate-Umgebungen erzwingen
		%funktioniert nicht so wie gewünscht, s.
		%https://tex.stackexchange.com/questions/104088/problem-with-enumitem-and-raggedright
		%Workaround weiter unten mit \AtBeginEnvironment{itemize}{\preto\item{\RaggedRight}}
		%Ränder setzen:
		%leftmargin=2em, %linkes Rand explizit setzen
		%rightmargin=1em, %rechtes Rand vergrößern
		rightmargin=2mm,
		itemsep=0.2\baselineskip,% 
		parsep=0.2\baselineskip%
		%topsep=0.2\baselineskip,
		%partopsep=0.2\baselineskip,%
}
%Flattersatz in itemize- und enumerate-Umgebungen erzwingen
% Vorgabe vom KIT-Verlag: keine Worttrennung in Aufzählungen, daher \raggedright statt \RaggedRight
%\AtBeginEnvironment{itemize}{\preto\item{\RaggedRight}}
\AtBeginEnvironment{itemize}{\preto\item{\raggedright}}
\AtBeginEnvironment{enumerate}{\preto\item{\raggedright}}
%\AtBeginEnvironment{description}{\preto\item{\raggedright}}
\AtBeginEnvironment{itemize*}{\preto\item{\raggedright}}
\AtBeginEnvironment{enumerate*}{\preto\item{\raggedright}}
%\AtBeginEnvironment{description*}{\preto\item{\raggedright}}
%

%for allowing hyphenation of words that contain a dash
%using shortcuts \-/, \=/, \--, \==, \---, and \===
\usepackage[shortcuts]{extdash}

%Zur besseren Silbentrennung
%s. http://de.wikibooks.org/wiki/LaTeX-W%C3%B6rterbuch:_Silbentrennung
\usepackage[ngerman=ngerman-x-latest]{hyphsubst}

\usepackage[super]{nth}

\usepackage{tikz}
\usetikzlibrary{calc,arrows,positioning,shadows,backgrounds,chains,topaths,matrix,scopes,decorations,decorations.pathmorphing,decorations.markings,fadings,shapes.geometric,shapes.multipart,shapes.misc,through,automata,fit}

%Einschliessen von Matlab-Grafiken:
\usepackage{pgfplots}
\pgfplotsset{compat=1.14}
%Externe Generierung von plots, s. https://tex.stackexchange.com/questions/7953/how-to-expand-texs-main-memory-size-pgfplots-memory-overload
%% Braucht -shell-escape Kompilieroption!!!
%\usepgfplotslibrary{external} 
%\tikzexternalize


%\usepackage{listings} \lstset{numbers=left, numberstyle=\tiny, numbersep=5pt} %\lstset{language=Perl}

%Achtung! Verwendung von Glossaries bzw. des Befehls \makeglossaries benötigt Perl!!!!
%Unter Linux ist Perl vorhanden, unter Windows muss es installiert werden, s.
%http://www.mrunix.de/forums/showthread.php?68892-Tip-Glossaries-Paket-was-oft-falsch-l%E4uft
%bzw.
%http://latex-community.org/know-how/latex/55-latex-general/263-glossaries-nomenclature-lists-of-symbols-and-acronyms#texniccenter
%\usepackage[ngerman]{translator}
\usepackage[%
xindy,
%translate=true,
%translate=babel, %verwende Babel zur Übersetzung der Überschriften
translate=false, %keine Übersetzung der Überschriften da ansonsten Zusatzpakete geladen werden müssen
nonumberlist,    %keine Seitenzahlen anzeigen
nopostdot,       %den Punkt am Ende jeder Beschreibung deaktivieren
acronym,         %ein Abkürzungsverzeichnis erstellen
symbols,					%ein Symbolverzeichnis
toc,             %Einträge im Inhaltsverzeichnis
section=chapter, %im Inhaltsverzeichnis auf chapter-Ebene erscheinen
%numberedsection=autolabel,%Kapitel nicht nummerieren  0> auskommentiert lassen!
%nohypertypes={glossary}, %nur den ersten Eintrag als Hyperlink anzeigen
nohypertypes={acronym,notation}, % keine Hyperlinks für Acronyme und Notation
%style=list
%style=long
%style=super
]%
{glossaries}
\usepackage{glossary-longragged} %zusätzlich laden um Glossareinträge linksbündig zu setzen (Wunsch KIT-Verlag)
%\usepackage{glossary-longbooktabs}
\setglossarystyle{longragged} % kann erst gesetzt werden, nachdem das Paket glossary-longragged geladen worden ist
%\renewcommand*{\glspostdescription}{} %Zusätzlichen Punkt am Ende jeder Beschreibung deaktivieren
\glsdisablehyper %Deaktivieren von Hyperlinks auf das Glossar.
%%\newglossary[alg]{acronym}{acr}{acn}{\acronymname} %Unnötig durch die Option "acronym" des glossaries-Pakets.
\newglossary[nlg]{notation}{not}{ntn}{Notation}
\newglossary[slg]{symbolslist}{syi}{syg}{List of Symbols}
%Glossarentries sollen fett sein
\renewcommand{\glsnamefont}[1]{\textbf{#1}}
%%%makeglossaries muss nach \newglossary eingebunden werden!
%\makeglossaries
