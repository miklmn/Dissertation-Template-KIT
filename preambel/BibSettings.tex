%% Settings for biblatex

\usepackage[backref=true,%
						%backend=bibtex8,%
						backend=biber,%
						style=alphabetic,%
						labelnumber,% zusätzlich labelnummer zur Verfügung stel
						defernumbers=true,%
						%maxnames=10,% Maximal 10 Autorennamen zulassen, wenn >10, dann lasse nur einen + et al. (ggf. änderbar mit  der Option "minnames"
						maxbibnames=10,%
						maxcitenames=1,%
						maxalphanames=1,% Anzahl der aufzuführenden Namen in Bib-Label
						%dashed=false,%will print recurring author/editor names instead of replacing them by a dash in case of authoryear, authortitle, and verbose bibliography styles)
						isbn=false,%die ISBN-Felder nicht drucken
						%doi=false,%
						%eprint=false,%
						block=none,
						%block=space,% Füge zusätzlichen horizontalen Raum zwischen den Einträgen hinzu.
						%block=par,% Beginne einen neuen Paragraphen für jeden Eintrag
						%block=nbpar,% Ähnlich der par-Option, aber verbietet Seitensprünge zwischen den Übergängen und innerhalb der Einträge.
						%block=ragged,% Fügt einen kleinen Strafraum ein, um Zeilenumbrüche an Blockgrenzen zu fördern und die Bibliografie rechts orientiert (Flattersatz) zu setzen
						backref=false%Entscheidet,ob die Endreferenzen (Seitenzahlen) in die Bibliografie geschrieben werden sollen. (default=false)
						]{biblatex}

%% Das "+"-Zeichen nach mehr als <maxcitenames> Authoren entfernen durch Umdefinieren von "labelalphaothers":
\renewcommand*{\labelalphaothers}{}

%% Reihenfolge umdrehen: Nachname, Vorname
%\DeclareNameAlias{default}{last-first} %deprecated
\DeclareNameAlias{default}{family-given}

%% Separator zwischen den Autornamen: Semikolon statt Komma
%\renewcommand*{\multinamedelim}{;\space}
\DeclareDelimFormat{multinamedelim}{\addsemicolon\space}

%% Separator vor dem letzten den Autornamen: ohne Komma vor dem "and" bei mehreren Authoren
%% da Komma den Nach- und Vornamen trennt
\renewcommand*{\finalnamedelim}{\space and\space}

%% Nach Autornamen: Doppelpunkt statt Punkt:
\renewcommand*{\labelnamepunct}{\addcolon\addspace}

%% Vergrößerung des hängenden Einzuges
\setlength{\bibhang}{5em}
%\addtolength{\bibhang}{2em}


%% Nachnamen in Kapitälchen
\AtBeginBibliography{\renewcommand*{\mkbibnamefamily}[1]{\textsc{#1}}}

%% Titel: nicht kursiv
\DeclareFieldFormat{title}{{#1}}

%%% Hack, damit man mehrere Bibliografien mit verschiedenen Zitationsstilen hat:
\DeclareFieldFormat{labelnumberwidth}{\mkbibbrackets{#1}}

%% Mehrere Zitate in einer Klammer, aber Komma statt Semikolon:
\renewcommand*{\multicitedelim}{\addcomma\addspace}
%% Nicht mehrere Zitate in einer Klammer, sondern eine Klammer pro Zitat:
%\renewcommand*{\multicitedelim}{\bibclosebracket\addcomma\addspace\bibopenbracket}

%% Definition für die allgemeine Literaturliste
\defbibenvironment{bibliography}
  {\list
     {\printtext[labelalphawidth]{%
        \printfield{prefixnumber}%
        \printfield{labelalpha}}}
     {\setlength{\labelwidth}{\labelalphawidth}%
      \setlength{\leftmargin}{\labelwidth}%
      \setlength{\labelsep}{\biblabelsep}%
      \addtolength{\leftmargin}{2\labelsep}%
      \setlength{\itemindent}{-\labelsep}%
      \setlength{\itemsep}{\bibitemsep}%
      \setlength{\parsep}{\bibparsep}%
      %\addtolength{\bibhang}{3em}%
      }%
      %% \hss für rechtsbündige Ausrichtung
      %\renewcommand*{\makelabel}[1]{\hss##1}}
      \renewcommand*{\makelabel}[1]{##1}}
  {\endlist}
  {\item}


%% Definition für die Liste eigener Publikationen
\defbibenvironment{bibliographyNUM}
  {\list
     {\printtext[labelnumberwidth]{%
        \printfield{prefixnumber}%
        \printfield{labelnumber}}}
     {\setlength{\labelwidth}{\labelnumberwidth}%
      \setlength{\leftmargin}{\labelwidth}%
      \setlength{\labelsep}{\biblabelsep}%
      \addtolength{\leftmargin}{2\labelsep}%
      \setlength{\itemindent}{-\labelsep}%
      \setlength{\itemsep}{\bibitemsep}%
      \setlength{\parsep}{\bibparsep}%
      %\addtolength{\bibhang}{3em}%
      }%
      %% \hss für rechtsbündige Ausrichtung
      %\renewcommand*{\makelabel}[1]{\hss##1}}
      \renewcommand*{\makelabel}[1]{##1}}
  {\endlist}
  {\item}