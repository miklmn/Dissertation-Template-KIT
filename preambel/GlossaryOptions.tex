%%%%%%%%%%%%%%%%%%%%%%%%%%%%%%%%%%%%%%%%%%%%%%%%%%%%%%%%%%%%%%%%%%%%%%%%%%%%%%%%%%%%%%%%%%%%%%%%
%% Hack for correct glossary width in case of too long entries:
%% define an own glossary style "mylongglossstyle"
%% (see https://tex.stackexchange.com/questions/25380/glossaries-printglossaries-prints-too-wide)
%%%%%%%%%%%%%%%%%%%%%%%%%%%%%%%%%%%%%%%%%%%%%%%%%%%%%%%%%%%%%%%%%%%%%%%%%%%%%%%%%%%%%%%%%%%%%%%%
\newlength{\myglstargetwidth}
\newlength{\myglshspace}
\newlength{\myglsdescwidth}
\newglossarystyle{mylongglossstyle}{%
% put the glossary in the longtable environment:
	\renewenvironment{theglossary}{%
		\setlength{\myglstargetwidth}{0.19\textwidth}%
		\setlength{\myglshspace}{0.02\textwidth}%
		\setlength{\myglsdescwidth}{0.79\textwidth}%
		\setlength{\tabcolsep}{0pt}%
		\setlength{\extrarowheight}{12pt}%
		\begin{longtable}{p{\myglstargetwidth} @{\hspace{\myglshspace}} p{\myglsdescwidth}}
	}{%
	  \end{longtable}%
	}%
	% have nothing after \begin{theglossary}:
	\renewcommand*{\glossaryheader}{}%
	% have nothing between glossary groups:
	\renewcommand*{\glsgroupheading}[1]{}%
	\renewcommand*{\glsgroupskip}{}%
	% set how each entry should appear:
	\renewcommand*{\glossaryentryfield}[5]{%
		\raggedright\strong{\glstarget{##1}{##2}}% the entry name
		##4% the symbol 
		&##3%,% the description
		%\space%
		%##5% the number list 
		\\%
	}%
	% set how sub-entries appear:
	\renewcommand*{\glossarysubentryfield}[6]{%
		\glossaryentryfield{##2}{##3}{##4}{##5}{##6}
	}%
}

% setglossarystyle must be issued before \printglossaries.
% the longragged style can be set only after the corresponding package (i.e. glossary-longragged) has been loaded
%\setglossarystyle{longragged}
\setglossarystyle{mylongglossstyle}

%% Glossarentries sollen fett sein
\renewcommand{\glsnamefont}[1]{\textbf{#1}}
%% Glossareintäge (auch Links) schwarz
\renewcommand*{\glstextformat}[1]{\textcolor{black}{#1}}%

%% Zusätzlichen Punkt am Ende jeder Beschreibung deaktivieren
%\renewcommand*{\glspostdescription}{}

%% Deaktivieren von Hyperlinks auf das Glossar.
\glsdisablehyper

% Hinzufügen verschieder Zusatz-Formen für Glossareinträge ermöglichen (Genitiv, Dativ + Plural)
% s. https://tex.stackexchange.com/questions/178725/how-to-use-glossaries-for-different-grammatical-acronym-forms
% genitive
\glsaddkey*
 {genitive}% key
 {\acrshort{\glslabel}}% default value
 {\glsentrygenitive}% no link cs
 {\Glsentrygenitive}% no link ucfirst cs
 {\glsgenitive}% link cs
 {\Glsgenitive}% link ucfirst cs
 {\GLSgenitive}% link all caps cs

% dative
\glsaddkey*
 {dative}% key
 {\acrshort{\glslabel}}% default value
 {\glsentrydative}% no link cs
 {\Glsentrydative}% no link ucfirst cs
 {\glsdative}% link cs
 {\Glsdative}% link ucfirst cs
 {\GLSdative}% link all caps cs

% accusative
\glsaddkey*
 {accusative}% key
 {\acrshort{\glslabel}}% default value
 {\glsentryaccusative}% no link cs
 {\Glsentryaccusative}% no link ucfirst cs
 {\glsaccusative}% link cs
 {\Glsaccusative}% link ucfirst cs
 {\GLSaccusative}% link all caps cs

% short genitive
\glsaddkey*
 {shortgenitive}% key
 {\acrshort{\glslabel}}% default value
 {\glsentryshortgenitive}% no link cs
 {\Glsentryshortgenitive}% no link ucfirst cs
 {\glsshortgenitive}% link cs
 {\Glsshortgenitive}% link ucfirst cs
 {\GLSshortgenitive}% link all caps cs

% short dative
\glsaddkey*
 {shortdative}% key
 {\acrshort{\glslabel}}% default value
 {\glsentryshortdative}% no link cs
 {\Glsentryshortdative}% no link ucfirst cs
 {\glsshortdative}% link cs
 {\Glsshortdative}% link ucfirst cs
 {\GLSshortdative}% link all caps cs

% short accusative
\glsaddkey*
 {shortaccusative}% key
 {\acrshort{\glslabel}}% default value
 {\glsentryshortaccusative}% no link cs
 {\Glsentryshortaccusative}% no link ucfirst cs
 {\glsshortaccusative}% link cs
 {\Glsshortaccusative}% link ucfirst cs
 {\GLSshortaccusative}% link all caps cs

% genitive plural
\glsaddkey*
 {pluralgenitive}% key
 {\acrshort{\glslabel}}% default value
 {\glsentrypluralgenitive}% no link cs
 {\Glsentrypluralgenitive}% no link ucfirst cs
 {\glsplgenitive}% link cs
 {\Glsplgenitive}% link ucfirst cs
 {\GLSplgenitive}% link all caps cs

% dative plural
\glsaddkey*
 {pluraldative}% key
 {\acrshort{\glslabel}}% default value
 {\glsentrypluraldative}% no link cs
 {\Glsentrypluraldative}% no link ucfirst cs
 {\glspldative}% link cs
 {\Glspldative}% link ucfirst cs
 {\GLSpldative}% link all caps cs

% accusative plural
\glsaddkey*
 {pluralaccusative}% key
 {\acrshort{\glslabel}}% default value
 {\glsentrypluralaccusative}% no link cs
 {\Glsentrypluralaccusative}% no link ucfirst cs
 {\glsplaccusative}% link cs
 {\Glsplaccusative}% link ucfirst cs
 {\GLSplaccusative}% link all caps cs

% short genitive plural
\glsaddkey*
 {shortpluralgenitive}% key
 {\acrshort{\glslabel}}% default value
 {\glsentryshortpluralgenitive}% no link cs
 {\Glsentryshortpluralgenitive}% no link ucfirst cs
 {\glssplgenitive}% link cs
 {\Glssplgenitive}% link ucfirst cs
 {\GLSsplgenitive}% link all caps cs

% short dative plural
\glsaddkey*
 {shortpluraldative}% key
 {\acrshort{\glslabel}}% default value
 {\glsentryshortpluraldative}% no link cs
 {\Glsentryshortpluraldative}% no link ucfirst cs
 {\glsspldative}% link cs
 {\Glsspldative}% link ucfirst cs
 {\GLSspldative}% link all caps cs

% short accusative plural
\glsaddkey*
 {shorttpluralaccusative}% key
 {\acrshort{\glslabel}}% default value
 {\glsentryshortpluralaccusative}% no link cs
 {\Glsentryshortpluralaccusative}% no link ucfirst cs
 {\glssplaccusative}% link cs
 {\Glssplaccusative}% link ucfirst cs
 {\GLSsplaccusative}% link all caps cs


% command for usage of genitive:
\newcommand{\glsgen}[1]{%
  \glsdoifexists{#1}{% do something only if the glossary entry has been defined
	\ifthenelse{\equal{\glsentrytype{#1}}{acronym}}{% if the entry is an acronym:
       \ifglsused{#1}{% if this acronym has been used:
	     \glsshortgenitive{#1}% use only the short form
       }{% esle (acronym has not been used yet):
         \glsgenitive{#1} (\glsshortgenitive{#1})% use the long genitive form followed by the short form
         \glsunset{#1}% unset the "first use" flag
       }%
    }{% else (the entry is not an acronym):
       \glsgenitive{#1}%
    }%% fi
  }%
}

% ommand for usage of dative
\newcommand{\glsdat}[1]{%
  \glsdoifexists{#1}{% do something only if the glossary entry has been defined
	\ifthenelse{\equal{\glsentrytype{#1}}{acronym}}{% if the entry is an acronym:
       \ifglsused{#1}{% if this acronym has been used:
	     \glsshortdative{#1}% use only the short form
       }{% esle (acronym has not been used yet):
         \glsdative{#1} (\glsshortdative{#1})% use the long genitive form followed by the short form
         \glsunset{#1}% unset the "first use" flag
       }%
    }{% else (the entry is not an acronym):
       \glsdative{#1}%
    }%% fi
  }%
}

% ommand for usage of accusative
\newcommand{\glsacc}[1]{%
  \glsdoifexists{#1}{% do something only if the glossary entry has been defined
	\ifthenelse{\equal{\glsentrytype{#1}}{acronym}}{% if the entry is an acronym:
       \ifglsused{#1}{% if this acronym has been used:
	     \glsshortaccusative{#1}% use only the short form
       }{% esle (acronym has not been used yet):
         \glsaccusative{#1} (\glsshortacccusative{#1})% use the long accusative form followed by the short form
         \glsunset{#1}% unset the "first use" flag
       }%
    }{% else (the entry is not an acronym):
       \glsaccusative{#1}%
    }%% fi
  }%
}

% command for usage of plural genitive:
\newcommand{\glsplgen}[1]{%
  \glsdoifexists{#1}{% do something only if the glossary entry has been defined
	\ifthenelse{\equal{\glsentrytype{#1}}{acronym}}{% if the entry is an acronym:
       \ifglsused{#1}{% if this acronym has been used:
	     \glssplgenitive{#1}% use only the short form
       }{% esle (acronym has not been used yet):
         \glsplgenitive{#1} (\glssplgenitive{#1})% use the long form followed by the short form
         \glsunset{#1}% unset the "first use" flag
       }%
    }{% else (the entry is not an acronym):
       \glsplgenitive{#1}%
    }%% fi
  }%
}

% command for usage of plural dative
\newcommand{\glspldat}[1]{%
  \glsdoifexists{#1}{% do something only if the glossary entry has been defined
	\ifthenelse{\equal{\glsentrytype{#1}}{acronym}}{% if the entry is an acronym:
       \ifglsused{#1}{% if this acronym has been used:
	     \glsspldative{#1}% use only the short form
       }{% esle (acronym has not been used yet):
         \glsspldative{#1} (\glspldative{#1})% use the long form followed by the short form
         \glsunset{#1}% unset the "first use" flag
       }%
    }{% else (the entry is not an acronym):
       \glspldative{#1}%
    }%% fi
  }%
}

% command for usage of plural accusative
\newcommand{\glsplacc}[1]{%
  \glsdoifexists{#1}{% do something only if the glossary entry has been defined
	\ifthenelse{\equal{\glsentrytype{#1}}{acronym}}{% if the entry is an acronym:
       \ifglsused{#1}{% if this acronym has been used:
	     \glssplaccusative{#1}% use only the short form
       }{% esle (acronym has not been used yet):
         \glssplaccusative{#1} (\glsplaccusative{#1})% use the long form followed by the short form
         \glsunset{#1}% unset the "first use" flag
       }%
    }{% else (the entry is not an acronym):
       \glsplaccusative{#1}%
    }%% fi
  }%
}

%Define shortcuts similar to \ac \acl, acf, acp etc.
\newcommand{\acgen}[1]{\glsgen{#1}}
\newcommand{\acdat}[1]{\glsdat{#1}}
\newcommand{\acacc}[1]{\glsacc{#1}}
\newcommand{\acpgen}[1]{\glsplgen{#1}}
\newcommand{\acpdat}[1]{\glspldat{#1}}
\newcommand{\acpacc}[1]{\glsplacc{#1}}
\newcommand{\acsgen}[1]{\glsshortgenitive{#1}}
\newcommand{\acsdat}[1]{\glsshortdative{#1}}
\newcommand{\acsacc}[1]{\glsshortacccusative{#1}}
\newcommand{\aclgen}[1]{\glsgenitive{#1}}
\newcommand{\acldat}[1]{\glsdative{#1}}
\newcommand{\aclacc}[1]{\glsaccusative{#1}}
% weitere noch zu definieren...


%%\newglossary[alg]{acronym}{acr}{acn}{\acronymname} %Unnötig durch die Option "acronym" des glossaries-Pakets.
\newglossary[nlg]{notation}{not}{ntn}{Notation}
%\newglossary[slg]{symbols}{sls}{slo}{\glssymbolsgroupname}

%%%makeglossaries muss nach \newglossary eingebunden werden!
%\makeglossaries
