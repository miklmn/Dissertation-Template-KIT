%%%%%%%%%% HAUPTDOKUMENT DER LATEX-VORLAGE DES IES %%%%%%%%%%%%%%%
%% Im wesentlichen basierend auf der Vorlage von Matthias Pospiech
%% http://www.matthiaspospiech.de/latex/vorlagen/allgemein/
%% für KOMA-Script 3.x
%% Erweitert und angepasst von Philipp Woock und Michael Grinberg
%% Version 3.0
%% Februar 2019
%%%%%%%%%%%%%%%%%%%%%%%%%%%%%%%%%%%%%%%%%%%%%%%%%%%%%%%%%%%%%%%%%%

%% PW: Paket silence unterdrückt Warnungen. Schreibt die unterdrückten Sachen aber in eine .sil Datei
%% Silence braucht für save auch ein TeX \write :-(
% \usepackage[debrief, save]{silence}
%\RequirePackage[options]{silence} vor \documentclass{}
% PW: Ausschalten bekannter Warnungen.
\RequirePackage{silence}
\WarningFilter{typearea}{Maybe no optimal type area settings}
\WarningFilter{typearea}{Bad type area settings}
\WarningFilter{Mathdots}{Redefining amsmath commands}
\WarningFilter{latexfont}{Font shape}
\WarningFilter{pdfpages}{I will use a dummy}
\WarningFilter{caption}{Unused}
\WarningFilter{hyperref}{Rerun to get /PageLabels}
\WarningFilter{scrwfile}{THIS IS AN ALPHA VERSION}
\WarningFilter*{ifplatform}{^^J \space\space\space shell escape is disabled}

%% Dokumentenklasse (Koma Script) -----------------------------------------
\documentclass[%
  % Sprachunterstützung:
  % (Dokumenthauptsprache ist die, die zuletzt genannt ist - sie wird 
  % für die Beschriftung von Abbildungen und Verzeichnissen verwendet)
%  ngerman,% Deutsch (neue Rechtschreibung)
%  english,% Englisch
  %draft,     % Entwurfsstadium
  final,      % fertiges Dokument
	% --- Paper Settings ---
	%% Für Dissertationen nimmt KIT Scientific Publishing sowohl A4 als auch A5.
	%% A4 wird einfach runterskaliert, muss aber entsprechend eine größere
	%% Schrift haben, was ggf. Nacharbei mit sich ziehen kann.
	%% Daher besser direkt im A5-Format
  %paper=a4,
  paper=a5,
	%% Hochformat/Querformat
  paper=portrait, % landscape
  pagesize=auto, % driver
  % --- Base Font Size ---
	%% Schriftgröße
  %fontsize=9pt,%  9pt ist dem KIT-Verlag bei A5 aktuell zu klein
  fontsize=10pt,%  bei A5 ist 10pt Mindest-Schriftgröße (Forderung des KSP-Verlages)
	%fontsize=14pt,%  % 11 bei A4.
	% --- Koma Script Version ---
  version=last %,
% ]{scrbook} % Mögliche Klassen scrartcl, scrreprt, scrbook
 ]{scrreprt} % 

\usepackage{scrhack}  %um Warnungen zu beheben

% Verwendung von ifthenelse
\usepackage{ifthen}

%%% Doc: ftp://tug.ctan.org/pub/tex-archive/macros/latex/contrib/xcolor/xcolor.pdf
% Farben
% Muss früher geladen werden, da ansonsten Optionen-Clash mit xelibertine?
% Incompatible: Do not load when using pstricks !
\usepackage[%
	table, % Load for using rowcolors command in tables
	%cmyk, % CMYK Farbraum
	dvipsnames % for using the extended name set
]{xcolor}
\usepackage{KAcolors}

% Encoding der Quellcode-Dateien (sonst funktionieren Umlaute in den Quellcodedateien nicht)
%%% Doc: ftp://tug.ctan.org/pub/tex-archive/macros/latex/required/amslatex/math/amsldoc.pdf
\usepackage{ifxetex}
%
%%% Doc: ftp://tug.ctan.org/pub/tex-archive/macros/latex/contrib/oberdiek/ifpdf.sty
% command for testing for pdf-creation
\usepackage{ifpdf} %\ifpdf \else \fi
%
% Amsmath - Mathematik Basispaket
%
% fuer pst-pdf displaymath Modus vor pst-pdf benoetigt.
\usepackage[%
   centertags, % (default) center tags vertically
   %tbtags,    % 'Top-or-bottom tags': For a split equation, place equation numbers level
               % with the last (resp. first) line, if numbers are on the right (resp. left).
   sumlimits,  %(default) Place the subscripts and superscripts of summation
               % symbols above and below
   %nosumlimits, % Always place the subscripts and superscripts of summation-type
               % symbols to the side, even in displayed equations.
   intlimits,  % Like sumlimits, but for integral symbols.
   %nointlimits, % (default) Opposite of intlimits.
   namelimits, % (default) Like sumlimits, but for certain 'operator names' such as
               % det, inf, lim, max, min, that traditionally have subscripts placed underneath
               % when they occur in a displayed equation.
   %nonamelimits, % Opposite of namelimits.
   %leqno,     % Place equation numbers on the left.
   %reqno,     % Place equation numbers on the right.
   fleqn,     % Position equations at a fixed indent from the left margin
   			  % rather than centered in the text column.
]{amsmath} %
%
\usepackage{amsfonts}
%\usepackage{amssymb} %bei Kompilierung mit LaTeX verursacht probleme mit tikz
\usepackage[fixamsmath,disallowspaces]{mathtools}
\usepackage{fixmath}
% eqnarray nicht zusammen mit amsmath benutzen, siehe l2tabu.pdf für
% Hintergruende.
%
%--------------------------------
% Hier die Schriftart auswählen:
%--------------------------------
%\def\UseLibertine{true} %Problem bei Bibliografie: fett und kursiv statt nur kursiv
%\def\UseLibertinus{true}
%\def\UseSTIX{true}
%\def\UseXITS{true}
\def\UseLibertinusSTIXMix{true}
%\def\UseLibertinusFrutigerSTIXMix{true}
%--------------------------------
%
%
\makeatletter
\ifxetex %Nutzung von XeLaTeX anstatt nur LaTeX:
  \usepackage{xltxtra}
	\usepackage{unicode-math}
	%--------------------------------------------------
	%  Libertine Fonts
	%--------------------------------------------------
	\ifdefined\UseLibertine
		\setmainfont[
								%Scale = 0.95,
								UprightFont = *_R_G,
								ItalicFont = *_RI_G,
								BoldFont = *_RB_G,
								BoldItalicFont = *_RBI_G,
								SmallCapsFeatures={Letters=SmallCaps},
								Path=./fonts/,
								Extension = .ttf
								]{LinLibertine}
		\setsansfont[
								%Scale = 0.85,
								UprightFont = *-regular,
								ItalicFont = *-italic,
								BoldFont = *-bold,
								%BoldItalicFont = *-bolditalic,
								SmallCapsFeatures={Letters=SmallCaps},
								Path=./fonts/,
								Extension = .otf
								]{libertinussans}
		\setmonofont[
								%Scale = 0.95,
								FakeStretch = 0.8,
								UprightFont = *-regular,
								%ItalicFont = *-italic,
								%BoldFont = *-bold,
								% AutoFakeBold = 1.5,
								Path=./fonts/,
								Extension = .otf
								]{libertinusmono}
		\setmathfont[
								%Scale = 0.95,
								Path=./fonts/,
								Extension = .otf
								]{libertinusmath-regular}
	\fi
	%--------------------------------------------------
	%  Libertinus Fonts
	%--------------------------------------------------
	\ifdefined\UseLibertinus
		\setmainfont[
								%Scale = 0.95,
								UprightFont = *-regular,
								ItalicFont = *-italic,
								BoldFont = *-bold,
								BoldItalicFont = *-bolditalic,
								SmallCapsFeatures={Letters=SmallCaps},
								Path=./fonts/,
								Extension = .otf
								]{libertinusserif}
		\setsansfont[
								%Scale = 0.85,
								UprightFont = *-regular,
								ItalicFont = *-italic,
								BoldFont = *-bold,
								%BoldItalicFont = *-bolditalic,
								SmallCapsFeatures={Letters=SmallCaps},
								Path=./fonts/,
								Extension = .otf
								]{libertinussans}
		\setmonofont[
								%Scale = 0.95,
								FakeStretch = 0.8,
								UprightFont = *-regular,
								%ItalicFont = *-italic,
								%BoldFont = *-bold,
								% AutoFakeBold = 1.5,
								Path=./fonts/,
								Extension = .otf
								]{libertinusmono}
		\setmathfont[
								%Scale = 0.95,
								Path=./fonts/,
								Extension = .otf
								]{libertinusmath-regular}
	\fi
	%--------------------------------------------------
	%  STIX Fonts; use STIX-Regular as SANS font
	%--------------------------------------------------
	\ifdefined\UseSTIX
		\setmainfont[
								UprightFont = *-Regular,
								ItalicFont = *-Italic,
								BoldFont = *-Bold,
								BoldItalicFont = *-BoldItalic,
								SmallCapsFeatures={Letters=SmallCaps},
								Extension = .otf
								]{STIX2Text}
		\setsansfont[
								Path=./fonts/,
								Extension = .otf
								]{STIX2Text-Regular}
		\setmathfont[
								Path=./fonts/,
								Extension = .otf
								]{STIX2Math}
	\fi
	%--------------------------------------------------
	%  XITS Fonts; use xits-regular as SANS font
	%--------------------------------------------------
	\ifdefined\UseXITS
		\setmainfont[
								UprightFont = *-regular,
								ItalicFont = *-italic,
								BoldFont = *-bold,
								BoldItalicFont = *-bolditalic,
								SmallCapsFeatures={Letters=SmallCaps},
								Path=./fonts/,
								Extension = .otf
								]{xits}
		\setsansfont[
								Path=./fonts/,
								Extension = .otf
								]{xits-regular}
		\setmathfont[
								Path=./fonts/,
								Extension = .otf
								]{xits-math}
	\fi
	%--------------------------------------------------
	%  Libertinus as the main font, STIX as math font
	%--------------------------------------------------
	\ifdefined\UseLibertinusSTIXMix
		\setmainfont[
								%Scale = 0.95,
								UprightFont = *-regular,
								ItalicFont = *-italic,
								BoldFont = *-bold,
								BoldItalicFont = *-bolditalic,
								SmallCapsFeatures={Letters=SmallCaps},
								Path=./fonts/,
								Extension = .otf
								]{libertinusserif}
		\setsansfont[
								%Scale = 0.85,
								UprightFont = *-regular,
								ItalicFont = *-italic,
								BoldFont = *-bold,
								%BoldItalicFont = *-bolditalic,
								SmallCapsFeatures={Letters=SmallCaps},
								Path=./fonts/,
								Extension = .otf
								]{libertinussans}
		\setmonofont[
								Scale = 0.89,
								FakeStretch = 0.89,
								UprightFont = *-regular,
								Path=./fonts/,
								Extension = .otf
								]{libertinusmono}
		\setmathfont[
								%Scale = 0.85,
								Path=./fonts/,
								Extension = .otf
								]{STIX2Math}
	\fi
	\ifdefined\UseLibertinusFrutigerSTIXMix
		\setmainfont[
								%Scale = 0.95,
								UprightFont = *-regular,
								ItalicFont = *-italic,
								BoldFont = *-bold,
								BoldItalicFont = *-bolditalic,
								SmallCapsFeatures={Letters=SmallCaps},
								Path=./fonts/,
								Extension = .otf
								]{libertinusserif}
		\setsansfont[
								%Scale = 0.90,
								UprightFont = *-Roman,
								ItalicFont = *-Italic,
								BoldFont = *-Bold,
								BoldItalicFont = *-BoldItalic,
								SmallCapsFeatures={Letters=SmallCaps},
								%Path=./fonts/,
								Extension = .ttf
								]{FrutigerLTCom}
		\setmonofont[
								Scale = 0.875,
								FakeStretch = 0.85,
								UprightFont = *-regular,
								Path=./fonts/,
								Extension = .otf
								]{libertinusmono}
		\setmathfont[
								%Scale = 0.85,
								Path=./fonts/,
								Extension = .otf
								]{STIX2Math}
	\fi
\else %falls doch LaTeX anstatt XeLaTeX verwendet wird soll:
  \GenericError{}{This template should be compiled with xelatex! Please use xelatex instead of pdflatex as compiler}{}{}%
	\usepackage[utf8]{inputenc} %Die Dateien liegen in der UTF-8-Kodierung vor
	\usepackage[T1]{fontenc} %Der Text enthält Umlaute
	\usepackage{sanitize-umlaut} %behebt Probleme mit Umlauten im Stichwortverzeichnis
	\usepackage[ttscale=.875]{libertine}
	\usepackage{libertinust1math}
	%% Alternative:
	%\usepackage{mathptmx}
	%\usepackage[scaled=.92]{helvet}
	%\usepackage{courier}
	%
	%% Workaround for a TikZ-related error :
	%% https://tex.stackexchange.com/questions/165929/semiverbatim-with-tikz-in-beamer/165937#165937
	%\global\let\tikz@ensure@dollar@catcode=\relax
\fi
%}
\makeatother

%%%%%%%%%%%%%%%% HIER EINSTELLEN, OB ENGLISCH ODER DEUTSCH USW.
%%% Angaben zum Dokument (Autor, Titel etc.)
% ******  Angaben zur Arbeit ******


%% Booleans, mit deren Hilfe wichtige Festlegungen gemacht werden.
%% Belegung mit "true" und "false" weiter unten
\newboolean{iesenglishs}     % Hauptsprache der Arbeit: Englisch oder Deutsch
\newboolean{useiosblogo}     % Soll zusätzlich das IOSB-Logo verwendet werden
\newboolean{isdissertation}  % Ist die Arbeit eine Dissertation?
\newboolean{printMuster}     % Die Seiten mit dem Wort "Muster" bedrucken
\newboolean{coloredlinks}    % Sollen Links farbig dargestellt werden (oder schwarz für den Druck)?
\newboolean{coloredlistings} % Sollen Listings farbig dargestellt werden (oder schwarz für den Druck)?
\newboolean{useCMYKcolors}   % Sollen alle Farben in den CMYK-Farbraum konvertiert werden (für den Druck)?
\newboolean{istgenehmigt}    % Bei Dissertationen: handelt es sich um die genehmigte Version?
\newboolean{showFrame}       % Sollen die Satzspiegel-Ränder angezeigt werden?
\newboolean{showGrid}        % soll ein Gitter angezeigt werden?

%% Hier true oder false auswählen!!!
\setboolean{iesenglishs}{false}
\setboolean{useiosblogo}{true}
\setboolean{isdissertation}{true}
\setboolean{printMuster}{false}
\setboolean{coloredlinks}{true}
\setboolean{coloredlistings}{true}
\setboolean{useCMYKcolors}{false}
\setboolean{istgenehmigt}{false}
\setboolean{showFrame}{false}
\setboolean{showGrid}{false}

% Titel der Arbeit
\newcommand{\Worktitle}{Hier den kompletten Titel der Arbeit angeben}
\newcommand{\WorktitleDivided}{Hier den manuell\\umbrochenen Titel der Arbeit\\für die Titelseite angeben}

% Typ der Arbeit angeben
\newcommand{\TypeOfThesis}{%
Dissertation%
%Seminararbeit%
%Masterarbeit%
%Bachelorarbeit%
%Diplomarbeit%
%Studienarbeit%
%Technischer Bericht%
%PhD thesis%
%Master thesis%
%Bachelor thesis%
%Diploma thesis%
%Term paper%
%Technical report%
}

\newcommand{\Authortitle}{Dipl.-Ing.~}
\newcommand{\AuthorFirstname}{Max}
\newcommand{\AuthorSurname}{Mustermann}
\newcommand{\Authorname}{\AuthorFirstname \  \AuthorSurname}
\newcommand{\AuthorFull}{\Authortitle \Authorname}
\newcommand{\PlaceOfBirth}{Kiel}
\newcommand{\DocDegreeVorgelegt}{Doktors der Ingenieurwissenschaften /\newline Doktors der Naturwissenschaften}
\newcommand{\DocDegreeGenehmigt}{Doktors der Ingenieurwissenschaften}
\ifthenelse{\boolean{istgenehmigt}}{%
		\newcommand{\DocDegree}{\DocDegreeGenehmigt}
}{%
		\newcommand{\DocDegree}{\DocDegreeVorgelegt}
}%

\newcommand{\Uni}{Karlsruhe Institut für Technologie (KIT)}
\newcommand{\Institut}{Institut für Anthropomatik}
\newcommand{\Lehrstuhl}{Lehrstuhl für Interaktive Echtzeitsysteme}
\newcommand{\Lehrstuhlinhaber}{Prof.~Dr.-Ing.~habil.~Jürgen~Beyerer}

\newcommand{\MainAdvisor}{Prof.~Dr.-Ing.~habil.~Jürgen~Beyerer}
\newcommand{\CoAdvisor}{Zweitgutachter}
\newcommand{\Reviewer}{\MainAdvisor}
\newcommand{\Supervisor}{\CoAdvisor}


\newcommand{\Submissiondate}{\today} %entsprechend ändern
\newcommand{\Signplace}{Karlsruhe}
\newcommand{\ExamDate}{12.12.2017}

%\newcommand{\Widmung}{Für meine Eltern}

\newcommand{\Strassenadresse}{Strassenadresse hier einfügen\\PLZ und Ort}
\newcommand{\eMailAdresse}{test@test.de}
\newcommand{\Adresse}{
\Verfasser \\
\Strassenadresse
}


\newcommand{\Arbeitsart}{\TypeOfThesis}
\newcommand{\Arbeitstitel}{\Worktitle}
\newcommand{\Verfassertitel}{\Authortitle}
\newcommand{\Verfasservorname}{\AuthorFirstname}
\newcommand{\Verfassernachname}{\AuthorSurname}
\newcommand{\Verfasser}{\Verfasservorname \  \Verfassernachname}
\newcommand{\Verfasserkomplett}{\Verfassertitel \Verfasser}
\newcommand{\Abgabedatum}{\Submissiondate}
\newcommand{\Ort}{\Signplace}
\newcommand{\Hauptreferent}{\MainAdvisor}
\newcommand{\Korreferent}{\CoAdvisor}
\newcommand{\Betreuer}{\Supervisor}


%%% Einstellungen zu Seitenlayout, Abständen etc.
%Tiefe des Inhaltsverzeichnisses setzen:
% bei Report: 3 geht bis subsection
\newcommand{\mytocdepth}{2}
%Tiefe des Inhaltsverzeichnisses in PDF-Lesezeichen setzen (evtl. höher als die des toc)
%\newcommand{\mybookmarkdepth}{\mytocdepth}
\newcommand{\mybookmarkdepth}{3}
%
% Hier Breite von marginnotes setzen
\newlength{\mymarginparwidth}
%\setlength{\mymarginparwidth}{0.15\paperwidth}
\setlength{\mymarginparwidth}{10mm}
%\setlength{\mymarginparwidth}{0pt}
% Abstand zwischen Seitennotizen und Text setzen
\newlength{\mymarginparsep}
%\setlength{\mymarginparsep}{}
\setlength{\mymarginparsep}{3mm}
%\setlength{\mymarginparsep}{5mm}
%\setlength{\mymarginparsep}{0pt}

% Abstand zwischen den Paragraphen
\newlength{\myparskip}
%\setlength{\myparskip}{0.5\baselineskip plus 0.5\baselineskip} % das wäre wohl default
%\setlength{\myparskip}{0.5\baselineskip} %ohne Pluswerte (= keine Dehnung)
%\setlength{\myparskip}{0.6\baselineskip} %ohne Pluswerte (= keine Dehnung)
\setlength{\myparskip}{0.6\baselineskip plus 1pt minus 1pt} %mit Pluswerten (= Dehnung) und Minuswerten (= Stauchung)

\newlength{\mychapterbeforeskip}
\newlength{\mychapterafterskip}
\newlength{\mysectionbeforeskip}
\newlength{\mysectionafterskip}
\newlength{\mysubsectionbeforeskip}
\newlength{\mysubsectionafterskip}
\newlength{\mysubsubsectionbeforeskip}
\newlength{\mysubsubsectionafterskip}
\newlength{\myparagraphbeforeskip}
\newlength{\myparagraphafterskip}

\setlength{\mychapterbeforeskip}{8\myparskip}
\setlength{\mychapterafterskip}{2.5\myparskip}
%
\setlength{\mysectionbeforeskip}{2\myparskip}
\setlength{\mysectionafterskip}{1.5\myparskip}
%
\setlength{\mysubsectionbeforeskip}{1.5\myparskip}
\setlength{\mysubsectionafterskip}{1\myparskip}
%
\setlength{\mysubsubsectionbeforeskip}{1.5\myparskip}
\setlength{\mysubsubsectionafterskip}{1\myparskip}
%
\setlength{\myparagraphbeforeskip}{1.5\myparskip}
\setlength{\myparagraphafterskip}{1\myparskip}

%Abstand zwischen der Abbilung/Tabelle und der Überschrift:
%funktioniert aus irgendeinem Grund nicht!
\newlength{\myaboveskip}
%\setlength{\myaboveskip}{10pt}
\setlength{\myaboveskip}{0pt}
%Abstand nach der Überschrift einer Abbilung/Tabelle:
\newlength{\mybelowskip}
%\setlength{\mybelowskip}{10pt}
\setlength{\mybelowskip}{0pt}
%Abstand zwischen der Abbilung/Tabelle und der Überschrift:
\newlength{\mysubcaptionskip}
\setlength{\mysubcaptionskip}{6pt}
%Abstand zwischen der Abbilung/Tabelle und der Überschrift:
\newlength{\mycaptionskip}
\setlength{\mycaptionskip}{5pt}
%Abstand zwischen zwei Subfloats (muss manuell hinzugefügt werden)
\newlength{\mysubfloatvskip}
\setlength{\mysubfloatvskip}{10pt} % (default 12pt plus 2pt minus 2pt)

%%% Vertikaler Abstand zwischen zwei Gleitobjekten
\newlength{\myfloatsep}
%\setlength{\myfloatsep}{\baselineskip} % (default 12pt plus 2pt minus 2pt)
\setlength{\myfloatsep}{\baselineskip}
%\setlength{\myfloatsep}{12pt}

%%% Vertikaler Abstand zwischen der Top- oder Bottom-Bereich und Text-Bereich
\newlength{\mytextfloatsep}
%\setlength{\mytextfloatsep}{\baselineskip} % (default 20pt plus 2pt minus 4pt)
%\setlength{\mytextfloatsep}{\myfloatsep}
\setlength{\mytextfloatsep}{1.5\baselineskip}
%\setlength{\mytextfloatsep}{12pt}

%% Abstand zwischen Inline-Gleitobjekten, die mit "here" platziert worden sind,
%% und dem darüber und darunter angeordneten Fließtext fest
%% Beispiel: \intextsep5mm plus3mm minus2mm
%\setlength{\intextsep}{0.5\baselineskip} % Platz ober- und unterhalb des Bildes
\newlength{\myintextsep}
\setlength{\myintextsep}{\baselineskip} % (default 12pt plus 2pt minus 2pt) 
%\setlength{\myintextsep}{10pt} % (default 12pt plus 2pt minus 2pt) 

%% Weitere Angaben:
%% Forderung des KSP-Verlages:
%usepackage[a5paper,headheight=1.5\baselineskip,top=25mm,lines=31,heightrounded=true,bindingoffset=15mm,textwidth=106mm]{geometry}
%\usepackage[a4paper,headheight=1.5\baselineskip,top=25mm,lines=46,heightrounded=true,bindingoffset=15mm,textwidth=160mm]{geometry}
% footskip: Abstand zwischen Textunterkante und Seitenzahl
% KIT-Verlag: mindestens 10mm bzw. 3 Zeilen
% Koma-Standard 3.5\baselineskip
% IES-Vorlage: 2\baselineskip
% Abstand zwischen Textkörper und Unterkante Fußzeile (Seitenzahlen)
\newlength{\myfootskip}
\setlength{\myfootskip}{11mm}
% Abstand zwischen Textkörper und Linie in der Kopfzeile
\newlength{\myheadsep}
%\setlength{\myheadsep}{5mm}
\setlength{\myheadsep}{7mm} %Da eine kleinere Schriftgröße verwendet wird

\newlength{\myheadheight}
\setlength{\myheadheight}{1.5\baselineskip}

\newlength{\mytop}
\setlength{\mytop}{25mm}

\newlength{\mybindingoffset}
\setlength{\mybindingoffset}{8mm}

\newlength{\myinner}
\setlength{\myinner}{15mm}

\newlength{\myouter}
\setlength{\myouter}{15mm}

\newlength{\mytextwidth}
\setlength{\mytextwidth}{106mm}


%vertical float alignment for the only-float pages:
%on float-only pages the figures/tables shall be aligned at the top of the page instead of being centered
\newboolean{SetFloatsVerticallyCentered}
\setboolean{SetFloatsVerticallyCentered}{false}

%%% LaTeX-Präambel
%%% Hier werden Pakete eingebunden, Teil I
%%% Internal Commands: ----------------------------------------------
\makeatletter
%
\providecommand{\IfPackageLoaded}[2]{\@ifpackageloaded{#1}{#2}{}}
\providecommand{\IfPackageNotLoaded}[2]{\@ifpackageloaded{#1}{}{#2}}
\providecommand{\IfElsePackageLoaded}[3]{\@ifpackageloaded{#1}{#2}{#3}}
%
\newboolean{partavailable}%
\newboolean{chapteravailable}%
\setboolean{partavailable}{false}%
\setboolean{chapteravailable}{false}%

\ifcsname part\endcsname
  \setboolean{partavailable}{true}%
\else
  \setboolean{partavailable}{false}%
\fi

\ifcsname chapter\endcsname
  \setboolean{chapteravailable}{true}%
\else
  \setboolean{chapteravailable}{false}%
\fi


\providecommand{\IfPartDefined}[1]{\ifthenelse{\boolean{partavailable}}{#1}{}}%
\providecommand{\IfChapterDefined}[1]{\ifthenelse{\boolean{chapteravailable}}{#1}{}}%
\providecommand{\IfElsePartDefined}[2]{\ifthenelse{\boolean{partavailable}}{#1}{#2}}%
\providecommand{\IfElseChapterDefined}[2]{\ifthenelse{\boolean{chapteravailable}}{#1}{#2}}%

\providecommand{\IfDefined}[2]{%
\ifcsname #1\endcsname
   #2 %
\else
     % do nothing
\fi
}

\providecommand{\IfElseDefined}[3]{%
\ifcsname #1\endcsname
   #2 %
\else
   #3 %
\fi
}

\providecommand{\IfElseUnDefined}[3]{%
\ifcsname #1\endcsname
   #3 %
\else
   #2 %
\fi
}


%
% Check for 'draft' mode - commands.
\newcommand{\IfNotDraft}[1]{\ifx\@draft\@undefined #1 \fi}
\newcommand{\IfNotDraftElse}[2]{\ifx\@draft\@undefined #1 \else #2 \fi}
\newcommand{\IfDraft}[1]{\ifx\@draft\@undefined \else #1 \fi}
%

% Define frontmatter, mainmatter and backmatter if not defined
\@ifundefined{prefrontmatter}{%
   \newcommand*{\prefrontmatter}{%
      %In lateinischen Kleinbuchstaben nummerieren (a, b, c)
      %\pagenumbering{roman}
			\hypersetup{pageanchor=false}
			\cleardoubleoddpage  %% M. Kohm sagt, das sollte man vor jedem Pagenumbering-Wechsel tun
			\pagenumbering{alph}%
			%\renewcommand*\thepage{\texorpdfstring{\arabic{page}}{prefrontP.\arabic{page}}}%
			%\renewcommand*{\theHpage}{prefront.\thepage} %statt front.\thepage ginge auch \arabic{chapter}.\thepage. Hauptsache eindeutig: http://de.authex.info/1132586-pdflatex-und-hyperref-mit-plainpages 
			% http://tex.stackexchange.com/questions/65182/cross-references-linking-to-wrong-equations-using-hyperref
			% oder auch: http://tex.stackexchange.com/questions/6098/wrong-hyper-references-after-resetting-chapter-counter
			%\renewcommand*\theHchapter{prefrontC.\arabic{chapter}}
    }
}{}
\@ifundefined{frontmatter}{%
   \newcommand*{\frontmatter}{%
      %In Römischen Grossbuchstaben nummerieren (I, II, III)
      %\pagenumbering{Roman}
			\cleardoubleoddpage  %% M. Kohm sagt, das sollte man vor jedem Pagenumbering-Wechsel tun
			\pagenumbering{Roman}%
			\hypersetup{pageanchor=false}
			%\renewcommand*\thepage{\texorpdfstring{\arabic{page}}{frontP.\arabic{page}}}%
			%%\renewcommand*{\theHpage}{front.\thepage} %statt front.\thepage ginge auch \arabic{chapter}.\thepage. Hauptsache eindeutig: http://de.authex.info/1132586-pdflatex-und-hyperref-mit-plainpages 
			%% http://tex.stackexchange.com/questions/65182/cross-references-linking-to-wrong-equations-using-hyperref
			%% oder auch: http://tex.stackexchange.com/questions/6098/wrong-hyper-references-after-resetting-chapter-counter
			%\renewcommand*\theHchapter{frontC.\arabic{chapter}}
    }
}{}
\@ifundefined{mainmatter}{%
   % scrpage2 benötigt den folgenden switch
   % wenn \mainmatter definiert ist.
   \newif\if@mainmatter\@mainmattertrue
   \newcommand*{\mainmatter}{%
      % -- Seitennummerierung auf Arabische Zahlen zurücksetzen (1,2,3)
			\cleardoubleoddpage  %% M. Kohm sagt, das sollte man vor jedem Pagenumbering-Wechsel tun
      \pagenumbering{arabic}%
      %\setcounter{page}{1}%
			\hypersetup{pageanchor=true}
			%\renewcommand*\thepage{\texorpdfstring{\arabic{page}}{mainP.\arabic{page}}}%
			%%\renewcommand*{\theHpage}{main.\thepage}
			%\renewcommand\theHchapter{mainC.\arabic{chapter}}
			%\renewcommand{\theHequation}{\theHsection.\equationgrouping\arabic{equation}}
   }
}{}
\@ifundefined{backmatter}{%
   \newcommand*{\backmatter}{
      %In Römischen Kleinbuchstaben nummerieren (i, ii, iii)
			\cleardoubleoddpage  %% M. Kohm sagt, das sollte man vor jedem Pagenumbering-Wechsel tun
			%\pagenumbering{Roman}%
     \pagenumbering{roman}
			%\renewcommand*\thepage{\texorpdfstring{\arabic{page}}{backP.\arabic{page}}}%
			%%\renewcommand*{\theHpage}{back.\thepage}
			%\renewcommand\theHchapter{backC.\arabic{chapter}}
   }
}{}

% Pakete speichern die später geladen werden sollen
\newcommand{\LoadPackagesNow}{}
\newcommand{\LoadPackageLater}[1]{%
   \g@addto@macro{\LoadPackagesNow}{%
      \usepackage{#1}%
   }%
}


\makeatother

%%% ----------------------------------------------------------------

%% Es werden jeweils eines der begrenzt verfügbaren TeX-\writes verwendet für
% Table of Contents
% List of figures
% List of tables
% List of listings
% List of theorems

%%% Einstellungen für KOMA-Script
%%% Hier werden die KOMAoptions gesetzt wie z.B. Ränder, Kopfzeilen, einseitig/zweiseitig, Absatzabstände, Inhaltsverzeichnis usw.
%%% Alle Optionen sind im scrguide.pdf erklärt, die ihr in Eurem LaTeX-Distributionsverzeichnis bei den Docs findet!
%%% === Textbody ==============================================================
\KOMAoptions{%
		%% Wird nicht verwendet, da typearea benutzt wird
   %DIV=14,% (Size of Text Body, higher values = greater textbody)  %DIV 14 für A5, DIV 12 für A4
%   DIV=11,% Alt (MG: Mein Hack um größere Margin-Fläche zu errechen??)
    DIV=15,%
    %DIV=calc, % (also areaset/classic/current/default/last) 
   % -> after setting of spacing necessary!   
   %BCOR=10mm% (Bindekorrektur) % 8 gibt einen Rand innen von 1,5 cm bei DIV16. Laut Frau Mehl entspricht 7 1,6 cm und sollte 1,8 bis 2,0 cm sein --> Jetzt 10.
   %BCOR=5mm% (Bindekorrektur)
   %BCOR=15mm
	BCOR=\mybindingoffset
}%
%\areaset[BCOR]{Breite}{Höhe}
%A4 hat 210mm x 297mm
%\areaset[15mm]{172mm}{267mm}
%\areaset[15mm]{100mm}{200mm}
%%% === Headings ==============================================================
\KOMAoptions{%
   %%%% headings
	% headings=small,  % Small Font Size, thin spacing above and below
   headings=normal, % Medium Font Size, medium spacing above and below
   %headings=big, % Big Font Size, large spacing above and below
   %
   %headings=noappendixprefix, % chapter in appendix as in body text
	%headings=nochapterprefix,  % no prefix at chapters
   % headings=appendixprefix,   % inverse of 'noappendixprefix'
   % headings=chapterprefix,    % inverse of 'nochapterprefix'
   % headings=openany,   % Chapters start at any side
   % headings=openleft,  % Chapters start at left side
   headings=openright, % Chapters start at right side      
   %%% Add/Dont/Auto Dot behind section numbers 
   %%% (see DUDEN as reference)
   % numbers=autoenddot
   % numbers=enddot
   numbers=noenddot
   % secnumdepth=3 % depth of sections numbering (???)
}%
%\setcounter{secnumdepth}{3} % Überschriften nur bis subsection-ebene nummerieren (subsubsections nicht mehr nummerieren)
\setcounter{secnumdepth}{4} % auch subsubsections nummerieren
%%% === Page Layout ===========================================================
\KOMAoptions{% (most options are for package typearea)
   twoside=true, % two side layout (alternating margins, standard in books)
   % twoside=false, % single side layout 
   % twoside=semi,  % two side layout (non alternating margins!)
   %
   twocolumn=false, % (true)
   %
   headinclude=true,%
   footinclude=true,%
   mpinclude=false,%      
   %
   % Die Option headlines setzt die Anzahl der Kopfzeilen.
   % Normalerweise arbeitet das typearea-Paket mit 1,25 Kopfzeilen
   %headlines=1.25,%
   headlines=1,
   %headlines=2.1,%
   % headheight=2em,%
   %% Die Option footlines setzt die Anzahl der Fußzeilen.
   %% Normalerweise arbeitet das typearea-Paket mit 1,25 Fußzeilen
   %footlines=1.25,%
   footlines=1,
   %footlines=1.6,%
   % footheight=2em,%
   headsepline=true,% %bedingt headinclude
   footsepline=false,% %bedingt footinclude
   %% wenn Head und Foot in Seitenspiegel inkludiert werden sollen:
   %headinclude=true,%
   %footinclude=true,%  %ändert die komisch tiefen Seitenzahlen
	%Vakatseiten werden mit dem Style gesetzt
   cleardoublepage=empty %plain, headings
}%
%%% === Paragraph Separation ==================================================
\KOMAoptions{%
	 %%% The first two require the TikZ workaround
	 %parskip=relative, % change indentation according to fontsize (recommended)
   parskip=absolute, % do not change indentation according to fontsize
   %%% The following doesn't need the TikZ Workaround
   % parskip=false    % indentation of 1em
   % parskip=true   % parksip of 1 line - with free space in last line of 1em
   % parskip=full-  % parksip of 1 line - no adjustment
   % parskip=full+  % parksip of 1 line - with free space in last line of 1/4
   % parskip=full*  % parksip of 1 line - with free space in last line of 1/3    %% TeX Grouping Capacity Fehler wenn TikZ-Bilder kommen. Seltsam.
   %parskip=half   % parksip of 1/2 line - with free space in last line of 1em
   parskip=half-  % parksip of 1/2 line - no adjustment
   % parskip=half+  % parksip of 1/2 line - with free space in last line of 1/3
   % parskip=half*  % parksip of 1/2 line - with free space in last line of 1em
}%
%%% === Table of Contents =====================================================
%Inhaltesverzeichnis mit größerer Tiefe: bei Report: 3 geht bis subsection
\setcounter{tocdepth}{\mytocdepth} % Depth of TOC Display
\KOMAoptions{%
   %%% Setting of 'Style' and 'Content' of TOC
   % toc=left, %
   toc=indented,%
   %
   toc=bib,
   % toc=nobib,
   % toc=bibnumbered,
   %
	% toc=index,%
   toc=noindex,
	 %
   toc=chapterentrydotfill, % Bei den Kapiteleinträgen sollen Text und Seitenzahl ebenfalls
	                       % durch eine punktierte Linie miteinander verbunden werden
	 %toc=sectionentrywithdots, bei Abschnittseinträgen der Klasse scrartcl
   % funktioniert nicht! Warum?
	 %
   % toc=listof,
   toc=nolistof
   % toc=listofnumbered,
   %   
}%  
%%% === Lists of figures, tables etc. =========================================
\KOMAoptions{%
   %%% Setting of 'Style' and 'Content' of Lists 
   %%% (figures, tables etc)
	% --- General List Style ---
   listof=left, % tabular styles
   %listof=indented, % hierarchical style
   % --- chapter highlighting ---
   % listof=chapterentry, % ??? Chapter starts are marked in figure/table
   % listof=chaptergapline, % New chapter starts are marked by a gap 
      		  			   	 % of a single line
	%listof=chaptergapsmall, % New chapter starts are marked by a gap 
   	    					   % of a smallsingle line
   % listof=nochaptergap, % No Gap between chapters
   %
   % listof=leveldown, % lists are moved one level down ???
   % --- Appearance of Lists in TOC
   % listof=notoc, % Lists are not part of the TOC
   listof=totoc % add Lists to TOC without number
   % listof=totocnumbered, % add Lists to TOC with number
}%  
%%% === Bibliography ==========================================================
%% Setting of 'Style' and 'Content' of Bibliography
\KOMAoptions{%
   %bibliography=oldstyle,% "Klassischer" Stil des Literaturverzeichnisses: jeder Eintrag abgesetzt, hängend
   bibliography=openstyle,% "Moderner" Stil: keine vertikale Absetzung, dafür horizontaler Zusatzeinzug
   % bibliography=nottotoc, % Bibliography is not part of the TOC
   % bibliography=totocnumbered, % add Bibliography to TOC with number
   bibliography=totoc % add Bibliography to TOC without number
}%
%%% === Index =================================================================
%% Setting of 'Style' and 'Content' of Index in TOC
\KOMAoptions{%
   index=nottotoc, % index is not part of the TOC
   % index=totoc, % add index to TOC without number
   %
   chapterentrydots=true % Verbindung der Kapiteleinträge im Inhaltverzeichnis mit der Seitenzahl durch Punkte
   % funktioniert nicht! Warum?
	 %
}%
%%% === Titlepage =============================================================
\KOMAoptions{%
   titlepage=true %
   %titlepage=false %
}%
%%% === Miscellaneous =========================================================
\KOMAoptions{% 	
   footnotes=multiple% nomultiple
   %open=any,%
   %open=left,%
   %open=right,%
   %chapterprefix=false,%
   %appendixprefix=false,%
   %chapteratlists=10pt,% entry
}%

%% Präambel Teil II
% ------------------------------------------------------------------------
% LaTeX - Preambel  ******************************************************
% ------------------------------------------------------------------------
% von: Matthias Pospiech
% ========================================================================

% Strukturierung dieser Praeambel:
%    1.  Pakete die vor anderen geladen werden müssen
%        (calc, babel, xcolor, graphicx, amsmath, pst-pdf, ragged2e, ...)
%    2.  Schriften
%    3.  Mathematik (mathtools, fixmath, onlyamsmath, braket,
%        cancel, empheq, exscale, icomma, ...)
%    4.  Tabellen (booktabs, multirow, dcolumn, tabularx, ltxtable, supertabular)
%    5.  Text
%        5.1 Auszeichnungen (ulem, soul, url)
%        5.2 Fussnoten (footmisc)
%        5.3 Verweise (varioref)
%        5.4 Listen (enumitem, paralist, declist)
%    6.  Zitieren (csquotes, jurabib, natbib)
%    7.  PDF (microtype, hyperref, backref, hypcap, pdfpages
%    8.  Graphiken (float, flafter, placeins, subfig, wrapfig,
%        floatflt, picins, psfrag, sidecap, pict2e, curve2e)
%    9.  Sonstiges (makeidx, isodate, numprint, nomencl, acronym)
%    10. Verbatim (upquote, verbatim, fancyvrb, listings, examplep)
%    11. Wissenschaft (units)
%    12. Fancy Stuff
%    13. Layout
%       13.1.  Diverse Pakete und Einstellungen (multicol, ellipsis)
%       13.2.  Zeilenabstand (setspace)
%       13.3.  Seitenlayout (typearea, geometry)
%       13.4.  Farben
%       13.5.  Aussehen der URLS
%       13.6.  Kopf und Fusszeilen (scrpage2)
%       13.7.  Fussnoten
%       13.8.  Schriften (Sections )
%       13.9.  UeberSchriften (Chapter und Sections) (titlesec, indentfirst)
%       13.10. Captions (Schrift, Aussehen)
%              (caption, subfig, capt-of, mcaption, tocloft, multitoc, minitoc)
%    14.  Auszufuehrende Befehle


% ~~~~~~~~~~~~~~~~~~~~~~~~~~~~~~~~~~~~~~~~~~~~~~~~~~~~~~~~~~~~~~~~~~~~~~~~
% Einige Pakete muessen unbedingt vor allen anderen geladen werden
% ~~~~~~~~~~~~~~~~~~~~~~~~~~~~~~~~~~~~~~~~~~~~~~~~~~~~~~~~~~~~~~~~~~~~~~~~
%%%% Internal Commands: ----------------------------------------------
\makeatletter
%
\providecommand{\IfPackageLoaded}[2]{\@ifpackageloaded{#1}{#2}{}}
\providecommand{\IfPackageNotLoaded}[2]{\@ifpackageloaded{#1}{}{#2}}
\providecommand{\IfElsePackageLoaded}[3]{\@ifpackageloaded{#1}{#2}{#3}}
%
\newboolean{partavailable}%
\newboolean{chapteravailable}%
\setboolean{partavailable}{false}%
\setboolean{chapteravailable}{false}%

\ifcsname part\endcsname
  \setboolean{partavailable}{true}%
\else
  \setboolean{partavailable}{false}%
\fi

\ifcsname chapter\endcsname
  \setboolean{chapteravailable}{true}%
\else
  \setboolean{chapteravailable}{false}%
\fi


\providecommand{\IfPartDefined}[1]{\ifthenelse{\boolean{partavailable}}{#1}{}}%
\providecommand{\IfChapterDefined}[1]{\ifthenelse{\boolean{chapteravailable}}{#1}{}}%
\providecommand{\IfElsePartDefined}[2]{\ifthenelse{\boolean{partavailable}}{#1}{#2}}%
\providecommand{\IfElseChapterDefined}[2]{\ifthenelse{\boolean{chapteravailable}}{#1}{#2}}%

\providecommand{\IfDefined}[2]{%
\ifcsname #1\endcsname
   #2 %
\else
     % do nothing
\fi
}

\providecommand{\IfElseDefined}[3]{%
\ifcsname #1\endcsname
   #2 %
\else
   #3 %
\fi
}

\providecommand{\IfElseUnDefined}[3]{%
\ifcsname #1\endcsname
   #3 %
\else
   #2 %
\fi
}


%
% Check for 'draft' mode - commands.
\newcommand{\IfNotDraft}[1]{\ifx\@draft\@undefined #1 \fi}
\newcommand{\IfNotDraftElse}[2]{\ifx\@draft\@undefined #1 \else #2 \fi}
\newcommand{\IfDraft}[1]{\ifx\@draft\@undefined \else #1 \fi}
%

% Define frontmatter, mainmatter and backmatter if not defined
\@ifundefined{prefrontmatter}{%
   \newcommand*{\prefrontmatter}{%
      %In lateinischen Kleinbuchstaben nummerieren (a, b, c)
      %\pagenumbering{roman}
			\hypersetup{pageanchor=false}
			\cleardoubleoddpage  %% M. Kohm sagt, das sollte man vor jedem Pagenumbering-Wechsel tun
			\pagenumbering{alph}%
			%\renewcommand*\thepage{\texorpdfstring{\arabic{page}}{prefrontP.\arabic{page}}}%
			%\renewcommand*{\theHpage}{prefront.\thepage} %statt front.\thepage ginge auch \arabic{chapter}.\thepage. Hauptsache eindeutig: http://de.authex.info/1132586-pdflatex-und-hyperref-mit-plainpages 
			% http://tex.stackexchange.com/questions/65182/cross-references-linking-to-wrong-equations-using-hyperref
			% oder auch: http://tex.stackexchange.com/questions/6098/wrong-hyper-references-after-resetting-chapter-counter
			%\renewcommand*\theHchapter{prefrontC.\arabic{chapter}}
    }
}{}
\@ifundefined{frontmatter}{%
   \newcommand*{\frontmatter}{%
      %In Römischen Grossbuchstaben nummerieren (I, II, III)
      %\pagenumbering{Roman}
			\cleardoubleoddpage  %% M. Kohm sagt, das sollte man vor jedem Pagenumbering-Wechsel tun
			\pagenumbering{Roman}%
			\hypersetup{pageanchor=false}
			%\renewcommand*\thepage{\texorpdfstring{\arabic{page}}{frontP.\arabic{page}}}%
			%%\renewcommand*{\theHpage}{front.\thepage} %statt front.\thepage ginge auch \arabic{chapter}.\thepage. Hauptsache eindeutig: http://de.authex.info/1132586-pdflatex-und-hyperref-mit-plainpages 
			%% http://tex.stackexchange.com/questions/65182/cross-references-linking-to-wrong-equations-using-hyperref
			%% oder auch: http://tex.stackexchange.com/questions/6098/wrong-hyper-references-after-resetting-chapter-counter
			%\renewcommand*\theHchapter{frontC.\arabic{chapter}}
    }
}{}
\@ifundefined{mainmatter}{%
   % scrpage2 benötigt den folgenden switch
   % wenn \mainmatter definiert ist.
   \newif\if@mainmatter\@mainmattertrue
   \newcommand*{\mainmatter}{%
      % -- Seitennummerierung auf Arabische Zahlen zurücksetzen (1,2,3)
			\cleardoubleoddpage  %% M. Kohm sagt, das sollte man vor jedem Pagenumbering-Wechsel tun
      \pagenumbering{arabic}%
      %\setcounter{page}{1}%
			\hypersetup{pageanchor=true}
			%\renewcommand*\thepage{\texorpdfstring{\arabic{page}}{mainP.\arabic{page}}}%
			%%\renewcommand*{\theHpage}{main.\thepage}
			%\renewcommand\theHchapter{mainC.\arabic{chapter}}
			%\renewcommand{\theHequation}{\theHsection.\equationgrouping\arabic{equation}}
   }
}{}
\@ifundefined{backmatter}{%
   \newcommand*{\backmatter}{
      %In Römischen Kleinbuchstaben nummerieren (i, ii, iii)
			\cleardoubleoddpage  %% M. Kohm sagt, das sollte man vor jedem Pagenumbering-Wechsel tun
			%\pagenumbering{Roman}%
     \pagenumbering{roman}
			%\renewcommand*\thepage{\texorpdfstring{\arabic{page}}{backP.\arabic{page}}}%
			%%\renewcommand*{\theHpage}{back.\thepage}
			%\renewcommand\theHchapter{backC.\arabic{chapter}}
   }
}{}

% Pakete speichern die später geladen werden sollen
\newcommand{\LoadPackagesNow}{}
\newcommand{\LoadPackageLater}[1]{%
   \g@addto@macro{\LoadPackagesNow}{%
      \usepackage{#1}%
   }%
}


\makeatother

%%% ----------------------------------------------------------------

%There are two packages that may be used for preventing "no room for a new \write" error,
% which occurs if too many external (intermediate) files need to be written simultaneously
%(table of contents, glossary, list of figures, abbreviations, acronyms, todos, etc.)
%the morewrites package caused a problem after an update in March 2017
%alternatively use scrwfile
%% PW: Macht in Zukunft "room for a new \write"
%% braucht l3kernel
%\usepackage{morewrites}
\usepackage{scrwfile}

%% PW: Tip von H Oberdiek: Volle Ausgabe der Fehlermeldungen
\errorcontextlines=\maxdimen


%% PW: Tip von H. Oberdiek:
%% Zählerwert und \thepage auf jeder Seite protokollieren lassen
\usepackage{atbegshi}
\AtBeginShipout{\typeout{* Page \the\value{page} (\thepage)}}

%
%
%%% Doc: www.cs.brown.edu/system/software/latex/doc/calc.pdf
% Calculation with LaTeX
\usepackage{calc}


%% PW: Befehle mit mehr als einem optionalen Argument. 
\usepackage{xargs}


%%% Doc: ftp://tug.ctan.org/pub/tex-archive/macros/latex/required/babel/babel.pdf
% Language setting.
% Main document language is listed at last
\ifthenelse{\boolean{englishAsMainLanguage}}%
{\usepackage[ngerman,english]{babel}}%
{\usepackage[english,ngerman]{babel}}

% The \babelhyphenation macro has been introduced
% only in the babel v.3.9 released in March 2013.
% For the sake of compatibility with older TeX distribution:
\providecommand{\babelhyphenation}[2][english]{%
\begin{hyphenrules}{#1}
\hyphenation{#2}
\end{hyphenrules}
}

%for allowing hyphenation of words that contain a dash
%using shortcuts \-/, \=/, \--, \==, \---, and \===
\usepackage[shortcuts]{extdash}

%Zur besseren Silbentrennung
%s. http://de.wikibooks.org/wiki/LaTeX-W%C3%B6rterbuch:_Silbentrennung
\usepackage[ngerman=ngerman-x-latest]{hyphsubst}

%% Paket zur Abfrage der aktuellen Sprache
\usepackage{iflang}


%%% Doc: ftp://tug.ctan.org/pub/tex-archive/macros/latex/required/graphics/grfguide.pdf
% Bilder
\usepackage[%
	%final,
	%draft, % do not include images (faster)
	%pdftex, %für asymptote  %% sorgt für PDF mode expected, but DVI mode detected!!!!! Keinen Treiber auswählen bei graphicx, sonst geht pstool nicht mehr!!!!!
]{graphicx}
% Nachfolgendes ausgelegert in eine Extra-Datei
% Achtung: Es gibt keine Warnung, falls Pfade an mehreren Stellen gesetzt werden!
%\graphicspath{\MyImagePathes}
%\DeclareGraphicsExtensions{\MyImageExtensions}

%%% Doc: ftp://tug.ctan.org/pub/tex-archive/macros/latex/contrib/oberdiek/epstopdf.pdf
%% If an eps image is detected, epstopdf is automatically called to convert it to pdf format.
%% Requires: graphicx loaded

\usepackage{ifplatform}

%%%PW: Tip von H. Oberdiek
\usepackage{ltxcmds}%[2010/04/26]
\makeatletter
\let\HashChar\ltx@hashchar
\makeatother

%%%PW: Tip von H. Oberdiek
%\begingroup
  %\catcode`\#=12 %
%\edef\x{\endgroup
  %\noexpand\newcommand*{\noexpand\HashChar}{#}%
%}\x

\ifxetex
	%bei Verwendung von XeLaTeX nicht verwenden, da nicht kompatibel
\else
	\usepackage{epstopdf}

\epstopdfDeclareGraphicsRule{.eps}{pdf}{.pdf}{%
  ps2pdf %
  -dEPSCrop %
	-dAutoRotatePages\HashChar /None %
  -dPDFSETTINGS\HashChar /prepress %
  -dCompatibilityLevel\HashChar 1.3 %
	-dEmbedAllFonts\HashChar true %
	-dSubsetFonts\HashChar true
  #1 \OutputFile
}

%%%%%%%%%%%%%%%%%%%%%%%%%%%%%%%%%%%%%%%%%%%%%%%%%%%%%%%%%%%%%%%%%%%%%%%%%%%%%%%%%%%%%%%%%%%%%%%%%%%%%%
\fi

%MG: vorverschoben, da bereits in marginnote verwendet
%% Doc: ftp://tug.ctan.org/pub/tex-archive/macros/latex/contrib/ms/ragged2e.pdf
% Besserer Flatternsatz (Linksbuendig, statt Blocksatz)
\usepackage{ragged2e}


%% Doc: ftp://tug.ctan.org/pub/tex-archive/graphics/pstricks/README
%% Im Beispiel auf der Homepage vor auto-pst-pdf
% load before graphicx
 %\usepackage{pstricks}  %%Funktioniert nicht zusammen mit pstool, weil pstool nur Unterstützung für psfrag bietet.
 %\usepackage{pst-plot, pst-node, pst-coil, pst-eps}


%%% Doc: http://www.ctan.org/tex-archive/macros/latex/contrib/pst-pdf/pst-pdf-DE.pdf
% Used to automatically integrate eps graphics in an pdf document using pdflatex.
% Requires ps4pdf macro !!!
% Download macro from http://www.ctan.org/tex-archive/macros/latex/contrib/pst-pdf/scripts/

%\usepackage[%
%   %active,       % Aktiviert den Extraktionsmodus (DVI-Ausgabe). Die explizite Angabe ist
%                  % normalerweise unnötig (Standard im LATEX-Modus).
%   %inactive,     % Das Paket wird deaktiviert, Zuätzlich werden die Pakete pstricks und
%                  % graphicx geladen
%   nopstricks,    % Das Paket pstricks wird nicht geladen.
%   %draft,        % Im pdfLATEX-Modus werden aus der Containerdatei eingefügte Grafiken nur
%                  % als Rahmen dargestellt.
%   %final,        % Im pdfLATEX-Modus werden aus der Containerdatei eingefügte Grafiken
%                  % vollständig dargestellt (Standard).
%   %tightpage,    % Die Abmessung Grafiken in der Containerdatei entsprechen denen der
%                  % zugehörigen TEX-Boxen (Standard).
%   %notightpage,  % die Grafiken in der Containerdatei nehmen
%                  % mindestens die Größe des gesamten Blattes einnehmen.
%   %displaymath,  % Es werden zusätzlich die mathematischen Umgebungen displaymath,
%                  % eqnarray und $$ extrahiert und im pdf-Modus als Grafik eingefügt.
%]{pst-pdf}

%% Geht zusammen mit pstricks, aber nicht zusammen mit pstool. Also entweder auto-pst-pdf oder pstool.
%\usepackage[
%	on,
%%	crop=off,
%	crop=on,
%%	cleanup={log,aux,dvi,ps,pdf},
%]{auto-pst-pdf}
%%
% Notwendiger Bugfix für natbib Paket bei Benutzung von pst-pdf (Version <= v1.1o)
\IfPackageLoaded{pst-pdf}{
   \providecommand\makeindex{}
   \providecommand\makeglossary{}
}{}


% This package implements a workaround for the LaTeX bug that marginpars
% sometimes appear on the wrong margin.
%% PW: Hilft auch nicht
\usepackage{mparhack}
% in some case this causes an error in the index together with package pdfpages
% the reason is unkown. Therefore I recommend to use the margins of marginnote

%% Doc: ftp://tug.ctan.org/pub/tex-archive/macros/latex/contrib/marginnote/marginnote.pdf
% Summary description: marginnote allows margin note, where \marginpar fails
\usepackage{marginnote}
\IfPackageLoaded{marginnote}{%
\renewcommand*{\raggedleftmarginnote}{\RaggedRight}
\renewcommand*{\raggedrightmarginnote}{\RaggedRight}
%\renewcommand*{\marginfont}{\color{gray}\sffamily\scshape}% serifenlos, mit Kapitälchen
\renewcommand*{\marginfont}{\color{gray}}% ohne Kapitälchen
}


%% Doc: (inside relsize.sty )
%% ftp://tug.ctan.org/pub/tex-archive/macros/latex/contrib/misc/relsize.sty
%  Set the font size relative to the current font size
\usepackage{relsize}

%% MG: Vorverschoben, da bereits in marginnote verwendet
%% Doc: ftp://tug.ctan.org/pub/tex-archive/macros/latex/contrib/ms/ragged2e.pdf
%% Besserer Flatternsatz (Linksbuendig, statt Blocksatz)
%\usepackage{ragged2e}

%%PW: Markus Kohms gridset zum registerhaltigen Satz. (Zeilen der Vorderseite matchen die Zeilen der Rückseite wenn man es gegen das Licht hält)
%% Man muss händisch an die Absätze \vskipnextgrid setzen und für jedes \vskipnextgrid braucht man (mehr od. weniger) einen extra-Durchlauf von pdfLaTeX :-(
%\usepackage{gridset}

% ~~~~~~~~~~~~~~~~~~~~~~~~~~~~~~~~~~~~~~~~~~~~~~~~~~~~~~~~~~~~~~~~~~~~~~~~
% Tables (Tabular)
% ~~~~~~~~~~~~~~~~~~~~~~~~~~~~~~~~~~~~~~~~~~~~~~~~~~~~~~~~~~~~~~~~~~~~~~~~

% Basispaket fuer alle Tabellenfunktionen
% -> wird automatisch durch andere Pakete geladen
% \usepackage{array}
%
% bessere Abstaende innerhalb der Tabelle (Layout))
% -------------------------------------------------
%%% Doc: ftp://tug.ctan.org/pub/tex-archive/macros/latex/contrib/booktabs/booktabs.pdf
\usepackage{booktabs}
%
% Farbige Tabellen
% ----------------
% Das Paket colortbl wird inzwischen automatisch durch xcolor geladen
%
% Erweiterte Funktionen innerhalb von Tabellen
% --------------------------------------------
%%% Doc: ftp://tug.ctan.org/pub/tex-archive/macros/latex/contrib/multirow/multirow.sty
\usepackage{multirow} % Mehrfachspalten
%
%%% Doc: Documentation inside dtx Package
\usepackage{dcolumn}  % Ausrichtung an Komma oder Punkt

\usepackage{tabulary}

%%% Neue Tabellen-Umgebungen:
% ---------------------------
% Spalten automatischer Breite:
%%% Doc: Documentation inside dtx Package
% \usepackage{tabularx}
% -> nach hyperref Laden
% -> wird von ltxtable geladen
% \LoadPackageLater{tabularx}


% Tabellen ueber mehere Seiten
% ----------------------------
%%% Doc: ftp://tug.ctan.org/pub/tex-archive/macros/latex/contrib/carlisle/ltxtable.pdf
% \usepackage{ltxtable} % Longtable + tabularx
                        % (multi-page tables) + (auto-sized columns in a fixed width table)
% -> nach hyperref laden
\LoadPackageLater{ltxtable}

%%% Doc: ftp://tug.ctan.org/pub/tex-archive/macros/latex/contrib/supertabular/supertabular.pdf
%\usepackage{supertabular}



% ~~~~~~~~~~~~~~~~~~~~~~~~~~~~~~~~~~~~~~~~~~~~~~~~~~~~~~~~~~~~~~~~~~~~~~~~
% text related packages
% ~~~~~~~~~~~~~~~~~~~~~~~~~~~~~~~~~~~~~~~~~~~~~~~~~~~~~~~~~~~~~~~~~~~~~~~~

%%% Textverzierungen/Auszeichnungen ======================================
%
%%% Doc: ftp://tug.ctan.org/pub/tex-archive/macros/latex/contrib/misc/ulem.sty
\usepackage[normalem]{ulem}      % Zum Unterstreichen
%%% Doc: ftp://tug.ctan.org/pub/tex-archive/macros/latex/contrib/soul/soul.pdf
\usepackage{soul}		            % Unterstreichen, Sperren
%%% Doc: ftp://tug.ctan.org/pub/tex-archive/macros/latex/contrib/misc/url.sty
\usepackage[hyphens]{url} % Setzen von URLs. In Verbindung mit hyperref sind diese auch aktive Links.

%%PW:  Textpos Erlaubt absolute Positionierung. Für Titelseite.
%%%%Beispiel:
%%%%\begin{textblock}{8}(10.8,0.2)
%%%%\includegraphics[width=8cm]{bild}
%%%%\end{textblock}
%%%%Dazu muss \usepackage[absolute]{textpos} eingebunden werden.
%%%%Die erste Zahl gibt die Breite des Textblocks an (wird für Bilder NICHT!!! ignoriert, legt die Bildbreite fest), die
%%%%zweite den Abstand zum linken und die dritte den Abstand zum oberen Seitenrand.
\usepackage[%
absolute,%
%showboxes,%   %Hilft gewaltig beim Verständnis!
%overlay,%
verbose,
]{textpos}  %% Definiert \begin{textblock}{hsize}(hpos,vpos)
\setlength{\TPHorizModule}{10mm}
\setlength{\TPVertModule}{\TPHorizModule}
\textblockorigin{0mm}{0mm} % start everything near the top-left corner

%\usepackage{eso-pic}  %% absolute Bildpositionierung


%%% Fussnoten/Endnoten ===================================================
%
%%% Doc: ftp://tug.ctan.org/pub/tex-archive/macros/latex/contrib/footmisc/footmisc.pdf
%
\usepackage[%
   bottom,      % Footnotes appear always on bottom. This is necessary
                % especially when floats are used
   stable,      % Make footnotes stable in section titles
   perpage,     % Reset on each page
   %para,       % Place footnotes side by side of in one paragraph.
   %side,       % Place footnotes in the margin
   ragged,      % Use RaggedRight
   %norule,     % suppress rule above footnotes
   multiple    % rearrange multiple footnotes intelligent in the text.
   %symbol,     % use symbols instead of numbers
]{footmisc}

\renewcommand*{\multfootsep}{,\nobreakspace}

%%Das passiert weiter unten nochmal, daher hier auskommentiert
%\deffootnote%
%   [1em]% width of marker
%   {1.5em}% indentation (general)
%   {1em}% indentation (par)
%   {\textsubscript{\thefootnotemark}}%


%% Einruecken der Fussnote einstellen
%\setlength\footnotemargin{10pt}

%--- footnote counter documentweit durchlaufend ------------------------------
%\usepackage{chngcntr}
%\counterwithout{footnote}{chapter}
%-----------------------------------------------------------------------------

%%% Doc: ftp://tug.ctan.org/pub/tex-archive/macros/latex/contrib/misc/endnotes.sty
%\usepackage{endnotes}
% From the Documentation:
% To turn all the footnotes in your documents into endnotes, say
%
%     \let\footnote=\endnote
%
%  in your preamble, and then add something like
%
%     \newpage
%     \begingroup
%     \parindent 0pt
%     \parskip 2ex
%     \def\enotesize{\normalsize}
%     \theendnotes
%     \endgroup
%
% as the last thing in your document.  (But \theendnotes all
% by itself will work.)

%%% Verweise =============================================================
%
%%% Doc: Documentation inside dtx File
\ifthenelse{\boolean{englishAsMainLanguage}}{%
		\usepackage[english]{varioref} % Intelligente Querverweise
}{%
		\usepackage[ngerman]{varioref} % Intelligente Querverweise
}

%%% Listen ===============================================================
%
%
%%% Doc: ftp://tug.ctan.org/pub/tex-archive/macros/latex/contrib/paralist/paralist.pdf
% \usepackage{paralist}
%
%%% Doc: ftp://tug.ctan.org/pub/tex-archive/macros/latex/contrib/enumitem/enumitem.pdf
% Better than 'paralist' and 'enumerate' because it uses a keyvalue interface !
% Do not load together with enumerate.
\IfPackageNotLoaded{enumerate}{
	\usepackage{enumitem}
}
%Verbesserter Abstand: später in MyPacakges.tex


%%% Doc: ftp://tug.ctan.org/pub/tex-archive/macros/latex/contrib/ncctools/doc/desclist.pdf
% Improved description environment
%\usepackage{declist}


\usepackage{blindtext}   %PW: Lorem ipsum deutsch.

% ~~~~~~~~~~~~~~~~~~~~~~~~~~~~~~~~~~~~~~~~~~~~~~~~~~~~~~~~~~~~~~~~~~~~~~~~
% Pakete zum Zitieren
% ~~~~~~~~~~~~~~~~~~~~~~~~~~~~~~~~~~~~~~~~~~~~~~~~~~~~~~~~~~~~~~~~~~~~~~~~

% Quotes =================================================================
%% Doc: ftp://tug.ctan.org/pub/tex-archive/macros/latex/contrib/csquotes/csquotes.pdf
% Advanced features for clever quotations
\usepackage[%
   %autostyle=true,% the style of all quotation marks will be adapted
                     %%% to the document language as chosen by 'babel' or polyglossia. Option babel is deprecated
   babel,            % the style of all quotation marks will be adapted
                     % to the document language as chosen by 'babel'
   german=quotes,		% Styles of quotes in each language
   english=american, %british,
   french=guillemets
]{csquotes}   %% csquotes und pstool zusammen macht Probleme.
%\defineshorthand{"`}{\openautoquote}
%\defineshorthand{"'}{\closeautoquote}
%

%\defineshorthand{"`}{\guillemotright}
%\defineshorthand{"'}{\guillemotleft}
%%%Nicht unbedingt notwendig, aber klappt
%\defineshorthand{"´}{\guilsinglright}
%\defineshorthand{"*}{\guilsinglleft}


%% Listings Paket ------------------------------------------------------
%%% Doc: ftp://tug.ctan.org/pub/tex-archive/macros/latex/contrib/listings/listings-1.3.pdf
 % for code listings
\ifxetex %Nutzung von XeLaTeX anstatt nur LaTeX
	\usepackage{listings} %ok bei XeLaTeX; bei LaTeX Probleme mit Umlauten bei UTF8-kodierten Dateien
\else
	\usepackage{listingsutf8} %Workaround, um UTF8-kodierte Listing-Dateien mit \lstinputlisting
	% einbinden zu können (kein Workaround für \lstinline, Environment lstlisting und
	% Environments, die durch \lstnewenvironment definiert sind.
	%Workaround für korrekte Interpretation der Umlaute auch bei lstlisting-Umgebung:
	\lstset{literate=%
		{ä}{{\"a}}1 {ë}{{\"e}}1 {ï}{{\"i}}1 {ö}{{\"o}}1 {ü}{{\"u}}1
		{Ä}{{\"A}}1 {Ë}{{\"E}}1 {Ï}{{\"I}}1 {Ö}{{\"O}}1 {Ü}{{\"U}}1
		{á}{{\'a}}1 {é}{{\'e}}1 {í}{{\'i}}1 {ó}{{\'o}}1 {ú}{{\'u}}1
		{Á}{{\'A}}1 {É}{{\'E}}1 {Í}{{\'I}}1 {Ó}{{\'O}}1 {Ú}{{\'U}}1
		{à}{{\`a}}1 {è}{{\`e}}1 {ì}{{\`i}}1 {ò}{{\`o}}1 {ù}{{\`u}}1
		{À}{{\`A}}1 {È}{{\'E}}1 {Ì}{{\`I}}1 {Ò}{{\`O}}1 {Ù}{{\`U}}1
		{â}{{\^a}}1 {ê}{{\^e}}1 {î}{{\^i}}1 {ô}{{\^o}}1 {û}{{\^u}}1
		{Â}{{\^A}}1 {Ê}{{\^E}}1 {Î}{{\^I}}1 {Ô}{{\^O}}1 {Û}{{\^U}}1
		{œ}{{\oe}}1 {Œ}{{\OE}}1 {æ}{{\ae}}1 {Æ}{{\AE}}1 {ß}{{\ss}}1
		{ű}{{\H{u}}}1 {Ű}{{\H{U}}}1 {ő}{{\H{o}}}1 {Ő}{{\H{O}}}1
		{ç}{{\c c}}1 {Ç}{{\c C}}1
		{ã}{{\~a}}1 {å}{{\r a}}1 {Å}{{\r A}}1
		{ø}{{\o}}1 {€}{{\EUR}}1 {£}{{\pounds}}1
		{~}{{\textasciitilde}}1
	}

\fi


\lstloadlanguages{% Check Dokumentation for further languages ...
         %[Visual]Basic
         %Pascal
         C,
         [Visual]C++,
         [ISO]C++,
         %XML,
         %HTML,
 }


\lstdefinestyle{nonumbers}{%
				numbers=none,%
				frame=none,
				xleftmargin=0pt,  % keine Einrückung
				xrightmargin=0pt,  
}

\lstdefinestyle{numbers}{%
				numbers=left,			          % Ort der Zeilennummern (falls gewünscht, 
				frame=none,
				xleftmargin=2\marginparsep,  % Einrückung, damit die Zeilennummern innerhalb des Textrahmens stehen!}
				xrightmargin=0pt,  
				numberstyle=\tiny,          % Stil der Zeilennummern
}

\lstdefinestyle{nonumbersframed}{%
			numbers=none,%
			frame = single,             % Rahmen um Listings zeichnen
			framesep=\mylistingframesep,
			framerule=\mylistingframerule,
			xleftmargin=\dimexpr\mylistingframesep+\mylistingframerule,
			xrightmargin=\dimexpr\mylistingframesep+\mylistingframerule,
}


\lstdefinestyle{numbersframed}{%
			numbers=left,			          % Ort der Zeilennummern (falls gewünscht, 
			frame = single,             % Rahmen um Listings zeichnen
			framesep=\mylistingframesep,
			framerule=\mylistingframerule,
			xleftmargin=\dimexpr\mylistingframesep+\mylistingframerule+2\marginparsep,
			xrightmargin=\dimexpr\mylistingframesep+\mylistingframerule,
			numberstyle=\tiny,          % Stil der Zeilennummern	
}

%%%%%%%%%%%%%%%%%%%%%%%%%%%%%%%%%%%%%%%%%%%%%%%%%%%%%%%%%%%%%
%% User-Defined styles for different programming languages %%
%%%%%%%%%%%%%%%%%%%%%%%%%%%%%%%%%%%%%%%%%%%%%%%%%%%%%%%%%%%%%
\lstdefinestyle{latex}{
				style=nonumbersframed,
				language=[LaTeX]TeX,
				morekeywords={%
					Acf,%
					acf,%
					acgen,%
					acdat,%
					acacc,%
					acplgen,%
					acpldat,%
					acplacc,%
					acrshort,%
					acrlong,%
					acrfull,%
					gls,%
					glsgen,%
					glsdat,%
					glsacc,%
					glsplgen,%
					glspldat,%
					glsplacc,%
					resizebox,%
					cref,%
					href,%
					url,%
					includegraphics,%
					toprule,%
					midrule,%
					bottomrule,%
					tikzsetnextfilename,%
					enquote,%
					myglossaryentry,%
					newglossaryentry,%
					newacronym,%
					caption,%
					longnewglossaryentry,%
					LTXtable,%
					subfloat,%
					lstset,%
					lstinline,%
					foreignlanguage,%
					textInGerman,%
					textInEnglish,%
					setUserDefinedBoolean,%
					myNotationTableEntryMath,%
					myNotationTableEntryText,%
				}%
}

\lstdefinestyle{java}{
				style=numbers,%
				%stile=nonumbers,
				%style=numbersframed,
				%style=nonumbersframed,
        language=Java,%
}

\lstdefinestyle{C++}{
				style=numbers,%
				%stile=nonumbers,%
				%style=numbersframed,%
				%style=nonumbersframed,%
        language=[Visual]C++,%
}


%%%%%%%%%%%%%%%%%%%%%%%%%%%%%%%%%%
%% Global settings for listings %%
%%%%%%%%%%%%%%%%%%%%%%%%%%%%%%%%%%
\lstset{%
				basicstyle=\ttfamily,%
				%style=numbers,
				%stile=nonumbers,
				%style=numbersframed,
				%style=nonumbersframed,
				style=latex,
				%style=java,
				%style=C++,
				%stepnumber=1,               % Abstand zwischen den Zeilennummern
				%numbersep=5pt,              % Abstand der Nummern zum Text
				tabsize=2,                  % Groesse von Tabs
				extendedchars=true,         %
				breaklines=true,            % Zeilen werden Umgebrochen
				keywordstyle=\color{keywordcolor}\bfseries,
%				keywordstyle=[1]\textbf,    % Stil der Keywords
%				keywordstyle=[2]\textbf,    %
%				keywordstyle=[3]\textbf,    %
%				keywordstyle=[4]\textbf,    %
				stringstyle=\color{stringcolor}, % Farbe der String
				showspaces=false,           % Leerzeichen anzeigen ?
				showtabs=false,             % Tabs anzeigen ?
				showstringspaces=false,      % Leerzeichen in Strings anzeigen ?
				commentstyle=\color{commentcolor},
				directivestyle=\color{keywordcolor}\bfseries,
				aboveskip=\mytextfloatsep,
				abovecaptionskip=\mycaptionskip,
				belowcaptionskip=\mytextfloatsep,
				captionpos=b, %Überschrift unterhalb (ggf. weitere Einstellung in captionsetup beachten!)
}



%%%%%%%%%%%%%%%%%%%%%%%%%%%%%%%%%%%%%%%%%%%%%%%%%%%%%%%%%%%%%%%%%%%%%%%%%%%%%%%
%% User-Defined environments for listings in different programming languages %%
%%%%%%%%%%%%%%%%%%%%%%%%%%%%%%%%%%%%%%%%%%%%%%%%%%%%%%%%%%%%%%%%%%%%%%%%%%%%%%%
\lstnewenvironment{latex}[1][]{%
	\lstset{style=latex,#1}%
}{}

\lstnewenvironment{java}[1][]{%
  \lstset{style=java,#1}%
}{}

\lstnewenvironment{C++}[1][]{%
  \lstset{style=C++,#1}%
}{}




%% PW
% Weitere Algorithmenumgebungen: http://en.wikibooks.org/wiki/LaTeX/Algorithms_and_Pseudocode
 
%%% Doc: ftp://tug.ctan.org/pub/tex-archive/macros/latex/contrib/examplep/eurotex_2005_examplep.pdf
% LaTeX Code und Ergebnis nebeneinander darstellen
%\usepackage{examplep}



%% Für rotierte Tabellen
\usepackage[figuresright]{rotating}


% ~~~~~~~~~~~~~~~~~~~~~~~~~~~~~~~~~~~~~~~~~~~~~~~~~~~~~~~~~~~~~~~~~~~~~~~~
% figures and placement
% ~~~~~~~~~~~~~~~~~~~~~~~~~~~~~~~~~~~~~~~~~~~~~~~~~~~~~~~~~~~~~~~~~~~~~~~~

%% Bilder und Graphiken ==================================================


%%% Doc: No Documentation
%\usepackage{latexrelease}     % wird wohl vom flafter gebraucht
%\usepackage{flafter}          % Floats immer erst nach der Referenz setzen  %% Erzeugt aber eine Warnung

% Defines a \FloatBarrier command, beyond which floats may not
% pass; useful, for example, to ensure all floats for a section
% appear before the next \section command.
\usepackage[
	section		% "\section" command will be redefined with "\FloatBarrier"
]{placeins}
%
%%% Doc: ftp://tug.ctan.org/pub/tex-archive/macros/latex/contrib/subfig/subfig.pdf
% Incompatible: loads package capt-of. Loading of 'capt-of' afterwards will fail therefor
\usepackage{subfig} % Layout wird weiter unten festgelegt !

%%% Bilder von Text Umfliessen lassen : (empfehle wrapfig)
%
%%% Doc: ftp://tug.ctan.org/pub/tex-archive/macros/latex/contrib/wrapfig/wrapfig.sty
\usepackage{wrapfig}	        % defines wrapfigure and wrapfloat
\setlength{\wrapoverhang}{\marginparwidth} % aeerlapp des Bildes ...
\addtolength{\wrapoverhang}{\marginparsep} % ... in den margin

%%% Doc: Documentation inside dtx Package
%\usepackage{floatflt}   	  % LaTeX2e Paket von 1996
                             % [rflt] - Standard float auf der rechten Seite

%%% Doc: ftp://tug.ctan.org/pub/tex-archive/macros/latex209/contrib/picins/picins.doc
%\usepackage{picins}          % LaTeX 2.09 Paket von 1992. aber Layout kombatibel


%% Make float placement easier
%minimum fraction of floatpage that should have floats (default: 0.5)
\renewcommand{\floatpagefraction}{.5} % vorher: .75 / .5
%% minimum fraction of page for text (default: 0.2)
\renewcommand{\textfraction}{0}       % vorher: .1 / .2
%% maximum fraction of page for floats at top (default: 0.7)
\renewcommand{\topfraction}{1}        % vorher: .8 / .7
%% maximum fraction of page for floats at bottom (default: 0.3)
\renewcommand{\bottomfraction}{1}     % vorher: .5 / .3
%% maximum number of floats at top of page (default: 2)
\setcounter{topnumber}{3}              % vorher: 2
%% maximum number of floats at bottom of page (default: 1)
\setcounter{bottomnumber}{3}           % vorher: 1
%% maximum number of floats on a page (default: 3)
\setcounter{totalnumber}{5}            % vorher: 3

% Abstand zwischen zwei Absätzen
\setlength{\parskip}{\myparskip}

%% Abstand zwischen Gleitobjekten und dem darüber und darunter angeordneten Fließtext fest
\setlength{\intextsep}{\myintextsep}
\setlength{\floatsep}{\myfloatsep}
\setlength{\textfloatsep}{\mytextfloatsep}
%\setlength{\columnsep}{1em} % Abstand zum Text


\makeatletter
% Positionierung von Gleitumgebungen defaultmäßig auf htbp statt tbp setzen.
\renewcommand{\fps@figure}{htb}
\renewcommand{\fps@table}{htb}

%% set vertical float alignment for the only-float pages:
\ifthenelse{\boolean{SetFloatsVerticallyCentered}}%
{% to get the float to be centered on a floats-only page:
\setlength{\@fptop}{0pt plus 1fil}
\setlength{\@fpbot}{0pt plus 1fil}
}%
{% to get the figure/table be aligned at the top instead:
\setlength{\@fptop}{0pt}
\setlength{\@fpsep}{\myfloatsep}
\setlength{\@fpbot}{0pt plus 1fil}
}
\makeatother

%%% Doc: ftp://tug.ctan.org/pub/tex-archive/macros/latex/contrib/psfrag/pfgguide.pdf
% \usepackage{psfrag}	% Ersetzen von Zeichen in eps Bildern  % 1998
 				%Usage: \psfrag{tag}[posn][psposn][scale][rot]{LATEX text}
% \usepackage{psfragx}	% Extension for psfrag, not a replacement. Weiß aber nicht, ob's das bringt. Ersetzen von Zeichen in eps Bildern  %%PW Ist aber alt: Dez 2004
%\ifpdf
%\usepackage[%
%cleanup={}%
%%crop=pdfcrop,%  %%PW: Wenn falsch gecroppt wird, es aber den Fehler "Cannot call ghostscript (mgs)" gibt, liegt das daran, dass die mgs.exe die Umgebungsvariable MIKTEX_GS_LIB abfragt, statt der üblichen GS_LIB. Das kommt daher, dass Miktex-Ghostscript benutzt wird, das ein wenig anders ist. Abhilfe schafft die Umgebungsvariable so zu setzen:
%%process,% Wenn man die psfrag-Sachen etc. immer durchgenudelt haben will
%%%PW:  set MIKTEX_GS_LIB=c:\Programme\MiKTeX28\ghostscript\base;c:\Programme\MiKTeX28\fonts\
%%%PW  Pfade natürlich anpassen
%%mode=errorstop%nonstop%batch%
%,mode=nonstop%
%%latex-options=
%]%
%{pstool}  %von Zebb Prime, eigentlich ein pst-pdf Ersatz

%% pstool hat aber noch tolle Unterstützung für Matlab-Figures
%\fi




%%PW: TikZ
%% Braucht anscheinend eines der begrenzt verfügbaren TeX-\writes
%\usepackage{tikz}
%\usepackage{tikz-3dplot}

%%PW: Damit KOMAscript: parskip = absolute und relative keine TeX grouping errors provozieren.
%\makeatletter
%\def\pgfutil@selectfont{\KOMAoptions{parskip=absolute}\selectfont}
%\makeatother


\usepackage{overpic}

%% Diagramme mit LaTeX ===================================================
%

%%% Doc: ftp://tug.ctan.org/pub/tex-archive/macros/latex/contrib/pict2e/pict2e.pdf
% Neuimplementation der Picture Umgebung.
%
% The new package extends the existing LaTeX picture environment, using
% the familiar technique (cf. the graphics and color packages) of driver
% files.  The package documentation (pict2e.dtx) has a fair number of
% examples of use, showing where things are improved by comparison with
% the LaTeX picture environment.
% \usepackage{pict2e}

%%% Doc: ftp://tug.ctan.org/pub/tex-archive/macros/latex/contrib/curve2e/curve2e.pdf
% Extensions for package pict2e.
%\usepackage{curve2e}
%


% ~~~~~~~~~~~~~~~~~~~~~~~~~~~~~~~~~~~~~~~~~~~~~~~~~~~~~~~~~~~~~~~~~~~~~~~~
% misc packages
% ~~~~~~~~~~~~~~~~~~~~~~~~~~~~~~~~~~~~~~~~~~~~~~~~~~~~~~~~~~~~~~~~~~~~~~~~


\IfDraft{
  \usepackage{showidx}    % Indizierte Begriffe am Rand (Korrekturlesen)
}
% Erweitertes Paket zur Generierung von Stichwortverzeichnissen
\usepackage{imakeidx}
%\usepackage[options=-s preambel/IndexStyle.tex]{imakeidx} % use this if you want to use makeindex for sorting
%\usepackage[program=truexindy, options={--language german-din --codepage utf8 --module indexStyle}]{imakeidx} % use xindy for sorting
%\usepackage[truexindy, options={--language german-din --codepage utf8 --module indexStyle}]{imakeidx} % use xindy for sorting
%\usepackage[xindy]{imakeidx}
\indexsetup{level=\chapter*,toclevel=chapter*}

%% PW: nameref: Vielleicht hilfts was gegen die falschen Index-Hyperlinks
%% Nein, tut es nicht.
%%\usepackage{nameref}

%%% Doc: ftp://tug.ctan.org/pub/tex-archive/macros/latex/contrib/isodate/README
%%% Incompatible: draftcopy
% Tune the output format of dates.
%\usepackage{isodate}

%%% Doc: ftp://tug.ctan.org/pub/tex-archive/macros/latex/contrib/numprint/numprint.pdf
% Modify printing of numbers
%\usepackage{numprint}

%%MG: nach hyperref verschoben
%%% Doc: ftp://tug.ctan.org/pub/tex-archive/macros/latex/contrib/nomencl/nomencl.pdf
%% Braucht anscheinend eines der begrenzt verfügbaren TeX-\writes
%\usepackage[%
%	german,
%	english
%]{nomencl}%[2005/09/22]

%%MG: nach hyperref verschoben
%\usepackage[
%%	footnote,	% Full names appear in the footnote
%%	smaller,		% Print acronym in smaller fontsize
%	printonlyused %
%]{acronym}

% ~~~~~~~~~~~~~~~~~~~~~~~~~~~~~~~~~~~~~~~~~~~~~~~~~~~~~~~~~~~~~~~~~~~~~~~~
% verbatim packages
% ~~~~~~~~~~~~~~~~~~~~~~~~~~~~~~~~~~~~~~~~~~~~~~~~~~~~~~~~~~~~~~~~~~~~~~~~

%%% Doc: ftp://tug.ctan.org/pub/tex-archive/macros/latex/contrib/upquote/upquote.sty
%\usepackage{upquote} % Setzt "richtige" Quotes in verbatim-Umgebung

%%% Doc: No Documentation
% \usepackage{verbatim} %Reimplemntation of the original verbatim

%%% Doc: http://www.cs.brown.edu/system/software/latex/doc/fancyvrb.pdf
\usepackage{fancyvrb} % Superior Verbatim Class, allows usage of \verb in footnotes


% ~~~~~~~~~~~~~~~~~~~~~~~~~~~~~~~~~~~~~~~~~~~~~~~~~~~~~~~~~~~~~~~~~~~~~~~~
% science packages
% ~~~~~~~~~~~~~~~~~~~~~~~~~~~~~~~~~~~~~~~~~~~~~~~~~~~~~~~~~~~~~~~~~~~~~~~~

\usepackage{units}


% ~~~~~~~~~~~~~~~~~~~~~~~~~~~~~~~~~~~~~~~~~~~~~~~~~~~~~~~~~~~~~~~~~~~~~~~~
% fancy packages
% ~~~~~~~~~~~~~~~~~~~~~~~~~~~~~~~~~~~~~~~~~~~~~~~~~~~~~~~~~~~~~~~~~~~~~~~~

%%% Doc: No documentation - documented in 'The LaTeX Companion'
% \usepackage{fancybox}   % for shadowbox, ovalbox

%%% Doc: ftp://tug.ctan.org/pub/tex-archive/macros/latex/contrib/misc/framed.sty
% \usepackage{framed}
% \renewcommand\FrameCommand{\fcolorbox{black}{shadecolor}}



%% PW: War so von M. Pospiech
\makeatletter
\IfPackageLoaded{framed}{%
   \IfPackageLoaded{marginnote}{%
      \begingroup
         \g@addto@macro\framed{%
            \let\marginnoteleftadjust\FrameSep
            \let\marginnoterightadjust\FrameSep
         }
       \endgroup 
  }
}
\makeatother



%%% Doc: No documentation - documented in 'The LaTeX Companion'
% \usepackage{boxedminipage}

%%% Doc: ftp://tug.ctan.org/pub/tex-archive/macros/latex/contrib/lettrine/doc/lettrine.pdf
% Dropping capitals
% \usepackage{lettrine}

%% PW: 3D- und Film-Unterstützung für PDF
%\usepackage[%
%3D,%
%%draft,%
%%final,%
%]{movie15}


% ~~~~~~~~~~~~~~~~~~~~~~~~~~~~~~~~~~~~~~~~~~~~~~~~~~~~~~~~~~~~~~~~~~~~~~~~
% layout packages
% ~~~~~~~~~~~~~~~~~~~~~~~~~~~~~~~~~~~~~~~~~~~~~~~~~~~~~~~~~~~~~~~~~~~~~~~~

%%% Diverse Pakete und Einstellungen =====================================

%%% Doc: Documentation inside dtx file
% Mehere Text-Spalten
\usepackage{multicol}

%\nonfrenchspacing     % liefert extra Platz hinter Satzenden.
                       % Fuer deutschen Text standardmaessig ausgeschaltet!


\usepackage{ellipsis}  % >>Intelligente<< \dots

%% Zeilenabstand =========================================================
%
%%% Doc: ftp://tug.ctan.org/pub/tex-archive/macros/latex/contrib/setspace/setspace.sty
\usepackage{setspace}
\setstretch{1.1}    % Aus der Datei Header.tex
%\onehalfspacing		% 1,5-facher Abstand
%\doublespacing		% 2-facher Abstand

\IfPackageLoaded{typearea}{% Wenn typearea geladen ist
	% hereafter load 'typearea' again
	\recalctypearea
}

%% Seitenlayout ==========================================================
%
% Layout laden um im Dokument den Befehl \layout nutzen zu koennen
%%% Doc: no documentation
%\usepackage[verbose]{layout}
%

% Layout mit 'geometry'
%%% Doc: ftp://tug.ctan.org/pub/tex-archive/macros/latex/contrib/geometry/manual.pdf
\usepackage{geometry}
%% Angaben in der KIT-Vorlage:
%\usepackage[a5paper,headheight=1.5\baselineskip,top=25mm,lines=31,heightrounded=true,bindingoffset=15mm,textwidth=106mm]{geometry}
%\usepackage[a4paper,headheight=1.5\baselineskip,top=25mm,lines=46,heightrounded=true,bindingoffset=15mm,textwidth=160mm]{geometry}

\IfPackageLoaded{geometry}{%
\geometry{%
%%% Paper Groesse
   a5paper, % Andere a0paper, a1paper, a2paper, a3paper, , a5paper, a6paper,
            % b0paper, b1paper, b2paper, b3paper, b4paper, b5paper, b6paper
            % letterpaper, executivepaper, legalpaper
   %screen,  % a special paper size with (W,H) = (225mm,180mm)
   %paperwidth=,
   %paperheight=,
   %papersize=, %{ width , height }
   %landscape,  % Querformat
   portrait,    % Hochformat
%%% Koerper Groesse
   %hscale=0.7,      % ratio of width of total body to \paperwidth
                  % hscale=0.8 is equivalent to width=0.8\paperwidth. (0.7 by default)
   %vscale=0.8,      % ratio of height of total body to \paperheight
                  % vscale=0.9 is equivalent to height=0.9\paperheight.
   %scale=,       % ratio of total body to the paper. scale={ h-scale , v-scale }
   %totalwidth=,    % width of total body % (Generally, width >= textwidth)
   %totalheight=,   % height of total body, excluding header and footer by default
   %total=,        % total={ width , height }
   %textwidth=,    % modifies \textwidth, the width of body
	 %textwidth=106mm, %KIT-Vorlage
	 %textwidth=\mytextwidth,
   %textheight=,   % modifies \textheight, the height of body
   %body=,        % { width , height } sets both \textwidth and \textheight of the body of page.
   %lines=,       % enables users to specify \textheight by the number of lines. 
   %lines=35,     % KIT-Vorlage für A5
   %lines=50,     % KIT-Vorlage für A4
   %includehead,  % includes the head of the page, \headheight and \headsep, into total body.
   %includefoot,  % includes the foot of the page, \footskip, into body.
   %includeheadfoot, % sets both includehead and includefoot to true
   %includemp,    % includes the margin notes, \marginparwidth and \marginparsep, into body
   %includeall,   % sets both includeheadfoot and includemp to true.
   %ignorehead,   % disregards the head of the page, headheight and headsep in determining vertical layout
   %ignorefoot,   % disregards the foot of page, footskip, in determining vertical layout
   %ignoreheadfoot, % sets both ignorehead and ignorefoot to true.
   %ignoremp,     % disregards the marginal notes in determining the horizontal margins
   %ignoreall,     % sets both ignoreheadfoot and ignoremp to true
   %heightrounded, % This option rounds \textheight to n-times (n: an integer) of \baselineskip
   %hdivide=,     % { left margin , width , right margin }
                  % Note that you should not specify all of the three parameters
   %vdivide=,     % { top margin , height , bottom margin }
   %divide=,      % ={A,B,C} %  is interpreted as hdivide={A,B,C} and vdivide={A,B,C}.
%%% Margin
   %left=,        % left margin (for oneside) or inner margin (for twoside) of total body
                  % alias: lmargin, inner
   inner=\myinner,  % 18mm
   %inner=2cm,     %
   %right=,       % right or outer margin of total body
                  % alias: rmargin outer
   outer=\myouter, % 15mm
	 %outer=4cm,     %
   %top=27mm,       % top margin of the page.
   %top=3cm,       % top margin of the page.
   top=\mytop,
                  % Alias : tmargin
	 %bottom=29mm,    % bottom margin of the page
   %bottom=4cm,    % bottom margin of the page
                  % Alias : bmargin
   %hmargin=,     % left and right margin. hmargin={ left margin , right margin }
   %vmargin=,     % top and bottom margin. vmargin={ top margin , bottom margin }
   %margin=,      % margin={A,B} is equivalent to hmargin={A,B} and vmargin={A,B}
   %hmarginratio, % horizontal margin ratio of left (inner) to right (outer).
   %vmarginratio, % vertical margin ratio of top to bottom.
   %marginratio,  % marginratio={ horizontal ratio , vertical ratio }
   %hcentering,   % sets auto-centering horizontally and is equivalent to hmarginratio=1:1
   %vcentering,   % sets auto-centering vertically and is equivalent to vmarginratio=1:1
   %centering,    % sets auto-centering and is equivalent to marginratio=1:1
   twoside,       % switches on twoside mode with left and right margins swapped on verso pages.
   %asymmetric,   % implements a twosided layout in which margins are not swapped on alternate pages
                  % and in which the marginal notes stay always on the same side.
   %bindingoffset=5mm,  % removes a specified space for binding
   bindingoffset=\mybindingoffset,
%%% Dimensionen
   %headheight=,  % Alias:  head
   headheight=\myheadheight,
   %headsep=,     % separation between header and text
   headsep=\myheadsep,
   %footskip=,    % distance separation between baseline of last line of text and baseline of footer
   footskip=\myfootskip,
                  % Alias: foot
   %nohead,       % eliminates spaces for the head of the page
                  % equivalent to both \headheight=0pt and \headsep=0pt.
   %nofoot,       % eliminates spaces for the foot of the page
                  % equivalent to \footskip=0pt.
   %noheadfoot,   % equivalent to nohead and nofoot.
   %footnotesep=, % changes the dimension \skip\footins,.
                  % separation between the bottom of text body and the top of footnote text
   marginparwidth=\mymarginparwidth, % width of the marginal notes
   %marginparwidth=0.15\paperwidth,%80pt, % width of the marginal notes
                  % Alias: marginpar
   %marginparsep=,% separation between body and marginal notes.
   marginparsep=\mymarginparsep,
   %nomarginpar,  % shrinks spaces for marginal notes to 0pt
   %columnsep=,   % the separation between two columns in twocolumn mode.
   %hoffset=,
   %voffset=,
   %offset=,      % horizontal and vertical offset.
                  % offset={ hoffset , voffset }
   %twocolumn,    % twocolumn=false denotes onecolumn
   %twoside,
   %textwidth=400pt,   % sets \textwidth directly
   %textheight=700pt,  % sets \textheight directly
   %reversemp,    % makes the marginal notes appear in the left (inner) margin
                  % Alias: reversemarginpar
}
} % Endif

% - Anzeigen des Layouts -
%%Rahmen um Elemente anzeigen
\ifthenelse{\boolean{showFrame}}{%
	\ifthenelse{\boolean{showGrid}}{%
		\usepackage[colorgrid,texcoord,gridunit=mm]{showframe}
	}{
		\IfElsePackageLoaded{geometry}{%
			\geometry{showframe}
		}{%
			\usepackage{showframe}
		}
	}
}


% Farben ================================================================
%
% Farbendefinition ausgelagert in die Datei ColorSettings.tex%% Aussehen der URLS======================================================

%fuer URL (nur wenn url geladen ist)
\IfDefined{urlstyle}{
  % URLs in gleicher Schrift wie der Fließtext statt in Schreibmaschinenschrift (default)
	\urlstyle{same} %tt %sf %rm
}

%% Kopf und Fusszeilen====================================================
%%% Doc: ftp://tug.ctan.org/pub/tex-archive/macros/latex/contrib/koma-script/scrguide.pdf

%\usepackage[%
   %% headtopline,
   %% plainheadtopline,
   %% headsepline,
   %% plainheadsepline,
   %% footsepline,
   %% plainfootsepline,
   %% footbotline,
   %% plainfootbotline,
   %% ilines,
   %% clines,
   %% olines,
	 %%headinclude,          % deprectated
	 %% headexclude,
	 %% footinclude,         % Hilft gegen die zu tief hängenden Seitenzahlen!  % deprectated
	 %% footexclude,
   %automark,              % automatische Aktualisierung der Kolumnentitel
   %% autooneside,         % ignore optional argument in automark at oneside
   %komastyle,             % Stil von Koma Script
   %% standardstyle,       % Stil der Standardklassen
   %% markuppercase,       % Grossbuchstaben erzwingen
   %% markusedcase,        % vordefinierten Stil beibehalten
   %nouppercase,           % Grossbuchstaben verhindern
%]{scrpage2}

\usepackage[%
	automark,%          % automatische Aktualisierung der Kolumnentitel
	headsepline=0.4pt,%  % Horizontale linie zwischen Body und Header, 0.4 pt dick und (defaultmäßig) so lang wie der Text
	%footsepline,%       % Horizontale linie zwischen Body und Footer
	%plainheadsepline,% % Horizontale linie zwischen Body und Header auf leeren Seiten
	%plainfootsepline,% % Horizontale linie zwischen Body und Footer auf leeren Seiten
	markcase=noupper,%  % Grossbuchstaben verhindern
	autooneside=false%
]{scrlayer-scrpage}


\IfElseChapterDefined{%  %werden erst durch scrpage2 definiert
   \pagestyle{scrheadings} % Seite mit Headern
}{
   \pagestyle{scrplain} % Seiten ohne Header
}
%\pagestyle{empty} % Seiten ohne Header
%
% loescht voreingestellte Stile
\clearscrheadings
\clearscrplain

% Angezeigte Abschnitte im Header
\IfElseChapterDefined{%Beim Buch und Report (Kapitel als Gliederungsebene vorhanden):
	 % Zunächst dafür sorgen, dass beidseitig Kapitelname (Chapter) angezeigt wird
	 \ohead{\headmark} % Oben außen: Setzt Kapitel und Section automatisch
   \automark[chapter]{chapter} %[rechts]{links}
   % sofern eine Section vorhanden, rechts ihren Namen anzeigen, ansonsten Kapitelnamen stehen lassen
   \automark*[section]{} %[rechts]{links leer = nichts ändern}
   %% Rechts Section, links Subsection. Falls keine Subsection vorhnanden, leer lassen
   %%\automark[section]{chapter} %[rechts]{links}
	 % Unten aussen: Seitenzahl
   \ofoot[\pagemark]{\pagemark}
}{%Beim Article (Höchste Gliederungsebene ist Section):
   % Zunächst dafür sorgen, dass beidseitig Kapitelname (Chapter) angezeigt wird
   \automark[section]{section} %[rechts]{links}
   % sofern eine Subection vorhanden, rechts ihren Namen anzeigen, ansonsten Section stehen lassen
   \automark*[subsection]{section} %[rechts]{links leer = nichts ändern}
   %% Rechts Section, links Subsection. Falls keine Subsection vorhnanden, leer lassen
   %%\automark[subsection]{section} %[rechts]{links}
	 % Setzt Seitenzahlen auf Kapitelstartseiten in die Mitte (cfoot) 
   \cfoot[\pagemark]{\pagemark} % war cfoot =  Mitte unten: Seitenzahlen bei plain
}
%
% Linien (moegliche Kombination mit Breiten)
\IfChapterDefined{
   %\setheadtopline{}     % modifiziert die Parameter fuer die Linie ueber dem
   							  %	 Seitenkopf
   %\setheadsepline{.4pt}[\color{black}] % Breite wird als Option des Pakets scrlayer-scrpage gesetzt
   \addtokomafont{headsepline}{\color{black}}
                         % modifiziert die Parameter fuer die Linie zwischen
                         % Kopf und Textkörper
   %\setfootsepline{}    % modifiziert die Parameter fuer die Linie zwischen
   							 % Text und Fuß
   %\setfootbotline{}    % modifiziert die Parameter fuer die Linie unter dem
   							 % Seitenfuss
}


%%%Die Einstellungen hier überprüfen!
% Groesse des Headers

% Breite von Kopf und Fusszeile einstellen
% \setheadwidth[Verschiebung]{Breite}
% \setfootwidth[Verschiebung]{Breite}
% mögliche Werte
% paper - die Breite des Papiers
% page - die Breite der Seite
% text - die Breite des Textbereichs
% textwithmarginpar - die Breite des Textbereichs inklusive dem Seitenrand
% head - die aktuelle Breite des Seitenkopfes
% foot - die aktuelle Breite des Seitenfusses
\setheadwidth[0pt]{text}
\setfootwidth[0pt]{text}

%%% PW: Typearea nach scrpage2 laden (scrguide.pdf), sonst werden Kopf-und Fußzeile nicht zur Satzspiegelberechnung einbezogen?!
% Layout mit 'typearea'
%%% Doc: ftp://tug.ctan.org/pub/tex-archive/macros/latex/contrib/koma-script/scrguide.pdf
\IfPackageLoaded{typearea}{% Wenn typearea geladen ist
   \IfPackageNotLoaded{geometry}{% aber nicht geometry
      \typearea[current]{last}
      %%show border lines of the margins etc
      %\usepackage{showframe}
   }
}
% BCOR
%    current  % Satzspiegelberechnung mit dem aktuell gültigen BCOR-Wert erneut
%             % durchführen.
% DIV
%    calc     % Satzspiegelberechnung einschließlich Ermittlung eines guten
%             % DIV-Wertes erneut durchführen.
%    classic  % Satzspiegelberechnung nach dem
%             % mittelalterlichen Buchseitenkanon
%             % (Kreisberechnung) erneut durchführen.
%    current  % Satzspiegelberechnung mit dem aktuell gültigen DIV-Wert erneut
%             % durchführen.
%    default  % Satzspiegelberechnung mit dem Standardwert für das aktuelle
%             % Seitenformat und die aktuelle Schriftgröße erneut durchführen.
%             % Falls kein Standardwert existiert calc anwenden.
%    last     % Satzspiegelberechnung mit demselben DIV -Argument, das beim
%             % letzten Aufruf angegeben wurde, erneut durchführen

% darf erst nach allen KOMAoptions und recalctypearea und areaset usw. stehen. --> Direkt vor begin document:
% Abstand zwischen Textunterkante und Seitenzahl kleiner.
% Forderung des KSP-Verlages: mindestens 10mm bzw. 3 Zeilen
\setlength{\footskip}{\myfootskip}

%% Großzügigere Abstände zwischen den Wörtern, dafür weniger Worttrennungen:
%% Abstandsvergrößerung innerhalb einer Zeile bei unschönem Zeilenumbruch groß genug setzen
\setlength{\emergencystretch}{3em}


%% Fussnoten =============================================================
\deffootnote%
%   [0.8em]% width of marker
   {0.8em}% indentation (general)
   {0.8em}% indentation (par)
   {\makebox[0.8em][l]{\textsuperscript{\thefootnotemark}}}%
%

%% Forderung des KSP-Verlages: Abstand Text <-> Fussnote auf mindestens 1,5 der normalen Leerzeile setzen
\setlength{\skip\footins}{\myfootnoteskip} % Abstand Text <-> Fussnote


\setlength{\dimen\footins}{10\baselineskip} % Beschraenkt den Platz von Fussnoten auf 10 Zeilen

\interfootnotelinepenalty=10000 % Verhindert das Fortsetzen von
                                % Fussnoten auf der gegenüberligenden Seite
%%%% Hurenkinder und Schusterjungen vermeiden
% Schusterjungen (einzelne Zeile unten auf der Seite) unterdrücken:
\clubpenalty=10000 % Alter Wert: 9000
% Hurenkinder (einzelne Zeile oben auf der Seite) unterdrücken
\widowpenalty=10000 % Alter Wert: 9000
% Trennung zwischen Text und Formel
\displaywidowpenalty=9000 % Alter Wert: 9000
% Silbentrennung zwischen zwei Seiten verhindern:
\brokenpenalty=10000
%% Schriften (Sections )==================================================



% -- Koma Schriften --

%% Forderung des KSP-Verlages: Anfang der Überschriften auslinieren
\renewcommand*{\chapterformat}{\makebox[12mm][l]{\thechapter\autodot}}
\renewcommand*{\sectionformat}{\makebox[12mm][l]{\thesection\autodot}}
\renewcommand*{\subsectionformat}{\makebox[12mm][l]{\thesubsection\autodot}}

%% Forderung des KSP-Verlages: Alle Überschriften fett!
%% Forderung des KSP-Verlagesg: Überschriften nicht in Blocksatz, sondern in Flattersatz
%% Forderung des KSP-Verlages: keine Worttrennung in Überschriften, daher \raggedright statt \RaggedRight
%\newcommand\SectionFontStyle{\bfseries \sffamily \RaggedRight}
%\newcommand\SectionFontStyle{\bfseries \rmfamily \RaggedRight}
\newcommand\SectionFontStyle{\raggedright \bfseries \rmfamily}

% Alle Überschriften fett, serifenlos und in Flattersatz ohne Silbentrennung
\setkomafont{sectioning}{\SectionFontStyle}

\IfPartDefined{
	\setkomafont{part}{\SectionFontStyle}
	\addtokomafont{part}{\huge}
	%% Eintrag für die Parts im Inhaltsverzeichnis:
	%\setkomafont{partentry}{\bfseries\large\raggedright}
	%\addtokomafont{partentrydots}{\bfseries}
}

\IfChapterDefined{%
	\setkomafont{chapter}{\LARGE\SectionFontStyle}    % Chapter
	%\setkomafont{chapter}{\huge\SectionFontStyle}    % Chapter
	%Zeilenabstand in mehrzeiligen Überschriften nicht entsprechend
	%dem setstretch-Angabe für den normalen Text vergrößern!
	\addtokomafont{chapter}{\linespread{1}\selectfont}
	%\addtokomafont{chapter}{\singlespacing\selectfont}
	% Kapitel-Einträge im Inhaltsverzeichnis:
	%\setkomafont{chapterentry}{\usekomafont{sectioning}\bfseries}
	\setkomafont{chapterentry}{\bfseries\raggedright} %Eintrag im Inhaltsverzeichnis
	\addtokomafont{chapterentrydots}{\bfseries} % Pünktchen zwischen dem Eintrag und Seitenzahl
}

\setkomafont{section}{\usekomafont{sectioning}}
%Zeilenabstand in mehrzeiligen Überschriften nicht entsprechend
%dem setstretch-Angabe für den normalen Text vergrößern!
\addtokomafont{section}{\Large\linespread{1}\selectfont}
%\setkomafont{sectionentry}{\usekomafont{sectioning}}
\setkomafont{subsection}{\usekomafont{sectioning}}
\addtokomafont{subsection}{\large\linespread{1}\selectfont}
\setkomafont{subsubsection}{\usekomafont{sectioning}}
%\setkomafont{subsubsection}{\bfseries}
\setkomafont{paragraph}{\usekomafont{sectioning}}
%\setkomafont{paragraph}{\bfseries \itshape}
\addtokomafont{paragraph}{\bfseries \itshape}
%\setkomafont{subparagraph}{\usekomafont{sectioning}}
\addtokomafont{subparagraph}{\itshape}

\setkomafont{descriptionlabel}{\itshape}


%% Forderung des KSP-Verlages: Kopfzeilen kleiner als der Fließtext
%\setkomafont{pageheadfoot}{\normalfont\normalcolor\small\sffamily}%warum eingentlich sans?
\setkomafont{pageheadfoot}{\normalfont\normalcolor\footnotesize}
%% Forderung des KSP-Verlages: Kopfzeilen in der gleichen Größe wie der Fließtext, nicht fett
\setkomafont{pagenumber}{\normalfont\normalcolor\normalsize}


%% UeberSchriften (Chapter und Sections) =================================

\addtokomafont{sectioning}{\color{sectioncolor}} % Farbe der Ueberschriften
\IfChapterDefined{%
	\addtokomafont{chapter}{\color{sectioncolor}} % Farbe der Ueberschriften
}
%% Forderung des KSP-Verlages: keine Worttrennung in Überschriften, daher \raggedright statt \RaggedRight
\renewcommand*{\raggedsection}{\raggedright} % Titelzeile linksbuendig, haengend
%% indirekte Auswirkung auch auf \raggedchapter

%%% Adjust space above and below chapter titles, section titles etc.

\RedeclareSectionCommand[%
%%beforeskip=-10pt plus -2pt minus -1pt,%
%%afterskip=1sp plus -1sp minus 1sp%
beforeskip=\mychapterbeforeskip,%
afterskip=\mychapterafterskip%
]{chapter}

\RedeclareSectionCommand[%
%beforeskip=-10pt plus -2pt minus -1pt,%
%afterskip=1sp plus -1sp minus 1sp%
beforeskip=\mysectionbeforeskip,%
afterskip=\mysectionafterskip%
]{section}

\RedeclareSectionCommand[%
%beforeskip=-10pt plus -2pt minus -1pt,%
%afterskip=1sp plus -1sp minus 1sp%
beforeskip=\mysubsectionbeforeskip,%
afterskip=\mysubsectionafterskip%
]{subsection}

\RedeclareSectionCommand[%
%beforeskip=-10pt plus -2pt minus -1pt,%
%afterskip=1sp plus -1sp minus 1sp%
beforeskip=\mysubsubsectionbeforeskip,%
afterskip=\mysubsubsectionafterskip%
]{subsubsection}

\RedeclareSectionCommand[%
%beforeskip=-10pt plus -2pt minus -1pt,%
%afterskip=1sp plus -1sp minus 1sp%
beforeskip=\myparagraphbeforeskip,%
afterskip=\myparagraphafterskip%
]{paragraph}

% -- Ueberschriften komlett Umdefinieren --
%%% Doc: ftp://tug.ctan.org/pub/tex-archive/macros/latex/contrib/titlesec/titlesec.pdf
% Wird gebraucht um die Chapter-Überschrift schön zu machen
% Inkompatibel mit biblatex und Koma-Skript
%\usepackage{titlesec}
%%PW aus http://texblog.net/latex-archive/layout/fancy-chapter-tikz/
%\usepackage[explicit]{titlesec}
%% paragraphs sehen aus wie subsubsubsections
%\titleformat{\paragraph}[hang]{\bf}{\thetitle\quad}{0pt}{}						
%\titlespacing{\paragraph}{0pt}{1em}{0.5em} 
%
%% subparagraphs sehen aus wie vorher paragraphs
%\titleformat{\subparagraph}[runin]{\bf}{}{0.5em}{}
%\titlespacing{\subparagraph}{0pt}{1em}{1em}

% -- Section Aussehen veraendern --
% --------------------------------
%% -> Section mit Unterstrich
% \titleformat{\section}
%   [hang]%[frame]display
%   {\usekomafont{sectioning}\Large}
%  {\thesection}
%   {6pt}
%   {}
%   [\titlerule \vspace{0.5\baselineskip}]
% --------------------------------

% -- Chapter Aussehen veraendern --
% --------------------------------
%--> Box mit (Kapitel + Nummer ) +  Name
% \titleformat{\chapter}[display]     % {command}[shape]
%   {\usekomafont{chapter}\filcenter} % format
%   {                                 % label
%   {\fcolorbox{black}{shadecolor}{
%   {\huge\chaptertitlename\mbox{\hspace{1mm}}\thechapter}
%   }}}
%   {1pc}                             % sep (from chapternumber)
%   {\vspace{1pc}}                    % {before}[after] (before chaptertitle and after)
% --------------------------------
%--> Kapitel + Nummer + Trennlinie + Name + Trennlinie
%\titleformat{\chapter}[display]	% {command}[shape]
%  {\usekomafont{chapter}\Large \color{black}}	% format
%  {   										% label
%  \LARGE\MakeUppercase{\chaptertitlename} \Huge \thechapter \filright%
%  }%}
%  {1pt}										% sep (from chapternumber)
%  {\titlerule \vspace{0.9pc} \filright \color{sectioncolor}}   % {before}[after] (before chaptertitle and after)
%  [\color{black} \vspace{0.9pc} \filright {\titlerule}]


%%% Doc: No documentation
% Indent first paragraph after section header
% \usepackage{indentfirst}


%for images at the title page
\usepackage{wallpaper}
%\setlength{\wpYoffset}{0cm} %negative Werte verschieben nach unten
%\addtolength{\wpXoffset}{0.0cm}







%% Captions (Schrift, Aussehen) ==========================================

% % Folgende Befehle werden durch das Paket caption und subfig ersetzt !
% \setcapindent{1em} % Einrueckung der Beschriftung
% \setkomafont{caption}{\color{black}\small\sffamily\RaggedRight}  % Schrift fuer Caption
% \setkomafont{captionlabel}{\color{black}\small}   % Schrift fuer 'Abbildung' usw.

%%% Doc: ftp://tug.ctan.org/pub/tex-archive/macros/latex/contrib/caption/caption.pdf
\usepackage{caption}
%\usepackage[aboveskip=\myaboveskip,belowskip=\mybelowskip]{caption}
% Caption fuer nicht fliessende Umgebungen
%%% Doc: ftp://tug.ctan.org/pub/tex-archive/macros/latex/contrib/misc/capt-of.sty
\IfPackageNotLoaded{caption}{
	\usepackage{capt-of} % only load when caption is not loaded. Otherwise compiling will fail.
	%Usage: \captionof{table}[short Titel]{long Titel}
}
%

%Funktioniert auch nicht...
%\setlength{\abovecaptionskip}{\mycaptionskip}
%\setlength{\belowcaptionskip}{\mycaptionskip}
%\setlength{\subfigtopskip}{0pt}
%\setlength{\subfigbottomskip}{pt}

% Aussehen der Captions
\captionsetup[figure]{
   margin = 0pt,
   aboveskip = \myaboveskip,
   belowskip = \mybelowskip,
   captionskip = \mycaptionskip,
   %skip = \mycaptionskip,
   font = {footnotesize,rm},
   labelfont = {footnotesize,bf},
   %format = plain, % oder 'hang'
   format = hang, %zweite und weitere Zeilen einrücken (an die erste ausrichten)
   indention = 0em,  % Einruecken der Beschriftung
   labelsep = colon, %period, space, quad, newline
   %justification = RaggedRight, % Falettersatz
   %justification = centering, % zentriert
   justification = justified, % Blocksatz
   singlelinecheck = true, % false (true=bei einer Zeile immer zentrieren)
   position = bottom %top
}
\captionsetup[table]{
   margin = 0pt,
   aboveskip = \myaboveskip,
   belowskip = \mybelowskip,
   captionskip = \mycaptionskip,
   %skip = \mycaptionskip,
   font = {footnotesize,rm},
   labelfont = {footnotesize,bf},
   %format = plain, % oder 'hang'
   format = hang, %zweite und weitere Zeilen einrücken (an die erste ausrichten)
   indention = 0em,  % Einruecken der Beschriftung
   labelsep = colon, %period, space, quad, newline
   %justification = RaggedRight, % Falettersatz
   %justification = centering, % zentriert
   justification = justified, % Blocksatz
   singlelinecheck = true, % false (true=bei einer Zeile immer zentrieren)
   position = top %bottom (wird durch die Einstellung \floatsetup[table]{style=plain,capposition=TOP} überrufen!!!)
}
\captionsetup[lstlisting]{
   margin = 0pt,
   aboveskip = \myaboveskip,
   belowskip = \mybelowskip,
   captionskip = \mycaptionskip,
   skip = \mycaptionskip,
   font = {footnotesize,rm},
   labelfont = {footnotesize,bf},
   %format = plain, % oder 'hang'
   format = hang, %zweite und weitere Zeilen einrücken (an die erste ausrichten)
   indention = 0em,  % Einruecken der Beschriftung
   labelsep = colon, %period, space, quad, newline
   %justification = RaggedRight, % Falettersatz
   %justification = centering, % zentriert
   justification = justified, % Blocksatz
   singlelinecheck = true, % false (true=bei einer Zeile immer zentrieren)
   position = bottom %top (ggf. weitere Einstellung in lstset beachten!)
}
%%% Bugfix Workaround
\DeclareCaptionOption{parskip}[]{}
\DeclareCaptionOption{parindent}[]{}

% Aussehen der Captions fuer subfigures (subfig-Paket)
\IfPackageLoaded{subfig}{
 \captionsetup[subfloat]{%
   %margin = 10pt,
   margin = 5pt,
   %margin = 0pt,
   aboveskip = \myaboveskip,
   belowskip = \mybelowskip,
   farskip = 0pt, %Glue placed opposite the sub-float caption (default: 10pt)
   nearskip = 0pt, %Glue placed opposite the caption from the sub-float (default: 0pt)
   captionskip = \mysubcaptionskip, %Glue placed between the sub-float and the caption (default: 4pt)
   %skip = \mycaptionskip,
   %topadjust=0pt, %Extra glue added to 'captionskip' when above the sub-float (default: 0pt)
   font = {footnotesize,rm},
   labelfont = {footnotesize,bf},
   %format = plain, % oder 'hang'
   format = hang, %zweite und weitere Zeilen einrücken (an die erste ausrichten)
   indention = 0em,  % Einruecken der Beschriftung
   labelsep = space, %period, space, quad, newline
   %justification = RaggedRight, % Flattersatz, mit Silbentrennung
   justification = raggedright, % Flattersatz ohne Silbentrennung
   %justification = centering, % zentriert
   %justification = justified, % Blocksatz
   singlelinecheck = true, % false (true=bei einer Zeile immer zentrieren)
   position = bottom, %top
   labelformat = parens % simple, empty % Wie die Bezeichnung gesetzt wird
 }
}

% \numberwithin{figure}{chapter} %Befehl zum Kapitelweise Nummerieren der Bilder, setzt `amsmath' vorraus
% \numberwithin{table}{chapter}  %Befehl zum Kapitelweise Nummerieren der Tabellen, setzt `amsmath' vorraus

%for better aligning images and captions
% has to be loaded after call of \captionsetup
\usepackage{floatrow}
\floatsetup[table]{style=plain,capposition=TOP}

%% Inhaltsverzeichnis (Schrift, Aussehen) sowie weitere Verzeichnisse ====

%Für TODO-Liste:
\usepackage[subfigure,titles]{tocloft}

%Wird in der Datei settings.tex gesetzt
%\setcounter{secnumdepth}{2}    % Abbildungsnummerierung mit groesserer Tiefe
%\setcounter{tocdepth}{3}		 % Inhaltsverzeichnis mit groesserer Tiefe
%Fonts in Kapiteln und sections...
\renewcommand\cftchapfont{\normalsize\bfseries\raggedright}
\renewcommand\cftsecfont{\normalsize\raggedright}
\renewcommand\cftsubsecfont{\normalsize\raggedright}
%
% Gepunktete Verbindung zwischen den Kapiteleinträgen und korrespondierenden Seitenzahlen:
% funktioniert!
\renewcommand{\cftchapdotsep}{\cftdotsep}
\renewcommand{\cftchapleader}{\cftdotfill{\cftchapdotsep}}

% Seitenzahlen bei Part unterdrücken
\cftpagenumbersoff{part}

% Einzug im Abbildungsverzeichnis zu Null setzen
\renewcommand{\cftfigindent}{0cm}
% Einzug im Tabellenverzeichnis zu Null setzen
\renewcommand{\cfttabindent}{0cm}

%\usepackage[tocgraduated]{tocstyle}
%\usetocstyle{allwithdot}

%Abstand zwischen Text und Seitenzahlen im Inhaltsverzeichnis vergrößern:
%\cftsetpnumwidth{2em}
%\cftsetrmarg{2.5em}

%% Flattersatz ohne Silbentrennung in Inhaltsverzeichnis
%(Lösung aus https://tex.stackexchange.com/questions/283730/left-align-toc-items-when-using-tableofcontents)
\makeatletter
\bgroup
\advance\@flushglue by \@tocrmarg
\xdef\@tocrmarg{\the\@flushglue}%
\egroup
\makeatother


% Workaround for correct setting of the list of listings taken from
% https://tex.stackexchange.com/questions/319372/customize-list-of-listings/319374#319374
\makeatletter
\AtBeginDocument{%
\renewcommand\lstlistoflistings{\bgroup
	\let\contentsname\lstlistlistingname
	%\def\l@lstlisting##1##2{\@dottedtocline{1}{1.5em}{2.3em}{\bfseries L ##1}{##2}}
	\def\l@lstlisting##1##2{\@dottedtocline{1}{0em}{2.3em}{##1}{##2}}
	\let\lst@temp\@starttoc \def\@starttoc##1{\lst@temp{lol}}%
	\tableofcontents 
	\addcontentsline{toc}{chapter}{\lstlistlistingname} % <-- diese Zeile selbst hinzugefügt
	\egroup}
}
\makeatother

%% Workaround for correct setting of the list of listings taken from
%% https://www.mrunix.de/forums/showthread.php?56028-List-of-Listings
%% (hat nicht funktioniert):
%\makeatletter
%\renewcommand{\l@lstlisting}[2]{\@dottedtocline{1}{0em}{2.3em}{#1}{#2}}
%\makeatother


% Inhalte von List of Figures
\IfPackageLoaded{subfig}{
	\setcounter{lofdepth}{1}  %1 = nur figures, 2 = figures + subfigures
}


\usepackage{here}

%% PW
%% geht nur grau
\ifthenelse{\boolean{printMuster}}%
{\usepackage{draftwatermark}%
\SetWatermarkText{\TransMuster}}{}


%% PW: Todo-Liste. Braucht anscheinend eines der begrenzt verfügbaren TeX-\writes
\ifthenelse{\boolean{showTODOs}}{%
	\usepackage[textsize=footnotesize,ngerman,colorinlistoftodos]{todonotes}
}{%
	\usepackage[disable]{todonotes}
}
% -------------------------------------------------------

% Aussehen des Inhaltsverzeichnisses: tocloft
%%% Doc: ftp://tug.ctan.org/pub/tex-archive/macros/latex/contrib/tocloft/tocloft.pdf
%% Laden mit Option subfigure in Abhaengigkeit vom Paket subfigure und subfig
% \IfElsePackageLoaded{subfig}
% 	% IF subfig
% 	{\usepackage[subfigure]{tocloft}}{
% 	% ELSE
% 	\IfElsePackageLoaded{subfigure}
% 		% IF subfigure
% 		{\usepackage[subfigure]{tocloft}}
% 	   % Else (No subfig nor subfigure)
% 		{\usepackage{tocloft}}
% 	}
%
% %TOCLOFT zerstoert Layout der Ueberschriften von TOC, LOT, LOF
% \IfPackageLoaded{tocloft}{
% %
% %%%% Layout Matthias Pospiech (alles serifenlos)
% \IfChapterDefined{%
% 	\renewcommand{\cftchappagefont}{\bfseries\sffamily}  % Kapitel Seiten Schrift
% 	\renewcommand{\cftchapfont}{\bfseries\sffamily}      % Kapitel Schrift
% }
% \renewcommand{\cftsecpagefont}{\sffamily}            % Section Seiten Schrift
% \renewcommand{\cftsubsecpagefont}{\sffamily}         % Subsectin Seiten Schrift
% \renewcommand{\cftsecfont}{\sffamily}                % Section Schrift
% \renewcommand{\cftsubsecfont}{\sffamily}             % Subsection Schrift
%
% %%%% Layout aus Typokurz:
% % % Seitenzahlen direkt hinter TOC-Eintrag:
% % % Ebene \chapter
% % \renewcommand{\cftchapleader}{}
% % \renewcommand{\cftchapafterpnum}{\cftparfillskip}
% % % Ebene \section
% % \renewcommand{\cftsecleader}{}
% % \renewcommand{\cftsecafterpnum}{\cftparfillskip}
% % % Ebene \subsection
% % \renewcommand{\cftsubsecleader}{}
% % \renewcommand{\cftsubsecafterpnum}{\cftparfillskip}
% % % Abstaende vor Eintraegen im TOC verkleinern
% % \setlength{\cftbeforesecskip}{.4\baselineskip}
% % \setlength{\cftbeforesubsecskip}{.1\baselineskip}
%
% % wenn subsubsections unnumeriert:
% % - Einzug der ersten Zeile manuell vergrößern
% % - zusätzlichen Einzug der nachfolgenden Zeilen 
% %   (bei einem eventuellen Zeilenumbruch) entfernen
% % (Zahlen aus der Dokumentation zu tocloft-Pakage):
%\setlength{\cftsubsubsecindent}{10.0em} %normalerweise: {7.0em}
%\setlength{\cftsubsubsecnumwidth}{0em} %normalerweise: {4.1em}
% }
% % Ende tocloft Einstellungen --------------

%%% Doc: ftp://tug.ctan.org/pub/tex-archive/macros/latex/contrib/ms/multitoc.dvi
% TOC in mehreren Spalten setzen
%\usepackage[toc]{multitoc}

% -------------------------

%%Schriften fuer Minitoc (Inhaltsverzeichnis vor jedem Kapitel)
%%% Doc: ftp://tug.ctan.org/pub/tex-archive/macros/latex/contrib/minitoc/minitoc.pdf
%\IfElseChapterDefined{%
% \usepackage{minitoc}
% \setlength{\mtcindent}{0em} % default: 24pt
% \setcounter{minitocdepth}{2}
% \setlength{\mtcskipamount}{\bigskipamount}
% \mtcsettitlefont{minitoc}{\normalsize\SectionFontStyle}
% \mtcsetfont{minitoc}{*}{\small\SectionFontStyle} %\color{textcolor}
% \mtcsetfont{minitoc}{section}{\small\SectionFontStyle}
% \mtcsetfont{minitoc}{subsection}{\small\SectionFontStyle}
% \mtcsetfont{minitoc}{subsubsection}{\small\SectionFontStyle}
%}{
% \usepackage{minitoc}
% \setlength{\stcindent}{0pt} %default
% \setcounter{secttocdepth}{2} %default
% \mtcsettitlefont{secttoc}{\SectionFontStyle}
% \mtcsetfont{secttoc}{*}{\small\SectionFontStyle}%
% \mtcsetfont{secttoc}{subsection}{\small\SectionFontStyle}
% \mtcsetfont{secttoc}{subsubsection}{\small\SectionFontStyle}
%}

% Packages that MUST be loaded before minitoc !
% hyperref, caption, sectsty, varsects, fncychap, hangcaption, quotchap, romannum, sfheaders, alnumsec, captcont


%% Index & Co. ===========================================================
% gibts dafuer noch eine sauberere Loesung ?
%%%%%%%% Index zweispaltig %%%%%%%
% \makeatletter
% \renewenvironment{theindex}{%
% \setlength{\columnsep}{2em}
% \begin{multicols}{2}[\section*{\indexname}]
% \parindent\z@
% \parskip\z@ \@plus .3\p@\relax
% \let\item\@idxitem}%
% {\end{multicols}\clearpage}
% \makeatother
%%%%%%%%%%%%%%%%%%%%%%%%%%%%%%%%%%


%% PW: idxlayout. Layout des Stichwortverzeichnisses (Index) verändern.
%\usepackage{idxlayout}

%% PW: glossaries.
%% Nach hyperref laden!!
%% Mehrere Verzeichnisse erstellen. Perl ist erwünscht, aber nicht zwingend. Bequemer ists mit.
%% Jedenfalls ist glossaries etwas umfangreicher und die Einträge zu erstellen hat etwas kompliziertere Syntax. z.B.:
%% \newglossaryentry{oesophagus}{name=\oe sophagus, description={canal from mouth to stomach}, plural=\oe sophagi, sort=oesophagus}
%% Mehr Infos, Tips und Anleitung zum Paket unter: http://www.mrunix.de/forums/showthread.php?t=68892
%% Glossaries Doku hat 260 Seiten...
%\usepackage{glossaries}


% ~~~~~~~~~~~~~~~~~~~~~~~~~~~~~~~~~~~~~~~~~~~~~~~~~~~~~~~~~~~~~~~~~~~~~~~~
% PDF related packages
% ~~~~~~~~~~~~~~~~~~~~~~~~~~~~~~~~~~~~~~~~~~~~~~~~~~~~~~~~~~~~~~~~~~~~~~~~

%%%%%% Doc: ftp://tug.ctan.org/pub/tex-archive/macros/latex/contrib/microtype/microtype.pdf
%%%%%% Optischer Randausgleich mit pdfTeX
\usepackage{microtype}
\microtypesetup{%
%	protrusion=true,
	protrusion=false,
}
\ifpdf
\microtypesetup{
	expansion=true, % better typography, but with much larger PDF file. --> mit pdfTeX 1.40 nicht mehr
%	expansion=false,
%	protrusion=true,
%	protrusion=false,
	kerning=true,
%	spacing=true,			%%stellt bei rechtsbündigem Satz manchmal Probleme dar (demo.tex, "`In id augue"', Rechtsbündig Ende)
%	tracking=false,   % >= pdfTeX 1.40.04
%	tracking=alltext,  % without specifying further: default expansion by 0.1 em
% tracking=true,   % >= pdfTeX 1.40.04
%	letterspace=20,  %for tracking. Leichte Sperrung,besonders bei Kapitälchen
%letterspace=-15,  %for tracking. %funzt mit tracking=alltext. Bereich von -1000 bis 1000 Teile von 1 em  % Für kpfonts sind -5 bis -10 ganz gut
%	letterspace= 5,  %for tracking. von -1000 bis 1000 Teile von 1 em  % Für Palatino
}
%\DeclareMicrotypeSet{pwtext}
   %{ encoding = {OT1,T1,T2A,LY1,OT4,QX,T5},
   %
     %family   = {rm*,sf*},
     %series   = {md*},
     %size     = {normalsize,footnotesize,small,large}
   %}
%was passiert hier? Ist das nötig?
\SetTracking[spacing={100*,-300, }]{encoding = *, family = jkp, shape = sc}{100}
\SetTracking{ encoding = *, family = jkp}{ -200 }
\SetTracking{ encoding = *, family = jkpss}{ -100 }
 
\SetTracking{ encoding = *, family = jkpl}{ -1000 }
\SetTracking{ encoding = *, family = jkpss, shape = rm*}{ -2000 }
\SetTracking[ no ligatures = f ]{ family=jkpss, encoding = *, shape = sc}{ 1000 }
\SetTracking{ encoding = *, size = -small}{ 4}
\SetTracking{ encoding = *, size = Large-}{ -4 }
\fi

%\microtypecontext{spacing=nonfrench}



% Zeilen auf der Seite verteilen
%( Es wird kein Ausgleich des unteren Seitenrandes durch Dehnung der Absatzabstände durchgeführt.) 
\raggedbottom     % Variable Seitenhoehen zulassen



%% Use only instead of hyperref !
% \usepackage[%
%    %ref,     % verweist auf Abschnitte
%    pageref, % verweist auf Seiten
% ]{backref} % Links in BiB back to Citation page/section (can be loaded by hyperref too)


%%%%% Doc: ftp://tug.ctan.org/pub/tex-archive/macros/latex/contrib/hyperref/doc/manual.pdf
\usepackage[%
   unicode,%                %hinzugefügt, da Fehler bei Verwendung von XeLaTeX
   pdfencoding=auto,        %hinzugefügt, da Fehler bei Verwendung von XeLaTeX
%%   									 % rechts im PDF-Viewer
%%%pdfx%   pdfa=true,%
%%   %pdfpagelayout=SinglePage, % einseitige Darstellung
%]{hyperref}  %% Braucht auch ohne backref anscheinend eines der begrenzt verfügbaren TeX-\writes, mit backref dann zwei
%
%\usepackage{hyperref}
%
%%%%PW: pdfx lädt seinerseits hyperref
%%%%Optionen a-1b oder x-1a
%\usepackage[a-1b]{pdfx}
%
%\hypersetup{
   % Farben fuer die Links
   colorlinks=true,         % Links erhalten Farben statt Kaeten
   urlcolor=pdfurlcolor,    % \href{...}{...} external (URL)
   filecolor=pdffilecolor,  % \href{...} local file
   linkcolor=pdflinkcolor,  %\ref{...} and \pageref{...}
   citecolor=pdfcitecolor,  %
   % Links
   raiselinks=true,			 % calculate real height of the link
   breaklinks=true,              % Links überstehen Zeilenumbruch
%   backref=page,            % Backlinks im Literaturverzeichnis (section, slide, page, none)
%   pagebackref=true,        % Backlinks im Literaturverzeichnis mit Seitenangabe
   verbose,
   hyperindex=true,         % backlinkex index
   linktocpage=true,        % Inhaltsverzeichnis verlinkt Seiten
   hyperfootnotes=false,     % Keine Links auf Fussnoten
   % Bookmarks
   bookmarks=true,          % Erzeugung von Bookmarks fuer PDF-Viewer
   bookmarksopenlevel=1,    % Gliederungstiefe der Bookmarks
   bookmarksopen=true,      % Expandierte Untermenues in Bookmarks
   bookmarksnumbered=true,  % Nummerierung der Bookmarks
   bookmarkstype=toc,       % Art der Verzeichnisses
   bookmarksdepth=\mybookmarkdepth,        %Tiefe des Inhaltsverzeichnisses in den Lesezeichen anders setzen
   % Anchors
   plainpages=false,        % Anchors even on plain pages ?
   pageanchor=true,         % Pages are linkable
   % PDF Informationen
   pdfinfo={
		Title={\WorkTitle},             % Titel
		Author={\AuthorName},            % Autor
		Subject={\TypeOfThesis},
		Creator={XeLaTeX, hyperref, KOMA-Script}%
	 }, % Ersteller
   %pdfproducer={pdfeTeX 1.10b-2.1} %Produzent
   pdfdisplaydoctitle=true, % Dokumententitel statt Dateiname im Fenstertitel
   pdfstartview=FitH,       % Dokument wird Fit Width geaefnet
   pdfpagemode=UseOutlines, % Bookmarks im Viewer anzeigen
   pdfpagelabels=true,           % set PDF page labels
   %pdfpagelayout=TwoPageRight, % zweiseitige Darstellung: ungerade Seiten
   									 % rechts im PDF-Viewer
   pdfpagelayout=SinglePage, % einseitige Darstellung
%}
]{hyperref}

% (replaces the hyperref functionality regarding pdf bookmarks)
\usepackage{bookmark}

%% PW: Wegen "destination with the same identifier"
%% http://tex.stackexchange.com/questions/65182/cross-references-linking-to-wrong-equations-using-hyperref
%\renewcommand{\theHequation}{\theHsection.\equationgrouping\arabic{equation}}


% PW: für bbordermatrix mit eckigen Klammern
% Alternativ: klassische plainTeX \bordermatrix mit eckigen Klammern (Redefinition von Befehlen):
\usepackage{etoolbox}


%% Zeilenumbrüche in urls nach folgenden Zeichen
%\appto\UrlBreaks{\do\a\do\b\do\c\do\d\do\e\do\f\do\g\do\h\do\i\do\j\do\k\do\l\do\m\do\n\do\o\do\p\do\q\do\r\do\s\do\t\do\u\do\v\do\w\do\x\do\y\do\z\do\/\do\.}
%\PassOptionsToPackage{hyphens}{url}

%%% Doc: ftp://tug.ctan.org/pub/tex-archive/macros/latex/contrib/oberdiek/hypcap.pdf
% Links auf Gleitumgebungen springen nicht zur Beschriftung,
% sondern zum Anfang der Gleitumgebung
\IfPackageLoaded{hyperref}{%
	\usepackage[figure,table]{hypcap}
}


%%DIN1505
%%\usepackage[backref=true,backend=bibtex8,style=alphabetic,maxnames=10]{biblatex}
\usepackage[backend=biber,%
						style=alphabetic,%
						labelnumber,% zusätzlich labelnummer zur Verfügung stel
						defernumbers=true,%
						%maxnames=10,% Maximal 10 Autorennamen zulassen, wenn >10, dann lasse nur einen + et al. (ggf. änderbar mit  der Option "minnames"
						maxbibnames=10,%
						maxcitenames=1,%
						maxalphanames=1,% Anzahl der aufzuführenden Namen in Bib-Label
						%dashed=false,%will print recurring author/editor names instead of replacing them by a dash in case of authoryear, authortitle, and verbose bibliography styles)
						isbn=false,%die ISBN-Felder nicht drucken
						%doi=false,%
						%eprint=false,%
						block=none,
						%block=space,% Füge zusätzlichen horizontalen Raum zwischen den Einträgen hinzu.
						%block=par,% Beginne einen neuen Paragraphen für jeden Eintrag
						%block=nbpar,% Ähnlich der par-Option, aber verbietet Seitensprünge zwischen den Übergängen und innerhalb der Einträge.
						%block=ragged,% Fügt einen kleinen Strafraum ein, um Zeilenumbrüche an Blockgrenzen zu fördern und die Bibliografie rechts orientiert (Flattersatz) zu setzen
						backref=true%Entscheidet,ob die Endreferenzen (Seitenzahlen) in die Bibliografie geschrieben werden sollen. (default=false)
						]{biblatex}

%% Das "+"-Zeichen nach mehr als maxcitenames Authoren entfernen durch Umdefinieren von "labelalphaothers":
\renewcommand*{\labelalphaothers}{}

%% Reihenfolge umdrehen: Nachname, Vorname
%\DeclareNameAlias{default}{last-first} %deprecated
\DeclareNameAlias{default}{family-given}

%% Separator zwischen den Autornamen: Semikolon statt Komma
%\renewcommand*{\multinamedelim}{;\space}
\DeclareDelimFormat{multinamedelim}{\addsemicolon\space}

%% Separator vor dem letzten den Autornamen: ohne Komma vor dem "and" bei mehreren Authoren
%% da Komma den Nach- und Vornamen trennt
\renewcommand*{\finalnamedelim}{\space and\space}

%% Nach Autornamen: Doppelpunkt statt Punkt:
\renewcommand*{\labelnamepunct}{\addcolon\addspace}

%% Vergrößerung des hängenden Einzuges
%\setlength{\bibhang}{5em}
%\addtolength{\bibhang}{2em}


%% Nachnamen in Kapitälchen
\AtBeginBibliography{\renewcommand*{\mkbibnamefamily}[1]{\textsc{#1}}}

%% Titel: nicht kursiv
\DeclareFieldFormat{title}{{#1}}

%%% Hack, damit man mehrere Bibliografien mit verschiedenen Zitationsstilen hat:
\DeclareFieldFormat{labelnumberwidth}{\mkbibbrackets{#1}}

%% Mehrere Zitate in einer Klammer, aber Komma statt Semikolon:
\renewcommand*{\multicitedelim}{\addcomma\addspace}
%% Nicht mehrere Zitate in einer Klammer, sondern eine Klammer pro Zitat:
%\renewcommand*{\multicitedelim}{\bibclosebracket\addcomma\addspace\bibopenbracket}

%% Definition für die allgemeine Literaturliste
\defbibenvironment{bibliography}
  {\list
     {\printtext[labelalphawidth]{%
        \printfield{labelprefix}%
        \printfield{labelalpha}}}
     {\setlength{\labelwidth}{\labelalphawidth}%
      \setlength{\leftmargin}{\labelwidth}%
      \setlength{\labelsep}{\biblabelsep}%
      \addtolength{\leftmargin}{\labelsep}%
      %\setlength{\itemindent}{-\labelsep}%
      \setlength{\itemsep}{\bibitemsep}%
      \setlength{\parsep}{\bibparsep}%
      %\addtolength{\bibhang}{3em}%
      }%
      %% \hss für rechtsbündige Ausrichtung
      %\renewcommand*{\makelabel}[1]{\hss##1}}
      \renewcommand*{\makelabel}[1]{##1}}
  {\endlist}
  {\item}


%% Definition für die Liste eigener Publikationen
\defbibenvironment{bibliographyNUM}
  {\list
     {\printtext[labelnumberwidth]{%
        \printfield{labelprefix}%
        \printfield{labelnumber}}}
     {\setlength{\labelwidth}{\labelnumberwidth}%
      \setlength{\leftmargin}{\labelwidth}%
      \setlength{\labelsep}{\biblabelsep}%
      \addtolength{\leftmargin}{\labelsep}%
      %\setlength{\itemindent}{-\labelsep}%
      \setlength{\itemsep}{\bibitemsep}%
      \setlength{\parsep}{\bibparsep}%
      %\addtolength{\bibhang}{3em}%
      }%
      %% \hss für rechtsbündige Ausrichtung
      %\renewcommand*{\makelabel}[1]{\hss##1}}
      \renewcommand*{\makelabel}[1]{##1}}
  {\endlist}
  {\item}

%\IfPackageLoaded{backref}{
   %% % Change Layout of Backref
   %\renewcommand*{\backref}[1]{%
   	%% default interface
   	%% #1: backref list
   	%%
   	%% We want to use the alternative interface,
   	%% therefore the definition is empty here.
   %}%
   %\renewcommand*{\backrefalt}[4]{%
   	%% alternative interface
   	%% #1: number of distinct back references
   	%% #2: backref list with distinct entries
   	%% #3: number of back references including duplicates
   	%% #4: backref list including duplicates
   	%\mbox{(Referenced on %Zitiert auf %
   	%\ifnum#1=1 %
		  %page~% 
			%%Seite~%
	   %\else
		  %pages~%
   		%%Seiten~%
   	%\fi
   	%#2)}%
   %}
%}



\IfPackageLoaded{backref}%
{
\ifthenelse{\boolean{englishAsMainLanguage}}{%
	\def\backrefpagesname{pages}%
	\def\backrefsectionsname{sections}%
	\def\backrefsep{, }%
	\def\backreftwosep{ and~}%
	\def\backreflastsep{, and~}%
}{%
	\def\backrefpagesname{Seiten}%
	\def\backrefsectionsname{Abschnitte}%
	\def\backrefsep{, }%
	\def\backreftwosep{ und~}%
	\def\backreflastsep{ und~}%
}

   % % Change Layout of Backref
   \renewcommand*{\backref}[1]%
	{%
   	% default interface
   	% #1: backref list
   	%
   	% We want to use the alternative interface,
   	% therefore the definition is empty here.
  }%
  \renewcommand*{\backrefalt}[4]%
	{%
   	% alternative interface
   	% #1: number of distinct back references
   	% #2: backref list with distinct entries
   	% #3: number of back references including duplicates
   	% #4: backref list including duplicates
   	\ifnum#1=0 %
	  {}%
		\else%
		\hfill\mbox{%
   		\ifthenelse{\boolean{englishAsMainLanguage}}{p.~}{S.~}%%
			#2}%
   	\fi%
		}
 }


\usepackage{cleveref}
\ifthenelse{\boolean{englishAsMainLanguage}}{%
	\crefname{chapter}{Chapter}{Chapters}
	\crefname{section}{Section}{Sections}
	\crefname{subsection}{Section}{Sections}
	\crefname{figure}{Figure}{Figures}
	\crefname{table}{Table}{Tables}
	\crefname{listing}{Listing}{Listing}
	\crefname{appendix}{Appendix}{Appendixes}

	\crefname{theorem}{Theorem}{Theorems}
	\crefname{definition}{Definition}{Definitions}
	\crefname{lemma}{Lemma}{Lemmata}
	\crefname{corollary}{Korollar}{Korollars}
	\crefname{proposition}{Proposition}{Propositions}

	\newcommand{\crefpairconjunction}{ and }
	\newcommand{\crefmiddleconjunction}{, }
	\newcommand{\creflastconjunction}{ and }
	\newcommand{\crefpairgroupconjunction}{ as well as }
	\newcommand{\crefmiddlegroupconjunction}{, }
	\newcommand{\creflastgroupconjunction}{ as well as }
}{%
	\crefname{chapter}{Kapitel}{Kapitel}
	\crefname{section}{Abschnitt}{Abschnitte}
	\crefname{subsection}{Abschnitt}{Abschnitte}
	\crefname{figure}{Abbildung}{Abbildungen}
	\crefname{table}{Tabelle}{Tabellen}
	\crefname{listing}{Listing}{Listing}
	\crefname{appendix}{Anhang}{Anhänge}

	\crefname{theorem}{Satz}{Sätze}
	\crefname{definition}{Definition}{Definitionen}
	\crefname{lemma}{Lemma}{Lemmata}
	\crefname{corollary}{Korollar}{Korollare}
	\crefname{proposition}{Proposition}{Propositionen}

	\newcommand{\crefpairconjunction}{ und }
	\newcommand{\crefmiddleconjunction}{, }
	\newcommand{\creflastconjunction}{ und }
	\newcommand{\crefpairgroupconjunction}{ sowie }
	\newcommand{\crefmiddlegroupconjunction}{, }
	\newcommand{\creflastgroupconjunction}{ sowie }
}


%% PW: ein paar Klimmzüge wegen der Raute statt Gleichheitszeichen für ps2pdf unter Windows.
\ifwindows
\ifxetex
%XeLaTeX unter Windows: hier passiert nichts
%\usepackage[cleanup={},mode=nonstop]{pstool}
\else
%pdfLaTeX unter Windows:
\usepackage{pstool}
\begingroup
\makeatletter
\catcode`\#=11
    \gdef\pstool@bitmap@opts{%
      -dAutoFilterColorImages#false
      -dAutoFilterGrayImages#false %
      -dColorImageFilter#/FlateEncode %
      -dGrayImageFilter#/FlateEncode % space
      }
     \gdef\pstool@pspdf@opts{%
				-dPDFSETTINGS#/prepress %
				-dCompatibilityLevel#1.3 %
				-dEmbedAllFonts#true %
				-dSubsetFonts#true
			}
\makeatother
\endgroup
\fi
\else
%pdfLaTeX oder XeLaTeX unter Linux:
%hier gibt es zur Zeit einen Fehler im IOSB-Cluster
%\usepackage[ps2pdf-options={-dPDFSETTINGS=/prepress -dCompatibilityLevel=1.3 -dEmbedAllFonts=true -dSubsetFonts=true}]{pstool}
\fi
%\usepackage[cleanup={},mode=nonstop,ps2pdf-options={"-dPDFSETTINGS=/prepress"}]{pstool}
%%PW: ps2pdf-options={"-dPDFSETTINGS=/prepress"} sorgt für Einbettung der Base14-Schriften, was laut KIT-SP-Richtlinien erwünscht ist.
%% Braucht anscheinend zwei der begrenzt verfügbaren TeX-\writes: Eins für psfrag und eins für pstool
%\usepackage[cleanup={},mode=nonstop,ps2pdf-options={"-dPDFSETTINGS=/prepress" "-dCompatibilityLevel=1.3"},latex-options={-interaction=nonstopmode -max-print-line=120  --enable-write18}]{pstool}
%\usepackage[cleanup={},mode=nonstop,ps2pdf-options={"-dPDFSETTINGS\#/prepress" "-dCompatibilityLevel\#1.3"}]{pstool}
%\usepackage[cleanup={}]{pstool}





% Auch Abbildung und nicht nur die Nummer wird zum Link (abgeleitet
% aus Posting von Heiko Oberdiek (d09n5p$9md$1@news.BelWue.DE);
% Verwendung: In \abbvref{label} ist ein Beispiel dargestellt
\providecommand*{\figrefname}{Abbildung }
\newcommand*{\figref}[1]{%
  \hyperref[fig:#1]{\figrefname{}}\ref{fig:#1}%
}
% ebenso bei Tabellen
\providecommand*{\tabrefname}{Tabelle~}
\newcommand*{\tabref}[1]{%
  \hyperref[tab:#1]{\tabrefname{}}\ref{tab:#1}%
}
% und Abschnitten
\providecommand*{\secrefname}{Abschnitt }
\newcommand*{\secref}[1]{%
  \hyperref[sec:#1]{\secrefname{}}\ref{sec:#1}%
}
% und Kapiteln
\providecommand*{\chaprefname}{Kapitel~}
\newcommand*{\chapref}[1]{%
  \hyperref[chap:#1]{\chaprefname{}}\ref{chap:#1}%
}

%%% Doc: ftp://tug.ctan.org/pub/tex-archive/macros/latex/contrib/pdfpages/pdfpages.pdf
\usepackage{pdfpages} % Include pages from external PDF documents in LaTeX documents

%%% Doc: ftp://tug.ctan.org/pub/tex-archive/macros/latex/contrib/oberdiek/pdflscape.sty
%% PW: Wird gebraut, wenn LSfigure-Environment definiert wird (newcommands.tex)
%\usepackage{pdflscape} %  Querformat mit PDF
%
% Pakete Laden die nach Hyperref geladen werden sollen
\LoadPackagesNow % (ltxtable, tabularx)

% ~~~~~~~~~~~~~~~~~~~~~~~~~~~~~~~~~~~~~~~~~~~~~~~~~~~~~~~~~~~~~~~~~~~~~~~~
% Zusätzliche Pakete
% ~~~~~~~~~~~~~~~~~~~~~~~~~~~~~~~~~~~~~~~~~~~~~~~~~~~~~~~~~~~~~~~~~~~~~~~~
%%% Doc: ftp://tug.ctan.org/pub/tex-archive/macros/latex/contrib/hyphenat/hyphenat.pdf
% According to documentation the font warnings can be ignored
%\usepackage[htt]{hyphenat} % enable hyphenation of typewriter text word (\textt).

%% Komprimierung von Bildern in PDF ausschalten
% \ifpdf
%    \pdfcompresslevel=0
% \fi



% ~~~~~~~~~~~~~~~~~~~~~~~~~~~~~~~~~~~~~~~~~~~~~~~~~~~~~~~~~~~~~~~~~~~~~~~~
% end of preambel
% ~~~~~~~~~~~~~~~~~~~~~~~~~~~~~~~~~~~~~~~~~~~~~~~~~~~~~~~~~~~~~~~~~~~~~~~~
%\IfPackageLoaded{fancyvrb}{
%	\DefineShortVerb{\|} % Nur mit fancyvrb zusammen laden!
%}
\VerbatimFootnotes

%%% PW Muster quer drüber
%\ifthenelse{\boolean{printMuster}}%
%{\ifthenelse{\boolean{englishAsMainLanguage}}%
%{\makeatletter%
%\AddToShipoutPicture{%
            %\setlength{\@tempdimb}{.5\paperwidth}%
            %\setlength{\@tempdimc}{.5\paperheight}%
            %\setlength{\unitlength}{1pt}%
            %\put(\strip@pt\@tempdimb,\strip@pt\@tempdimc){%
        %\makebox(0,0){\rotatebox{45}{\textcolor[gray]{0.75}%
        %{\fontsize{6cm}{6cm}\selectfont{DRAFT}}}}%
            %}%
%}%
%\makeatother}%
%{\makeatletter%
%\AddToShipoutPicture{%
            %\setlength{\@tempdimb}{.5\paperwidth}%
            %\setlength{\@tempdimc}{.5\paperheight}%
            %\setlength{\unitlength}{1pt}%
            %\put(\strip@pt\@tempdimb,\strip@pt\@tempdimc){%
        %\makebox(0,0){\rotatebox{45}{\textcolor[gray]{0.75}%
        %{\fontsize{6cm}{6cm}\selectfont{MUSTER}}}}%
            %}%
%}%
%\makeatother}%
%}{}


%%%% PW
%%%% Scrhack macht float@addtolists-Warnung weg und verbessert Kompatibilität zu anderen Paketen
%%% Braucht anscheinend eines der begrenzt verfügbaren TeX-\writes :-(
%\usepackage[%
%%hyperref=false
%]{scrhack}

%%MG: von oben verschoben
%%% Doc: ftp://tug.ctan.org/pub/tex-archive/macros/latex/contrib/nomencl/nomencl.pdf
%% Braucht anscheinend eines der begrenzt verfügbaren TeX-\writes
%\usepackage[%
%	german,
%	english
%]{nomencl}%[2005/09/22]

%%MG: von oben verschoben
%\usepackage[
%%	footnote,	% Full names appear in the footnote
%%	smaller,		% Print acronym in smaller fontsize
%	printonlyused %
%]{acronym}


\IfPackageLoaded{minitoc}{\IfElseUnDefined{chapter}{\dosecttoc}{\dominitoc}}

% Auszufuehrende Befehle  ------------------------------------------------
%% Verschoben in Header.tex
%\IfDefined{makeindex}{\makeindex[options=-s preambel/IndexStyle.tex]}
%\IfDefined{makenomenclature}{\makenomenclature}
%\listfiles
%------------------------------------------------------------------------

\ifpdf
%% um Warnungen vom Typ "PDF inclusion: found PDF version 1.6, but at most version 1.5 allowed" zu unterdrücken
\pdfoptionpdfminorversion=6
%% Zur Vermeidung von Transparenzen Beschränkung auf PDF 1.3!
\pdfminorversion=3
%\pdfminorversion=5


\fi


%%%Alle Pakete, die ich selbst eingebunden habe
%% Define commands that don't eat spaces.
\usepackage{xspace}

%% IfThenElse: muss frueher eingebunden werden wegen Abfragen zur Sprache etc.
%\usepackage{ifthen}

%% for adding invisible comments
%% not needed anymore since included in package "versions"
%% Braucht ein \write. Vielleicht eins zuviel, Asymptote braucht auch welche, es gibt ing. nur 16.
%\usepackage{comment}
%\includecomment{showcomment}
%\excludecomment{hidecomment}

%for conditional text inclusions
\usepackage[nogroup]{versions}

%for compact lists
\usepackage{mdwlist}

% Für verzierte Überschriften
\usepackage{lettrine}

%% Einstellungen für die Aufzählungen
%% Kein Abstand zwischen den Item-Einträgen
%\setlist{noitemsep}
%%Verkleinerter Abstand zwischen den Item-Einträgen
\setlist{%
		%before=\RaggedRight ,%Flattersatz in itemize- und enumerate-Umgebungen erzwingen
		%funktioniert nicht so wie gewünscht, s.
		%https://tex.stackexchange.com/questions/104088/problem-with-enumitem-and-raggedright
		%Workaround weiter unten mit \AtBeginEnvironment{itemize}{\preto\item{\RaggedRight}}
		%Ränder setzen:
		%leftmargin=2em, %linkes Rand explizit setzen
		%rightmargin=1em, %rechtes Rand vergrößern
		rightmargin=2mm,
		itemsep=0.2\baselineskip,% 
		parsep=0.2\baselineskip%
		%topsep=0.2\baselineskip,
		%partopsep=0.2\baselineskip,%
}
%% Flattersatz in itemize- und enumerate-Umgebungen erzwingen
%% Forderung des KSP-Verlages: keine Worttrennung in Aufzählungen, daher \raggedright statt \RaggedRight
%\AtBeginEnvironment{itemize}{\preto\item{\RaggedRight}}
\AtBeginEnvironment{itemize}{\preto\item{\raggedright}}
\AtBeginEnvironment{enumerate}{\preto\item{\raggedright}}
%\AtBeginEnvironment{description}{\preto\item{\raggedright}}
\AtBeginEnvironment{itemize*}{\preto\item{\raggedright}}
\AtBeginEnvironment{enumerate*}{\preto\item{\raggedright}}
%\AtBeginEnvironment{description*}{\preto\item{\raggedright}}
%

%for allowing hyphenation of words that contain a dash
%using shortcuts \-/, \=/, \--, \==, \---, and \===
\usepackage[shortcuts]{extdash}

%Zur besseren Silbentrennung
%s. http://de.wikibooks.org/wiki/LaTeX-W%C3%B6rterbuch:_Silbentrennung
\usepackage[ngerman=ngerman-x-latest]{hyphsubst}

\usepackage[super]{nth}

\usepackage{tikz}
\usetikzlibrary{%
				automata,%
				arrows,%
				backgrounds,%
				calc,%
				chains,%
				decorations,%
				decorations.markings,%
				decorations.pathmorphing,%
				external,%
				fadings,%
				fit,%
				matrix,%
				positioning,%
				scopes,%
				shadows,%
				shapes.geometric,%
				shapes.misc,%
				shapes.multipart,%
				through,%
				topaths}
\usepackage{tikz-3dplot}

%Einschliessen von Matlab-Grafiken:
\usepackage{pgfplots}
%Externe Generierung von plots, s. https://tex.stackexchange.com/questions/7953/how-to-expand-texs-main-memory-size-pgfplots-memory-overload
%% Braucht -shell-escape Kompilieroption!!!

\pgfplotsset{
  compat=1.14,%
  plot coordinates/math parser=false,%
  tick label style={font=\footnotesize},%
  label style={font=\small},%
	scale only axis,%
	axis lines=center,%
	axis on top,%
  every axis legend/.append style={cells={anchor=west},draw=none,font=\small},%
  every axis plot/.append style={semithick},%
	KIT scatter plot A/.style={%
    draw=none,only marks,mark=*,mark options={draw=KITblue,fill=KITblue}},%
	KIT scatter plot B/.style={%
    draw=none,only marks,mark=square*,mark options={draw=KITred,fill=KITred}},%
	KIT scatter plot C/.style={%
    draw=none,only marks,mark=diamond*,mark options={draw=KITorange,fill=KITorange}},%
	KIT scatter plot explicit/.style={%
    scatter,%
    scatter/classes={%
      a={mark=*,KITblue},%
      b={mark=square*,KITred},%
      c={mark=diamond*,KITorange}},%
    only marks,%
    scatter src=explicit symbolic,%
    z buffer=sort},%
  KIT ybar plot A/.style={%
    ybar,fill=KITblue,draw=none},%
  KIT ybar plot B/.style={%
    ybar,fill=KITred,draw=none},%
  KIT ybar plot C/.style={%
    ybar,fill=KITorange,draw=none},%
  KIT xbar plot A/.style={%
    ybar,fill=KITblue,draw=none},%
  KIT xbar plot B/.style={%
    ybar,fill=KITred,draw=none},%
  KIT xbar plot C/.style={%
    ybar,fill=KITorange,draw=none},%
  KIT line plot A/.style={%
    KITblue,semithick},%
  KIT line plot B/.style={%
    KITred,semithick},%
  KIT line plot C/.style={%
    KITorange,semithick},%
  KIT line plot D/.style={%
    KITlilac,semithick},%
  KIT line plot E/.style={%
    KITbrown,semithick},%
  KIT line plot F/.style={%
    KITblue,semithick,dashed},%
  KIT line plot G/.style={%
    KITred,semithick,dashed},%
  KIT line plot H/.style={%
    KITorange,semithick,dashed},%
  KIT line plot I/.style={%
    KITlilac,semithick,dashed},%
  KIT line plot J/.style={%
    KITbrown,semithick,dashed},%
	KIT smooth plot A/.style={%
    KITblue,semithick,smooth},%
  KIT smooth plot B/.style={%
    KITred,semithick,smooth},%
  KIT smooth plot C/.style={%
    KITorange,semithick,smooth},%
  KIT smooth plot D/.style={%
    KITlilac,semithick,smooth},%
  KIT smooth plot E/.style={%
    KITbrown,semithick,smooth},%
  KIT smooth plot F/.style={%
    KITblue,semithick,dashed,smooth},%
  KIT smooth plot G/.style={%
    KITred,semithick,dashed,smooth},%
  KIT smooth plot H/.style={%
    KITorange,semithick,dashed,smooth},%
  KIT smooth plot I/.style={%
    KITlilac,semithick,dashed,smooth},%
  KIT smooth plot J/.style={%
    KITbrown,semithick,dashed,smooth},%
}

%\usepgfplotslibrary{external} 
%\tikzexternalize
%\usepgfplotslibrary{ternary}
%\tikzsetexternalprefix{./figures-compiled/}
%\tikzset{external/aux in dpth={false},external/disable dependency files}
%\tikzexternalize[shell escape=-enable-write18]

%\usepackage{listings} \lstset{numbers=left, numberstyle=\tiny, numbersep=5pt} %\lstset{language=Perl}

%Achtung! Verwendung von Glossaries bzw. des Befehls \makeglossaries benötigt Perl!!!!
%Unter Linux ist Perl vorhanden, unter Windows muss es installiert werden, s.
%http://www.mrunix.de/forums/showthread.php?68892-Tip-Glossaries-Paket-was-oft-falsch-l%E4uft
%bzw.
%http://latex-community.org/know-how/latex/55-latex-general/263-glossaries-nomenclature-lists-of-symbols-and-acronyms#texniccenter
%\usepackage[ngerman]{translator}
\usepackage[%
xindy,
%translate=true, % verwendet das translator-paket
translate=babel, % verwende Babel zur Übersetzung der Überschriften
%translate=false, % keine Übersetzung der Überschriften da ansonsten Zusatzpakete geladen werden müssen
nonumberlist,    % keine Seitenzahlen anzeigen
nopostdot,       % den Punkt am Ende jeder Beschreibung deaktivieren
acronym,         % ein Abkürzungsverzeichnis erstellen
shortcuts,       % Shortcuts aus dem Acronym-Package verwenden, z.B. as \ac, \acs, \acl, \acf, etc.
symbols,         % ein Symbolverzeichnis erstellen
toc,             % Einträge im Inhaltsverzeichnis erzeugen
section=chapter, % Einträge im Inhaltsverzeichnis auf chapter-Ebene erscheinen lassen
%numberedsection=autolabel,%Kapitel nicht nummerieren  => auskommentiert lassen!
%nohypertypes={glossary}, %nur den ersten Eintrag als Hyperlink anzeigen
nohypertypes={acronym,notation}, % keine Hyperlinks für Acronyme und Notation
%style=list      % Es wird später ein eigener Stil definiert mylongglosstyle!
%style=long
%style=super
]%
{glossaries}

%% Zusatzpakete für Glossar-Stile "longragged" und "longbooktabs"
%% Hack: stattdessen eigenes Stil "mylongglossstyle" definieren (s. Datei GlossaryOptions.tex)
%\usepackage{glossary-longragged} %zusätzlich laden um Glossareinträge linksbündig zu setzen (Forderung des KSP-Verlages)
%%\usepackage{glossary-longbooktabs}



%% Mathe-Pakete laden, Sondersymbole und Mathe-Macros definieren
% *** Mathematik **************************************
%
% Eine Definition eigener mathematischer Befehle ist besonders sinnvoll, wenn diese im Dokument oft verwendet werden.
% Man kann dann hier an zentraler Stelle z.B. alle Vektoren mit Pfeil statt fett formatieren.
%
%




% amsmath schon vorher geladen da es vor pst-pdf geladen werden muss

\ifxetex
	% hier nichts tun
\else
%%% Doc: ftp://tug.ctan.org/pub/tex-archive/macros/latex/contrib/mh/doc/mathtools.pdf
% Erweitert amsmath und behebt einige Bugs.
% Muss vor ntheorem geladen werden!
%\usepackage{mathtools}
%\usepackage[disallowspaces]{mathtools}
%\usepackage[fixamsmath,disallowspaces]{mathtools}


%%% Doc: http://www.ctan.org/info?id=fixmath
% LaTeX's default style of typesetting mathematics does not comply
% with the International Standards ISO31-0:1992 to ISO31-13:1992
% which indicate that uppercase Greek letters always be typset
% upright, as opposed to italic (even though they usually
% represent variables) and allow for typsetting of variables in a
% boldface italic style (even though the required fonts are
% available). This package ensures that uppercase Greek be typeset
% in italic style, that upright $\Delta$ and $\Omega$ symbols are
% available through the commands \upDelta and \upOmega; and
% provides a new math alphabet \mathbold for boldface
% italic letters, including Greek.


%\usepackage{fixmath}
%%%%%PW: fixmath ist schuld, dass mathbold mit eulervm nicht mehr funktioniert


	%bei Verwendung von xelatex folgendes nicht einbinden:
	
	%%% Doc: ftp://tug.ctan.org/pub/tex-archive/macros/latex/contrib/onlyamsmath/onlyamsmath.dvi
	% Warnt bei Benutzung von Befehlen die mit amsmath inkompatibel sind.
	% verursacht Probleme mit XeLaTeX
	\usepackage[%
		all,%
		warning%
	]{onlyamsmath}


%
%%PW:
% Ohne (!) bm-Paket gehen Formeln in Überschriften nicht mehr --> too many math alphabets used in version normal
\newcommand\hmmax{0}
\newcommand\bmmax{1}  %%2 ist schon zuviel
\usepackage{bm}
% bm an sich gibt aber auch schnell "too many math alphabets" Fehler. Insg. kann TeX 16 Mathe-Alphabete
%% Kann man wegbekommen, indem man die heavy families reduziert (hmmax) und die bold families (bmmax)
%% http://www.tex.ac.uk/cgi-bin/texfaq2html?label=manymathalph
%\renewcommand{\hmmax}{0}
%\renewcommand{\bmmax}{3}
%Boldmath mit \bm Befehl. Empfohlen im LaTeX-Begleiter. Laden nach allen Pakten, die Mathefont-Einstellungen machen.
\fi

%Alternative for XeLaTeX-Befehlen zur Textauszeichnung \symbf und \mathbfcal
\makeatletter
\@ifpackageloaded{unicode-math}{%
%nichts tun
}{%ansonsten fehlende Makros definieren
%
%\usepackage{amssymb}   %for using \mathbb{}, \nexists etc.
%
%Folgende Makros, die in unicode-math definiert sind, brauchen das bm-Paket
\newcommand{\symbf}[1]{\bm{#1}}
\newcommand{\symbfit}[1]{\bm{#1}}
\newcommand{\symbfcal}[1]{\bm{\mathcal{#1}}}
\newcommand{\mathbfcal}[1]{\bm{\mathcal{#1}}}
}
\makeatother




%%PW: Für Beispiele, Sätze, Lemmata: ntheorem
\usepackage{amsthm}
%%PW: amsthm gibt es auch noch, ist aber älter
% Muss nach mathtools geladen werden!
%\usepackage{ntheorem}
%\usepackage[standard]{ntheorem}  %funktioniert nicht, da mit der Option "Standard" asmsymb eingebunden wird, das Probleme ergibt
%\usepackage[thmmarks,standard,hyperref]{ntheorem} 
%\usepackage[amsmath,thmmarks,standard,hyperref]{ntheorem}   %hyperref-Option macht Probleme  %im Minimaltest aber nicht
%\usepackage{thmtools}   %Fuer \listoftheorems  %Nicht in MikTeX 2.8 %Wird nicht benutzt
%\usepackage{theoremref}  %Für bessere Referenzierung
\newtheorem{theorem}{\protect\theoremname}[chapter]
\newtheorem{definition}[theorem]{\protect\definitionname}
\newtheorem{proposition}[theorem]{\protect\propositionname}
\newtheorem{lemma}[theorem]{\protect\lemmaname}
\newtheorem{example}{\protect\examplename}
%\newtheorem{proof}{\protect\proofname}
\newtheorem{cor}{\protect\corollaryname}

%% Changing the qed symbol from white square to something else:
%\renewcommand\qedsymbol{QED}% just "QED"
%\renewcommand\qedsymbol{$\blacksquare$}% black square, needs amssymb package!


%\fi


%% PW: Asymptote Unterstützung. Asymptote braucht auch ein TeX-\write von denen es nur 16 gibt. Bin schon am Limit, deswegen Asymptote XOR comment (braucht auch eins).
%\usepackage[inline]{asymptote}  %%gibts erst nach Installation von Hand
%\usepackage[% %%gehört zu MikTeX, macht aber nicht das, was man sich erhofft.
%process=all,%
%%process=none,
%]{asyfig}

%------------------------------------------------------

% -- Vektor fett darstellen -----------------
% \let\oldvec\vec
% \def\vec#1{{\boldsymbol{#1}}} %Fetter Vektor
% \newcommand{\ve}{\vec} %
% -------------------------------------------


%%% Doc: ftp://tug.ctan.org/pub/tex-archive/macros/latex/contrib/misc/braket.sty
%\usepackage{braket}  % Quantenmechanik Bracket Schreibweise

%%% Doc: ftp://tug.ctan.org/pub/tex-archive/macros/latex/contrib/misc/cancel.sty
\usepackage{cancel}  % Durchstreichen

%%% Doc: ftp://tug.ctan.org/pub/tex-archive/macros/latex/contrib/mh/doc/empheq.pdf
\usepackage{empheq}  % Hervorheben

%%% Doc: ftp://tug.ctan.org/pub/tex-archive/info/math/voss/mathmode/Mathmode.pdf
%\usepackage{exscale} % Skaliert Mathe-Modus Ausgaben in allen Umgebungen richtig.

%%% Doc: ftp://tug.ctan.org/pub/tex-archive/macros/latex/contrib/was/icomma.dtx
% Erlaubt die Benutzung von Kommas im Mathematikmodus
\usepackage{icomma}


%%PW stackrel
\usepackage{stackrel} %provides enhanced \stackrel and \stackbin
%\stackrel [subscript] {superscript} {rel}
%\stackbin [subscript] {superscript} {bin}
%Example:
%A \stackbin[\text{and}]{}{+} B \stackrel[x]{!}{=} C

%%% Doc: http://www.ctex.org/documents/packages/special/units.pdf
%\usepackage[nice]{nicefrac}
\usepackage{xfrac}

%%PW commath
%commath -- A LaTeX class which provides some commands which help you to format formulas flexibly.
%Differentiale aufrecht setzen mit vorgefertigten Operatoren, Intervalle, Senkrechter Evaluationsstrich, Norm
%%% PW: Gibt aber leider Probleme mit figureref-Befehl und wenn das draußen ist mit "`too many math alphabets"'
%\usepackage{commath}


%%% Tauschen von Epsilon und andere:
% \let\ORGvarrho=\varrho
% \let\varrho=\rho
% \let\rho=\ORGvarrho
%
\let\ORGvarepsilon=\varepsilon
\let\varepsilon=\epsilon
\let\epsilon=\ORGvarepsilon
%
% \let\ORGvartheta=\vartheta
% \let\vartheta=\theta
% \let\theta=\ORGvartheta
%
% \let\ORGvarphi=\varphi
% \let\varphi=\phi
% \let\phi=\ORGvarphi

% Vier Indizes an beliebigen Zeichen (2 links, 2 rechts)
\usepackage{fouridx}

% Matrizen mit Randbeschriftung
\usepackage{blkarray}


% ~~~~~~~~~~~~~~~~~~~~~~~~~~~~~~~~~~~~~~~~~~~~~~~~~~~~~~~~~~~~~~~~~~~~~~~~
% Symbole
% ~~~~~~~~~~~~~~~~~~~~~~~~~~~~~~~~~~~~~~~~~~~~~~~~~~~~~~~~~~~~~~~~~~~~~~~~
%
%%% General Doc: http://www.ctan.org/tex-archive/info/symbols/comprehensive/symbols-a4.pdf
%
%% Symbole für Mathematiksatz
%\usepackage{mathrsfs} %% Schreibschriftbuchstaben für den Mathematiksatz (nur Großbuchstaben)   %%%%% PW: Auch einfach mal reingemacht
%\usepackage{dsfont}   %% Double Stroke Fonts   %%%% PW: Für Mengensymbole R Q Z N
%\usepackage[mathcal]{euscript} %% adds euler mathcal font    %%%% PW: Mathcal ist zwar bei Eulervm schon irgendwie drin. Ist zuviel. Gibt dann "`too many math alphabets"' Fehler.
%\usepackage{amssymb}      %%%% knallt mit eufrak, was das gleiche macht
%\usepackage[Symbolsmallscale]{upgreek} % upright symbols from euler package [Euler] or Adobe Symbols [Symbol]
%\usepackage[upmu]{gensymb}             % Option upmu  %% bekomme Fehler: command \upmu is undefined

%% Allgemeine Symbole
%\usepackage{wasysym}  %% Doc: http://www.ctan.org/tex-archive/macros/latex/contrib/wasysym/wasysym.pdf
%\usepackage{marvosym} %% Symbole aus der marvosym Schrift
%\usepackage{pifont}   %% ZapfDingbats


\newcommand{\mb}[1]{\ensuremath{\symbfit{#1}}}



\newcommand{\Ast}{\ensuremath{\mathord{\ast}}}
\newcommand{\Sim}{\ensuremath{\mathord{\sim}}}
\newcommand{\Cdot}{\ensuremath{\mathord{\,\cdot\,}}}
\newcommand{\Tr}{\ensuremath{\mathsf{T}}}
\newcommand{\const}{\ensuremath{\mathord{\mathrm{const}}}}

\newcommand{\SkalProd}[2]{\langle #1,#2 \rangle}
\newcommand{\SkalProdD}[2]{\left\langle #1,#2 \right\rangle}

\DeclareMathOperator{\supp}{supp}
\DeclareMathOperator{\rect}{rect}
\DeclareMathOperator{\ld}{ld}
\DeclareMathOperator{\SO}{SO}
\DeclareMathOperator{\Var}{Var}
\DeclareMathOperator{\Cov}{Cov}
\DeclareMathOperator{\vol}{vol}
\DeclareMathOperator{\tr}{tr}
\DeclareMathOperator*{\argmin}{arg\,min}
\DeclareMathOperator*{\argmax}{arg\,max}
\DeclareMathOperator{\grad}{grad}
\DeclareMathOperator{\Arg}{Arg}
\DeclareMathOperator{\col}{col}
\DeclareMathOperator{\spn}{span}
\DeclareMathOperator{\aff}{aff}
% http://tex.stackexchange.com/questions/84302/what-is-the-difference-of-mathop-operatorname-and-declaremathoperator
\newcommand{\diff}{\mathop{}\!\mathrm{d}}
\newcommand{\ceq}{\mathrel{\mathop:}=}


% ------------------------------------------------------------------------------
% Expectation value:
% Erwartungswert:
% ------------------------------------------------------------------------------
%\DeclareMathOperator{\E}{E}
\DeclareMathOperator{\E}{\mathbb{E}}
%\newcommand{\E}{\mathbb{E}}
\newcommand{\Erw}[1]{ \E[#1] }
\newcommand{\ERW}[1]{ \E\bigl[#1\bigr] } %Mit großen Klammern
% ------------------------------------------------------------------------------

% ------------------------------------------------------------------------------
% Random variables, command is small "rv" followed by letter
% ------------------------------------------------------------------------------
\newcommand{\randomvar}[1]{#1}

\newcommand{\rvA}{\randomvar{A}}
\newcommand{\rvB}{\randomvar{B}}
\newcommand{\rvC}{\randomvar{C}}
\newcommand{\rvD}{\randomvar{D}}
\newcommand{\rvE}{\randomvar{E}}
\newcommand{\rvF}{\randomvar{F}}
\newcommand{\rvG}{\randomvar{G}}
\newcommand{\rvH}{\randomvar{H}}
\newcommand{\rvI}{\randomvar{I}}
\newcommand{\rvJ}{\randomvar{J}}
\newcommand{\rvK}{\randomvar{K}}
\newcommand{\rvL}{\randomvar{L}}
\newcommand{\rvM}{\randomvar{M}}
\newcommand{\rvN}{\randomvar{N}}
\newcommand{\rvO}{\randomvar{O}}
\newcommand{\rvP}{\randomvar{P}}
\newcommand{\rvQ}{\randomvar{Q}}
\newcommand{\rvR}{\randomvar{R}}
\newcommand{\rvS}{\randomvar{S}}
\newcommand{\rvT}{\randomvar{T}}
\newcommand{\rvU}{\randomvar{U}}
\newcommand{\rvV}{\randomvar{V}}
\newcommand{\rvW}{\randomvar{W}}
\newcommand{\rvX}{\randomvar{X}}
\newcommand{\rvY}{\randomvar{Y}}
\newcommand{\rvZ}{\randomvar{Z}}
\newcommand{\rvGamma}{\randomvar{\Gamma}}
% ------------------------------------------------------------------------------


% ------------------------------------------------------------------------------
% Multidimensionale Zufallsvariablen:
% Multidimensional random variables, command is small "mrv" followed by letter
% ------------------------------------------------------------------------------
\newcommand{\mdrv}[1]{\symbfit{#1}}

\newcommand{\mdrvA}{\mdrv{A}}
\newcommand{\mdrvB}{\mdrv{B}}
\newcommand{\mdrvC}{\mdrv{C}}
\newcommand{\mdrvD}{\mdrv{D}}
\newcommand{\mdrvE}{\mdrv{E}}
\newcommand{\mdrvF}{\mdrv{F}}
\newcommand{\mdrvG}{\mdrv{G}}
\newcommand{\mdrvH}{\mdrv{H}}
\newcommand{\mdrvI}{\mdrv{I}}
\newcommand{\mdrvJ}{\mdrv{J}}
\newcommand{\mdrvK}{\mdrv{K}}
\newcommand{\mdrvL}{\mdrv{L}}
\newcommand{\mdrvM}{\mdrv{M}}
\newcommand{\mdrvN}{\mdrv{N}}
\newcommand{\mdrvO}{\mdrv{O}}
\newcommand{\mdrvP}{\mdrv{P}}
\newcommand{\mdrvQ}{\mdrv{Q}}
\newcommand{\mdrvR}{\mdrv{R}}
\newcommand{\mdrvS}{\mdrv{S}}
\newcommand{\mdrvT}{\mdrv{T}}
\newcommand{\mdrvU}{\mdrv{U}}
\newcommand{\mdrvV}{\mdrv{V}}
\newcommand{\mdrvW}{\mdrv{W}}
\newcommand{\mdrvX}{\mdrv{X}}
\newcommand{\mdrvY}{\mdrv{Y}}
\newcommand{\mdrvZ}{\mdrv{Z}}
\newcommand{\mdrvGamma}{\mdrv{\Gamma}}
% ------------------------------------------------------------------------------


% ------------------------------------------------------------------------------
% Estimation, command is mall "es" followed by letter
% ------------------------------------------------------------------------------
\newcommand{\esti}[1]{\hat{#1}}

\newcommand{\esa}{\esti{a}}
\newcommand{\esb}{\esti{b}}
\newcommand{\esc}{\esti{c}}
\newcommand{\esd}{\esti{d}}
\newcommand{\ese}{\esti{e}}
\newcommand{\esf}{\esti{f}}
\newcommand{\esg}{\esti{g}}
\newcommand{\esh}{\esti{h}}
\newcommand{\esi}{\esti{i}}
\newcommand{\esj}{\esti{j}}
\newcommand{\esk}{\esti{k}}
\newcommand{\esl}{\esti{l}}
\newcommand{\esm}{\esti{m}}
\newcommand{\esn}{\esti{n}}
\newcommand{\eso}{\esti{o}}
\newcommand{\esp}{\esti{p}}
\newcommand{\esq}{\esti{q}}
\newcommand{\esr}{\esti{r}}
\newcommand{\ess}{\esti{s}}
\newcommand{\est}{\esti{t}}
\newcommand{\esu}{\esti{u}}
\newcommand{\esv}{\esti{v}}
\newcommand{\esw}{\esti{w}}
\newcommand{\esx}{\esti{x}}
\newcommand{\esy}{\esti{y}}
\newcommand{\esz}{\esti{z}}
\newcommand{\esgamma}{\esti{\gamma}}
\newcommand{\essigma}{\esti{\sigma}}
\newcommand{\esomega}{\esti{\omega}}
\newcommand{\eskappa}{\esti{\kappa}}
\newcommand{\esmu}{\esti{\mu}}
\newcommand{\esSigma}{\esti{\Sigma}}
% ------------------------------------------------------------------------------



%-------------------------------------------------------------------------------
% Matrices and vectors:
% Matrizen und Vektoren:
% 
% Vectors, command is "v" followed by lowercase letter
% Matrices, command is "m" followed by capital letter
%-------------------------------------------------------------------------------
\newcommand{\vecfont}[1]{\symbf{#1}}
\newcommand{\matfont}[1]{\symbf{#1}}

\newcommand{\Vektor}[1]{\vecfont{#1}}
\newcommand{\Matrix}[1]{\matfont{#1}}
\newcommand{\Vek}[1]{\vecfont{#1}}
\newcommand{\Mat}[1]{\matfont{#1}}

\newcommand{\va}{\vecfont{a}}
\newcommand{\vb}{\vecfont{b}}
\newcommand{\vc}{\vecfont{c}}
\newcommand{\vd}{\vecfont{d}}
\newcommand{\ve}{\vecfont{e}}
\newcommand{\vf}{\vecfont{f}}
\newcommand{\vg}{\vecfont{g}}
\newcommand{\vh}{\vecfont{h}}
\newcommand{\vi}{\vecfont{i}}
\newcommand{\vj}{\vecfont{j}}
\newcommand{\vk}{\vecfont{k}}
\newcommand{\vl}{\vecfont{l}}
\newcommand{\vm}{\vecfont{m}}
\newcommand{\vn}{\vecfont{n}}
\newcommand{\vo}{\vecfont{o}}
\newcommand{\vp}{\vecfont{p}}
\newcommand{\vq}{\vecfont{q}}
\newcommand{\vr}{\vecfont{r}}
\newcommand{\vs}{\vecfont{s}}
\newcommand{\vt}{\vecfont{t}}
\newcommand{\vu}{\vecfont{u}}
\newcommand{\vv}{\vecfont{v}}
\newcommand{\vw}{\vecfont{w}}
\newcommand{\vx}{\vecfont{x}}
\newcommand{\vy}{\vecfont{y}}
\newcommand{\vz}{\vecfont{z}}

\newcommand{\valpha}{\vecfont{\alpha}}
\newcommand{\vepsilon}{\vecfont{\varepsilon}}
\newcommand{\vgamma}{\vecfont{\gamma}}
\newcommand{\veta}{\vecfont{\eta}}
\newcommand{\vmu}{\vecfont{\mu}}

\newcommand{\mA}{\matfont{A}}
\newcommand{\mB}{\matfont{B}}
\newcommand{\mC}{\matfont{C}}
\newcommand{\mD}{\matfont{D}}
\newcommand{\mE}{\matfont{E}}
\newcommand{\mF}{\matfont{F}}
\newcommand{\mG}{\matfont{G}}
\newcommand{\mH}{\matfont{H}}
\newcommand{\mI}{\matfont{I}}
\newcommand{\mJ}{\matfont{J}}
\newcommand{\mK}{\matfont{K}}
\newcommand{\mL}{\matfont{L}}
\newcommand{\mM}{\matfont{M}}
\newcommand{\mN}{\matfont{N}}
\newcommand{\mO}{\matfont{O}}
\newcommand{\mP}{\matfont{P}}
\newcommand{\mQ}{\matfont{Q}}
\newcommand{\mR}{\matfont{R}}
\newcommand{\mS}{\matfont{S}}
\newcommand{\mT}{\matfont{T}}
\newcommand{\mU}{\matfont{U}}
\newcommand{\mV}{\matfont{V}}
\newcommand{\mW}{\matfont{W}}
\newcommand{\mX}{\matfont{X}}
\newcommand{\mY}{\matfont{Y}}
\newcommand{\mZ}{\matfont{Z}}




%-------------------------------------------------------------------------------
% Spaces, command is "sp" followed by capital letter
%-------------------------------------------------------------------------------
\newcommand{\spacefont}[1]{\symbfcal{#1}}

\newcommand{\spA}{\spacefont{A}}
\newcommand{\spB}{\spacefont{B}}
\newcommand{\spC}{\spacefont{C}}
\newcommand{\spD}{\spacefont{D}}
\newcommand{\spE}{\spacefont{E}}
\newcommand{\spF}{\spacefont{F}}
\newcommand{\spG}{\spacefont{G}}
\newcommand{\spH}{\spacefont{H}}
\newcommand{\spI}{\spacefont{I}}
\newcommand{\spJ}{\spacefont{J}}
\newcommand{\spK}{\spacefont{K}}
\newcommand{\spL}{\spacefont{L}}
\newcommand{\spM}{\spacefont{M}}
\newcommand{\spN}{\spacefont{N}}
\newcommand{\spO}{\spacefont{O}}
\newcommand{\spP}{\spacefont{P}}
\newcommand{\spQ}{\spacefont{Q}}
\newcommand{\spR}{\spacefont{R}}
\newcommand{\spS}{\spacefont{S}}
\newcommand{\spT}{\spacefont{T}}
\newcommand{\spU}{\spacefont{U}}
\newcommand{\spV}{\spacefont{V}}
\newcommand{\spW}{\spacefont{W}}
\newcommand{\spX}{\spacefont{X}}
\newcommand{\spY}{\spacefont{Y}}
\newcommand{\spZ}{\spacefont{Z}}
%-------------------------------------------------------------------------------
% State space and measurement space:
% Zustandsraum und Messraum:
%-------------------------------------------------------------------------------
\newcommand{\statespace}{\spX}
\newcommand{\measurementspace}{\spZ}
%-------------------------------------------------------------------------------




%-------------------------------------------------------------------------------
% Sets, command is "s" followed by capital letter
%-------------------------------------------------------------------------------
\newcommand{\setfont}[1]{\mathcal{#1}}

\newcommand{\sA}{\setfont{A}}
\newcommand{\sB}{\setfont{B}}
\newcommand{\sC}{\setfont{C}}
\newcommand{\sD}{\setfont{D}}
\newcommand{\sE}{\setfont{E}}
\newcommand{\sF}{\setfont{F}}
\newcommand{\sG}{\setfont{G}}
\newcommand{\sH}{\setfont{H}}
\newcommand{\sI}{\setfont{I}}
\newcommand{\sJ}{\setfont{J}}
\newcommand{\sK}{\setfont{K}}
\newcommand{\sL}{\setfont{L}}
\newcommand{\sM}{\setfont{M}}
\newcommand{\sN}{\setfont{N}}
\newcommand{\sO}{\setfont{O}}
\newcommand{\sP}{\setfont{P}}
\newcommand{\sQ}{\setfont{Q}}
\newcommand{\sR}{\setfont{R}}
\newcommand{\sS}{\setfont{S}}
\newcommand{\sT}{\setfont{T}}
\newcommand{\sU}{\setfont{U}}
\newcommand{\sV}{\setfont{V}}
\newcommand{\sW}{\setfont{W}}
\newcommand{\sX}{\setfont{X}}
\newcommand{\sY}{\setfont{Y}}
\newcommand{\sZ}{\setfont{Z}}

%-------------------------------------
% Special sets:
%-------------------------------------
% Short forms for well-known sets ("ds" = double stroke, \mathbb is correct
% because it is redefined by libertine to print nice double stroke letters)
\newcommand{\dsA}{\mathbb{A}}
\newcommand{\dsB}{\mathbb{B}}
\newcommand{\dsC}{\mathbb{C}}
\newcommand{\dsD}{\mathbb{D}}
\newcommand{\dsE}{\mathbb{E}}
\newcommand{\dsF}{\mathbb{F}}
\newcommand{\dsG}{\mathbb{G}}
\newcommand{\dsH}{\mathbb{H}}
\newcommand{\dsI}{\mathbb{I}}
\newcommand{\dsJ}{\mathbb{J}}
\newcommand{\dsK}{\mathbb{K}}
\newcommand{\dsL}{\mathbb{L}}
\newcommand{\dsM}{\mathbb{M}}
\newcommand{\dsN}{\mathbb{N}}
\newcommand{\dsO}{\mathbb{O}}
\newcommand{\dsP}{\mathbb{P}}
\newcommand{\dsQ}{\mathbb{Q}}
\newcommand{\dsR}{\mathbb{R}}
\newcommand{\dsS}{\mathbb{S}}
\newcommand{\dsT}{\mathbb{T}}
\newcommand{\dsU}{\mathbb{U}}
\newcommand{\dsV}{\mathbb{V}}
\newcommand{\dsW}{\mathbb{W}}
\newcommand{\dsX}{\mathbb{X}}
\newcommand{\dsY}{\mathbb{Y}}
\newcommand{\dsZ}{\mathbb{Z}}
%
% Set of natural numbers, set of real numbers, set of rational numbers:
% Menge natürlicher Zahlen, Menge reeller Zahlen, menge rationaler Zahlen:
%-------------------------------------
\newcommand{\NatNum}{\mathbb{N}}
\newcommand{\RealNum}{\mathbb{R}}
\newcommand{\RatNum}{\mathbb{Q}}
%-------------------------------------------------------------------------------




%-------------------------------------------------------------------------------
% Distributions
%-------------------------------------------------------------------------------
% Gauss-Verteilung + Gauss-Verteilung an einer Stelle:
\newcommand{\GaussDist}{\mathcal{N}}
\newcommand{\NormDist}{\mathcal{N}}
\newcommand{\GaussDistValue}[3]{\mathcal{N}({#1}; {#2}, {#3})}
\newcommand{\NormDistValue}[3]{\mathcal{N}({#1}; {#2}, {#3})}
% Chi^2-Verteilung:
\newcommand{\ChiSqDist}{\chi^2}
%Dirac-Verteilung
\newcommand{\DiracDist}{\delta}
%-------------------------------------------------------------------------------



% ~~~~~~~~~~~~~~~~~~~~~~~~~~~~~~~~~~~~~~~~~~~~~~~~~~~~~~~~~~~~~~~~~~~~~~~~
% Special symbols:
% Sondersymbole:
% ~~~~~~~~~~~~~~~~~~~~~~~~~~~~~~~~~~~~~~~~~~~~~~~~~~~~~~~~~~~~~~~~~~~~~~~~
%Rotation matrix
\newcommand{\RotMat}{\mR}
%Translation vector
\newcommand{\transVec}{\vt}
%
\newcommand{\uVec}{\vu} % Control vector (vector with control parameters)
\newcommand{\wVec}{\vw} % 
\newcommand{\vVec}{\vv} % 
%-------------------------------------------------------------------------------
\newcommand{\Imat}{\mI} % Einheitsmatrix
\newcommand{\ZeroMat}{\Mat{0}} % zero matrix
\newcommand{\Fmat}{\mF} % system matrix of the Kalman Filter
\newcommand{\Gmat}{\mG} % control matrix of the Kalman Filter
\newcommand{\Hmat}{\mH} % measurement matrix of the Kalman Filter
\newcommand{\Qmat}{\mQ} % system noise covariance matrix of the Kalman Filter
\newcommand{\Rmat}{\mR} % measurement noise covariance matrix of the Kalman Filter
\newcommand{\Wmat}{\mW} % Jacobian matrix in the Extended Kalman Filter
\newcommand{\Vmat}{\mV} % Jacobian matrix in the Extended Kalman Filter
\newcommand{\KG}{\mK} % Kalman Gain
%-------------------------------------------------------------------------------
\newcommand{\uk}{\vu_k} % Control vector (vector with control parameters) at time $k$
\newcommand{\Qk}{\mQ_k} % System noise covariance matrix at time $k$
\newcommand{\Rk}{\mR_k} % Measurement noise covariance matrix at time $k$
\newcommand{\KGk}{\mK_k} % Kalman gain at time $k$ at time $k$
%
%
%
% point in image coordinates
\newcommand{\pkt}{\vp}
\newcommand{\pImage}{\pkt}
\newcommand{\pImageUntransf}{\vecfont{p'}}
% points in camera coordinates
\newcommand{\PCam}{\pkt_{\textnormal{C}}}
\newcommand{\pLeft}{\pkt_{\textnormal{L}}}
\newcommand{\pRight}{\pkt_{\textnormal{R}}}
% point in world coordinates
\newcommand{\PWorld}{\pkt_{\textnormal{W}}}
% homogenious coordinates
\newcommand{\pHomogen}{\check{\pkt}}





%% Hauptsprache des Dokuments setzen
\ifthenelse{\boolean{iesenglishs}}{%
	\selectlanguage{english}
}{%
	\selectlanguage{ngerman}
}

%%PW: Glyphtounicode für pdfx, PDF/A-Kompatibilität. In Ruhe lassen!
%% this file was converted from the following files:
%   - glyphlist.txt       (Adobe Glyph List v2.0)
%   - zapfdingbats.txt    (ITC Zapf Dingbats Glyph List)
%   - texglyphlist.txt    (lcdf-typetools texglyphlist.txt, v2.33)
%   - additional.tex      (additional entries)
%
% Notes:
% - entries containing duplicates in glyphlist.txt like
%   'dalethatafpatah;05D3 05B2' are ignored (commented out)
%
% - entries containing duplicates in texglyphlist.txt like
%   'angbracketleft;27E8,2329' are changed so that only the first
%   choice remains, ie 'angbracketleft;27E8'
%
% - a few entries in texglyphlist.txt like Delta, Omega, etc. are
%   commented out (they are already in glyphlist.txt)

% entries from glyphlist.txt:
\pdfglyphtounicode{A}{0041}
\pdfglyphtounicode{AE}{00C6}
\pdfglyphtounicode{AEacute}{01FC}
\pdfglyphtounicode{AEmacron}{01E2}
\pdfglyphtounicode{AEsmall}{F7E6}
\pdfglyphtounicode{Aacute}{00C1}
\pdfglyphtounicode{Aacutesmall}{F7E1}
\pdfglyphtounicode{Abreve}{0102}
\pdfglyphtounicode{Abreveacute}{1EAE}
\pdfglyphtounicode{Abrevecyrillic}{04D0}
\pdfglyphtounicode{Abrevedotbelow}{1EB6}
\pdfglyphtounicode{Abrevegrave}{1EB0}
\pdfglyphtounicode{Abrevehookabove}{1EB2}
\pdfglyphtounicode{Abrevetilde}{1EB4}
\pdfglyphtounicode{Acaron}{01CD}
\pdfglyphtounicode{Acircle}{24B6}
\pdfglyphtounicode{Acircumflex}{00C2}
\pdfglyphtounicode{Acircumflexacute}{1EA4}
\pdfglyphtounicode{Acircumflexdotbelow}{1EAC}
\pdfglyphtounicode{Acircumflexgrave}{1EA6}
\pdfglyphtounicode{Acircumflexhookabove}{1EA8}
\pdfglyphtounicode{Acircumflexsmall}{F7E2}
\pdfglyphtounicode{Acircumflextilde}{1EAA}
\pdfglyphtounicode{Acute}{F6C9}
\pdfglyphtounicode{Acutesmall}{F7B4}
\pdfglyphtounicode{Acyrillic}{0410}
\pdfglyphtounicode{Adblgrave}{0200}
\pdfglyphtounicode{Adieresis}{00C4}
\pdfglyphtounicode{Adieresiscyrillic}{04D2}
\pdfglyphtounicode{Adieresismacron}{01DE}
\pdfglyphtounicode{Adieresissmall}{F7E4}
\pdfglyphtounicode{Adotbelow}{1EA0}
\pdfglyphtounicode{Adotmacron}{01E0}
\pdfglyphtounicode{Agrave}{00C0}
\pdfglyphtounicode{Agravesmall}{F7E0}
\pdfglyphtounicode{Ahookabove}{1EA2}
\pdfglyphtounicode{Aiecyrillic}{04D4}
\pdfglyphtounicode{Ainvertedbreve}{0202}
\pdfglyphtounicode{Alpha}{0391}
\pdfglyphtounicode{Alphatonos}{0386}
\pdfglyphtounicode{Amacron}{0100}
\pdfglyphtounicode{Amonospace}{FF21}
\pdfglyphtounicode{Aogonek}{0104}
\pdfglyphtounicode{Aring}{00C5}
\pdfglyphtounicode{Aringacute}{01FA}
\pdfglyphtounicode{Aringbelow}{1E00}
\pdfglyphtounicode{Aringsmall}{F7E5}
\pdfglyphtounicode{Asmall}{F761}
\pdfglyphtounicode{Atilde}{00C3}
\pdfglyphtounicode{Atildesmall}{F7E3}
\pdfglyphtounicode{Aybarmenian}{0531}
\pdfglyphtounicode{B}{0042}
\pdfglyphtounicode{Bcircle}{24B7}
\pdfglyphtounicode{Bdotaccent}{1E02}
\pdfglyphtounicode{Bdotbelow}{1E04}
\pdfglyphtounicode{Becyrillic}{0411}
\pdfglyphtounicode{Benarmenian}{0532}
\pdfglyphtounicode{Beta}{0392}
\pdfglyphtounicode{Bhook}{0181}
\pdfglyphtounicode{Blinebelow}{1E06}
\pdfglyphtounicode{Bmonospace}{FF22}
\pdfglyphtounicode{Brevesmall}{F6F4}
\pdfglyphtounicode{Bsmall}{F762}
\pdfglyphtounicode{Btopbar}{0182}
\pdfglyphtounicode{C}{0043}
\pdfglyphtounicode{Caarmenian}{053E}
\pdfglyphtounicode{Cacute}{0106}
\pdfglyphtounicode{Caron}{F6CA}
\pdfglyphtounicode{Caronsmall}{F6F5}
\pdfglyphtounicode{Ccaron}{010C}
\pdfglyphtounicode{Ccedilla}{00C7}
\pdfglyphtounicode{Ccedillaacute}{1E08}
\pdfglyphtounicode{Ccedillasmall}{F7E7}
\pdfglyphtounicode{Ccircle}{24B8}
\pdfglyphtounicode{Ccircumflex}{0108}
\pdfglyphtounicode{Cdot}{010A}
\pdfglyphtounicode{Cdotaccent}{010A}
\pdfglyphtounicode{Cedillasmall}{F7B8}
\pdfglyphtounicode{Chaarmenian}{0549}
\pdfglyphtounicode{Cheabkhasiancyrillic}{04BC}
\pdfglyphtounicode{Checyrillic}{0427}
\pdfglyphtounicode{Chedescenderabkhasiancyrillic}{04BE}
\pdfglyphtounicode{Chedescendercyrillic}{04B6}
\pdfglyphtounicode{Chedieresiscyrillic}{04F4}
\pdfglyphtounicode{Cheharmenian}{0543}
\pdfglyphtounicode{Chekhakassiancyrillic}{04CB}
\pdfglyphtounicode{Cheverticalstrokecyrillic}{04B8}
\pdfglyphtounicode{Chi}{03A7}
\pdfglyphtounicode{Chook}{0187}
\pdfglyphtounicode{Circumflexsmall}{F6F6}
\pdfglyphtounicode{Cmonospace}{FF23}
\pdfglyphtounicode{Coarmenian}{0551}
\pdfglyphtounicode{Csmall}{F763}
\pdfglyphtounicode{D}{0044}
\pdfglyphtounicode{DZ}{01F1}
\pdfglyphtounicode{DZcaron}{01C4}
\pdfglyphtounicode{Daarmenian}{0534}
\pdfglyphtounicode{Dafrican}{0189}
\pdfglyphtounicode{Dcaron}{010E}
\pdfglyphtounicode{Dcedilla}{1E10}
\pdfglyphtounicode{Dcircle}{24B9}
\pdfglyphtounicode{Dcircumflexbelow}{1E12}
\pdfglyphtounicode{Dcroat}{0110}
\pdfglyphtounicode{Ddotaccent}{1E0A}
\pdfglyphtounicode{Ddotbelow}{1E0C}
\pdfglyphtounicode{Decyrillic}{0414}
\pdfglyphtounicode{Deicoptic}{03EE}
\pdfglyphtounicode{Delta}{2206}
\pdfglyphtounicode{Deltagreek}{0394}
\pdfglyphtounicode{Dhook}{018A}
\pdfglyphtounicode{Dieresis}{F6CB}
\pdfglyphtounicode{DieresisAcute}{F6CC}
\pdfglyphtounicode{DieresisGrave}{F6CD}
\pdfglyphtounicode{Dieresissmall}{F7A8}
\pdfglyphtounicode{Digammagreek}{03DC}
\pdfglyphtounicode{Djecyrillic}{0402}
\pdfglyphtounicode{Dlinebelow}{1E0E}
\pdfglyphtounicode{Dmonospace}{FF24}
\pdfglyphtounicode{Dotaccentsmall}{F6F7}
\pdfglyphtounicode{Dslash}{0110}
\pdfglyphtounicode{Dsmall}{F764}
\pdfglyphtounicode{Dtopbar}{018B}
\pdfglyphtounicode{Dz}{01F2}
\pdfglyphtounicode{Dzcaron}{01C5}
\pdfglyphtounicode{Dzeabkhasiancyrillic}{04E0}
\pdfglyphtounicode{Dzecyrillic}{0405}
\pdfglyphtounicode{Dzhecyrillic}{040F}
\pdfglyphtounicode{E}{0045}
\pdfglyphtounicode{Eacute}{00C9}
\pdfglyphtounicode{Eacutesmall}{F7E9}
\pdfglyphtounicode{Ebreve}{0114}
\pdfglyphtounicode{Ecaron}{011A}
\pdfglyphtounicode{Ecedillabreve}{1E1C}
\pdfglyphtounicode{Echarmenian}{0535}
\pdfglyphtounicode{Ecircle}{24BA}
\pdfglyphtounicode{Ecircumflex}{00CA}
\pdfglyphtounicode{Ecircumflexacute}{1EBE}
\pdfglyphtounicode{Ecircumflexbelow}{1E18}
\pdfglyphtounicode{Ecircumflexdotbelow}{1EC6}
\pdfglyphtounicode{Ecircumflexgrave}{1EC0}
\pdfglyphtounicode{Ecircumflexhookabove}{1EC2}
\pdfglyphtounicode{Ecircumflexsmall}{F7EA}
\pdfglyphtounicode{Ecircumflextilde}{1EC4}
\pdfglyphtounicode{Ecyrillic}{0404}
\pdfglyphtounicode{Edblgrave}{0204}
\pdfglyphtounicode{Edieresis}{00CB}
\pdfglyphtounicode{Edieresissmall}{F7EB}
\pdfglyphtounicode{Edot}{0116}
\pdfglyphtounicode{Edotaccent}{0116}
\pdfglyphtounicode{Edotbelow}{1EB8}
\pdfglyphtounicode{Efcyrillic}{0424}
\pdfglyphtounicode{Egrave}{00C8}
\pdfglyphtounicode{Egravesmall}{F7E8}
\pdfglyphtounicode{Eharmenian}{0537}
\pdfglyphtounicode{Ehookabove}{1EBA}
\pdfglyphtounicode{Eightroman}{2167}
\pdfglyphtounicode{Einvertedbreve}{0206}
\pdfglyphtounicode{Eiotifiedcyrillic}{0464}
\pdfglyphtounicode{Elcyrillic}{041B}
\pdfglyphtounicode{Elevenroman}{216A}
\pdfglyphtounicode{Emacron}{0112}
\pdfglyphtounicode{Emacronacute}{1E16}
\pdfglyphtounicode{Emacrongrave}{1E14}
\pdfglyphtounicode{Emcyrillic}{041C}
\pdfglyphtounicode{Emonospace}{FF25}
\pdfglyphtounicode{Encyrillic}{041D}
\pdfglyphtounicode{Endescendercyrillic}{04A2}
\pdfglyphtounicode{Eng}{014A}
\pdfglyphtounicode{Enghecyrillic}{04A4}
\pdfglyphtounicode{Enhookcyrillic}{04C7}
\pdfglyphtounicode{Eogonek}{0118}
\pdfglyphtounicode{Eopen}{0190}
\pdfglyphtounicode{Epsilon}{0395}
\pdfglyphtounicode{Epsilontonos}{0388}
\pdfglyphtounicode{Ercyrillic}{0420}
\pdfglyphtounicode{Ereversed}{018E}
\pdfglyphtounicode{Ereversedcyrillic}{042D}
\pdfglyphtounicode{Escyrillic}{0421}
\pdfglyphtounicode{Esdescendercyrillic}{04AA}
\pdfglyphtounicode{Esh}{01A9}
\pdfglyphtounicode{Esmall}{F765}
\pdfglyphtounicode{Eta}{0397}
\pdfglyphtounicode{Etarmenian}{0538}
\pdfglyphtounicode{Etatonos}{0389}
\pdfglyphtounicode{Eth}{00D0}
\pdfglyphtounicode{Ethsmall}{F7F0}
\pdfglyphtounicode{Etilde}{1EBC}
\pdfglyphtounicode{Etildebelow}{1E1A}
\pdfglyphtounicode{Euro}{20AC}
\pdfglyphtounicode{Ezh}{01B7}
\pdfglyphtounicode{Ezhcaron}{01EE}
\pdfglyphtounicode{Ezhreversed}{01B8}
\pdfglyphtounicode{F}{0046}
\pdfglyphtounicode{Fcircle}{24BB}
\pdfglyphtounicode{Fdotaccent}{1E1E}
\pdfglyphtounicode{Feharmenian}{0556}
\pdfglyphtounicode{Feicoptic}{03E4}
\pdfglyphtounicode{Fhook}{0191}
\pdfglyphtounicode{Fitacyrillic}{0472}
\pdfglyphtounicode{Fiveroman}{2164}
\pdfglyphtounicode{Fmonospace}{FF26}
\pdfglyphtounicode{Fourroman}{2163}
\pdfglyphtounicode{Fsmall}{F766}
\pdfglyphtounicode{G}{0047}
\pdfglyphtounicode{GBsquare}{3387}
\pdfglyphtounicode{Gacute}{01F4}
\pdfglyphtounicode{Gamma}{0393}
\pdfglyphtounicode{Gammaafrican}{0194}
\pdfglyphtounicode{Gangiacoptic}{03EA}
\pdfglyphtounicode{Gbreve}{011E}
\pdfglyphtounicode{Gcaron}{01E6}
\pdfglyphtounicode{Gcedilla}{0122}
\pdfglyphtounicode{Gcircle}{24BC}
\pdfglyphtounicode{Gcircumflex}{011C}
\pdfglyphtounicode{Gcommaaccent}{0122}
\pdfglyphtounicode{Gdot}{0120}
\pdfglyphtounicode{Gdotaccent}{0120}
\pdfglyphtounicode{Gecyrillic}{0413}
\pdfglyphtounicode{Ghadarmenian}{0542}
\pdfglyphtounicode{Ghemiddlehookcyrillic}{0494}
\pdfglyphtounicode{Ghestrokecyrillic}{0492}
\pdfglyphtounicode{Gheupturncyrillic}{0490}
\pdfglyphtounicode{Ghook}{0193}
\pdfglyphtounicode{Gimarmenian}{0533}
\pdfglyphtounicode{Gjecyrillic}{0403}
\pdfglyphtounicode{Gmacron}{1E20}
\pdfglyphtounicode{Gmonospace}{FF27}
\pdfglyphtounicode{Grave}{F6CE}
\pdfglyphtounicode{Gravesmall}{F760}
\pdfglyphtounicode{Gsmall}{F767}
\pdfglyphtounicode{Gsmallhook}{029B}
\pdfglyphtounicode{Gstroke}{01E4}
\pdfglyphtounicode{H}{0048}
\pdfglyphtounicode{H18533}{25CF}
\pdfglyphtounicode{H18543}{25AA}
\pdfglyphtounicode{H18551}{25AB}
\pdfglyphtounicode{H22073}{25A1}
\pdfglyphtounicode{HPsquare}{33CB}
\pdfglyphtounicode{Haabkhasiancyrillic}{04A8}
\pdfglyphtounicode{Hadescendercyrillic}{04B2}
\pdfglyphtounicode{Hardsigncyrillic}{042A}
\pdfglyphtounicode{Hbar}{0126}
\pdfglyphtounicode{Hbrevebelow}{1E2A}
\pdfglyphtounicode{Hcedilla}{1E28}
\pdfglyphtounicode{Hcircle}{24BD}
\pdfglyphtounicode{Hcircumflex}{0124}
\pdfglyphtounicode{Hdieresis}{1E26}
\pdfglyphtounicode{Hdotaccent}{1E22}
\pdfglyphtounicode{Hdotbelow}{1E24}
\pdfglyphtounicode{Hmonospace}{FF28}
\pdfglyphtounicode{Hoarmenian}{0540}
\pdfglyphtounicode{Horicoptic}{03E8}
\pdfglyphtounicode{Hsmall}{F768}
\pdfglyphtounicode{Hungarumlaut}{F6CF}
\pdfglyphtounicode{Hungarumlautsmall}{F6F8}
\pdfglyphtounicode{Hzsquare}{3390}
\pdfglyphtounicode{I}{0049}
\pdfglyphtounicode{IAcyrillic}{042F}
\pdfglyphtounicode{IJ}{0132}
\pdfglyphtounicode{IUcyrillic}{042E}
\pdfglyphtounicode{Iacute}{00CD}
\pdfglyphtounicode{Iacutesmall}{F7ED}
\pdfglyphtounicode{Ibreve}{012C}
\pdfglyphtounicode{Icaron}{01CF}
\pdfglyphtounicode{Icircle}{24BE}
\pdfglyphtounicode{Icircumflex}{00CE}
\pdfglyphtounicode{Icircumflexsmall}{F7EE}
\pdfglyphtounicode{Icyrillic}{0406}
\pdfglyphtounicode{Idblgrave}{0208}
\pdfglyphtounicode{Idieresis}{00CF}
\pdfglyphtounicode{Idieresisacute}{1E2E}
\pdfglyphtounicode{Idieresiscyrillic}{04E4}
\pdfglyphtounicode{Idieresissmall}{F7EF}
\pdfglyphtounicode{Idot}{0130}
\pdfglyphtounicode{Idotaccent}{0130}
\pdfglyphtounicode{Idotbelow}{1ECA}
\pdfglyphtounicode{Iebrevecyrillic}{04D6}
\pdfglyphtounicode{Iecyrillic}{0415}
\pdfglyphtounicode{Ifraktur}{2111}
\pdfglyphtounicode{Igrave}{00CC}
\pdfglyphtounicode{Igravesmall}{F7EC}
\pdfglyphtounicode{Ihookabove}{1EC8}
\pdfglyphtounicode{Iicyrillic}{0418}
\pdfglyphtounicode{Iinvertedbreve}{020A}
\pdfglyphtounicode{Iishortcyrillic}{0419}
\pdfglyphtounicode{Imacron}{012A}
\pdfglyphtounicode{Imacroncyrillic}{04E2}
\pdfglyphtounicode{Imonospace}{FF29}
\pdfglyphtounicode{Iniarmenian}{053B}
\pdfglyphtounicode{Iocyrillic}{0401}
\pdfglyphtounicode{Iogonek}{012E}
\pdfglyphtounicode{Iota}{0399}
\pdfglyphtounicode{Iotaafrican}{0196}
\pdfglyphtounicode{Iotadieresis}{03AA}
\pdfglyphtounicode{Iotatonos}{038A}
\pdfglyphtounicode{Ismall}{F769}
\pdfglyphtounicode{Istroke}{0197}
\pdfglyphtounicode{Itilde}{0128}
\pdfglyphtounicode{Itildebelow}{1E2C}
\pdfglyphtounicode{Izhitsacyrillic}{0474}
\pdfglyphtounicode{Izhitsadblgravecyrillic}{0476}
\pdfglyphtounicode{J}{004A}
\pdfglyphtounicode{Jaarmenian}{0541}
\pdfglyphtounicode{Jcircle}{24BF}
\pdfglyphtounicode{Jcircumflex}{0134}
\pdfglyphtounicode{Jecyrillic}{0408}
\pdfglyphtounicode{Jheharmenian}{054B}
\pdfglyphtounicode{Jmonospace}{FF2A}
\pdfglyphtounicode{Jsmall}{F76A}
\pdfglyphtounicode{K}{004B}
\pdfglyphtounicode{KBsquare}{3385}
\pdfglyphtounicode{KKsquare}{33CD}
\pdfglyphtounicode{Kabashkircyrillic}{04A0}
\pdfglyphtounicode{Kacute}{1E30}
\pdfglyphtounicode{Kacyrillic}{041A}
\pdfglyphtounicode{Kadescendercyrillic}{049A}
\pdfglyphtounicode{Kahookcyrillic}{04C3}
\pdfglyphtounicode{Kappa}{039A}
\pdfglyphtounicode{Kastrokecyrillic}{049E}
\pdfglyphtounicode{Kaverticalstrokecyrillic}{049C}
\pdfglyphtounicode{Kcaron}{01E8}
\pdfglyphtounicode{Kcedilla}{0136}
\pdfglyphtounicode{Kcircle}{24C0}
\pdfglyphtounicode{Kcommaaccent}{0136}
\pdfglyphtounicode{Kdotbelow}{1E32}
\pdfglyphtounicode{Keharmenian}{0554}
\pdfglyphtounicode{Kenarmenian}{053F}
\pdfglyphtounicode{Khacyrillic}{0425}
\pdfglyphtounicode{Kheicoptic}{03E6}
\pdfglyphtounicode{Khook}{0198}
\pdfglyphtounicode{Kjecyrillic}{040C}
\pdfglyphtounicode{Klinebelow}{1E34}
\pdfglyphtounicode{Kmonospace}{FF2B}
\pdfglyphtounicode{Koppacyrillic}{0480}
\pdfglyphtounicode{Koppagreek}{03DE}
\pdfglyphtounicode{Ksicyrillic}{046E}
\pdfglyphtounicode{Ksmall}{F76B}
\pdfglyphtounicode{L}{004C}
\pdfglyphtounicode{LJ}{01C7}
\pdfglyphtounicode{LL}{F6BF}
\pdfglyphtounicode{Lacute}{0139}
\pdfglyphtounicode{Lambda}{039B}
\pdfglyphtounicode{Lcaron}{013D}
\pdfglyphtounicode{Lcedilla}{013B}
\pdfglyphtounicode{Lcircle}{24C1}
\pdfglyphtounicode{Lcircumflexbelow}{1E3C}
\pdfglyphtounicode{Lcommaaccent}{013B}
\pdfglyphtounicode{Ldot}{013F}
\pdfglyphtounicode{Ldotaccent}{013F}
\pdfglyphtounicode{Ldotbelow}{1E36}
\pdfglyphtounicode{Ldotbelowmacron}{1E38}
\pdfglyphtounicode{Liwnarmenian}{053C}
\pdfglyphtounicode{Lj}{01C8}
\pdfglyphtounicode{Ljecyrillic}{0409}
\pdfglyphtounicode{Llinebelow}{1E3A}
\pdfglyphtounicode{Lmonospace}{FF2C}
\pdfglyphtounicode{Lslash}{0141}
\pdfglyphtounicode{Lslashsmall}{F6F9}
\pdfglyphtounicode{Lsmall}{F76C}
\pdfglyphtounicode{M}{004D}
\pdfglyphtounicode{MBsquare}{3386}
\pdfglyphtounicode{Macron}{F6D0}
\pdfglyphtounicode{Macronsmall}{F7AF}
\pdfglyphtounicode{Macute}{1E3E}
\pdfglyphtounicode{Mcircle}{24C2}
\pdfglyphtounicode{Mdotaccent}{1E40}
\pdfglyphtounicode{Mdotbelow}{1E42}
\pdfglyphtounicode{Menarmenian}{0544}
\pdfglyphtounicode{Mmonospace}{FF2D}
\pdfglyphtounicode{Msmall}{F76D}
\pdfglyphtounicode{Mturned}{019C}
\pdfglyphtounicode{Mu}{039C}
\pdfglyphtounicode{N}{004E}
\pdfglyphtounicode{NJ}{01CA}
\pdfglyphtounicode{Nacute}{0143}
\pdfglyphtounicode{Ncaron}{0147}
\pdfglyphtounicode{Ncedilla}{0145}
\pdfglyphtounicode{Ncircle}{24C3}
\pdfglyphtounicode{Ncircumflexbelow}{1E4A}
\pdfglyphtounicode{Ncommaaccent}{0145}
\pdfglyphtounicode{Ndotaccent}{1E44}
\pdfglyphtounicode{Ndotbelow}{1E46}
\pdfglyphtounicode{Nhookleft}{019D}
\pdfglyphtounicode{Nineroman}{2168}
\pdfglyphtounicode{Nj}{01CB}
\pdfglyphtounicode{Njecyrillic}{040A}
\pdfglyphtounicode{Nlinebelow}{1E48}
\pdfglyphtounicode{Nmonospace}{FF2E}
\pdfglyphtounicode{Nowarmenian}{0546}
\pdfglyphtounicode{Nsmall}{F76E}
\pdfglyphtounicode{Ntilde}{00D1}
\pdfglyphtounicode{Ntildesmall}{F7F1}
\pdfglyphtounicode{Nu}{039D}
\pdfglyphtounicode{O}{004F}
\pdfglyphtounicode{OE}{0152}
\pdfglyphtounicode{OEsmall}{F6FA}
\pdfglyphtounicode{Oacute}{00D3}
\pdfglyphtounicode{Oacutesmall}{F7F3}
\pdfglyphtounicode{Obarredcyrillic}{04E8}
\pdfglyphtounicode{Obarreddieresiscyrillic}{04EA}
\pdfglyphtounicode{Obreve}{014E}
\pdfglyphtounicode{Ocaron}{01D1}
\pdfglyphtounicode{Ocenteredtilde}{019F}
\pdfglyphtounicode{Ocircle}{24C4}
\pdfglyphtounicode{Ocircumflex}{00D4}
\pdfglyphtounicode{Ocircumflexacute}{1ED0}
\pdfglyphtounicode{Ocircumflexdotbelow}{1ED8}
\pdfglyphtounicode{Ocircumflexgrave}{1ED2}
\pdfglyphtounicode{Ocircumflexhookabove}{1ED4}
\pdfglyphtounicode{Ocircumflexsmall}{F7F4}
\pdfglyphtounicode{Ocircumflextilde}{1ED6}
\pdfglyphtounicode{Ocyrillic}{041E}
\pdfglyphtounicode{Odblacute}{0150}
\pdfglyphtounicode{Odblgrave}{020C}
\pdfglyphtounicode{Odieresis}{00D6}
\pdfglyphtounicode{Odieresiscyrillic}{04E6}
\pdfglyphtounicode{Odieresissmall}{F7F6}
\pdfglyphtounicode{Odotbelow}{1ECC}
\pdfglyphtounicode{Ogoneksmall}{F6FB}
\pdfglyphtounicode{Ograve}{00D2}
\pdfglyphtounicode{Ogravesmall}{F7F2}
\pdfglyphtounicode{Oharmenian}{0555}
\pdfglyphtounicode{Ohm}{2126}
\pdfglyphtounicode{Ohookabove}{1ECE}
\pdfglyphtounicode{Ohorn}{01A0}
\pdfglyphtounicode{Ohornacute}{1EDA}
\pdfglyphtounicode{Ohorndotbelow}{1EE2}
\pdfglyphtounicode{Ohorngrave}{1EDC}
\pdfglyphtounicode{Ohornhookabove}{1EDE}
\pdfglyphtounicode{Ohorntilde}{1EE0}
\pdfglyphtounicode{Ohungarumlaut}{0150}
\pdfglyphtounicode{Oi}{01A2}
\pdfglyphtounicode{Oinvertedbreve}{020E}
\pdfglyphtounicode{Omacron}{014C}
\pdfglyphtounicode{Omacronacute}{1E52}
\pdfglyphtounicode{Omacrongrave}{1E50}
\pdfglyphtounicode{Omega}{2126}
\pdfglyphtounicode{Omegacyrillic}{0460}
\pdfglyphtounicode{Omegagreek}{03A9}
\pdfglyphtounicode{Omegaroundcyrillic}{047A}
\pdfglyphtounicode{Omegatitlocyrillic}{047C}
\pdfglyphtounicode{Omegatonos}{038F}
\pdfglyphtounicode{Omicron}{039F}
\pdfglyphtounicode{Omicrontonos}{038C}
\pdfglyphtounicode{Omonospace}{FF2F}
\pdfglyphtounicode{Oneroman}{2160}
\pdfglyphtounicode{Oogonek}{01EA}
\pdfglyphtounicode{Oogonekmacron}{01EC}
\pdfglyphtounicode{Oopen}{0186}
\pdfglyphtounicode{Oslash}{00D8}
\pdfglyphtounicode{Oslashacute}{01FE}
\pdfglyphtounicode{Oslashsmall}{F7F8}
\pdfglyphtounicode{Osmall}{F76F}
\pdfglyphtounicode{Ostrokeacute}{01FE}
\pdfglyphtounicode{Otcyrillic}{047E}
\pdfglyphtounicode{Otilde}{00D5}
\pdfglyphtounicode{Otildeacute}{1E4C}
\pdfglyphtounicode{Otildedieresis}{1E4E}
\pdfglyphtounicode{Otildesmall}{F7F5}
\pdfglyphtounicode{P}{0050}
\pdfglyphtounicode{Pacute}{1E54}
\pdfglyphtounicode{Pcircle}{24C5}
\pdfglyphtounicode{Pdotaccent}{1E56}
\pdfglyphtounicode{Pecyrillic}{041F}
\pdfglyphtounicode{Peharmenian}{054A}
\pdfglyphtounicode{Pemiddlehookcyrillic}{04A6}
\pdfglyphtounicode{Phi}{03A6}
\pdfglyphtounicode{Phook}{01A4}
\pdfglyphtounicode{Pi}{03A0}
\pdfglyphtounicode{Piwrarmenian}{0553}
\pdfglyphtounicode{Pmonospace}{FF30}
\pdfglyphtounicode{Psi}{03A8}
\pdfglyphtounicode{Psicyrillic}{0470}
\pdfglyphtounicode{Psmall}{F770}
\pdfglyphtounicode{Q}{0051}
\pdfglyphtounicode{Qcircle}{24C6}
\pdfglyphtounicode{Qmonospace}{FF31}
\pdfglyphtounicode{Qsmall}{F771}
\pdfglyphtounicode{R}{0052}
\pdfglyphtounicode{Raarmenian}{054C}
\pdfglyphtounicode{Racute}{0154}
\pdfglyphtounicode{Rcaron}{0158}
\pdfglyphtounicode{Rcedilla}{0156}
\pdfglyphtounicode{Rcircle}{24C7}
\pdfglyphtounicode{Rcommaaccent}{0156}
\pdfglyphtounicode{Rdblgrave}{0210}
\pdfglyphtounicode{Rdotaccent}{1E58}
\pdfglyphtounicode{Rdotbelow}{1E5A}
\pdfglyphtounicode{Rdotbelowmacron}{1E5C}
\pdfglyphtounicode{Reharmenian}{0550}
\pdfglyphtounicode{Rfraktur}{211C}
\pdfglyphtounicode{Rho}{03A1}
\pdfglyphtounicode{Ringsmall}{F6FC}
\pdfglyphtounicode{Rinvertedbreve}{0212}
\pdfglyphtounicode{Rlinebelow}{1E5E}
\pdfglyphtounicode{Rmonospace}{FF32}
\pdfglyphtounicode{Rsmall}{F772}
\pdfglyphtounicode{Rsmallinverted}{0281}
\pdfglyphtounicode{Rsmallinvertedsuperior}{02B6}
\pdfglyphtounicode{S}{0053}
\pdfglyphtounicode{SF010000}{250C}
\pdfglyphtounicode{SF020000}{2514}
\pdfglyphtounicode{SF030000}{2510}
\pdfglyphtounicode{SF040000}{2518}
\pdfglyphtounicode{SF050000}{253C}
\pdfglyphtounicode{SF060000}{252C}
\pdfglyphtounicode{SF070000}{2534}
\pdfglyphtounicode{SF080000}{251C}
\pdfglyphtounicode{SF090000}{2524}
\pdfglyphtounicode{SF100000}{2500}
\pdfglyphtounicode{SF110000}{2502}
\pdfglyphtounicode{SF190000}{2561}
\pdfglyphtounicode{SF200000}{2562}
\pdfglyphtounicode{SF210000}{2556}
\pdfglyphtounicode{SF220000}{2555}
\pdfglyphtounicode{SF230000}{2563}
\pdfglyphtounicode{SF240000}{2551}
\pdfglyphtounicode{SF250000}{2557}
\pdfglyphtounicode{SF260000}{255D}
\pdfglyphtounicode{SF270000}{255C}
\pdfglyphtounicode{SF280000}{255B}
\pdfglyphtounicode{SF360000}{255E}
\pdfglyphtounicode{SF370000}{255F}
\pdfglyphtounicode{SF380000}{255A}
\pdfglyphtounicode{SF390000}{2554}
\pdfglyphtounicode{SF400000}{2569}
\pdfglyphtounicode{SF410000}{2566}
\pdfglyphtounicode{SF420000}{2560}
\pdfglyphtounicode{SF430000}{2550}
\pdfglyphtounicode{SF440000}{256C}
\pdfglyphtounicode{SF450000}{2567}
\pdfglyphtounicode{SF460000}{2568}
\pdfglyphtounicode{SF470000}{2564}
\pdfglyphtounicode{SF480000}{2565}
\pdfglyphtounicode{SF490000}{2559}
\pdfglyphtounicode{SF500000}{2558}
\pdfglyphtounicode{SF510000}{2552}
\pdfglyphtounicode{SF520000}{2553}
\pdfglyphtounicode{SF530000}{256B}
\pdfglyphtounicode{SF540000}{256A}
\pdfglyphtounicode{Sacute}{015A}
\pdfglyphtounicode{Sacutedotaccent}{1E64}
\pdfglyphtounicode{Sampigreek}{03E0}
\pdfglyphtounicode{Scaron}{0160}
\pdfglyphtounicode{Scarondotaccent}{1E66}
\pdfglyphtounicode{Scaronsmall}{F6FD}
\pdfglyphtounicode{Scedilla}{015E}
\pdfglyphtounicode{Schwa}{018F}
\pdfglyphtounicode{Schwacyrillic}{04D8}
\pdfglyphtounicode{Schwadieresiscyrillic}{04DA}
\pdfglyphtounicode{Scircle}{24C8}
\pdfglyphtounicode{Scircumflex}{015C}
\pdfglyphtounicode{Scommaaccent}{0218}
\pdfglyphtounicode{Sdotaccent}{1E60}
\pdfglyphtounicode{Sdotbelow}{1E62}
\pdfglyphtounicode{Sdotbelowdotaccent}{1E68}
\pdfglyphtounicode{Seharmenian}{054D}
\pdfglyphtounicode{Sevenroman}{2166}
\pdfglyphtounicode{Shaarmenian}{0547}
\pdfglyphtounicode{Shacyrillic}{0428}
\pdfglyphtounicode{Shchacyrillic}{0429}
\pdfglyphtounicode{Sheicoptic}{03E2}
\pdfglyphtounicode{Shhacyrillic}{04BA}
\pdfglyphtounicode{Shimacoptic}{03EC}
\pdfglyphtounicode{Sigma}{03A3}
\pdfglyphtounicode{Sixroman}{2165}
\pdfglyphtounicode{Smonospace}{FF33}
\pdfglyphtounicode{Softsigncyrillic}{042C}
\pdfglyphtounicode{Ssmall}{F773}
\pdfglyphtounicode{Stigmagreek}{03DA}
\pdfglyphtounicode{T}{0054}
\pdfglyphtounicode{Tau}{03A4}
\pdfglyphtounicode{Tbar}{0166}
\pdfglyphtounicode{Tcaron}{0164}
\pdfglyphtounicode{Tcedilla}{0162}
\pdfglyphtounicode{Tcircle}{24C9}
\pdfglyphtounicode{Tcircumflexbelow}{1E70}
\pdfglyphtounicode{Tcommaaccent}{0162}
\pdfglyphtounicode{Tdotaccent}{1E6A}
\pdfglyphtounicode{Tdotbelow}{1E6C}
\pdfglyphtounicode{Tecyrillic}{0422}
\pdfglyphtounicode{Tedescendercyrillic}{04AC}
\pdfglyphtounicode{Tenroman}{2169}
\pdfglyphtounicode{Tetsecyrillic}{04B4}
\pdfglyphtounicode{Theta}{0398}
\pdfglyphtounicode{Thook}{01AC}
\pdfglyphtounicode{Thorn}{00DE}
\pdfglyphtounicode{Thornsmall}{F7FE}
\pdfglyphtounicode{Threeroman}{2162}
\pdfglyphtounicode{Tildesmall}{F6FE}
\pdfglyphtounicode{Tiwnarmenian}{054F}
\pdfglyphtounicode{Tlinebelow}{1E6E}
\pdfglyphtounicode{Tmonospace}{FF34}
\pdfglyphtounicode{Toarmenian}{0539}
\pdfglyphtounicode{Tonefive}{01BC}
\pdfglyphtounicode{Tonesix}{0184}
\pdfglyphtounicode{Tonetwo}{01A7}
\pdfglyphtounicode{Tretroflexhook}{01AE}
\pdfglyphtounicode{Tsecyrillic}{0426}
\pdfglyphtounicode{Tshecyrillic}{040B}
\pdfglyphtounicode{Tsmall}{F774}
\pdfglyphtounicode{Twelveroman}{216B}
\pdfglyphtounicode{Tworoman}{2161}
\pdfglyphtounicode{U}{0055}
\pdfglyphtounicode{Uacute}{00DA}
\pdfglyphtounicode{Uacutesmall}{F7FA}
\pdfglyphtounicode{Ubreve}{016C}
\pdfglyphtounicode{Ucaron}{01D3}
\pdfglyphtounicode{Ucircle}{24CA}
\pdfglyphtounicode{Ucircumflex}{00DB}
\pdfglyphtounicode{Ucircumflexbelow}{1E76}
\pdfglyphtounicode{Ucircumflexsmall}{F7FB}
\pdfglyphtounicode{Ucyrillic}{0423}
\pdfglyphtounicode{Udblacute}{0170}
\pdfglyphtounicode{Udblgrave}{0214}
\pdfglyphtounicode{Udieresis}{00DC}
\pdfglyphtounicode{Udieresisacute}{01D7}
\pdfglyphtounicode{Udieresisbelow}{1E72}
\pdfglyphtounicode{Udieresiscaron}{01D9}
\pdfglyphtounicode{Udieresiscyrillic}{04F0}
\pdfglyphtounicode{Udieresisgrave}{01DB}
\pdfglyphtounicode{Udieresismacron}{01D5}
\pdfglyphtounicode{Udieresissmall}{F7FC}
\pdfglyphtounicode{Udotbelow}{1EE4}
\pdfglyphtounicode{Ugrave}{00D9}
\pdfglyphtounicode{Ugravesmall}{F7F9}
\pdfglyphtounicode{Uhookabove}{1EE6}
\pdfglyphtounicode{Uhorn}{01AF}
\pdfglyphtounicode{Uhornacute}{1EE8}
\pdfglyphtounicode{Uhorndotbelow}{1EF0}
\pdfglyphtounicode{Uhorngrave}{1EEA}
\pdfglyphtounicode{Uhornhookabove}{1EEC}
\pdfglyphtounicode{Uhorntilde}{1EEE}
\pdfglyphtounicode{Uhungarumlaut}{0170}
\pdfglyphtounicode{Uhungarumlautcyrillic}{04F2}
\pdfglyphtounicode{Uinvertedbreve}{0216}
\pdfglyphtounicode{Ukcyrillic}{0478}
\pdfglyphtounicode{Umacron}{016A}
\pdfglyphtounicode{Umacroncyrillic}{04EE}
\pdfglyphtounicode{Umacrondieresis}{1E7A}
\pdfglyphtounicode{Umonospace}{FF35}
\pdfglyphtounicode{Uogonek}{0172}
\pdfglyphtounicode{Upsilon}{03A5}
\pdfglyphtounicode{Upsilon1}{03D2}
\pdfglyphtounicode{Upsilonacutehooksymbolgreek}{03D3}
\pdfglyphtounicode{Upsilonafrican}{01B1}
\pdfglyphtounicode{Upsilondieresis}{03AB}
\pdfglyphtounicode{Upsilondieresishooksymbolgreek}{03D4}
\pdfglyphtounicode{Upsilonhooksymbol}{03D2}
\pdfglyphtounicode{Upsilontonos}{038E}
\pdfglyphtounicode{Uring}{016E}
\pdfglyphtounicode{Ushortcyrillic}{040E}
\pdfglyphtounicode{Usmall}{F775}
\pdfglyphtounicode{Ustraightcyrillic}{04AE}
\pdfglyphtounicode{Ustraightstrokecyrillic}{04B0}
\pdfglyphtounicode{Utilde}{0168}
\pdfglyphtounicode{Utildeacute}{1E78}
\pdfglyphtounicode{Utildebelow}{1E74}
\pdfglyphtounicode{V}{0056}
\pdfglyphtounicode{Vcircle}{24CB}
\pdfglyphtounicode{Vdotbelow}{1E7E}
\pdfglyphtounicode{Vecyrillic}{0412}
\pdfglyphtounicode{Vewarmenian}{054E}
\pdfglyphtounicode{Vhook}{01B2}
\pdfglyphtounicode{Vmonospace}{FF36}
\pdfglyphtounicode{Voarmenian}{0548}
\pdfglyphtounicode{Vsmall}{F776}
\pdfglyphtounicode{Vtilde}{1E7C}
\pdfglyphtounicode{W}{0057}
\pdfglyphtounicode{Wacute}{1E82}
\pdfglyphtounicode{Wcircle}{24CC}
\pdfglyphtounicode{Wcircumflex}{0174}
\pdfglyphtounicode{Wdieresis}{1E84}
\pdfglyphtounicode{Wdotaccent}{1E86}
\pdfglyphtounicode{Wdotbelow}{1E88}
\pdfglyphtounicode{Wgrave}{1E80}
\pdfglyphtounicode{Wmonospace}{FF37}
\pdfglyphtounicode{Wsmall}{F777}
\pdfglyphtounicode{X}{0058}
\pdfglyphtounicode{Xcircle}{24CD}
\pdfglyphtounicode{Xdieresis}{1E8C}
\pdfglyphtounicode{Xdotaccent}{1E8A}
\pdfglyphtounicode{Xeharmenian}{053D}
\pdfglyphtounicode{Xi}{039E}
\pdfglyphtounicode{Xmonospace}{FF38}
\pdfglyphtounicode{Xsmall}{F778}
\pdfglyphtounicode{Y}{0059}
\pdfglyphtounicode{Yacute}{00DD}
\pdfglyphtounicode{Yacutesmall}{F7FD}
\pdfglyphtounicode{Yatcyrillic}{0462}
\pdfglyphtounicode{Ycircle}{24CE}
\pdfglyphtounicode{Ycircumflex}{0176}
\pdfglyphtounicode{Ydieresis}{0178}
\pdfglyphtounicode{Ydieresissmall}{F7FF}
\pdfglyphtounicode{Ydotaccent}{1E8E}
\pdfglyphtounicode{Ydotbelow}{1EF4}
\pdfglyphtounicode{Yericyrillic}{042B}
\pdfglyphtounicode{Yerudieresiscyrillic}{04F8}
\pdfglyphtounicode{Ygrave}{1EF2}
\pdfglyphtounicode{Yhook}{01B3}
\pdfglyphtounicode{Yhookabove}{1EF6}
\pdfglyphtounicode{Yiarmenian}{0545}
\pdfglyphtounicode{Yicyrillic}{0407}
\pdfglyphtounicode{Yiwnarmenian}{0552}
\pdfglyphtounicode{Ymonospace}{FF39}
\pdfglyphtounicode{Ysmall}{F779}
\pdfglyphtounicode{Ytilde}{1EF8}
\pdfglyphtounicode{Yusbigcyrillic}{046A}
\pdfglyphtounicode{Yusbigiotifiedcyrillic}{046C}
\pdfglyphtounicode{Yuslittlecyrillic}{0466}
\pdfglyphtounicode{Yuslittleiotifiedcyrillic}{0468}
\pdfglyphtounicode{Z}{005A}
\pdfglyphtounicode{Zaarmenian}{0536}
\pdfglyphtounicode{Zacute}{0179}
\pdfglyphtounicode{Zcaron}{017D}
\pdfglyphtounicode{Zcaronsmall}{F6FF}
\pdfglyphtounicode{Zcircle}{24CF}
\pdfglyphtounicode{Zcircumflex}{1E90}
\pdfglyphtounicode{Zdot}{017B}
\pdfglyphtounicode{Zdotaccent}{017B}
\pdfglyphtounicode{Zdotbelow}{1E92}
\pdfglyphtounicode{Zecyrillic}{0417}
\pdfglyphtounicode{Zedescendercyrillic}{0498}
\pdfglyphtounicode{Zedieresiscyrillic}{04DE}
\pdfglyphtounicode{Zeta}{0396}
\pdfglyphtounicode{Zhearmenian}{053A}
\pdfglyphtounicode{Zhebrevecyrillic}{04C1}
\pdfglyphtounicode{Zhecyrillic}{0416}
\pdfglyphtounicode{Zhedescendercyrillic}{0496}
\pdfglyphtounicode{Zhedieresiscyrillic}{04DC}
\pdfglyphtounicode{Zlinebelow}{1E94}
\pdfglyphtounicode{Zmonospace}{FF3A}
\pdfglyphtounicode{Zsmall}{F77A}
\pdfglyphtounicode{Zstroke}{01B5}
\pdfglyphtounicode{a}{0061}
\pdfglyphtounicode{aabengali}{0986}
\pdfglyphtounicode{aacute}{00E1}
\pdfglyphtounicode{aadeva}{0906}
\pdfglyphtounicode{aagujarati}{0A86}
\pdfglyphtounicode{aagurmukhi}{0A06}
\pdfglyphtounicode{aamatragurmukhi}{0A3E}
\pdfglyphtounicode{aarusquare}{3303}
\pdfglyphtounicode{aavowelsignbengali}{09BE}
\pdfglyphtounicode{aavowelsigndeva}{093E}
\pdfglyphtounicode{aavowelsigngujarati}{0ABE}
\pdfglyphtounicode{abbreviationmarkarmenian}{055F}
\pdfglyphtounicode{abbreviationsigndeva}{0970}
\pdfglyphtounicode{abengali}{0985}
\pdfglyphtounicode{abopomofo}{311A}
\pdfglyphtounicode{abreve}{0103}
\pdfglyphtounicode{abreveacute}{1EAF}
\pdfglyphtounicode{abrevecyrillic}{04D1}
\pdfglyphtounicode{abrevedotbelow}{1EB7}
\pdfglyphtounicode{abrevegrave}{1EB1}
\pdfglyphtounicode{abrevehookabove}{1EB3}
\pdfglyphtounicode{abrevetilde}{1EB5}
\pdfglyphtounicode{acaron}{01CE}
\pdfglyphtounicode{acircle}{24D0}
\pdfglyphtounicode{acircumflex}{00E2}
\pdfglyphtounicode{acircumflexacute}{1EA5}
\pdfglyphtounicode{acircumflexdotbelow}{1EAD}
\pdfglyphtounicode{acircumflexgrave}{1EA7}
\pdfglyphtounicode{acircumflexhookabove}{1EA9}
\pdfglyphtounicode{acircumflextilde}{1EAB}
\pdfglyphtounicode{acute}{00B4}
\pdfglyphtounicode{acutebelowcmb}{0317}
\pdfglyphtounicode{acutecmb}{0301}
\pdfglyphtounicode{acutecomb}{0301}
\pdfglyphtounicode{acutedeva}{0954}
\pdfglyphtounicode{acutelowmod}{02CF}
\pdfglyphtounicode{acutetonecmb}{0341}
\pdfglyphtounicode{acyrillic}{0430}
\pdfglyphtounicode{adblgrave}{0201}
\pdfglyphtounicode{addakgurmukhi}{0A71}
\pdfglyphtounicode{adeva}{0905}
\pdfglyphtounicode{adieresis}{00E4}
\pdfglyphtounicode{adieresiscyrillic}{04D3}
\pdfglyphtounicode{adieresismacron}{01DF}
\pdfglyphtounicode{adotbelow}{1EA1}
\pdfglyphtounicode{adotmacron}{01E1}
\pdfglyphtounicode{ae}{00E6}
\pdfglyphtounicode{aeacute}{01FD}
\pdfglyphtounicode{aekorean}{3150}
\pdfglyphtounicode{aemacron}{01E3}
\pdfglyphtounicode{afii00208}{2015}
\pdfglyphtounicode{afii08941}{20A4}
\pdfglyphtounicode{afii10017}{0410}
\pdfglyphtounicode{afii10018}{0411}
\pdfglyphtounicode{afii10019}{0412}
\pdfglyphtounicode{afii10020}{0413}
\pdfglyphtounicode{afii10021}{0414}
\pdfglyphtounicode{afii10022}{0415}
\pdfglyphtounicode{afii10023}{0401}
\pdfglyphtounicode{afii10024}{0416}
\pdfglyphtounicode{afii10025}{0417}
\pdfglyphtounicode{afii10026}{0418}
\pdfglyphtounicode{afii10027}{0419}
\pdfglyphtounicode{afii10028}{041A}
\pdfglyphtounicode{afii10029}{041B}
\pdfglyphtounicode{afii10030}{041C}
\pdfglyphtounicode{afii10031}{041D}
\pdfglyphtounicode{afii10032}{041E}
\pdfglyphtounicode{afii10033}{041F}
\pdfglyphtounicode{afii10034}{0420}
\pdfglyphtounicode{afii10035}{0421}
\pdfglyphtounicode{afii10036}{0422}
\pdfglyphtounicode{afii10037}{0423}
\pdfglyphtounicode{afii10038}{0424}
\pdfglyphtounicode{afii10039}{0425}
\pdfglyphtounicode{afii10040}{0426}
\pdfglyphtounicode{afii10041}{0427}
\pdfglyphtounicode{afii10042}{0428}
\pdfglyphtounicode{afii10043}{0429}
\pdfglyphtounicode{afii10044}{042A}
\pdfglyphtounicode{afii10045}{042B}
\pdfglyphtounicode{afii10046}{042C}
\pdfglyphtounicode{afii10047}{042D}
\pdfglyphtounicode{afii10048}{042E}
\pdfglyphtounicode{afii10049}{042F}
\pdfglyphtounicode{afii10050}{0490}
\pdfglyphtounicode{afii10051}{0402}
\pdfglyphtounicode{afii10052}{0403}
\pdfglyphtounicode{afii10053}{0404}
\pdfglyphtounicode{afii10054}{0405}
\pdfglyphtounicode{afii10055}{0406}
\pdfglyphtounicode{afii10056}{0407}
\pdfglyphtounicode{afii10057}{0408}
\pdfglyphtounicode{afii10058}{0409}
\pdfglyphtounicode{afii10059}{040A}
\pdfglyphtounicode{afii10060}{040B}
\pdfglyphtounicode{afii10061}{040C}
\pdfglyphtounicode{afii10062}{040E}
\pdfglyphtounicode{afii10063}{F6C4}
\pdfglyphtounicode{afii10064}{F6C5}
\pdfglyphtounicode{afii10065}{0430}
\pdfglyphtounicode{afii10066}{0431}
\pdfglyphtounicode{afii10067}{0432}
\pdfglyphtounicode{afii10068}{0433}
\pdfglyphtounicode{afii10069}{0434}
\pdfglyphtounicode{afii10070}{0435}
\pdfglyphtounicode{afii10071}{0451}
\pdfglyphtounicode{afii10072}{0436}
\pdfglyphtounicode{afii10073}{0437}
\pdfglyphtounicode{afii10074}{0438}
\pdfglyphtounicode{afii10075}{0439}
\pdfglyphtounicode{afii10076}{043A}
\pdfglyphtounicode{afii10077}{043B}
\pdfglyphtounicode{afii10078}{043C}
\pdfglyphtounicode{afii10079}{043D}
\pdfglyphtounicode{afii10080}{043E}
\pdfglyphtounicode{afii10081}{043F}
\pdfglyphtounicode{afii10082}{0440}
\pdfglyphtounicode{afii10083}{0441}
\pdfglyphtounicode{afii10084}{0442}
\pdfglyphtounicode{afii10085}{0443}
\pdfglyphtounicode{afii10086}{0444}
\pdfglyphtounicode{afii10087}{0445}
\pdfglyphtounicode{afii10088}{0446}
\pdfglyphtounicode{afii10089}{0447}
\pdfglyphtounicode{afii10090}{0448}
\pdfglyphtounicode{afii10091}{0449}
\pdfglyphtounicode{afii10092}{044A}
\pdfglyphtounicode{afii10093}{044B}
\pdfglyphtounicode{afii10094}{044C}
\pdfglyphtounicode{afii10095}{044D}
\pdfglyphtounicode{afii10096}{044E}
\pdfglyphtounicode{afii10097}{044F}
\pdfglyphtounicode{afii10098}{0491}
\pdfglyphtounicode{afii10099}{0452}
\pdfglyphtounicode{afii10100}{0453}
\pdfglyphtounicode{afii10101}{0454}
\pdfglyphtounicode{afii10102}{0455}
\pdfglyphtounicode{afii10103}{0456}
\pdfglyphtounicode{afii10104}{0457}
\pdfglyphtounicode{afii10105}{0458}
\pdfglyphtounicode{afii10106}{0459}
\pdfglyphtounicode{afii10107}{045A}
\pdfglyphtounicode{afii10108}{045B}
\pdfglyphtounicode{afii10109}{045C}
\pdfglyphtounicode{afii10110}{045E}
\pdfglyphtounicode{afii10145}{040F}
\pdfglyphtounicode{afii10146}{0462}
\pdfglyphtounicode{afii10147}{0472}
\pdfglyphtounicode{afii10148}{0474}
\pdfglyphtounicode{afii10192}{F6C6}
\pdfglyphtounicode{afii10193}{045F}
\pdfglyphtounicode{afii10194}{0463}
\pdfglyphtounicode{afii10195}{0473}
\pdfglyphtounicode{afii10196}{0475}
\pdfglyphtounicode{afii10831}{F6C7}
\pdfglyphtounicode{afii10832}{F6C8}
\pdfglyphtounicode{afii10846}{04D9}
\pdfglyphtounicode{afii299}{200E}
\pdfglyphtounicode{afii300}{200F}
\pdfglyphtounicode{afii301}{200D}
\pdfglyphtounicode{afii57381}{066A}
\pdfglyphtounicode{afii57388}{060C}
\pdfglyphtounicode{afii57392}{0660}
\pdfglyphtounicode{afii57393}{0661}
\pdfglyphtounicode{afii57394}{0662}
\pdfglyphtounicode{afii57395}{0663}
\pdfglyphtounicode{afii57396}{0664}
\pdfglyphtounicode{afii57397}{0665}
\pdfglyphtounicode{afii57398}{0666}
\pdfglyphtounicode{afii57399}{0667}
\pdfglyphtounicode{afii57400}{0668}
\pdfglyphtounicode{afii57401}{0669}
\pdfglyphtounicode{afii57403}{061B}
\pdfglyphtounicode{afii57407}{061F}
\pdfglyphtounicode{afii57409}{0621}
\pdfglyphtounicode{afii57410}{0622}
\pdfglyphtounicode{afii57411}{0623}
\pdfglyphtounicode{afii57412}{0624}
\pdfglyphtounicode{afii57413}{0625}
\pdfglyphtounicode{afii57414}{0626}
\pdfglyphtounicode{afii57415}{0627}
\pdfglyphtounicode{afii57416}{0628}
\pdfglyphtounicode{afii57417}{0629}
\pdfglyphtounicode{afii57418}{062A}
\pdfglyphtounicode{afii57419}{062B}
\pdfglyphtounicode{afii57420}{062C}
\pdfglyphtounicode{afii57421}{062D}
\pdfglyphtounicode{afii57422}{062E}
\pdfglyphtounicode{afii57423}{062F}
\pdfglyphtounicode{afii57424}{0630}
\pdfglyphtounicode{afii57425}{0631}
\pdfglyphtounicode{afii57426}{0632}
\pdfglyphtounicode{afii57427}{0633}
\pdfglyphtounicode{afii57428}{0634}
\pdfglyphtounicode{afii57429}{0635}
\pdfglyphtounicode{afii57430}{0636}
\pdfglyphtounicode{afii57431}{0637}
\pdfglyphtounicode{afii57432}{0638}
\pdfglyphtounicode{afii57433}{0639}
\pdfglyphtounicode{afii57434}{063A}
\pdfglyphtounicode{afii57440}{0640}
\pdfglyphtounicode{afii57441}{0641}
\pdfglyphtounicode{afii57442}{0642}
\pdfglyphtounicode{afii57443}{0643}
\pdfglyphtounicode{afii57444}{0644}
\pdfglyphtounicode{afii57445}{0645}
\pdfglyphtounicode{afii57446}{0646}
\pdfglyphtounicode{afii57448}{0648}
\pdfglyphtounicode{afii57449}{0649}
\pdfglyphtounicode{afii57450}{064A}
\pdfglyphtounicode{afii57451}{064B}
\pdfglyphtounicode{afii57452}{064C}
\pdfglyphtounicode{afii57453}{064D}
\pdfglyphtounicode{afii57454}{064E}
\pdfglyphtounicode{afii57455}{064F}
\pdfglyphtounicode{afii57456}{0650}
\pdfglyphtounicode{afii57457}{0651}
\pdfglyphtounicode{afii57458}{0652}
\pdfglyphtounicode{afii57470}{0647}
\pdfglyphtounicode{afii57505}{06A4}
\pdfglyphtounicode{afii57506}{067E}
\pdfglyphtounicode{afii57507}{0686}
\pdfglyphtounicode{afii57508}{0698}
\pdfglyphtounicode{afii57509}{06AF}
\pdfglyphtounicode{afii57511}{0679}
\pdfglyphtounicode{afii57512}{0688}
\pdfglyphtounicode{afii57513}{0691}
\pdfglyphtounicode{afii57514}{06BA}
\pdfglyphtounicode{afii57519}{06D2}
\pdfglyphtounicode{afii57534}{06D5}
\pdfglyphtounicode{afii57636}{20AA}
\pdfglyphtounicode{afii57645}{05BE}
\pdfglyphtounicode{afii57658}{05C3}
\pdfglyphtounicode{afii57664}{05D0}
\pdfglyphtounicode{afii57665}{05D1}
\pdfglyphtounicode{afii57666}{05D2}
\pdfglyphtounicode{afii57667}{05D3}
\pdfglyphtounicode{afii57668}{05D4}
\pdfglyphtounicode{afii57669}{05D5}
\pdfglyphtounicode{afii57670}{05D6}
\pdfglyphtounicode{afii57671}{05D7}
\pdfglyphtounicode{afii57672}{05D8}
\pdfglyphtounicode{afii57673}{05D9}
\pdfglyphtounicode{afii57674}{05DA}
\pdfglyphtounicode{afii57675}{05DB}
\pdfglyphtounicode{afii57676}{05DC}
\pdfglyphtounicode{afii57677}{05DD}
\pdfglyphtounicode{afii57678}{05DE}
\pdfglyphtounicode{afii57679}{05DF}
\pdfglyphtounicode{afii57680}{05E0}
\pdfglyphtounicode{afii57681}{05E1}
\pdfglyphtounicode{afii57682}{05E2}
\pdfglyphtounicode{afii57683}{05E3}
\pdfglyphtounicode{afii57684}{05E4}
\pdfglyphtounicode{afii57685}{05E5}
\pdfglyphtounicode{afii57686}{05E6}
\pdfglyphtounicode{afii57687}{05E7}
\pdfglyphtounicode{afii57688}{05E8}
\pdfglyphtounicode{afii57689}{05E9}
\pdfglyphtounicode{afii57690}{05EA}
\pdfglyphtounicode{afii57694}{FB2A}
\pdfglyphtounicode{afii57695}{FB2B}
\pdfglyphtounicode{afii57700}{FB4B}
\pdfglyphtounicode{afii57705}{FB1F}
\pdfglyphtounicode{afii57716}{05F0}
\pdfglyphtounicode{afii57717}{05F1}
\pdfglyphtounicode{afii57718}{05F2}
\pdfglyphtounicode{afii57723}{FB35}
\pdfglyphtounicode{afii57793}{05B4}
\pdfglyphtounicode{afii57794}{05B5}
\pdfglyphtounicode{afii57795}{05B6}
\pdfglyphtounicode{afii57796}{05BB}
\pdfglyphtounicode{afii57797}{05B8}
\pdfglyphtounicode{afii57798}{05B7}
\pdfglyphtounicode{afii57799}{05B0}
\pdfglyphtounicode{afii57800}{05B2}
\pdfglyphtounicode{afii57801}{05B1}
\pdfglyphtounicode{afii57802}{05B3}
\pdfglyphtounicode{afii57803}{05C2}
\pdfglyphtounicode{afii57804}{05C1}
\pdfglyphtounicode{afii57806}{05B9}
\pdfglyphtounicode{afii57807}{05BC}
\pdfglyphtounicode{afii57839}{05BD}
\pdfglyphtounicode{afii57841}{05BF}
\pdfglyphtounicode{afii57842}{05C0}
\pdfglyphtounicode{afii57929}{02BC}
\pdfglyphtounicode{afii61248}{2105}
\pdfglyphtounicode{afii61289}{2113}
\pdfglyphtounicode{afii61352}{2116}
\pdfglyphtounicode{afii61573}{202C}
\pdfglyphtounicode{afii61574}{202D}
\pdfglyphtounicode{afii61575}{202E}
\pdfglyphtounicode{afii61664}{200C}
\pdfglyphtounicode{afii63167}{066D}
\pdfglyphtounicode{afii64937}{02BD}
\pdfglyphtounicode{agrave}{00E0}
\pdfglyphtounicode{agujarati}{0A85}
\pdfglyphtounicode{agurmukhi}{0A05}
\pdfglyphtounicode{ahiragana}{3042}
\pdfglyphtounicode{ahookabove}{1EA3}
\pdfglyphtounicode{aibengali}{0990}
\pdfglyphtounicode{aibopomofo}{311E}
\pdfglyphtounicode{aideva}{0910}
\pdfglyphtounicode{aiecyrillic}{04D5}
\pdfglyphtounicode{aigujarati}{0A90}
\pdfglyphtounicode{aigurmukhi}{0A10}
\pdfglyphtounicode{aimatragurmukhi}{0A48}
\pdfglyphtounicode{ainarabic}{0639}
\pdfglyphtounicode{ainfinalarabic}{FECA}
\pdfglyphtounicode{aininitialarabic}{FECB}
\pdfglyphtounicode{ainmedialarabic}{FECC}
\pdfglyphtounicode{ainvertedbreve}{0203}
\pdfglyphtounicode{aivowelsignbengali}{09C8}
\pdfglyphtounicode{aivowelsigndeva}{0948}
\pdfglyphtounicode{aivowelsigngujarati}{0AC8}
\pdfglyphtounicode{akatakana}{30A2}
\pdfglyphtounicode{akatakanahalfwidth}{FF71}
\pdfglyphtounicode{akorean}{314F}
\pdfglyphtounicode{alef}{05D0}
\pdfglyphtounicode{alefarabic}{0627}
\pdfglyphtounicode{alefdageshhebrew}{FB30}
\pdfglyphtounicode{aleffinalarabic}{FE8E}
\pdfglyphtounicode{alefhamzaabovearabic}{0623}
\pdfglyphtounicode{alefhamzaabovefinalarabic}{FE84}
\pdfglyphtounicode{alefhamzabelowarabic}{0625}
\pdfglyphtounicode{alefhamzabelowfinalarabic}{FE88}
\pdfglyphtounicode{alefhebrew}{05D0}
\pdfglyphtounicode{aleflamedhebrew}{FB4F}
\pdfglyphtounicode{alefmaddaabovearabic}{0622}
\pdfglyphtounicode{alefmaddaabovefinalarabic}{FE82}
\pdfglyphtounicode{alefmaksuraarabic}{0649}
\pdfglyphtounicode{alefmaksurafinalarabic}{FEF0}
\pdfglyphtounicode{alefmaksurainitialarabic}{FEF3}
\pdfglyphtounicode{alefmaksuramedialarabic}{FEF4}
\pdfglyphtounicode{alefpatahhebrew}{FB2E}
\pdfglyphtounicode{alefqamatshebrew}{FB2F}
\pdfglyphtounicode{aleph}{2135}
\pdfglyphtounicode{allequal}{224C}
\pdfglyphtounicode{alpha}{03B1}
\pdfglyphtounicode{alphatonos}{03AC}
\pdfglyphtounicode{amacron}{0101}
\pdfglyphtounicode{amonospace}{FF41}
\pdfglyphtounicode{ampersand}{0026}
\pdfglyphtounicode{ampersandmonospace}{FF06}
\pdfglyphtounicode{ampersandsmall}{F726}
\pdfglyphtounicode{amsquare}{33C2}
\pdfglyphtounicode{anbopomofo}{3122}
\pdfglyphtounicode{angbopomofo}{3124}
\pdfglyphtounicode{angkhankhuthai}{0E5A}
\pdfglyphtounicode{angle}{2220}
\pdfglyphtounicode{anglebracketleft}{3008}
\pdfglyphtounicode{anglebracketleftvertical}{FE3F}
\pdfglyphtounicode{anglebracketright}{3009}
\pdfglyphtounicode{anglebracketrightvertical}{FE40}
\pdfglyphtounicode{angleleft}{2329}
\pdfglyphtounicode{angleright}{232A}
\pdfglyphtounicode{angstrom}{212B}
\pdfglyphtounicode{anoteleia}{0387}
\pdfglyphtounicode{anudattadeva}{0952}
\pdfglyphtounicode{anusvarabengali}{0982}
\pdfglyphtounicode{anusvaradeva}{0902}
\pdfglyphtounicode{anusvaragujarati}{0A82}
\pdfglyphtounicode{aogonek}{0105}
\pdfglyphtounicode{apaatosquare}{3300}
\pdfglyphtounicode{aparen}{249C}
\pdfglyphtounicode{apostrophearmenian}{055A}
\pdfglyphtounicode{apostrophemod}{02BC}
\pdfglyphtounicode{apple}{F8FF}
\pdfglyphtounicode{approaches}{2250}
\pdfglyphtounicode{approxequal}{2248}
\pdfglyphtounicode{approxequalorimage}{2252}
\pdfglyphtounicode{approximatelyequal}{2245}
\pdfglyphtounicode{araeaekorean}{318E}
\pdfglyphtounicode{araeakorean}{318D}
\pdfglyphtounicode{arc}{2312}
\pdfglyphtounicode{arighthalfring}{1E9A}
\pdfglyphtounicode{aring}{00E5}
\pdfglyphtounicode{aringacute}{01FB}
\pdfglyphtounicode{aringbelow}{1E01}
\pdfglyphtounicode{arrowboth}{2194}
\pdfglyphtounicode{arrowdashdown}{21E3}
\pdfglyphtounicode{arrowdashleft}{21E0}
\pdfglyphtounicode{arrowdashright}{21E2}
\pdfglyphtounicode{arrowdashup}{21E1}
\pdfglyphtounicode{arrowdblboth}{21D4}
\pdfglyphtounicode{arrowdbldown}{21D3}
\pdfglyphtounicode{arrowdblleft}{21D0}
\pdfglyphtounicode{arrowdblright}{21D2}
\pdfglyphtounicode{arrowdblup}{21D1}
\pdfglyphtounicode{arrowdown}{2193}
\pdfglyphtounicode{arrowdownleft}{2199}
\pdfglyphtounicode{arrowdownright}{2198}
\pdfglyphtounicode{arrowdownwhite}{21E9}
\pdfglyphtounicode{arrowheaddownmod}{02C5}
\pdfglyphtounicode{arrowheadleftmod}{02C2}
\pdfglyphtounicode{arrowheadrightmod}{02C3}
\pdfglyphtounicode{arrowheadupmod}{02C4}
\pdfglyphtounicode{arrowhorizex}{F8E7}
\pdfglyphtounicode{arrowleft}{2190}
\pdfglyphtounicode{arrowleftdbl}{21D0}
\pdfglyphtounicode{arrowleftdblstroke}{21CD}
\pdfglyphtounicode{arrowleftoverright}{21C6}
\pdfglyphtounicode{arrowleftwhite}{21E6}
\pdfglyphtounicode{arrowright}{2192}
\pdfglyphtounicode{arrowrightdblstroke}{21CF}
\pdfglyphtounicode{arrowrightheavy}{279E}
\pdfglyphtounicode{arrowrightoverleft}{21C4}
\pdfglyphtounicode{arrowrightwhite}{21E8}
\pdfglyphtounicode{arrowtableft}{21E4}
\pdfglyphtounicode{arrowtabright}{21E5}
\pdfglyphtounicode{arrowup}{2191}
\pdfglyphtounicode{arrowupdn}{2195}
\pdfglyphtounicode{arrowupdnbse}{21A8}
\pdfglyphtounicode{arrowupdownbase}{21A8}
\pdfglyphtounicode{arrowupleft}{2196}
\pdfglyphtounicode{arrowupleftofdown}{21C5}
\pdfglyphtounicode{arrowupright}{2197}
\pdfglyphtounicode{arrowupwhite}{21E7}
\pdfglyphtounicode{arrowvertex}{F8E6}
\pdfglyphtounicode{asciicircum}{005E}
\pdfglyphtounicode{asciicircummonospace}{FF3E}
\pdfglyphtounicode{asciitilde}{007E}
\pdfglyphtounicode{asciitildemonospace}{FF5E}
\pdfglyphtounicode{ascript}{0251}
\pdfglyphtounicode{ascriptturned}{0252}
\pdfglyphtounicode{asmallhiragana}{3041}
\pdfglyphtounicode{asmallkatakana}{30A1}
\pdfglyphtounicode{asmallkatakanahalfwidth}{FF67}
\pdfglyphtounicode{asterisk}{002A}
\pdfglyphtounicode{asteriskaltonearabic}{066D}
\pdfglyphtounicode{asteriskarabic}{066D}
\pdfglyphtounicode{asteriskmath}{2217}
\pdfglyphtounicode{asteriskmonospace}{FF0A}
\pdfglyphtounicode{asterisksmall}{FE61}
\pdfglyphtounicode{asterism}{2042}
\pdfglyphtounicode{asuperior}{F6E9}
\pdfglyphtounicode{asymptoticallyequal}{2243}
\pdfglyphtounicode{at}{0040}
\pdfglyphtounicode{atilde}{00E3}
\pdfglyphtounicode{atmonospace}{FF20}
\pdfglyphtounicode{atsmall}{FE6B}
\pdfglyphtounicode{aturned}{0250}
\pdfglyphtounicode{aubengali}{0994}
\pdfglyphtounicode{aubopomofo}{3120}
\pdfglyphtounicode{audeva}{0914}
\pdfglyphtounicode{augujarati}{0A94}
\pdfglyphtounicode{augurmukhi}{0A14}
\pdfglyphtounicode{aulengthmarkbengali}{09D7}
\pdfglyphtounicode{aumatragurmukhi}{0A4C}
\pdfglyphtounicode{auvowelsignbengali}{09CC}
\pdfglyphtounicode{auvowelsigndeva}{094C}
\pdfglyphtounicode{auvowelsigngujarati}{0ACC}
\pdfglyphtounicode{avagrahadeva}{093D}
\pdfglyphtounicode{aybarmenian}{0561}
\pdfglyphtounicode{ayin}{05E2}
\pdfglyphtounicode{ayinaltonehebrew}{FB20}
\pdfglyphtounicode{ayinhebrew}{05E2}
\pdfglyphtounicode{b}{0062}
\pdfglyphtounicode{babengali}{09AC}
\pdfglyphtounicode{backslash}{005C}
\pdfglyphtounicode{backslashmonospace}{FF3C}
\pdfglyphtounicode{badeva}{092C}
\pdfglyphtounicode{bagujarati}{0AAC}
\pdfglyphtounicode{bagurmukhi}{0A2C}
\pdfglyphtounicode{bahiragana}{3070}
\pdfglyphtounicode{bahtthai}{0E3F}
\pdfglyphtounicode{bakatakana}{30D0}
\pdfglyphtounicode{bar}{007C}
\pdfglyphtounicode{barmonospace}{FF5C}
\pdfglyphtounicode{bbopomofo}{3105}
\pdfglyphtounicode{bcircle}{24D1}
\pdfglyphtounicode{bdotaccent}{1E03}
\pdfglyphtounicode{bdotbelow}{1E05}
\pdfglyphtounicode{beamedsixteenthnotes}{266C}
\pdfglyphtounicode{because}{2235}
\pdfglyphtounicode{becyrillic}{0431}
\pdfglyphtounicode{beharabic}{0628}
\pdfglyphtounicode{behfinalarabic}{FE90}
\pdfglyphtounicode{behinitialarabic}{FE91}
\pdfglyphtounicode{behiragana}{3079}
\pdfglyphtounicode{behmedialarabic}{FE92}
\pdfglyphtounicode{behmeeminitialarabic}{FC9F}
\pdfglyphtounicode{behmeemisolatedarabic}{FC08}
\pdfglyphtounicode{behnoonfinalarabic}{FC6D}
\pdfglyphtounicode{bekatakana}{30D9}
\pdfglyphtounicode{benarmenian}{0562}
\pdfglyphtounicode{bet}{05D1}
\pdfglyphtounicode{beta}{03B2}
\pdfglyphtounicode{betasymbolgreek}{03D0}
\pdfglyphtounicode{betdagesh}{FB31}
\pdfglyphtounicode{betdageshhebrew}{FB31}
\pdfglyphtounicode{bethebrew}{05D1}
\pdfglyphtounicode{betrafehebrew}{FB4C}
\pdfglyphtounicode{bhabengali}{09AD}
\pdfglyphtounicode{bhadeva}{092D}
\pdfglyphtounicode{bhagujarati}{0AAD}
\pdfglyphtounicode{bhagurmukhi}{0A2D}
\pdfglyphtounicode{bhook}{0253}
\pdfglyphtounicode{bihiragana}{3073}
\pdfglyphtounicode{bikatakana}{30D3}
\pdfglyphtounicode{bilabialclick}{0298}
\pdfglyphtounicode{bindigurmukhi}{0A02}
\pdfglyphtounicode{birusquare}{3331}
\pdfglyphtounicode{blackcircle}{25CF}
\pdfglyphtounicode{blackdiamond}{25C6}
\pdfglyphtounicode{blackdownpointingtriangle}{25BC}
\pdfglyphtounicode{blackleftpointingpointer}{25C4}
\pdfglyphtounicode{blackleftpointingtriangle}{25C0}
\pdfglyphtounicode{blacklenticularbracketleft}{3010}
\pdfglyphtounicode{blacklenticularbracketleftvertical}{FE3B}
\pdfglyphtounicode{blacklenticularbracketright}{3011}
\pdfglyphtounicode{blacklenticularbracketrightvertical}{FE3C}
\pdfglyphtounicode{blacklowerlefttriangle}{25E3}
\pdfglyphtounicode{blacklowerrighttriangle}{25E2}
\pdfglyphtounicode{blackrectangle}{25AC}
\pdfglyphtounicode{blackrightpointingpointer}{25BA}
\pdfglyphtounicode{blackrightpointingtriangle}{25B6}
\pdfglyphtounicode{blacksmallsquare}{25AA}
\pdfglyphtounicode{blacksmilingface}{263B}
\pdfglyphtounicode{blacksquare}{25A0}
\pdfglyphtounicode{blackstar}{2605}
\pdfglyphtounicode{blackupperlefttriangle}{25E4}
\pdfglyphtounicode{blackupperrighttriangle}{25E5}
\pdfglyphtounicode{blackuppointingsmalltriangle}{25B4}
\pdfglyphtounicode{blackuppointingtriangle}{25B2}
\pdfglyphtounicode{blank}{2423}
\pdfglyphtounicode{blinebelow}{1E07}
\pdfglyphtounicode{block}{2588}
\pdfglyphtounicode{bmonospace}{FF42}
\pdfglyphtounicode{bobaimaithai}{0E1A}
\pdfglyphtounicode{bohiragana}{307C}
\pdfglyphtounicode{bokatakana}{30DC}
\pdfglyphtounicode{bparen}{249D}
\pdfglyphtounicode{bqsquare}{33C3}
\pdfglyphtounicode{braceex}{F8F4}
\pdfglyphtounicode{braceleft}{007B}
\pdfglyphtounicode{braceleftbt}{F8F3}
\pdfglyphtounicode{braceleftmid}{F8F2}
\pdfglyphtounicode{braceleftmonospace}{FF5B}
\pdfglyphtounicode{braceleftsmall}{FE5B}
\pdfglyphtounicode{bracelefttp}{F8F1}
\pdfglyphtounicode{braceleftvertical}{FE37}
\pdfglyphtounicode{braceright}{007D}
\pdfglyphtounicode{bracerightbt}{F8FE}
\pdfglyphtounicode{bracerightmid}{F8FD}
\pdfglyphtounicode{bracerightmonospace}{FF5D}
\pdfglyphtounicode{bracerightsmall}{FE5C}
\pdfglyphtounicode{bracerighttp}{F8FC}
\pdfglyphtounicode{bracerightvertical}{FE38}
\pdfglyphtounicode{bracketleft}{005B}
\pdfglyphtounicode{bracketleftbt}{F8F0}
\pdfglyphtounicode{bracketleftex}{F8EF}
\pdfglyphtounicode{bracketleftmonospace}{FF3B}
\pdfglyphtounicode{bracketlefttp}{F8EE}
\pdfglyphtounicode{bracketright}{005D}
\pdfglyphtounicode{bracketrightbt}{F8FB}
\pdfglyphtounicode{bracketrightex}{F8FA}
\pdfglyphtounicode{bracketrightmonospace}{FF3D}
\pdfglyphtounicode{bracketrighttp}{F8F9}
\pdfglyphtounicode{breve}{02D8}
\pdfglyphtounicode{brevebelowcmb}{032E}
\pdfglyphtounicode{brevecmb}{0306}
\pdfglyphtounicode{breveinvertedbelowcmb}{032F}
\pdfglyphtounicode{breveinvertedcmb}{0311}
\pdfglyphtounicode{breveinverteddoublecmb}{0361}
\pdfglyphtounicode{bridgebelowcmb}{032A}
\pdfglyphtounicode{bridgeinvertedbelowcmb}{033A}
\pdfglyphtounicode{brokenbar}{00A6}
\pdfglyphtounicode{bstroke}{0180}
\pdfglyphtounicode{bsuperior}{F6EA}
\pdfglyphtounicode{btopbar}{0183}
\pdfglyphtounicode{buhiragana}{3076}
\pdfglyphtounicode{bukatakana}{30D6}
\pdfglyphtounicode{bullet}{2022}
\pdfglyphtounicode{bulletinverse}{25D8}
\pdfglyphtounicode{bulletoperator}{2219}
\pdfglyphtounicode{bullseye}{25CE}
\pdfglyphtounicode{c}{0063}
\pdfglyphtounicode{caarmenian}{056E}
\pdfglyphtounicode{cabengali}{099A}
\pdfglyphtounicode{cacute}{0107}
\pdfglyphtounicode{cadeva}{091A}
\pdfglyphtounicode{cagujarati}{0A9A}
\pdfglyphtounicode{cagurmukhi}{0A1A}
\pdfglyphtounicode{calsquare}{3388}
\pdfglyphtounicode{candrabindubengali}{0981}
\pdfglyphtounicode{candrabinducmb}{0310}
\pdfglyphtounicode{candrabindudeva}{0901}
\pdfglyphtounicode{candrabindugujarati}{0A81}
\pdfglyphtounicode{capslock}{21EA}
\pdfglyphtounicode{careof}{2105}
\pdfglyphtounicode{caron}{02C7}
\pdfglyphtounicode{caronbelowcmb}{032C}
\pdfglyphtounicode{caroncmb}{030C}
\pdfglyphtounicode{carriagereturn}{21B5}
\pdfglyphtounicode{cbopomofo}{3118}
\pdfglyphtounicode{ccaron}{010D}
\pdfglyphtounicode{ccedilla}{00E7}
\pdfglyphtounicode{ccedillaacute}{1E09}
\pdfglyphtounicode{ccircle}{24D2}
\pdfglyphtounicode{ccircumflex}{0109}
\pdfglyphtounicode{ccurl}{0255}
\pdfglyphtounicode{cdot}{010B}
\pdfglyphtounicode{cdotaccent}{010B}
\pdfglyphtounicode{cdsquare}{33C5}
\pdfglyphtounicode{cedilla}{00B8}
\pdfglyphtounicode{cedillacmb}{0327}
\pdfglyphtounicode{cent}{00A2}
\pdfglyphtounicode{centigrade}{2103}
\pdfglyphtounicode{centinferior}{F6DF}
\pdfglyphtounicode{centmonospace}{FFE0}
\pdfglyphtounicode{centoldstyle}{F7A2}
\pdfglyphtounicode{centsuperior}{F6E0}
\pdfglyphtounicode{chaarmenian}{0579}
\pdfglyphtounicode{chabengali}{099B}
\pdfglyphtounicode{chadeva}{091B}
\pdfglyphtounicode{chagujarati}{0A9B}
\pdfglyphtounicode{chagurmukhi}{0A1B}
\pdfglyphtounicode{chbopomofo}{3114}
\pdfglyphtounicode{cheabkhasiancyrillic}{04BD}
\pdfglyphtounicode{checkmark}{2713}
\pdfglyphtounicode{checyrillic}{0447}
\pdfglyphtounicode{chedescenderabkhasiancyrillic}{04BF}
\pdfglyphtounicode{chedescendercyrillic}{04B7}
\pdfglyphtounicode{chedieresiscyrillic}{04F5}
\pdfglyphtounicode{cheharmenian}{0573}
\pdfglyphtounicode{chekhakassiancyrillic}{04CC}
\pdfglyphtounicode{cheverticalstrokecyrillic}{04B9}
\pdfglyphtounicode{chi}{03C7}
\pdfglyphtounicode{chieuchacirclekorean}{3277}
\pdfglyphtounicode{chieuchaparenkorean}{3217}
\pdfglyphtounicode{chieuchcirclekorean}{3269}
\pdfglyphtounicode{chieuchkorean}{314A}
\pdfglyphtounicode{chieuchparenkorean}{3209}
\pdfglyphtounicode{chochangthai}{0E0A}
\pdfglyphtounicode{chochanthai}{0E08}
\pdfglyphtounicode{chochingthai}{0E09}
\pdfglyphtounicode{chochoethai}{0E0C}
\pdfglyphtounicode{chook}{0188}
\pdfglyphtounicode{cieucacirclekorean}{3276}
\pdfglyphtounicode{cieucaparenkorean}{3216}
\pdfglyphtounicode{cieuccirclekorean}{3268}
\pdfglyphtounicode{cieuckorean}{3148}
\pdfglyphtounicode{cieucparenkorean}{3208}
\pdfglyphtounicode{cieucuparenkorean}{321C}
\pdfglyphtounicode{circle}{25CB}
\pdfglyphtounicode{circlemultiply}{2297}
\pdfglyphtounicode{circleot}{2299}
\pdfglyphtounicode{circleplus}{2295}
\pdfglyphtounicode{circlepostalmark}{3036}
\pdfglyphtounicode{circlewithlefthalfblack}{25D0}
\pdfglyphtounicode{circlewithrighthalfblack}{25D1}
\pdfglyphtounicode{circumflex}{02C6}
\pdfglyphtounicode{circumflexbelowcmb}{032D}
\pdfglyphtounicode{circumflexcmb}{0302}
\pdfglyphtounicode{clear}{2327}
\pdfglyphtounicode{clickalveolar}{01C2}
\pdfglyphtounicode{clickdental}{01C0}
\pdfglyphtounicode{clicklateral}{01C1}
\pdfglyphtounicode{clickretroflex}{01C3}
\pdfglyphtounicode{club}{2663}
\pdfglyphtounicode{clubsuitblack}{2663}
\pdfglyphtounicode{clubsuitwhite}{2667}
\pdfglyphtounicode{cmcubedsquare}{33A4}
\pdfglyphtounicode{cmonospace}{FF43}
\pdfglyphtounicode{cmsquaredsquare}{33A0}
\pdfglyphtounicode{coarmenian}{0581}
\pdfglyphtounicode{colon}{003A}
\pdfglyphtounicode{colonmonetary}{20A1}
\pdfglyphtounicode{colonmonospace}{FF1A}
\pdfglyphtounicode{colonsign}{20A1}
\pdfglyphtounicode{colonsmall}{FE55}
\pdfglyphtounicode{colontriangularhalfmod}{02D1}
\pdfglyphtounicode{colontriangularmod}{02D0}
\pdfglyphtounicode{comma}{002C}
\pdfglyphtounicode{commaabovecmb}{0313}
\pdfglyphtounicode{commaaboverightcmb}{0315}
\pdfglyphtounicode{commaaccent}{F6C3}
\pdfglyphtounicode{commaarabic}{060C}
\pdfglyphtounicode{commaarmenian}{055D}
\pdfglyphtounicode{commainferior}{F6E1}
\pdfglyphtounicode{commamonospace}{FF0C}
\pdfglyphtounicode{commareversedabovecmb}{0314}
\pdfglyphtounicode{commareversedmod}{02BD}
\pdfglyphtounicode{commasmall}{FE50}
\pdfglyphtounicode{commasuperior}{F6E2}
\pdfglyphtounicode{commaturnedabovecmb}{0312}
\pdfglyphtounicode{commaturnedmod}{02BB}
\pdfglyphtounicode{compass}{263C}
\pdfglyphtounicode{congruent}{2245}
\pdfglyphtounicode{contourintegral}{222E}
\pdfglyphtounicode{control}{2303}
\pdfglyphtounicode{controlACK}{0006}
\pdfglyphtounicode{controlBEL}{0007}
\pdfglyphtounicode{controlBS}{0008}
\pdfglyphtounicode{controlCAN}{0018}
\pdfglyphtounicode{controlCR}{000D}
\pdfglyphtounicode{controlDC1}{0011}
\pdfglyphtounicode{controlDC2}{0012}
\pdfglyphtounicode{controlDC3}{0013}
\pdfglyphtounicode{controlDC4}{0014}
\pdfglyphtounicode{controlDEL}{007F}
\pdfglyphtounicode{controlDLE}{0010}
\pdfglyphtounicode{controlEM}{0019}
\pdfglyphtounicode{controlENQ}{0005}
\pdfglyphtounicode{controlEOT}{0004}
\pdfglyphtounicode{controlESC}{001B}
\pdfglyphtounicode{controlETB}{0017}
\pdfglyphtounicode{controlETX}{0003}
\pdfglyphtounicode{controlFF}{000C}
\pdfglyphtounicode{controlFS}{001C}
\pdfglyphtounicode{controlGS}{001D}
\pdfglyphtounicode{controlHT}{0009}
\pdfglyphtounicode{controlLF}{000A}
\pdfglyphtounicode{controlNAK}{0015}
\pdfglyphtounicode{controlRS}{001E}
\pdfglyphtounicode{controlSI}{000F}
\pdfglyphtounicode{controlSO}{000E}
\pdfglyphtounicode{controlSOT}{0002}
\pdfglyphtounicode{controlSTX}{0001}
\pdfglyphtounicode{controlSUB}{001A}
\pdfglyphtounicode{controlSYN}{0016}
\pdfglyphtounicode{controlUS}{001F}
\pdfglyphtounicode{controlVT}{000B}
\pdfglyphtounicode{copyright}{00A9}
\pdfglyphtounicode{copyrightsans}{F8E9}
\pdfglyphtounicode{copyrightserif}{F6D9}
\pdfglyphtounicode{cornerbracketleft}{300C}
\pdfglyphtounicode{cornerbracketlefthalfwidth}{FF62}
\pdfglyphtounicode{cornerbracketleftvertical}{FE41}
\pdfglyphtounicode{cornerbracketright}{300D}
\pdfglyphtounicode{cornerbracketrighthalfwidth}{FF63}
\pdfglyphtounicode{cornerbracketrightvertical}{FE42}
\pdfglyphtounicode{corporationsquare}{337F}
\pdfglyphtounicode{cosquare}{33C7}
\pdfglyphtounicode{coverkgsquare}{33C6}
\pdfglyphtounicode{cparen}{249E}
\pdfglyphtounicode{cruzeiro}{20A2}
\pdfglyphtounicode{cstretched}{0297}
\pdfglyphtounicode{curlyand}{22CF}
\pdfglyphtounicode{curlyor}{22CE}
\pdfglyphtounicode{currency}{00A4}
\pdfglyphtounicode{cyrBreve}{F6D1}
\pdfglyphtounicode{cyrFlex}{F6D2}
\pdfglyphtounicode{cyrbreve}{F6D4}
\pdfglyphtounicode{cyrflex}{F6D5}
\pdfglyphtounicode{d}{0064}
\pdfglyphtounicode{daarmenian}{0564}
\pdfglyphtounicode{dabengali}{09A6}
\pdfglyphtounicode{dadarabic}{0636}
\pdfglyphtounicode{dadeva}{0926}
\pdfglyphtounicode{dadfinalarabic}{FEBE}
\pdfglyphtounicode{dadinitialarabic}{FEBF}
\pdfglyphtounicode{dadmedialarabic}{FEC0}
\pdfglyphtounicode{dagesh}{05BC}
\pdfglyphtounicode{dageshhebrew}{05BC}
\pdfglyphtounicode{dagger}{2020}
\pdfglyphtounicode{daggerdbl}{2021}
\pdfglyphtounicode{dagujarati}{0AA6}
\pdfglyphtounicode{dagurmukhi}{0A26}
\pdfglyphtounicode{dahiragana}{3060}
\pdfglyphtounicode{dakatakana}{30C0}
\pdfglyphtounicode{dalarabic}{062F}
\pdfglyphtounicode{dalet}{05D3}
\pdfglyphtounicode{daletdagesh}{FB33}
\pdfglyphtounicode{daletdageshhebrew}{FB33}
% dalethatafpatah;05D3 05B2
% dalethatafpatahhebrew;05D3 05B2
% dalethatafsegol;05D3 05B1
% dalethatafsegolhebrew;05D3 05B1
\pdfglyphtounicode{dalethebrew}{05D3}
% dalethiriq;05D3 05B4
% dalethiriqhebrew;05D3 05B4
% daletholam;05D3 05B9
% daletholamhebrew;05D3 05B9
% daletpatah;05D3 05B7
% daletpatahhebrew;05D3 05B7
% daletqamats;05D3 05B8
% daletqamatshebrew;05D3 05B8
% daletqubuts;05D3 05BB
% daletqubutshebrew;05D3 05BB
% daletsegol;05D3 05B6
% daletsegolhebrew;05D3 05B6
% daletsheva;05D3 05B0
% daletshevahebrew;05D3 05B0
% dalettsere;05D3 05B5
% dalettserehebrew;05D3 05B5
\pdfglyphtounicode{dalfinalarabic}{FEAA}
\pdfglyphtounicode{dammaarabic}{064F}
\pdfglyphtounicode{dammalowarabic}{064F}
\pdfglyphtounicode{dammatanaltonearabic}{064C}
\pdfglyphtounicode{dammatanarabic}{064C}
\pdfglyphtounicode{danda}{0964}
\pdfglyphtounicode{dargahebrew}{05A7}
\pdfglyphtounicode{dargalefthebrew}{05A7}
\pdfglyphtounicode{dasiapneumatacyrilliccmb}{0485}
\pdfglyphtounicode{dblGrave}{F6D3}
\pdfglyphtounicode{dblanglebracketleft}{300A}
\pdfglyphtounicode{dblanglebracketleftvertical}{FE3D}
\pdfglyphtounicode{dblanglebracketright}{300B}
\pdfglyphtounicode{dblanglebracketrightvertical}{FE3E}
\pdfglyphtounicode{dblarchinvertedbelowcmb}{032B}
\pdfglyphtounicode{dblarrowleft}{21D4}
\pdfglyphtounicode{dblarrowright}{21D2}
\pdfglyphtounicode{dbldanda}{0965}
\pdfglyphtounicode{dblgrave}{F6D6}
\pdfglyphtounicode{dblgravecmb}{030F}
\pdfglyphtounicode{dblintegral}{222C}
\pdfglyphtounicode{dbllowline}{2017}
\pdfglyphtounicode{dbllowlinecmb}{0333}
\pdfglyphtounicode{dbloverlinecmb}{033F}
\pdfglyphtounicode{dblprimemod}{02BA}
\pdfglyphtounicode{dblverticalbar}{2016}
\pdfglyphtounicode{dblverticallineabovecmb}{030E}
\pdfglyphtounicode{dbopomofo}{3109}
\pdfglyphtounicode{dbsquare}{33C8}
\pdfglyphtounicode{dcaron}{010F}
\pdfglyphtounicode{dcedilla}{1E11}
\pdfglyphtounicode{dcircle}{24D3}
\pdfglyphtounicode{dcircumflexbelow}{1E13}
\pdfglyphtounicode{dcroat}{0111}
\pdfglyphtounicode{ddabengali}{09A1}
\pdfglyphtounicode{ddadeva}{0921}
\pdfglyphtounicode{ddagujarati}{0AA1}
\pdfglyphtounicode{ddagurmukhi}{0A21}
\pdfglyphtounicode{ddalarabic}{0688}
\pdfglyphtounicode{ddalfinalarabic}{FB89}
\pdfglyphtounicode{dddhadeva}{095C}
\pdfglyphtounicode{ddhabengali}{09A2}
\pdfglyphtounicode{ddhadeva}{0922}
\pdfglyphtounicode{ddhagujarati}{0AA2}
\pdfglyphtounicode{ddhagurmukhi}{0A22}
\pdfglyphtounicode{ddotaccent}{1E0B}
\pdfglyphtounicode{ddotbelow}{1E0D}
\pdfglyphtounicode{decimalseparatorarabic}{066B}
\pdfglyphtounicode{decimalseparatorpersian}{066B}
\pdfglyphtounicode{decyrillic}{0434}
\pdfglyphtounicode{degree}{00B0}
\pdfglyphtounicode{dehihebrew}{05AD}
\pdfglyphtounicode{dehiragana}{3067}
\pdfglyphtounicode{deicoptic}{03EF}
\pdfglyphtounicode{dekatakana}{30C7}
\pdfglyphtounicode{deleteleft}{232B}
\pdfglyphtounicode{deleteright}{2326}
\pdfglyphtounicode{delta}{03B4}
\pdfglyphtounicode{deltaturned}{018D}
\pdfglyphtounicode{denominatorminusonenumeratorbengali}{09F8}
\pdfglyphtounicode{dezh}{02A4}
\pdfglyphtounicode{dhabengali}{09A7}
\pdfglyphtounicode{dhadeva}{0927}
\pdfglyphtounicode{dhagujarati}{0AA7}
\pdfglyphtounicode{dhagurmukhi}{0A27}
\pdfglyphtounicode{dhook}{0257}
\pdfglyphtounicode{dialytikatonos}{0385}
\pdfglyphtounicode{dialytikatonoscmb}{0344}
\pdfglyphtounicode{diamond}{2666}
\pdfglyphtounicode{diamondsuitwhite}{2662}
\pdfglyphtounicode{dieresis}{00A8}
\pdfglyphtounicode{dieresisacute}{F6D7}
\pdfglyphtounicode{dieresisbelowcmb}{0324}
\pdfglyphtounicode{dieresiscmb}{0308}
\pdfglyphtounicode{dieresisgrave}{F6D8}
\pdfglyphtounicode{dieresistonos}{0385}
\pdfglyphtounicode{dihiragana}{3062}
\pdfglyphtounicode{dikatakana}{30C2}
\pdfglyphtounicode{dittomark}{3003}
\pdfglyphtounicode{divide}{00F7}
\pdfglyphtounicode{divides}{2223}
\pdfglyphtounicode{divisionslash}{2215}
\pdfglyphtounicode{djecyrillic}{0452}
\pdfglyphtounicode{dkshade}{2593}
\pdfglyphtounicode{dlinebelow}{1E0F}
\pdfglyphtounicode{dlsquare}{3397}
\pdfglyphtounicode{dmacron}{0111}
\pdfglyphtounicode{dmonospace}{FF44}
\pdfglyphtounicode{dnblock}{2584}
\pdfglyphtounicode{dochadathai}{0E0E}
\pdfglyphtounicode{dodekthai}{0E14}
\pdfglyphtounicode{dohiragana}{3069}
\pdfglyphtounicode{dokatakana}{30C9}
\pdfglyphtounicode{dollar}{0024}
\pdfglyphtounicode{dollarinferior}{F6E3}
\pdfglyphtounicode{dollarmonospace}{FF04}
\pdfglyphtounicode{dollaroldstyle}{F724}
\pdfglyphtounicode{dollarsmall}{FE69}
\pdfglyphtounicode{dollarsuperior}{F6E4}
\pdfglyphtounicode{dong}{20AB}
\pdfglyphtounicode{dorusquare}{3326}
\pdfglyphtounicode{dotaccent}{02D9}
\pdfglyphtounicode{dotaccentcmb}{0307}
\pdfglyphtounicode{dotbelowcmb}{0323}
\pdfglyphtounicode{dotbelowcomb}{0323}
\pdfglyphtounicode{dotkatakana}{30FB}
\pdfglyphtounicode{dotlessi}{0131}
\pdfglyphtounicode{dotlessj}{F6BE}
\pdfglyphtounicode{dotlessjstrokehook}{0284}
\pdfglyphtounicode{dotmath}{22C5}
\pdfglyphtounicode{dottedcircle}{25CC}
\pdfglyphtounicode{doubleyodpatah}{FB1F}
\pdfglyphtounicode{doubleyodpatahhebrew}{FB1F}
\pdfglyphtounicode{downtackbelowcmb}{031E}
\pdfglyphtounicode{downtackmod}{02D5}
\pdfglyphtounicode{dparen}{249F}
\pdfglyphtounicode{dsuperior}{F6EB}
\pdfglyphtounicode{dtail}{0256}
\pdfglyphtounicode{dtopbar}{018C}
\pdfglyphtounicode{duhiragana}{3065}
\pdfglyphtounicode{dukatakana}{30C5}
\pdfglyphtounicode{dz}{01F3}
\pdfglyphtounicode{dzaltone}{02A3}
\pdfglyphtounicode{dzcaron}{01C6}
\pdfglyphtounicode{dzcurl}{02A5}
\pdfglyphtounicode{dzeabkhasiancyrillic}{04E1}
\pdfglyphtounicode{dzecyrillic}{0455}
\pdfglyphtounicode{dzhecyrillic}{045F}
\pdfglyphtounicode{e}{0065}
\pdfglyphtounicode{eacute}{00E9}
\pdfglyphtounicode{earth}{2641}
\pdfglyphtounicode{ebengali}{098F}
\pdfglyphtounicode{ebopomofo}{311C}
\pdfglyphtounicode{ebreve}{0115}
\pdfglyphtounicode{ecandradeva}{090D}
\pdfglyphtounicode{ecandragujarati}{0A8D}
\pdfglyphtounicode{ecandravowelsigndeva}{0945}
\pdfglyphtounicode{ecandravowelsigngujarati}{0AC5}
\pdfglyphtounicode{ecaron}{011B}
\pdfglyphtounicode{ecedillabreve}{1E1D}
\pdfglyphtounicode{echarmenian}{0565}
\pdfglyphtounicode{echyiwnarmenian}{0587}
\pdfglyphtounicode{ecircle}{24D4}
\pdfglyphtounicode{ecircumflex}{00EA}
\pdfglyphtounicode{ecircumflexacute}{1EBF}
\pdfglyphtounicode{ecircumflexbelow}{1E19}
\pdfglyphtounicode{ecircumflexdotbelow}{1EC7}
\pdfglyphtounicode{ecircumflexgrave}{1EC1}
\pdfglyphtounicode{ecircumflexhookabove}{1EC3}
\pdfglyphtounicode{ecircumflextilde}{1EC5}
\pdfglyphtounicode{ecyrillic}{0454}
\pdfglyphtounicode{edblgrave}{0205}
\pdfglyphtounicode{edeva}{090F}
\pdfglyphtounicode{edieresis}{00EB}
\pdfglyphtounicode{edot}{0117}
\pdfglyphtounicode{edotaccent}{0117}
\pdfglyphtounicode{edotbelow}{1EB9}
\pdfglyphtounicode{eegurmukhi}{0A0F}
\pdfglyphtounicode{eematragurmukhi}{0A47}
\pdfglyphtounicode{efcyrillic}{0444}
\pdfglyphtounicode{egrave}{00E8}
\pdfglyphtounicode{egujarati}{0A8F}
\pdfglyphtounicode{eharmenian}{0567}
\pdfglyphtounicode{ehbopomofo}{311D}
\pdfglyphtounicode{ehiragana}{3048}
\pdfglyphtounicode{ehookabove}{1EBB}
\pdfglyphtounicode{eibopomofo}{311F}
\pdfglyphtounicode{eight}{0038}
\pdfglyphtounicode{eightarabic}{0668}
\pdfglyphtounicode{eightbengali}{09EE}
\pdfglyphtounicode{eightcircle}{2467}
\pdfglyphtounicode{eightcircleinversesansserif}{2791}
\pdfglyphtounicode{eightdeva}{096E}
\pdfglyphtounicode{eighteencircle}{2471}
\pdfglyphtounicode{eighteenparen}{2485}
\pdfglyphtounicode{eighteenperiod}{2499}
\pdfglyphtounicode{eightgujarati}{0AEE}
\pdfglyphtounicode{eightgurmukhi}{0A6E}
\pdfglyphtounicode{eighthackarabic}{0668}
\pdfglyphtounicode{eighthangzhou}{3028}
\pdfglyphtounicode{eighthnotebeamed}{266B}
\pdfglyphtounicode{eightideographicparen}{3227}
\pdfglyphtounicode{eightinferior}{2088}
\pdfglyphtounicode{eightmonospace}{FF18}
\pdfglyphtounicode{eightoldstyle}{F738}
\pdfglyphtounicode{eightparen}{247B}
\pdfglyphtounicode{eightperiod}{248F}
\pdfglyphtounicode{eightpersian}{06F8}
\pdfglyphtounicode{eightroman}{2177}
\pdfglyphtounicode{eightsuperior}{2078}
\pdfglyphtounicode{eightthai}{0E58}
\pdfglyphtounicode{einvertedbreve}{0207}
\pdfglyphtounicode{eiotifiedcyrillic}{0465}
\pdfglyphtounicode{ekatakana}{30A8}
\pdfglyphtounicode{ekatakanahalfwidth}{FF74}
\pdfglyphtounicode{ekonkargurmukhi}{0A74}
\pdfglyphtounicode{ekorean}{3154}
\pdfglyphtounicode{elcyrillic}{043B}
\pdfglyphtounicode{element}{2208}
\pdfglyphtounicode{elevencircle}{246A}
\pdfglyphtounicode{elevenparen}{247E}
\pdfglyphtounicode{elevenperiod}{2492}
\pdfglyphtounicode{elevenroman}{217A}
\pdfglyphtounicode{ellipsis}{2026}
\pdfglyphtounicode{ellipsisvertical}{22EE}
\pdfglyphtounicode{emacron}{0113}
\pdfglyphtounicode{emacronacute}{1E17}
\pdfglyphtounicode{emacrongrave}{1E15}
\pdfglyphtounicode{emcyrillic}{043C}
\pdfglyphtounicode{emdash}{2014}
\pdfglyphtounicode{emdashvertical}{FE31}
\pdfglyphtounicode{emonospace}{FF45}
\pdfglyphtounicode{emphasismarkarmenian}{055B}
\pdfglyphtounicode{emptyset}{2205}
\pdfglyphtounicode{enbopomofo}{3123}
\pdfglyphtounicode{encyrillic}{043D}
\pdfglyphtounicode{endash}{2013}
\pdfglyphtounicode{endashvertical}{FE32}
\pdfglyphtounicode{endescendercyrillic}{04A3}
\pdfglyphtounicode{eng}{014B}
\pdfglyphtounicode{engbopomofo}{3125}
\pdfglyphtounicode{enghecyrillic}{04A5}
\pdfglyphtounicode{enhookcyrillic}{04C8}
\pdfglyphtounicode{enspace}{2002}
\pdfglyphtounicode{eogonek}{0119}
\pdfglyphtounicode{eokorean}{3153}
\pdfglyphtounicode{eopen}{025B}
\pdfglyphtounicode{eopenclosed}{029A}
\pdfglyphtounicode{eopenreversed}{025C}
\pdfglyphtounicode{eopenreversedclosed}{025E}
\pdfglyphtounicode{eopenreversedhook}{025D}
\pdfglyphtounicode{eparen}{24A0}
\pdfglyphtounicode{epsilon}{03B5}
\pdfglyphtounicode{epsilontonos}{03AD}
\pdfglyphtounicode{equal}{003D}
\pdfglyphtounicode{equalmonospace}{FF1D}
\pdfglyphtounicode{equalsmall}{FE66}
\pdfglyphtounicode{equalsuperior}{207C}
\pdfglyphtounicode{equivalence}{2261}
\pdfglyphtounicode{erbopomofo}{3126}
\pdfglyphtounicode{ercyrillic}{0440}
\pdfglyphtounicode{ereversed}{0258}
\pdfglyphtounicode{ereversedcyrillic}{044D}
\pdfglyphtounicode{escyrillic}{0441}
\pdfglyphtounicode{esdescendercyrillic}{04AB}
\pdfglyphtounicode{esh}{0283}
\pdfglyphtounicode{eshcurl}{0286}
\pdfglyphtounicode{eshortdeva}{090E}
\pdfglyphtounicode{eshortvowelsigndeva}{0946}
\pdfglyphtounicode{eshreversedloop}{01AA}
\pdfglyphtounicode{eshsquatreversed}{0285}
\pdfglyphtounicode{esmallhiragana}{3047}
\pdfglyphtounicode{esmallkatakana}{30A7}
\pdfglyphtounicode{esmallkatakanahalfwidth}{FF6A}
\pdfglyphtounicode{estimated}{212E}
\pdfglyphtounicode{esuperior}{F6EC}
\pdfglyphtounicode{eta}{03B7}
\pdfglyphtounicode{etarmenian}{0568}
\pdfglyphtounicode{etatonos}{03AE}
\pdfglyphtounicode{eth}{00F0}
\pdfglyphtounicode{etilde}{1EBD}
\pdfglyphtounicode{etildebelow}{1E1B}
\pdfglyphtounicode{etnahtafoukhhebrew}{0591}
\pdfglyphtounicode{etnahtafoukhlefthebrew}{0591}
\pdfglyphtounicode{etnahtahebrew}{0591}
\pdfglyphtounicode{etnahtalefthebrew}{0591}
\pdfglyphtounicode{eturned}{01DD}
\pdfglyphtounicode{eukorean}{3161}
\pdfglyphtounicode{euro}{20AC}
\pdfglyphtounicode{evowelsignbengali}{09C7}
\pdfglyphtounicode{evowelsigndeva}{0947}
\pdfglyphtounicode{evowelsigngujarati}{0AC7}
\pdfglyphtounicode{exclam}{0021}
\pdfglyphtounicode{exclamarmenian}{055C}
\pdfglyphtounicode{exclamdbl}{203C}
\pdfglyphtounicode{exclamdown}{00A1}
\pdfglyphtounicode{exclamdownsmall}{F7A1}
\pdfglyphtounicode{exclammonospace}{FF01}
\pdfglyphtounicode{exclamsmall}{F721}
\pdfglyphtounicode{existential}{2203}
\pdfglyphtounicode{ezh}{0292}
\pdfglyphtounicode{ezhcaron}{01EF}
\pdfglyphtounicode{ezhcurl}{0293}
\pdfglyphtounicode{ezhreversed}{01B9}
\pdfglyphtounicode{ezhtail}{01BA}
\pdfglyphtounicode{f}{0066}
\pdfglyphtounicode{fadeva}{095E}
\pdfglyphtounicode{fagurmukhi}{0A5E}
\pdfglyphtounicode{fahrenheit}{2109}
\pdfglyphtounicode{fathaarabic}{064E}
\pdfglyphtounicode{fathalowarabic}{064E}
\pdfglyphtounicode{fathatanarabic}{064B}
\pdfglyphtounicode{fbopomofo}{3108}
\pdfglyphtounicode{fcircle}{24D5}
\pdfglyphtounicode{fdotaccent}{1E1F}
\pdfglyphtounicode{feharabic}{0641}
\pdfglyphtounicode{feharmenian}{0586}
\pdfglyphtounicode{fehfinalarabic}{FED2}
\pdfglyphtounicode{fehinitialarabic}{FED3}
\pdfglyphtounicode{fehmedialarabic}{FED4}
\pdfglyphtounicode{feicoptic}{03E5}
\pdfglyphtounicode{female}{2640}
\pdfglyphtounicode{ff}{FB00}
\pdfglyphtounicode{ffi}{FB03}
\pdfglyphtounicode{ffl}{FB04}
\pdfglyphtounicode{fi}{FB01}
\pdfglyphtounicode{fifteencircle}{246E}
\pdfglyphtounicode{fifteenparen}{2482}
\pdfglyphtounicode{fifteenperiod}{2496}
\pdfglyphtounicode{figuredash}{2012}
\pdfglyphtounicode{filledbox}{25A0}
\pdfglyphtounicode{filledrect}{25AC}
\pdfglyphtounicode{finalkaf}{05DA}
\pdfglyphtounicode{finalkafdagesh}{FB3A}
\pdfglyphtounicode{finalkafdageshhebrew}{FB3A}
\pdfglyphtounicode{finalkafhebrew}{05DA}
% finalkafqamats;05DA 05B8
% finalkafqamatshebrew;05DA 05B8
% finalkafsheva;05DA 05B0
% finalkafshevahebrew;05DA 05B0
\pdfglyphtounicode{finalmem}{05DD}
\pdfglyphtounicode{finalmemhebrew}{05DD}
\pdfglyphtounicode{finalnun}{05DF}
\pdfglyphtounicode{finalnunhebrew}{05DF}
\pdfglyphtounicode{finalpe}{05E3}
\pdfglyphtounicode{finalpehebrew}{05E3}
\pdfglyphtounicode{finaltsadi}{05E5}
\pdfglyphtounicode{finaltsadihebrew}{05E5}
\pdfglyphtounicode{firsttonechinese}{02C9}
\pdfglyphtounicode{fisheye}{25C9}
\pdfglyphtounicode{fitacyrillic}{0473}
\pdfglyphtounicode{five}{0035}
\pdfglyphtounicode{fivearabic}{0665}
\pdfglyphtounicode{fivebengali}{09EB}
\pdfglyphtounicode{fivecircle}{2464}
\pdfglyphtounicode{fivecircleinversesansserif}{278E}
\pdfglyphtounicode{fivedeva}{096B}
\pdfglyphtounicode{fiveeighths}{215D}
\pdfglyphtounicode{fivegujarati}{0AEB}
\pdfglyphtounicode{fivegurmukhi}{0A6B}
\pdfglyphtounicode{fivehackarabic}{0665}
\pdfglyphtounicode{fivehangzhou}{3025}
\pdfglyphtounicode{fiveideographicparen}{3224}
\pdfglyphtounicode{fiveinferior}{2085}
\pdfglyphtounicode{fivemonospace}{FF15}
\pdfglyphtounicode{fiveoldstyle}{F735}
\pdfglyphtounicode{fiveparen}{2478}
\pdfglyphtounicode{fiveperiod}{248C}
\pdfglyphtounicode{fivepersian}{06F5}
\pdfglyphtounicode{fiveroman}{2174}
\pdfglyphtounicode{fivesuperior}{2075}
\pdfglyphtounicode{fivethai}{0E55}
\pdfglyphtounicode{fl}{FB02}
\pdfglyphtounicode{florin}{0192}
\pdfglyphtounicode{fmonospace}{FF46}
\pdfglyphtounicode{fmsquare}{3399}
\pdfglyphtounicode{fofanthai}{0E1F}
\pdfglyphtounicode{fofathai}{0E1D}
\pdfglyphtounicode{fongmanthai}{0E4F}
\pdfglyphtounicode{forall}{2200}
\pdfglyphtounicode{four}{0034}
\pdfglyphtounicode{fourarabic}{0664}
\pdfglyphtounicode{fourbengali}{09EA}
\pdfglyphtounicode{fourcircle}{2463}
\pdfglyphtounicode{fourcircleinversesansserif}{278D}
\pdfglyphtounicode{fourdeva}{096A}
\pdfglyphtounicode{fourgujarati}{0AEA}
\pdfglyphtounicode{fourgurmukhi}{0A6A}
\pdfglyphtounicode{fourhackarabic}{0664}
\pdfglyphtounicode{fourhangzhou}{3024}
\pdfglyphtounicode{fourideographicparen}{3223}
\pdfglyphtounicode{fourinferior}{2084}
\pdfglyphtounicode{fourmonospace}{FF14}
\pdfglyphtounicode{fournumeratorbengali}{09F7}
\pdfglyphtounicode{fouroldstyle}{F734}
\pdfglyphtounicode{fourparen}{2477}
\pdfglyphtounicode{fourperiod}{248B}
\pdfglyphtounicode{fourpersian}{06F4}
\pdfglyphtounicode{fourroman}{2173}
\pdfglyphtounicode{foursuperior}{2074}
\pdfglyphtounicode{fourteencircle}{246D}
\pdfglyphtounicode{fourteenparen}{2481}
\pdfglyphtounicode{fourteenperiod}{2495}
\pdfglyphtounicode{fourthai}{0E54}
\pdfglyphtounicode{fourthtonechinese}{02CB}
\pdfglyphtounicode{fparen}{24A1}
\pdfglyphtounicode{fraction}{2044}
\pdfglyphtounicode{franc}{20A3}
\pdfglyphtounicode{g}{0067}
\pdfglyphtounicode{gabengali}{0997}
\pdfglyphtounicode{gacute}{01F5}
\pdfglyphtounicode{gadeva}{0917}
\pdfglyphtounicode{gafarabic}{06AF}
\pdfglyphtounicode{gaffinalarabic}{FB93}
\pdfglyphtounicode{gafinitialarabic}{FB94}
\pdfglyphtounicode{gafmedialarabic}{FB95}
\pdfglyphtounicode{gagujarati}{0A97}
\pdfglyphtounicode{gagurmukhi}{0A17}
\pdfglyphtounicode{gahiragana}{304C}
\pdfglyphtounicode{gakatakana}{30AC}
\pdfglyphtounicode{gamma}{03B3}
\pdfglyphtounicode{gammalatinsmall}{0263}
\pdfglyphtounicode{gammasuperior}{02E0}
\pdfglyphtounicode{gangiacoptic}{03EB}
\pdfglyphtounicode{gbopomofo}{310D}
\pdfglyphtounicode{gbreve}{011F}
\pdfglyphtounicode{gcaron}{01E7}
\pdfglyphtounicode{gcedilla}{0123}
\pdfglyphtounicode{gcircle}{24D6}
\pdfglyphtounicode{gcircumflex}{011D}
\pdfglyphtounicode{gcommaaccent}{0123}
\pdfglyphtounicode{gdot}{0121}
\pdfglyphtounicode{gdotaccent}{0121}
\pdfglyphtounicode{gecyrillic}{0433}
\pdfglyphtounicode{gehiragana}{3052}
\pdfglyphtounicode{gekatakana}{30B2}
\pdfglyphtounicode{geometricallyequal}{2251}
\pdfglyphtounicode{gereshaccenthebrew}{059C}
\pdfglyphtounicode{gereshhebrew}{05F3}
\pdfglyphtounicode{gereshmuqdamhebrew}{059D}
\pdfglyphtounicode{germandbls}{00DF}
\pdfglyphtounicode{gershayimaccenthebrew}{059E}
\pdfglyphtounicode{gershayimhebrew}{05F4}
\pdfglyphtounicode{getamark}{3013}
\pdfglyphtounicode{ghabengali}{0998}
\pdfglyphtounicode{ghadarmenian}{0572}
\pdfglyphtounicode{ghadeva}{0918}
\pdfglyphtounicode{ghagujarati}{0A98}
\pdfglyphtounicode{ghagurmukhi}{0A18}
\pdfglyphtounicode{ghainarabic}{063A}
\pdfglyphtounicode{ghainfinalarabic}{FECE}
\pdfglyphtounicode{ghaininitialarabic}{FECF}
\pdfglyphtounicode{ghainmedialarabic}{FED0}
\pdfglyphtounicode{ghemiddlehookcyrillic}{0495}
\pdfglyphtounicode{ghestrokecyrillic}{0493}
\pdfglyphtounicode{gheupturncyrillic}{0491}
\pdfglyphtounicode{ghhadeva}{095A}
\pdfglyphtounicode{ghhagurmukhi}{0A5A}
\pdfglyphtounicode{ghook}{0260}
\pdfglyphtounicode{ghzsquare}{3393}
\pdfglyphtounicode{gihiragana}{304E}
\pdfglyphtounicode{gikatakana}{30AE}
\pdfglyphtounicode{gimarmenian}{0563}
\pdfglyphtounicode{gimel}{05D2}
\pdfglyphtounicode{gimeldagesh}{FB32}
\pdfglyphtounicode{gimeldageshhebrew}{FB32}
\pdfglyphtounicode{gimelhebrew}{05D2}
\pdfglyphtounicode{gjecyrillic}{0453}
\pdfglyphtounicode{glottalinvertedstroke}{01BE}
\pdfglyphtounicode{glottalstop}{0294}
\pdfglyphtounicode{glottalstopinverted}{0296}
\pdfglyphtounicode{glottalstopmod}{02C0}
\pdfglyphtounicode{glottalstopreversed}{0295}
\pdfglyphtounicode{glottalstopreversedmod}{02C1}
\pdfglyphtounicode{glottalstopreversedsuperior}{02E4}
\pdfglyphtounicode{glottalstopstroke}{02A1}
\pdfglyphtounicode{glottalstopstrokereversed}{02A2}
\pdfglyphtounicode{gmacron}{1E21}
\pdfglyphtounicode{gmonospace}{FF47}
\pdfglyphtounicode{gohiragana}{3054}
\pdfglyphtounicode{gokatakana}{30B4}
\pdfglyphtounicode{gparen}{24A2}
\pdfglyphtounicode{gpasquare}{33AC}
\pdfglyphtounicode{gradient}{2207}
\pdfglyphtounicode{grave}{0060}
\pdfglyphtounicode{gravebelowcmb}{0316}
\pdfglyphtounicode{gravecmb}{0300}
\pdfglyphtounicode{gravecomb}{0300}
\pdfglyphtounicode{gravedeva}{0953}
\pdfglyphtounicode{gravelowmod}{02CE}
\pdfglyphtounicode{gravemonospace}{FF40}
\pdfglyphtounicode{gravetonecmb}{0340}
\pdfglyphtounicode{greater}{003E}
\pdfglyphtounicode{greaterequal}{2265}
\pdfglyphtounicode{greaterequalorless}{22DB}
\pdfglyphtounicode{greatermonospace}{FF1E}
\pdfglyphtounicode{greaterorequivalent}{2273}
\pdfglyphtounicode{greaterorless}{2277}
\pdfglyphtounicode{greateroverequal}{2267}
\pdfglyphtounicode{greatersmall}{FE65}
\pdfglyphtounicode{gscript}{0261}
\pdfglyphtounicode{gstroke}{01E5}
\pdfglyphtounicode{guhiragana}{3050}
\pdfglyphtounicode{guillemotleft}{00AB}
\pdfglyphtounicode{guillemotright}{00BB}
\pdfglyphtounicode{guilsinglleft}{2039}
\pdfglyphtounicode{guilsinglright}{203A}
\pdfglyphtounicode{gukatakana}{30B0}
\pdfglyphtounicode{guramusquare}{3318}
\pdfglyphtounicode{gysquare}{33C9}
\pdfglyphtounicode{h}{0068}
\pdfglyphtounicode{haabkhasiancyrillic}{04A9}
\pdfglyphtounicode{haaltonearabic}{06C1}
\pdfglyphtounicode{habengali}{09B9}
\pdfglyphtounicode{hadescendercyrillic}{04B3}
\pdfglyphtounicode{hadeva}{0939}
\pdfglyphtounicode{hagujarati}{0AB9}
\pdfglyphtounicode{hagurmukhi}{0A39}
\pdfglyphtounicode{haharabic}{062D}
\pdfglyphtounicode{hahfinalarabic}{FEA2}
\pdfglyphtounicode{hahinitialarabic}{FEA3}
\pdfglyphtounicode{hahiragana}{306F}
\pdfglyphtounicode{hahmedialarabic}{FEA4}
\pdfglyphtounicode{haitusquare}{332A}
\pdfglyphtounicode{hakatakana}{30CF}
\pdfglyphtounicode{hakatakanahalfwidth}{FF8A}
\pdfglyphtounicode{halantgurmukhi}{0A4D}
\pdfglyphtounicode{hamzaarabic}{0621}
% hamzadammaarabic;0621 064F
% hamzadammatanarabic;0621 064C
% hamzafathaarabic;0621 064E
% hamzafathatanarabic;0621 064B
\pdfglyphtounicode{hamzalowarabic}{0621}
% hamzalowkasraarabic;0621 0650
% hamzalowkasratanarabic;0621 064D
% hamzasukunarabic;0621 0652
\pdfglyphtounicode{hangulfiller}{3164}
\pdfglyphtounicode{hardsigncyrillic}{044A}
\pdfglyphtounicode{harpoonleftbarbup}{21BC}
\pdfglyphtounicode{harpoonrightbarbup}{21C0}
\pdfglyphtounicode{hasquare}{33CA}
\pdfglyphtounicode{hatafpatah}{05B2}
\pdfglyphtounicode{hatafpatah16}{05B2}
\pdfglyphtounicode{hatafpatah23}{05B2}
\pdfglyphtounicode{hatafpatah2f}{05B2}
\pdfglyphtounicode{hatafpatahhebrew}{05B2}
\pdfglyphtounicode{hatafpatahnarrowhebrew}{05B2}
\pdfglyphtounicode{hatafpatahquarterhebrew}{05B2}
\pdfglyphtounicode{hatafpatahwidehebrew}{05B2}
\pdfglyphtounicode{hatafqamats}{05B3}
\pdfglyphtounicode{hatafqamats1b}{05B3}
\pdfglyphtounicode{hatafqamats28}{05B3}
\pdfglyphtounicode{hatafqamats34}{05B3}
\pdfglyphtounicode{hatafqamatshebrew}{05B3}
\pdfglyphtounicode{hatafqamatsnarrowhebrew}{05B3}
\pdfglyphtounicode{hatafqamatsquarterhebrew}{05B3}
\pdfglyphtounicode{hatafqamatswidehebrew}{05B3}
\pdfglyphtounicode{hatafsegol}{05B1}
\pdfglyphtounicode{hatafsegol17}{05B1}
\pdfglyphtounicode{hatafsegol24}{05B1}
\pdfglyphtounicode{hatafsegol30}{05B1}
\pdfglyphtounicode{hatafsegolhebrew}{05B1}
\pdfglyphtounicode{hatafsegolnarrowhebrew}{05B1}
\pdfglyphtounicode{hatafsegolquarterhebrew}{05B1}
\pdfglyphtounicode{hatafsegolwidehebrew}{05B1}
\pdfglyphtounicode{hbar}{0127}
\pdfglyphtounicode{hbopomofo}{310F}
\pdfglyphtounicode{hbrevebelow}{1E2B}
\pdfglyphtounicode{hcedilla}{1E29}
\pdfglyphtounicode{hcircle}{24D7}
\pdfglyphtounicode{hcircumflex}{0125}
\pdfglyphtounicode{hdieresis}{1E27}
\pdfglyphtounicode{hdotaccent}{1E23}
\pdfglyphtounicode{hdotbelow}{1E25}
\pdfglyphtounicode{he}{05D4}
\pdfglyphtounicode{heart}{2665}
\pdfglyphtounicode{heartsuitblack}{2665}
\pdfglyphtounicode{heartsuitwhite}{2661}
\pdfglyphtounicode{hedagesh}{FB34}
\pdfglyphtounicode{hedageshhebrew}{FB34}
\pdfglyphtounicode{hehaltonearabic}{06C1}
\pdfglyphtounicode{heharabic}{0647}
\pdfglyphtounicode{hehebrew}{05D4}
\pdfglyphtounicode{hehfinalaltonearabic}{FBA7}
\pdfglyphtounicode{hehfinalalttwoarabic}{FEEA}
\pdfglyphtounicode{hehfinalarabic}{FEEA}
\pdfglyphtounicode{hehhamzaabovefinalarabic}{FBA5}
\pdfglyphtounicode{hehhamzaaboveisolatedarabic}{FBA4}
\pdfglyphtounicode{hehinitialaltonearabic}{FBA8}
\pdfglyphtounicode{hehinitialarabic}{FEEB}
\pdfglyphtounicode{hehiragana}{3078}
\pdfglyphtounicode{hehmedialaltonearabic}{FBA9}
\pdfglyphtounicode{hehmedialarabic}{FEEC}
\pdfglyphtounicode{heiseierasquare}{337B}
\pdfglyphtounicode{hekatakana}{30D8}
\pdfglyphtounicode{hekatakanahalfwidth}{FF8D}
\pdfglyphtounicode{hekutaarusquare}{3336}
\pdfglyphtounicode{henghook}{0267}
\pdfglyphtounicode{herutusquare}{3339}
\pdfglyphtounicode{het}{05D7}
\pdfglyphtounicode{hethebrew}{05D7}
\pdfglyphtounicode{hhook}{0266}
\pdfglyphtounicode{hhooksuperior}{02B1}
\pdfglyphtounicode{hieuhacirclekorean}{327B}
\pdfglyphtounicode{hieuhaparenkorean}{321B}
\pdfglyphtounicode{hieuhcirclekorean}{326D}
\pdfglyphtounicode{hieuhkorean}{314E}
\pdfglyphtounicode{hieuhparenkorean}{320D}
\pdfglyphtounicode{hihiragana}{3072}
\pdfglyphtounicode{hikatakana}{30D2}
\pdfglyphtounicode{hikatakanahalfwidth}{FF8B}
\pdfglyphtounicode{hiriq}{05B4}
\pdfglyphtounicode{hiriq14}{05B4}
\pdfglyphtounicode{hiriq21}{05B4}
\pdfglyphtounicode{hiriq2d}{05B4}
\pdfglyphtounicode{hiriqhebrew}{05B4}
\pdfglyphtounicode{hiriqnarrowhebrew}{05B4}
\pdfglyphtounicode{hiriqquarterhebrew}{05B4}
\pdfglyphtounicode{hiriqwidehebrew}{05B4}
\pdfglyphtounicode{hlinebelow}{1E96}
\pdfglyphtounicode{hmonospace}{FF48}
\pdfglyphtounicode{hoarmenian}{0570}
\pdfglyphtounicode{hohipthai}{0E2B}
\pdfglyphtounicode{hohiragana}{307B}
\pdfglyphtounicode{hokatakana}{30DB}
\pdfglyphtounicode{hokatakanahalfwidth}{FF8E}
\pdfglyphtounicode{holam}{05B9}
\pdfglyphtounicode{holam19}{05B9}
\pdfglyphtounicode{holam26}{05B9}
\pdfglyphtounicode{holam32}{05B9}
\pdfglyphtounicode{holamhebrew}{05B9}
\pdfglyphtounicode{holamnarrowhebrew}{05B9}
\pdfglyphtounicode{holamquarterhebrew}{05B9}
\pdfglyphtounicode{holamwidehebrew}{05B9}
\pdfglyphtounicode{honokhukthai}{0E2E}
\pdfglyphtounicode{hookabovecomb}{0309}
\pdfglyphtounicode{hookcmb}{0309}
\pdfglyphtounicode{hookpalatalizedbelowcmb}{0321}
\pdfglyphtounicode{hookretroflexbelowcmb}{0322}
\pdfglyphtounicode{hoonsquare}{3342}
\pdfglyphtounicode{horicoptic}{03E9}
\pdfglyphtounicode{horizontalbar}{2015}
\pdfglyphtounicode{horncmb}{031B}
\pdfglyphtounicode{hotsprings}{2668}
\pdfglyphtounicode{house}{2302}
\pdfglyphtounicode{hparen}{24A3}
\pdfglyphtounicode{hsuperior}{02B0}
\pdfglyphtounicode{hturned}{0265}
\pdfglyphtounicode{huhiragana}{3075}
\pdfglyphtounicode{huiitosquare}{3333}
\pdfglyphtounicode{hukatakana}{30D5}
\pdfglyphtounicode{hukatakanahalfwidth}{FF8C}
\pdfglyphtounicode{hungarumlaut}{02DD}
\pdfglyphtounicode{hungarumlautcmb}{030B}
\pdfglyphtounicode{hv}{0195}
\pdfglyphtounicode{hyphen}{002D}
\pdfglyphtounicode{hypheninferior}{F6E5}
\pdfglyphtounicode{hyphenmonospace}{FF0D}
\pdfglyphtounicode{hyphensmall}{FE63}
\pdfglyphtounicode{hyphensuperior}{F6E6}
\pdfglyphtounicode{hyphentwo}{2010}
\pdfglyphtounicode{i}{0069}
\pdfglyphtounicode{iacute}{00ED}
\pdfglyphtounicode{iacyrillic}{044F}
\pdfglyphtounicode{ibengali}{0987}
\pdfglyphtounicode{ibopomofo}{3127}
\pdfglyphtounicode{ibreve}{012D}
\pdfglyphtounicode{icaron}{01D0}
\pdfglyphtounicode{icircle}{24D8}
\pdfglyphtounicode{icircumflex}{00EE}
\pdfglyphtounicode{icyrillic}{0456}
\pdfglyphtounicode{idblgrave}{0209}
\pdfglyphtounicode{ideographearthcircle}{328F}
\pdfglyphtounicode{ideographfirecircle}{328B}
\pdfglyphtounicode{ideographicallianceparen}{323F}
\pdfglyphtounicode{ideographiccallparen}{323A}
\pdfglyphtounicode{ideographiccentrecircle}{32A5}
\pdfglyphtounicode{ideographicclose}{3006}
\pdfglyphtounicode{ideographiccomma}{3001}
\pdfglyphtounicode{ideographiccommaleft}{FF64}
\pdfglyphtounicode{ideographiccongratulationparen}{3237}
\pdfglyphtounicode{ideographiccorrectcircle}{32A3}
\pdfglyphtounicode{ideographicearthparen}{322F}
\pdfglyphtounicode{ideographicenterpriseparen}{323D}
\pdfglyphtounicode{ideographicexcellentcircle}{329D}
\pdfglyphtounicode{ideographicfestivalparen}{3240}
\pdfglyphtounicode{ideographicfinancialcircle}{3296}
\pdfglyphtounicode{ideographicfinancialparen}{3236}
\pdfglyphtounicode{ideographicfireparen}{322B}
\pdfglyphtounicode{ideographichaveparen}{3232}
\pdfglyphtounicode{ideographichighcircle}{32A4}
\pdfglyphtounicode{ideographiciterationmark}{3005}
\pdfglyphtounicode{ideographiclaborcircle}{3298}
\pdfglyphtounicode{ideographiclaborparen}{3238}
\pdfglyphtounicode{ideographicleftcircle}{32A7}
\pdfglyphtounicode{ideographiclowcircle}{32A6}
\pdfglyphtounicode{ideographicmedicinecircle}{32A9}
\pdfglyphtounicode{ideographicmetalparen}{322E}
\pdfglyphtounicode{ideographicmoonparen}{322A}
\pdfglyphtounicode{ideographicnameparen}{3234}
\pdfglyphtounicode{ideographicperiod}{3002}
\pdfglyphtounicode{ideographicprintcircle}{329E}
\pdfglyphtounicode{ideographicreachparen}{3243}
\pdfglyphtounicode{ideographicrepresentparen}{3239}
\pdfglyphtounicode{ideographicresourceparen}{323E}
\pdfglyphtounicode{ideographicrightcircle}{32A8}
\pdfglyphtounicode{ideographicsecretcircle}{3299}
\pdfglyphtounicode{ideographicselfparen}{3242}
\pdfglyphtounicode{ideographicsocietyparen}{3233}
\pdfglyphtounicode{ideographicspace}{3000}
\pdfglyphtounicode{ideographicspecialparen}{3235}
\pdfglyphtounicode{ideographicstockparen}{3231}
\pdfglyphtounicode{ideographicstudyparen}{323B}
\pdfglyphtounicode{ideographicsunparen}{3230}
\pdfglyphtounicode{ideographicsuperviseparen}{323C}
\pdfglyphtounicode{ideographicwaterparen}{322C}
\pdfglyphtounicode{ideographicwoodparen}{322D}
\pdfglyphtounicode{ideographiczero}{3007}
\pdfglyphtounicode{ideographmetalcircle}{328E}
\pdfglyphtounicode{ideographmooncircle}{328A}
\pdfglyphtounicode{ideographnamecircle}{3294}
\pdfglyphtounicode{ideographsuncircle}{3290}
\pdfglyphtounicode{ideographwatercircle}{328C}
\pdfglyphtounicode{ideographwoodcircle}{328D}
\pdfglyphtounicode{ideva}{0907}
\pdfglyphtounicode{idieresis}{00EF}
\pdfglyphtounicode{idieresisacute}{1E2F}
\pdfglyphtounicode{idieresiscyrillic}{04E5}
\pdfglyphtounicode{idotbelow}{1ECB}
\pdfglyphtounicode{iebrevecyrillic}{04D7}
\pdfglyphtounicode{iecyrillic}{0435}
\pdfglyphtounicode{ieungacirclekorean}{3275}
\pdfglyphtounicode{ieungaparenkorean}{3215}
\pdfglyphtounicode{ieungcirclekorean}{3267}
\pdfglyphtounicode{ieungkorean}{3147}
\pdfglyphtounicode{ieungparenkorean}{3207}
\pdfglyphtounicode{igrave}{00EC}
\pdfglyphtounicode{igujarati}{0A87}
\pdfglyphtounicode{igurmukhi}{0A07}
\pdfglyphtounicode{ihiragana}{3044}
\pdfglyphtounicode{ihookabove}{1EC9}
\pdfglyphtounicode{iibengali}{0988}
\pdfglyphtounicode{iicyrillic}{0438}
\pdfglyphtounicode{iideva}{0908}
\pdfglyphtounicode{iigujarati}{0A88}
\pdfglyphtounicode{iigurmukhi}{0A08}
\pdfglyphtounicode{iimatragurmukhi}{0A40}
\pdfglyphtounicode{iinvertedbreve}{020B}
\pdfglyphtounicode{iishortcyrillic}{0439}
\pdfglyphtounicode{iivowelsignbengali}{09C0}
\pdfglyphtounicode{iivowelsigndeva}{0940}
\pdfglyphtounicode{iivowelsigngujarati}{0AC0}
\pdfglyphtounicode{ij}{0133}
\pdfglyphtounicode{ikatakana}{30A4}
\pdfglyphtounicode{ikatakanahalfwidth}{FF72}
\pdfglyphtounicode{ikorean}{3163}
\pdfglyphtounicode{ilde}{02DC}
\pdfglyphtounicode{iluyhebrew}{05AC}
\pdfglyphtounicode{imacron}{012B}
\pdfglyphtounicode{imacroncyrillic}{04E3}
\pdfglyphtounicode{imageorapproximatelyequal}{2253}
\pdfglyphtounicode{imatragurmukhi}{0A3F}
\pdfglyphtounicode{imonospace}{FF49}
\pdfglyphtounicode{increment}{2206}
\pdfglyphtounicode{infinity}{221E}
\pdfglyphtounicode{iniarmenian}{056B}
\pdfglyphtounicode{integral}{222B}
\pdfglyphtounicode{integralbottom}{2321}
\pdfglyphtounicode{integralbt}{2321}
\pdfglyphtounicode{integralex}{F8F5}
\pdfglyphtounicode{integraltop}{2320}
\pdfglyphtounicode{integraltp}{2320}
\pdfglyphtounicode{intersection}{2229}
\pdfglyphtounicode{intisquare}{3305}
\pdfglyphtounicode{invbullet}{25D8}
\pdfglyphtounicode{invcircle}{25D9}
\pdfglyphtounicode{invsmileface}{263B}
\pdfglyphtounicode{iocyrillic}{0451}
\pdfglyphtounicode{iogonek}{012F}
\pdfglyphtounicode{iota}{03B9}
\pdfglyphtounicode{iotadieresis}{03CA}
\pdfglyphtounicode{iotadieresistonos}{0390}
\pdfglyphtounicode{iotalatin}{0269}
\pdfglyphtounicode{iotatonos}{03AF}
\pdfglyphtounicode{iparen}{24A4}
\pdfglyphtounicode{irigurmukhi}{0A72}
\pdfglyphtounicode{ismallhiragana}{3043}
\pdfglyphtounicode{ismallkatakana}{30A3}
\pdfglyphtounicode{ismallkatakanahalfwidth}{FF68}
\pdfglyphtounicode{issharbengali}{09FA}
\pdfglyphtounicode{istroke}{0268}
\pdfglyphtounicode{isuperior}{F6ED}
\pdfglyphtounicode{iterationhiragana}{309D}
\pdfglyphtounicode{iterationkatakana}{30FD}
\pdfglyphtounicode{itilde}{0129}
\pdfglyphtounicode{itildebelow}{1E2D}
\pdfglyphtounicode{iubopomofo}{3129}
\pdfglyphtounicode{iucyrillic}{044E}
\pdfglyphtounicode{ivowelsignbengali}{09BF}
\pdfglyphtounicode{ivowelsigndeva}{093F}
\pdfglyphtounicode{ivowelsigngujarati}{0ABF}
\pdfglyphtounicode{izhitsacyrillic}{0475}
\pdfglyphtounicode{izhitsadblgravecyrillic}{0477}
\pdfglyphtounicode{j}{006A}
\pdfglyphtounicode{jaarmenian}{0571}
\pdfglyphtounicode{jabengali}{099C}
\pdfglyphtounicode{jadeva}{091C}
\pdfglyphtounicode{jagujarati}{0A9C}
\pdfglyphtounicode{jagurmukhi}{0A1C}
\pdfglyphtounicode{jbopomofo}{3110}
\pdfglyphtounicode{jcaron}{01F0}
\pdfglyphtounicode{jcircle}{24D9}
\pdfglyphtounicode{jcircumflex}{0135}
\pdfglyphtounicode{jcrossedtail}{029D}
\pdfglyphtounicode{jdotlessstroke}{025F}
\pdfglyphtounicode{jecyrillic}{0458}
\pdfglyphtounicode{jeemarabic}{062C}
\pdfglyphtounicode{jeemfinalarabic}{FE9E}
\pdfglyphtounicode{jeeminitialarabic}{FE9F}
\pdfglyphtounicode{jeemmedialarabic}{FEA0}
\pdfglyphtounicode{jeharabic}{0698}
\pdfglyphtounicode{jehfinalarabic}{FB8B}
\pdfglyphtounicode{jhabengali}{099D}
\pdfglyphtounicode{jhadeva}{091D}
\pdfglyphtounicode{jhagujarati}{0A9D}
\pdfglyphtounicode{jhagurmukhi}{0A1D}
\pdfglyphtounicode{jheharmenian}{057B}
\pdfglyphtounicode{jis}{3004}
\pdfglyphtounicode{jmonospace}{FF4A}
\pdfglyphtounicode{jparen}{24A5}
\pdfglyphtounicode{jsuperior}{02B2}
\pdfglyphtounicode{k}{006B}
\pdfglyphtounicode{kabashkircyrillic}{04A1}
\pdfglyphtounicode{kabengali}{0995}
\pdfglyphtounicode{kacute}{1E31}
\pdfglyphtounicode{kacyrillic}{043A}
\pdfglyphtounicode{kadescendercyrillic}{049B}
\pdfglyphtounicode{kadeva}{0915}
\pdfglyphtounicode{kaf}{05DB}
\pdfglyphtounicode{kafarabic}{0643}
\pdfglyphtounicode{kafdagesh}{FB3B}
\pdfglyphtounicode{kafdageshhebrew}{FB3B}
\pdfglyphtounicode{kaffinalarabic}{FEDA}
\pdfglyphtounicode{kafhebrew}{05DB}
\pdfglyphtounicode{kafinitialarabic}{FEDB}
\pdfglyphtounicode{kafmedialarabic}{FEDC}
\pdfglyphtounicode{kafrafehebrew}{FB4D}
\pdfglyphtounicode{kagujarati}{0A95}
\pdfglyphtounicode{kagurmukhi}{0A15}
\pdfglyphtounicode{kahiragana}{304B}
\pdfglyphtounicode{kahookcyrillic}{04C4}
\pdfglyphtounicode{kakatakana}{30AB}
\pdfglyphtounicode{kakatakanahalfwidth}{FF76}
\pdfglyphtounicode{kappa}{03BA}
\pdfglyphtounicode{kappasymbolgreek}{03F0}
\pdfglyphtounicode{kapyeounmieumkorean}{3171}
\pdfglyphtounicode{kapyeounphieuphkorean}{3184}
\pdfglyphtounicode{kapyeounpieupkorean}{3178}
\pdfglyphtounicode{kapyeounssangpieupkorean}{3179}
\pdfglyphtounicode{karoriisquare}{330D}
\pdfglyphtounicode{kashidaautoarabic}{0640}
\pdfglyphtounicode{kashidaautonosidebearingarabic}{0640}
\pdfglyphtounicode{kasmallkatakana}{30F5}
\pdfglyphtounicode{kasquare}{3384}
\pdfglyphtounicode{kasraarabic}{0650}
\pdfglyphtounicode{kasratanarabic}{064D}
\pdfglyphtounicode{kastrokecyrillic}{049F}
\pdfglyphtounicode{katahiraprolongmarkhalfwidth}{FF70}
\pdfglyphtounicode{kaverticalstrokecyrillic}{049D}
\pdfglyphtounicode{kbopomofo}{310E}
\pdfglyphtounicode{kcalsquare}{3389}
\pdfglyphtounicode{kcaron}{01E9}
\pdfglyphtounicode{kcedilla}{0137}
\pdfglyphtounicode{kcircle}{24DA}
\pdfglyphtounicode{kcommaaccent}{0137}
\pdfglyphtounicode{kdotbelow}{1E33}
\pdfglyphtounicode{keharmenian}{0584}
\pdfglyphtounicode{kehiragana}{3051}
\pdfglyphtounicode{kekatakana}{30B1}
\pdfglyphtounicode{kekatakanahalfwidth}{FF79}
\pdfglyphtounicode{kenarmenian}{056F}
\pdfglyphtounicode{kesmallkatakana}{30F6}
\pdfglyphtounicode{kgreenlandic}{0138}
\pdfglyphtounicode{khabengali}{0996}
\pdfglyphtounicode{khacyrillic}{0445}
\pdfglyphtounicode{khadeva}{0916}
\pdfglyphtounicode{khagujarati}{0A96}
\pdfglyphtounicode{khagurmukhi}{0A16}
\pdfglyphtounicode{khaharabic}{062E}
\pdfglyphtounicode{khahfinalarabic}{FEA6}
\pdfglyphtounicode{khahinitialarabic}{FEA7}
\pdfglyphtounicode{khahmedialarabic}{FEA8}
\pdfglyphtounicode{kheicoptic}{03E7}
\pdfglyphtounicode{khhadeva}{0959}
\pdfglyphtounicode{khhagurmukhi}{0A59}
\pdfglyphtounicode{khieukhacirclekorean}{3278}
\pdfglyphtounicode{khieukhaparenkorean}{3218}
\pdfglyphtounicode{khieukhcirclekorean}{326A}
\pdfglyphtounicode{khieukhkorean}{314B}
\pdfglyphtounicode{khieukhparenkorean}{320A}
\pdfglyphtounicode{khokhaithai}{0E02}
\pdfglyphtounicode{khokhonthai}{0E05}
\pdfglyphtounicode{khokhuatthai}{0E03}
\pdfglyphtounicode{khokhwaithai}{0E04}
\pdfglyphtounicode{khomutthai}{0E5B}
\pdfglyphtounicode{khook}{0199}
\pdfglyphtounicode{khorakhangthai}{0E06}
\pdfglyphtounicode{khzsquare}{3391}
\pdfglyphtounicode{kihiragana}{304D}
\pdfglyphtounicode{kikatakana}{30AD}
\pdfglyphtounicode{kikatakanahalfwidth}{FF77}
\pdfglyphtounicode{kiroguramusquare}{3315}
\pdfglyphtounicode{kiromeetorusquare}{3316}
\pdfglyphtounicode{kirosquare}{3314}
\pdfglyphtounicode{kiyeokacirclekorean}{326E}
\pdfglyphtounicode{kiyeokaparenkorean}{320E}
\pdfglyphtounicode{kiyeokcirclekorean}{3260}
\pdfglyphtounicode{kiyeokkorean}{3131}
\pdfglyphtounicode{kiyeokparenkorean}{3200}
\pdfglyphtounicode{kiyeoksioskorean}{3133}
\pdfglyphtounicode{kjecyrillic}{045C}
\pdfglyphtounicode{klinebelow}{1E35}
\pdfglyphtounicode{klsquare}{3398}
\pdfglyphtounicode{kmcubedsquare}{33A6}
\pdfglyphtounicode{kmonospace}{FF4B}
\pdfglyphtounicode{kmsquaredsquare}{33A2}
\pdfglyphtounicode{kohiragana}{3053}
\pdfglyphtounicode{kohmsquare}{33C0}
\pdfglyphtounicode{kokaithai}{0E01}
\pdfglyphtounicode{kokatakana}{30B3}
\pdfglyphtounicode{kokatakanahalfwidth}{FF7A}
\pdfglyphtounicode{kooposquare}{331E}
\pdfglyphtounicode{koppacyrillic}{0481}
\pdfglyphtounicode{koreanstandardsymbol}{327F}
\pdfglyphtounicode{koroniscmb}{0343}
\pdfglyphtounicode{kparen}{24A6}
\pdfglyphtounicode{kpasquare}{33AA}
\pdfglyphtounicode{ksicyrillic}{046F}
\pdfglyphtounicode{ktsquare}{33CF}
\pdfglyphtounicode{kturned}{029E}
\pdfglyphtounicode{kuhiragana}{304F}
\pdfglyphtounicode{kukatakana}{30AF}
\pdfglyphtounicode{kukatakanahalfwidth}{FF78}
\pdfglyphtounicode{kvsquare}{33B8}
\pdfglyphtounicode{kwsquare}{33BE}
\pdfglyphtounicode{l}{006C}
\pdfglyphtounicode{labengali}{09B2}
\pdfglyphtounicode{lacute}{013A}
\pdfglyphtounicode{ladeva}{0932}
\pdfglyphtounicode{lagujarati}{0AB2}
\pdfglyphtounicode{lagurmukhi}{0A32}
\pdfglyphtounicode{lakkhangyaothai}{0E45}
\pdfglyphtounicode{lamaleffinalarabic}{FEFC}
\pdfglyphtounicode{lamalefhamzaabovefinalarabic}{FEF8}
\pdfglyphtounicode{lamalefhamzaaboveisolatedarabic}{FEF7}
\pdfglyphtounicode{lamalefhamzabelowfinalarabic}{FEFA}
\pdfglyphtounicode{lamalefhamzabelowisolatedarabic}{FEF9}
\pdfglyphtounicode{lamalefisolatedarabic}{FEFB}
\pdfglyphtounicode{lamalefmaddaabovefinalarabic}{FEF6}
\pdfglyphtounicode{lamalefmaddaaboveisolatedarabic}{FEF5}
\pdfglyphtounicode{lamarabic}{0644}
\pdfglyphtounicode{lambda}{03BB}
\pdfglyphtounicode{lambdastroke}{019B}
\pdfglyphtounicode{lamed}{05DC}
\pdfglyphtounicode{lameddagesh}{FB3C}
\pdfglyphtounicode{lameddageshhebrew}{FB3C}
\pdfglyphtounicode{lamedhebrew}{05DC}
% lamedholam;05DC 05B9
% lamedholamdagesh;05DC 05B9 05BC
% lamedholamdageshhebrew;05DC 05B9 05BC
% lamedholamhebrew;05DC 05B9
\pdfglyphtounicode{lamfinalarabic}{FEDE}
\pdfglyphtounicode{lamhahinitialarabic}{FCCA}
\pdfglyphtounicode{laminitialarabic}{FEDF}
\pdfglyphtounicode{lamjeeminitialarabic}{FCC9}
\pdfglyphtounicode{lamkhahinitialarabic}{FCCB}
\pdfglyphtounicode{lamlamhehisolatedarabic}{FDF2}
\pdfglyphtounicode{lammedialarabic}{FEE0}
\pdfglyphtounicode{lammeemhahinitialarabic}{FD88}
\pdfglyphtounicode{lammeeminitialarabic}{FCCC}
% lammeemjeeminitialarabic;FEDF FEE4 FEA0
% lammeemkhahinitialarabic;FEDF FEE4 FEA8
\pdfglyphtounicode{largecircle}{25EF}
\pdfglyphtounicode{lbar}{019A}
\pdfglyphtounicode{lbelt}{026C}
\pdfglyphtounicode{lbopomofo}{310C}
\pdfglyphtounicode{lcaron}{013E}
\pdfglyphtounicode{lcedilla}{013C}
\pdfglyphtounicode{lcircle}{24DB}
\pdfglyphtounicode{lcircumflexbelow}{1E3D}
\pdfglyphtounicode{lcommaaccent}{013C}
\pdfglyphtounicode{ldot}{0140}
\pdfglyphtounicode{ldotaccent}{0140}
\pdfglyphtounicode{ldotbelow}{1E37}
\pdfglyphtounicode{ldotbelowmacron}{1E39}
\pdfglyphtounicode{leftangleabovecmb}{031A}
\pdfglyphtounicode{lefttackbelowcmb}{0318}
\pdfglyphtounicode{less}{003C}
\pdfglyphtounicode{lessequal}{2264}
\pdfglyphtounicode{lessequalorgreater}{22DA}
\pdfglyphtounicode{lessmonospace}{FF1C}
\pdfglyphtounicode{lessorequivalent}{2272}
\pdfglyphtounicode{lessorgreater}{2276}
\pdfglyphtounicode{lessoverequal}{2266}
\pdfglyphtounicode{lesssmall}{FE64}
\pdfglyphtounicode{lezh}{026E}
\pdfglyphtounicode{lfblock}{258C}
\pdfglyphtounicode{lhookretroflex}{026D}
\pdfglyphtounicode{lira}{20A4}
\pdfglyphtounicode{liwnarmenian}{056C}
\pdfglyphtounicode{lj}{01C9}
\pdfglyphtounicode{ljecyrillic}{0459}
\pdfglyphtounicode{ll}{F6C0}
\pdfglyphtounicode{lladeva}{0933}
\pdfglyphtounicode{llagujarati}{0AB3}
\pdfglyphtounicode{llinebelow}{1E3B}
\pdfglyphtounicode{llladeva}{0934}
\pdfglyphtounicode{llvocalicbengali}{09E1}
\pdfglyphtounicode{llvocalicdeva}{0961}
\pdfglyphtounicode{llvocalicvowelsignbengali}{09E3}
\pdfglyphtounicode{llvocalicvowelsigndeva}{0963}
\pdfglyphtounicode{lmiddletilde}{026B}
\pdfglyphtounicode{lmonospace}{FF4C}
\pdfglyphtounicode{lmsquare}{33D0}
\pdfglyphtounicode{lochulathai}{0E2C}
\pdfglyphtounicode{logicaland}{2227}
\pdfglyphtounicode{logicalnot}{00AC}
\pdfglyphtounicode{logicalnotreversed}{2310}
\pdfglyphtounicode{logicalor}{2228}
\pdfglyphtounicode{lolingthai}{0E25}
\pdfglyphtounicode{longs}{017F}
\pdfglyphtounicode{lowlinecenterline}{FE4E}
\pdfglyphtounicode{lowlinecmb}{0332}
\pdfglyphtounicode{lowlinedashed}{FE4D}
\pdfglyphtounicode{lozenge}{25CA}
\pdfglyphtounicode{lparen}{24A7}
\pdfglyphtounicode{lslash}{0142}
\pdfglyphtounicode{lsquare}{2113}
\pdfglyphtounicode{lsuperior}{F6EE}
\pdfglyphtounicode{ltshade}{2591}
\pdfglyphtounicode{luthai}{0E26}
\pdfglyphtounicode{lvocalicbengali}{098C}
\pdfglyphtounicode{lvocalicdeva}{090C}
\pdfglyphtounicode{lvocalicvowelsignbengali}{09E2}
\pdfglyphtounicode{lvocalicvowelsigndeva}{0962}
\pdfglyphtounicode{lxsquare}{33D3}
\pdfglyphtounicode{m}{006D}
\pdfglyphtounicode{mabengali}{09AE}
\pdfglyphtounicode{macron}{00AF}
\pdfglyphtounicode{macronbelowcmb}{0331}
\pdfglyphtounicode{macroncmb}{0304}
\pdfglyphtounicode{macronlowmod}{02CD}
\pdfglyphtounicode{macronmonospace}{FFE3}
\pdfglyphtounicode{macute}{1E3F}
\pdfglyphtounicode{madeva}{092E}
\pdfglyphtounicode{magujarati}{0AAE}
\pdfglyphtounicode{magurmukhi}{0A2E}
\pdfglyphtounicode{mahapakhhebrew}{05A4}
\pdfglyphtounicode{mahapakhlefthebrew}{05A4}
\pdfglyphtounicode{mahiragana}{307E}
\pdfglyphtounicode{maichattawalowleftthai}{F895}
\pdfglyphtounicode{maichattawalowrightthai}{F894}
\pdfglyphtounicode{maichattawathai}{0E4B}
\pdfglyphtounicode{maichattawaupperleftthai}{F893}
\pdfglyphtounicode{maieklowleftthai}{F88C}
\pdfglyphtounicode{maieklowrightthai}{F88B}
\pdfglyphtounicode{maiekthai}{0E48}
\pdfglyphtounicode{maiekupperleftthai}{F88A}
\pdfglyphtounicode{maihanakatleftthai}{F884}
\pdfglyphtounicode{maihanakatthai}{0E31}
\pdfglyphtounicode{maitaikhuleftthai}{F889}
\pdfglyphtounicode{maitaikhuthai}{0E47}
\pdfglyphtounicode{maitholowleftthai}{F88F}
\pdfglyphtounicode{maitholowrightthai}{F88E}
\pdfglyphtounicode{maithothai}{0E49}
\pdfglyphtounicode{maithoupperleftthai}{F88D}
\pdfglyphtounicode{maitrilowleftthai}{F892}
\pdfglyphtounicode{maitrilowrightthai}{F891}
\pdfglyphtounicode{maitrithai}{0E4A}
\pdfglyphtounicode{maitriupperleftthai}{F890}
\pdfglyphtounicode{maiyamokthai}{0E46}
\pdfglyphtounicode{makatakana}{30DE}
\pdfglyphtounicode{makatakanahalfwidth}{FF8F}
\pdfglyphtounicode{male}{2642}
\pdfglyphtounicode{mansyonsquare}{3347}
\pdfglyphtounicode{maqafhebrew}{05BE}
\pdfglyphtounicode{mars}{2642}
\pdfglyphtounicode{masoracirclehebrew}{05AF}
\pdfglyphtounicode{masquare}{3383}
\pdfglyphtounicode{mbopomofo}{3107}
\pdfglyphtounicode{mbsquare}{33D4}
\pdfglyphtounicode{mcircle}{24DC}
\pdfglyphtounicode{mcubedsquare}{33A5}
\pdfglyphtounicode{mdotaccent}{1E41}
\pdfglyphtounicode{mdotbelow}{1E43}
\pdfglyphtounicode{meemarabic}{0645}
\pdfglyphtounicode{meemfinalarabic}{FEE2}
\pdfglyphtounicode{meeminitialarabic}{FEE3}
\pdfglyphtounicode{meemmedialarabic}{FEE4}
\pdfglyphtounicode{meemmeeminitialarabic}{FCD1}
\pdfglyphtounicode{meemmeemisolatedarabic}{FC48}
\pdfglyphtounicode{meetorusquare}{334D}
\pdfglyphtounicode{mehiragana}{3081}
\pdfglyphtounicode{meizierasquare}{337E}
\pdfglyphtounicode{mekatakana}{30E1}
\pdfglyphtounicode{mekatakanahalfwidth}{FF92}
\pdfglyphtounicode{mem}{05DE}
\pdfglyphtounicode{memdagesh}{FB3E}
\pdfglyphtounicode{memdageshhebrew}{FB3E}
\pdfglyphtounicode{memhebrew}{05DE}
\pdfglyphtounicode{menarmenian}{0574}
\pdfglyphtounicode{merkhahebrew}{05A5}
\pdfglyphtounicode{merkhakefulahebrew}{05A6}
\pdfglyphtounicode{merkhakefulalefthebrew}{05A6}
\pdfglyphtounicode{merkhalefthebrew}{05A5}
\pdfglyphtounicode{mhook}{0271}
\pdfglyphtounicode{mhzsquare}{3392}
\pdfglyphtounicode{middledotkatakanahalfwidth}{FF65}
\pdfglyphtounicode{middot}{00B7}
\pdfglyphtounicode{mieumacirclekorean}{3272}
\pdfglyphtounicode{mieumaparenkorean}{3212}
\pdfglyphtounicode{mieumcirclekorean}{3264}
\pdfglyphtounicode{mieumkorean}{3141}
\pdfglyphtounicode{mieumpansioskorean}{3170}
\pdfglyphtounicode{mieumparenkorean}{3204}
\pdfglyphtounicode{mieumpieupkorean}{316E}
\pdfglyphtounicode{mieumsioskorean}{316F}
\pdfglyphtounicode{mihiragana}{307F}
\pdfglyphtounicode{mikatakana}{30DF}
\pdfglyphtounicode{mikatakanahalfwidth}{FF90}
\pdfglyphtounicode{minus}{2212}
\pdfglyphtounicode{minusbelowcmb}{0320}
\pdfglyphtounicode{minuscircle}{2296}
\pdfglyphtounicode{minusmod}{02D7}
\pdfglyphtounicode{minusplus}{2213}
\pdfglyphtounicode{minute}{2032}
\pdfglyphtounicode{miribaarusquare}{334A}
\pdfglyphtounicode{mirisquare}{3349}
\pdfglyphtounicode{mlonglegturned}{0270}
\pdfglyphtounicode{mlsquare}{3396}
\pdfglyphtounicode{mmcubedsquare}{33A3}
\pdfglyphtounicode{mmonospace}{FF4D}
\pdfglyphtounicode{mmsquaredsquare}{339F}
\pdfglyphtounicode{mohiragana}{3082}
\pdfglyphtounicode{mohmsquare}{33C1}
\pdfglyphtounicode{mokatakana}{30E2}
\pdfglyphtounicode{mokatakanahalfwidth}{FF93}
\pdfglyphtounicode{molsquare}{33D6}
\pdfglyphtounicode{momathai}{0E21}
\pdfglyphtounicode{moverssquare}{33A7}
\pdfglyphtounicode{moverssquaredsquare}{33A8}
\pdfglyphtounicode{mparen}{24A8}
\pdfglyphtounicode{mpasquare}{33AB}
\pdfglyphtounicode{mssquare}{33B3}
\pdfglyphtounicode{msuperior}{F6EF}
\pdfglyphtounicode{mturned}{026F}
\pdfglyphtounicode{mu}{00B5}
\pdfglyphtounicode{mu1}{00B5}
\pdfglyphtounicode{muasquare}{3382}
\pdfglyphtounicode{muchgreater}{226B}
\pdfglyphtounicode{muchless}{226A}
\pdfglyphtounicode{mufsquare}{338C}
\pdfglyphtounicode{mugreek}{03BC}
\pdfglyphtounicode{mugsquare}{338D}
\pdfglyphtounicode{muhiragana}{3080}
\pdfglyphtounicode{mukatakana}{30E0}
\pdfglyphtounicode{mukatakanahalfwidth}{FF91}
\pdfglyphtounicode{mulsquare}{3395}
\pdfglyphtounicode{multiply}{00D7}
\pdfglyphtounicode{mumsquare}{339B}
\pdfglyphtounicode{munahhebrew}{05A3}
\pdfglyphtounicode{munahlefthebrew}{05A3}
\pdfglyphtounicode{musicalnote}{266A}
\pdfglyphtounicode{musicalnotedbl}{266B}
\pdfglyphtounicode{musicflatsign}{266D}
\pdfglyphtounicode{musicsharpsign}{266F}
\pdfglyphtounicode{mussquare}{33B2}
\pdfglyphtounicode{muvsquare}{33B6}
\pdfglyphtounicode{muwsquare}{33BC}
\pdfglyphtounicode{mvmegasquare}{33B9}
\pdfglyphtounicode{mvsquare}{33B7}
\pdfglyphtounicode{mwmegasquare}{33BF}
\pdfglyphtounicode{mwsquare}{33BD}
\pdfglyphtounicode{n}{006E}
\pdfglyphtounicode{nabengali}{09A8}
\pdfglyphtounicode{nabla}{2207}
\pdfglyphtounicode{nacute}{0144}
\pdfglyphtounicode{nadeva}{0928}
\pdfglyphtounicode{nagujarati}{0AA8}
\pdfglyphtounicode{nagurmukhi}{0A28}
\pdfglyphtounicode{nahiragana}{306A}
\pdfglyphtounicode{nakatakana}{30CA}
\pdfglyphtounicode{nakatakanahalfwidth}{FF85}
\pdfglyphtounicode{napostrophe}{0149}
\pdfglyphtounicode{nasquare}{3381}
\pdfglyphtounicode{nbopomofo}{310B}
\pdfglyphtounicode{nbspace}{00A0}
\pdfglyphtounicode{ncaron}{0148}
\pdfglyphtounicode{ncedilla}{0146}
\pdfglyphtounicode{ncircle}{24DD}
\pdfglyphtounicode{ncircumflexbelow}{1E4B}
\pdfglyphtounicode{ncommaaccent}{0146}
\pdfglyphtounicode{ndotaccent}{1E45}
\pdfglyphtounicode{ndotbelow}{1E47}
\pdfglyphtounicode{nehiragana}{306D}
\pdfglyphtounicode{nekatakana}{30CD}
\pdfglyphtounicode{nekatakanahalfwidth}{FF88}
\pdfglyphtounicode{newsheqelsign}{20AA}
\pdfglyphtounicode{nfsquare}{338B}
\pdfglyphtounicode{ngabengali}{0999}
\pdfglyphtounicode{ngadeva}{0919}
\pdfglyphtounicode{ngagujarati}{0A99}
\pdfglyphtounicode{ngagurmukhi}{0A19}
\pdfglyphtounicode{ngonguthai}{0E07}
\pdfglyphtounicode{nhiragana}{3093}
\pdfglyphtounicode{nhookleft}{0272}
\pdfglyphtounicode{nhookretroflex}{0273}
\pdfglyphtounicode{nieunacirclekorean}{326F}
\pdfglyphtounicode{nieunaparenkorean}{320F}
\pdfglyphtounicode{nieuncieuckorean}{3135}
\pdfglyphtounicode{nieuncirclekorean}{3261}
\pdfglyphtounicode{nieunhieuhkorean}{3136}
\pdfglyphtounicode{nieunkorean}{3134}
\pdfglyphtounicode{nieunpansioskorean}{3168}
\pdfglyphtounicode{nieunparenkorean}{3201}
\pdfglyphtounicode{nieunsioskorean}{3167}
\pdfglyphtounicode{nieuntikeutkorean}{3166}
\pdfglyphtounicode{nihiragana}{306B}
\pdfglyphtounicode{nikatakana}{30CB}
\pdfglyphtounicode{nikatakanahalfwidth}{FF86}
\pdfglyphtounicode{nikhahitleftthai}{F899}
\pdfglyphtounicode{nikhahitthai}{0E4D}
\pdfglyphtounicode{nine}{0039}
\pdfglyphtounicode{ninearabic}{0669}
\pdfglyphtounicode{ninebengali}{09EF}
\pdfglyphtounicode{ninecircle}{2468}
\pdfglyphtounicode{ninecircleinversesansserif}{2792}
\pdfglyphtounicode{ninedeva}{096F}
\pdfglyphtounicode{ninegujarati}{0AEF}
\pdfglyphtounicode{ninegurmukhi}{0A6F}
\pdfglyphtounicode{ninehackarabic}{0669}
\pdfglyphtounicode{ninehangzhou}{3029}
\pdfglyphtounicode{nineideographicparen}{3228}
\pdfglyphtounicode{nineinferior}{2089}
\pdfglyphtounicode{ninemonospace}{FF19}
\pdfglyphtounicode{nineoldstyle}{F739}
\pdfglyphtounicode{nineparen}{247C}
\pdfglyphtounicode{nineperiod}{2490}
\pdfglyphtounicode{ninepersian}{06F9}
\pdfglyphtounicode{nineroman}{2178}
\pdfglyphtounicode{ninesuperior}{2079}
\pdfglyphtounicode{nineteencircle}{2472}
\pdfglyphtounicode{nineteenparen}{2486}
\pdfglyphtounicode{nineteenperiod}{249A}
\pdfglyphtounicode{ninethai}{0E59}
\pdfglyphtounicode{nj}{01CC}
\pdfglyphtounicode{njecyrillic}{045A}
\pdfglyphtounicode{nkatakana}{30F3}
\pdfglyphtounicode{nkatakanahalfwidth}{FF9D}
\pdfglyphtounicode{nlegrightlong}{019E}
\pdfglyphtounicode{nlinebelow}{1E49}
\pdfglyphtounicode{nmonospace}{FF4E}
\pdfglyphtounicode{nmsquare}{339A}
\pdfglyphtounicode{nnabengali}{09A3}
\pdfglyphtounicode{nnadeva}{0923}
\pdfglyphtounicode{nnagujarati}{0AA3}
\pdfglyphtounicode{nnagurmukhi}{0A23}
\pdfglyphtounicode{nnnadeva}{0929}
\pdfglyphtounicode{nohiragana}{306E}
\pdfglyphtounicode{nokatakana}{30CE}
\pdfglyphtounicode{nokatakanahalfwidth}{FF89}
\pdfglyphtounicode{nonbreakingspace}{00A0}
\pdfglyphtounicode{nonenthai}{0E13}
\pdfglyphtounicode{nonuthai}{0E19}
\pdfglyphtounicode{noonarabic}{0646}
\pdfglyphtounicode{noonfinalarabic}{FEE6}
\pdfglyphtounicode{noonghunnaarabic}{06BA}
\pdfglyphtounicode{noonghunnafinalarabic}{FB9F}
% noonhehinitialarabic;FEE7 FEEC
\pdfglyphtounicode{nooninitialarabic}{FEE7}
\pdfglyphtounicode{noonjeeminitialarabic}{FCD2}
\pdfglyphtounicode{noonjeemisolatedarabic}{FC4B}
\pdfglyphtounicode{noonmedialarabic}{FEE8}
\pdfglyphtounicode{noonmeeminitialarabic}{FCD5}
\pdfglyphtounicode{noonmeemisolatedarabic}{FC4E}
\pdfglyphtounicode{noonnoonfinalarabic}{FC8D}
\pdfglyphtounicode{notcontains}{220C}
\pdfglyphtounicode{notelement}{2209}
\pdfglyphtounicode{notelementof}{2209}
\pdfglyphtounicode{notequal}{2260}
\pdfglyphtounicode{notgreater}{226F}
\pdfglyphtounicode{notgreaternorequal}{2271}
\pdfglyphtounicode{notgreaternorless}{2279}
\pdfglyphtounicode{notidentical}{2262}
\pdfglyphtounicode{notless}{226E}
\pdfglyphtounicode{notlessnorequal}{2270}
\pdfglyphtounicode{notparallel}{2226}
\pdfglyphtounicode{notprecedes}{2280}
\pdfglyphtounicode{notsubset}{2284}
\pdfglyphtounicode{notsucceeds}{2281}
\pdfglyphtounicode{notsuperset}{2285}
\pdfglyphtounicode{nowarmenian}{0576}
\pdfglyphtounicode{nparen}{24A9}
\pdfglyphtounicode{nssquare}{33B1}
\pdfglyphtounicode{nsuperior}{207F}
\pdfglyphtounicode{ntilde}{00F1}
\pdfglyphtounicode{nu}{03BD}
\pdfglyphtounicode{nuhiragana}{306C}
\pdfglyphtounicode{nukatakana}{30CC}
\pdfglyphtounicode{nukatakanahalfwidth}{FF87}
\pdfglyphtounicode{nuktabengali}{09BC}
\pdfglyphtounicode{nuktadeva}{093C}
\pdfglyphtounicode{nuktagujarati}{0ABC}
\pdfglyphtounicode{nuktagurmukhi}{0A3C}
\pdfglyphtounicode{numbersign}{0023}
\pdfglyphtounicode{numbersignmonospace}{FF03}
\pdfglyphtounicode{numbersignsmall}{FE5F}
\pdfglyphtounicode{numeralsigngreek}{0374}
\pdfglyphtounicode{numeralsignlowergreek}{0375}
\pdfglyphtounicode{numero}{2116}
\pdfglyphtounicode{nun}{05E0}
\pdfglyphtounicode{nundagesh}{FB40}
\pdfglyphtounicode{nundageshhebrew}{FB40}
\pdfglyphtounicode{nunhebrew}{05E0}
\pdfglyphtounicode{nvsquare}{33B5}
\pdfglyphtounicode{nwsquare}{33BB}
\pdfglyphtounicode{nyabengali}{099E}
\pdfglyphtounicode{nyadeva}{091E}
\pdfglyphtounicode{nyagujarati}{0A9E}
\pdfglyphtounicode{nyagurmukhi}{0A1E}
\pdfglyphtounicode{o}{006F}
\pdfglyphtounicode{oacute}{00F3}
\pdfglyphtounicode{oangthai}{0E2D}
\pdfglyphtounicode{obarred}{0275}
\pdfglyphtounicode{obarredcyrillic}{04E9}
\pdfglyphtounicode{obarreddieresiscyrillic}{04EB}
\pdfglyphtounicode{obengali}{0993}
\pdfglyphtounicode{obopomofo}{311B}
\pdfglyphtounicode{obreve}{014F}
\pdfglyphtounicode{ocandradeva}{0911}
\pdfglyphtounicode{ocandragujarati}{0A91}
\pdfglyphtounicode{ocandravowelsigndeva}{0949}
\pdfglyphtounicode{ocandravowelsigngujarati}{0AC9}
\pdfglyphtounicode{ocaron}{01D2}
\pdfglyphtounicode{ocircle}{24DE}
\pdfglyphtounicode{ocircumflex}{00F4}
\pdfglyphtounicode{ocircumflexacute}{1ED1}
\pdfglyphtounicode{ocircumflexdotbelow}{1ED9}
\pdfglyphtounicode{ocircumflexgrave}{1ED3}
\pdfglyphtounicode{ocircumflexhookabove}{1ED5}
\pdfglyphtounicode{ocircumflextilde}{1ED7}
\pdfglyphtounicode{ocyrillic}{043E}
\pdfglyphtounicode{odblacute}{0151}
\pdfglyphtounicode{odblgrave}{020D}
\pdfglyphtounicode{odeva}{0913}
\pdfglyphtounicode{odieresis}{00F6}
\pdfglyphtounicode{odieresiscyrillic}{04E7}
\pdfglyphtounicode{odotbelow}{1ECD}
\pdfglyphtounicode{oe}{0153}
\pdfglyphtounicode{oekorean}{315A}
\pdfglyphtounicode{ogonek}{02DB}
\pdfglyphtounicode{ogonekcmb}{0328}
\pdfglyphtounicode{ograve}{00F2}
\pdfglyphtounicode{ogujarati}{0A93}
\pdfglyphtounicode{oharmenian}{0585}
\pdfglyphtounicode{ohiragana}{304A}
\pdfglyphtounicode{ohookabove}{1ECF}
\pdfglyphtounicode{ohorn}{01A1}
\pdfglyphtounicode{ohornacute}{1EDB}
\pdfglyphtounicode{ohorndotbelow}{1EE3}
\pdfglyphtounicode{ohorngrave}{1EDD}
\pdfglyphtounicode{ohornhookabove}{1EDF}
\pdfglyphtounicode{ohorntilde}{1EE1}
\pdfglyphtounicode{ohungarumlaut}{0151}
\pdfglyphtounicode{oi}{01A3}
\pdfglyphtounicode{oinvertedbreve}{020F}
\pdfglyphtounicode{okatakana}{30AA}
\pdfglyphtounicode{okatakanahalfwidth}{FF75}
\pdfglyphtounicode{okorean}{3157}
\pdfglyphtounicode{olehebrew}{05AB}
\pdfglyphtounicode{omacron}{014D}
\pdfglyphtounicode{omacronacute}{1E53}
\pdfglyphtounicode{omacrongrave}{1E51}
\pdfglyphtounicode{omdeva}{0950}
\pdfglyphtounicode{omega}{03C9}
\pdfglyphtounicode{omega1}{03D6}
\pdfglyphtounicode{omegacyrillic}{0461}
\pdfglyphtounicode{omegalatinclosed}{0277}
\pdfglyphtounicode{omegaroundcyrillic}{047B}
\pdfglyphtounicode{omegatitlocyrillic}{047D}
\pdfglyphtounicode{omegatonos}{03CE}
\pdfglyphtounicode{omgujarati}{0AD0}
\pdfglyphtounicode{omicron}{03BF}
\pdfglyphtounicode{omicrontonos}{03CC}
\pdfglyphtounicode{omonospace}{FF4F}
\pdfglyphtounicode{one}{0031}
\pdfglyphtounicode{onearabic}{0661}
\pdfglyphtounicode{onebengali}{09E7}
\pdfglyphtounicode{onecircle}{2460}
\pdfglyphtounicode{onecircleinversesansserif}{278A}
\pdfglyphtounicode{onedeva}{0967}
\pdfglyphtounicode{onedotenleader}{2024}
\pdfglyphtounicode{oneeighth}{215B}
\pdfglyphtounicode{onefitted}{F6DC}
\pdfglyphtounicode{onegujarati}{0AE7}
\pdfglyphtounicode{onegurmukhi}{0A67}
\pdfglyphtounicode{onehackarabic}{0661}
\pdfglyphtounicode{onehalf}{00BD}
\pdfglyphtounicode{onehangzhou}{3021}
\pdfglyphtounicode{oneideographicparen}{3220}
\pdfglyphtounicode{oneinferior}{2081}
\pdfglyphtounicode{onemonospace}{FF11}
\pdfglyphtounicode{onenumeratorbengali}{09F4}
\pdfglyphtounicode{oneoldstyle}{F731}
\pdfglyphtounicode{oneparen}{2474}
\pdfglyphtounicode{oneperiod}{2488}
\pdfglyphtounicode{onepersian}{06F1}
\pdfglyphtounicode{onequarter}{00BC}
\pdfglyphtounicode{oneroman}{2170}
\pdfglyphtounicode{onesuperior}{00B9}
\pdfglyphtounicode{onethai}{0E51}
\pdfglyphtounicode{onethird}{2153}
\pdfglyphtounicode{oogonek}{01EB}
\pdfglyphtounicode{oogonekmacron}{01ED}
\pdfglyphtounicode{oogurmukhi}{0A13}
\pdfglyphtounicode{oomatragurmukhi}{0A4B}
\pdfglyphtounicode{oopen}{0254}
\pdfglyphtounicode{oparen}{24AA}
\pdfglyphtounicode{openbullet}{25E6}
\pdfglyphtounicode{option}{2325}
\pdfglyphtounicode{ordfeminine}{00AA}
\pdfglyphtounicode{ordmasculine}{00BA}
\pdfglyphtounicode{orthogonal}{221F}
\pdfglyphtounicode{oshortdeva}{0912}
\pdfglyphtounicode{oshortvowelsigndeva}{094A}
\pdfglyphtounicode{oslash}{00F8}
\pdfglyphtounicode{oslashacute}{01FF}
\pdfglyphtounicode{osmallhiragana}{3049}
\pdfglyphtounicode{osmallkatakana}{30A9}
\pdfglyphtounicode{osmallkatakanahalfwidth}{FF6B}
\pdfglyphtounicode{ostrokeacute}{01FF}
\pdfglyphtounicode{osuperior}{F6F0}
\pdfglyphtounicode{otcyrillic}{047F}
\pdfglyphtounicode{otilde}{00F5}
\pdfglyphtounicode{otildeacute}{1E4D}
\pdfglyphtounicode{otildedieresis}{1E4F}
\pdfglyphtounicode{oubopomofo}{3121}
\pdfglyphtounicode{overline}{203E}
\pdfglyphtounicode{overlinecenterline}{FE4A}
\pdfglyphtounicode{overlinecmb}{0305}
\pdfglyphtounicode{overlinedashed}{FE49}
\pdfglyphtounicode{overlinedblwavy}{FE4C}
\pdfglyphtounicode{overlinewavy}{FE4B}
\pdfglyphtounicode{overscore}{00AF}
\pdfglyphtounicode{ovowelsignbengali}{09CB}
\pdfglyphtounicode{ovowelsigndeva}{094B}
\pdfglyphtounicode{ovowelsigngujarati}{0ACB}
\pdfglyphtounicode{p}{0070}
\pdfglyphtounicode{paampssquare}{3380}
\pdfglyphtounicode{paasentosquare}{332B}
\pdfglyphtounicode{pabengali}{09AA}
\pdfglyphtounicode{pacute}{1E55}
\pdfglyphtounicode{padeva}{092A}
\pdfglyphtounicode{pagedown}{21DF}
\pdfglyphtounicode{pageup}{21DE}
\pdfglyphtounicode{pagujarati}{0AAA}
\pdfglyphtounicode{pagurmukhi}{0A2A}
\pdfglyphtounicode{pahiragana}{3071}
\pdfglyphtounicode{paiyannoithai}{0E2F}
\pdfglyphtounicode{pakatakana}{30D1}
\pdfglyphtounicode{palatalizationcyrilliccmb}{0484}
\pdfglyphtounicode{palochkacyrillic}{04C0}
\pdfglyphtounicode{pansioskorean}{317F}
\pdfglyphtounicode{paragraph}{00B6}
\pdfglyphtounicode{parallel}{2225}
\pdfglyphtounicode{parenleft}{0028}
\pdfglyphtounicode{parenleftaltonearabic}{FD3E}
\pdfglyphtounicode{parenleftbt}{F8ED}
\pdfglyphtounicode{parenleftex}{F8EC}
\pdfglyphtounicode{parenleftinferior}{208D}
\pdfglyphtounicode{parenleftmonospace}{FF08}
\pdfglyphtounicode{parenleftsmall}{FE59}
\pdfglyphtounicode{parenleftsuperior}{207D}
\pdfglyphtounicode{parenlefttp}{F8EB}
\pdfglyphtounicode{parenleftvertical}{FE35}
\pdfglyphtounicode{parenright}{0029}
\pdfglyphtounicode{parenrightaltonearabic}{FD3F}
\pdfglyphtounicode{parenrightbt}{F8F8}
\pdfglyphtounicode{parenrightex}{F8F7}
\pdfglyphtounicode{parenrightinferior}{208E}
\pdfglyphtounicode{parenrightmonospace}{FF09}
\pdfglyphtounicode{parenrightsmall}{FE5A}
\pdfglyphtounicode{parenrightsuperior}{207E}
\pdfglyphtounicode{parenrighttp}{F8F6}
\pdfglyphtounicode{parenrightvertical}{FE36}
\pdfglyphtounicode{partialdiff}{2202}
\pdfglyphtounicode{paseqhebrew}{05C0}
\pdfglyphtounicode{pashtahebrew}{0599}
\pdfglyphtounicode{pasquare}{33A9}
\pdfglyphtounicode{patah}{05B7}
\pdfglyphtounicode{patah11}{05B7}
\pdfglyphtounicode{patah1d}{05B7}
\pdfglyphtounicode{patah2a}{05B7}
\pdfglyphtounicode{patahhebrew}{05B7}
\pdfglyphtounicode{patahnarrowhebrew}{05B7}
\pdfglyphtounicode{patahquarterhebrew}{05B7}
\pdfglyphtounicode{patahwidehebrew}{05B7}
\pdfglyphtounicode{pazerhebrew}{05A1}
\pdfglyphtounicode{pbopomofo}{3106}
\pdfglyphtounicode{pcircle}{24DF}
\pdfglyphtounicode{pdotaccent}{1E57}
\pdfglyphtounicode{pe}{05E4}
\pdfglyphtounicode{pecyrillic}{043F}
\pdfglyphtounicode{pedagesh}{FB44}
\pdfglyphtounicode{pedageshhebrew}{FB44}
\pdfglyphtounicode{peezisquare}{333B}
\pdfglyphtounicode{pefinaldageshhebrew}{FB43}
\pdfglyphtounicode{peharabic}{067E}
\pdfglyphtounicode{peharmenian}{057A}
\pdfglyphtounicode{pehebrew}{05E4}
\pdfglyphtounicode{pehfinalarabic}{FB57}
\pdfglyphtounicode{pehinitialarabic}{FB58}
\pdfglyphtounicode{pehiragana}{307A}
\pdfglyphtounicode{pehmedialarabic}{FB59}
\pdfglyphtounicode{pekatakana}{30DA}
\pdfglyphtounicode{pemiddlehookcyrillic}{04A7}
\pdfglyphtounicode{perafehebrew}{FB4E}
\pdfglyphtounicode{percent}{0025}
\pdfglyphtounicode{percentarabic}{066A}
\pdfglyphtounicode{percentmonospace}{FF05}
\pdfglyphtounicode{percentsmall}{FE6A}
\pdfglyphtounicode{period}{002E}
\pdfglyphtounicode{periodarmenian}{0589}
\pdfglyphtounicode{periodcentered}{00B7}
\pdfglyphtounicode{periodhalfwidth}{FF61}
\pdfglyphtounicode{periodinferior}{F6E7}
\pdfglyphtounicode{periodmonospace}{FF0E}
\pdfglyphtounicode{periodsmall}{FE52}
\pdfglyphtounicode{periodsuperior}{F6E8}
\pdfglyphtounicode{perispomenigreekcmb}{0342}
\pdfglyphtounicode{perpendicular}{22A5}
\pdfglyphtounicode{perthousand}{2030}
\pdfglyphtounicode{peseta}{20A7}
\pdfglyphtounicode{pfsquare}{338A}
\pdfglyphtounicode{phabengali}{09AB}
\pdfglyphtounicode{phadeva}{092B}
\pdfglyphtounicode{phagujarati}{0AAB}
\pdfglyphtounicode{phagurmukhi}{0A2B}
\pdfglyphtounicode{phi}{03C6}
\pdfglyphtounicode{phi1}{03D5}
\pdfglyphtounicode{phieuphacirclekorean}{327A}
\pdfglyphtounicode{phieuphaparenkorean}{321A}
\pdfglyphtounicode{phieuphcirclekorean}{326C}
\pdfglyphtounicode{phieuphkorean}{314D}
\pdfglyphtounicode{phieuphparenkorean}{320C}
\pdfglyphtounicode{philatin}{0278}
\pdfglyphtounicode{phinthuthai}{0E3A}
\pdfglyphtounicode{phisymbolgreek}{03D5}
\pdfglyphtounicode{phook}{01A5}
\pdfglyphtounicode{phophanthai}{0E1E}
\pdfglyphtounicode{phophungthai}{0E1C}
\pdfglyphtounicode{phosamphaothai}{0E20}
\pdfglyphtounicode{pi}{03C0}
\pdfglyphtounicode{pieupacirclekorean}{3273}
\pdfglyphtounicode{pieupaparenkorean}{3213}
\pdfglyphtounicode{pieupcieuckorean}{3176}
\pdfglyphtounicode{pieupcirclekorean}{3265}
\pdfglyphtounicode{pieupkiyeokkorean}{3172}
\pdfglyphtounicode{pieupkorean}{3142}
\pdfglyphtounicode{pieupparenkorean}{3205}
\pdfglyphtounicode{pieupsioskiyeokkorean}{3174}
\pdfglyphtounicode{pieupsioskorean}{3144}
\pdfglyphtounicode{pieupsiostikeutkorean}{3175}
\pdfglyphtounicode{pieupthieuthkorean}{3177}
\pdfglyphtounicode{pieuptikeutkorean}{3173}
\pdfglyphtounicode{pihiragana}{3074}
\pdfglyphtounicode{pikatakana}{30D4}
\pdfglyphtounicode{pisymbolgreek}{03D6}
\pdfglyphtounicode{piwrarmenian}{0583}
\pdfglyphtounicode{plus}{002B}
\pdfglyphtounicode{plusbelowcmb}{031F}
\pdfglyphtounicode{pluscircle}{2295}
\pdfglyphtounicode{plusminus}{00B1}
\pdfglyphtounicode{plusmod}{02D6}
\pdfglyphtounicode{plusmonospace}{FF0B}
\pdfglyphtounicode{plussmall}{FE62}
\pdfglyphtounicode{plussuperior}{207A}
\pdfglyphtounicode{pmonospace}{FF50}
\pdfglyphtounicode{pmsquare}{33D8}
\pdfglyphtounicode{pohiragana}{307D}
\pdfglyphtounicode{pointingindexdownwhite}{261F}
\pdfglyphtounicode{pointingindexleftwhite}{261C}
\pdfglyphtounicode{pointingindexrightwhite}{261E}
\pdfglyphtounicode{pointingindexupwhite}{261D}
\pdfglyphtounicode{pokatakana}{30DD}
\pdfglyphtounicode{poplathai}{0E1B}
\pdfglyphtounicode{postalmark}{3012}
\pdfglyphtounicode{postalmarkface}{3020}
\pdfglyphtounicode{pparen}{24AB}
\pdfglyphtounicode{precedes}{227A}
\pdfglyphtounicode{prescription}{211E}
\pdfglyphtounicode{primemod}{02B9}
\pdfglyphtounicode{primereversed}{2035}
\pdfglyphtounicode{product}{220F}
\pdfglyphtounicode{projective}{2305}
\pdfglyphtounicode{prolongedkana}{30FC}
\pdfglyphtounicode{propellor}{2318}
\pdfglyphtounicode{propersubset}{2282}
\pdfglyphtounicode{propersuperset}{2283}
\pdfglyphtounicode{proportion}{2237}
\pdfglyphtounicode{proportional}{221D}
\pdfglyphtounicode{psi}{03C8}
\pdfglyphtounicode{psicyrillic}{0471}
\pdfglyphtounicode{psilipneumatacyrilliccmb}{0486}
\pdfglyphtounicode{pssquare}{33B0}
\pdfglyphtounicode{puhiragana}{3077}
\pdfglyphtounicode{pukatakana}{30D7}
\pdfglyphtounicode{pvsquare}{33B4}
\pdfglyphtounicode{pwsquare}{33BA}
\pdfglyphtounicode{q}{0071}
\pdfglyphtounicode{qadeva}{0958}
\pdfglyphtounicode{qadmahebrew}{05A8}
\pdfglyphtounicode{qafarabic}{0642}
\pdfglyphtounicode{qaffinalarabic}{FED6}
\pdfglyphtounicode{qafinitialarabic}{FED7}
\pdfglyphtounicode{qafmedialarabic}{FED8}
\pdfglyphtounicode{qamats}{05B8}
\pdfglyphtounicode{qamats10}{05B8}
\pdfglyphtounicode{qamats1a}{05B8}
\pdfglyphtounicode{qamats1c}{05B8}
\pdfglyphtounicode{qamats27}{05B8}
\pdfglyphtounicode{qamats29}{05B8}
\pdfglyphtounicode{qamats33}{05B8}
\pdfglyphtounicode{qamatsde}{05B8}
\pdfglyphtounicode{qamatshebrew}{05B8}
\pdfglyphtounicode{qamatsnarrowhebrew}{05B8}
\pdfglyphtounicode{qamatsqatanhebrew}{05B8}
\pdfglyphtounicode{qamatsqatannarrowhebrew}{05B8}
\pdfglyphtounicode{qamatsqatanquarterhebrew}{05B8}
\pdfglyphtounicode{qamatsqatanwidehebrew}{05B8}
\pdfglyphtounicode{qamatsquarterhebrew}{05B8}
\pdfglyphtounicode{qamatswidehebrew}{05B8}
\pdfglyphtounicode{qarneyparahebrew}{059F}
\pdfglyphtounicode{qbopomofo}{3111}
\pdfglyphtounicode{qcircle}{24E0}
\pdfglyphtounicode{qhook}{02A0}
\pdfglyphtounicode{qmonospace}{FF51}
\pdfglyphtounicode{qof}{05E7}
\pdfglyphtounicode{qofdagesh}{FB47}
\pdfglyphtounicode{qofdageshhebrew}{FB47}
% qofhatafpatah;05E7 05B2
% qofhatafpatahhebrew;05E7 05B2
% qofhatafsegol;05E7 05B1
% qofhatafsegolhebrew;05E7 05B1
\pdfglyphtounicode{qofhebrew}{05E7}
% qofhiriq;05E7 05B4
% qofhiriqhebrew;05E7 05B4
% qofholam;05E7 05B9
% qofholamhebrew;05E7 05B9
% qofpatah;05E7 05B7
% qofpatahhebrew;05E7 05B7
% qofqamats;05E7 05B8
% qofqamatshebrew;05E7 05B8
% qofqubuts;05E7 05BB
% qofqubutshebrew;05E7 05BB
% qofsegol;05E7 05B6
% qofsegolhebrew;05E7 05B6
% qofsheva;05E7 05B0
% qofshevahebrew;05E7 05B0
% qoftsere;05E7 05B5
% qoftserehebrew;05E7 05B5
\pdfglyphtounicode{qparen}{24AC}
\pdfglyphtounicode{quarternote}{2669}
\pdfglyphtounicode{qubuts}{05BB}
\pdfglyphtounicode{qubuts18}{05BB}
\pdfglyphtounicode{qubuts25}{05BB}
\pdfglyphtounicode{qubuts31}{05BB}
\pdfglyphtounicode{qubutshebrew}{05BB}
\pdfglyphtounicode{qubutsnarrowhebrew}{05BB}
\pdfglyphtounicode{qubutsquarterhebrew}{05BB}
\pdfglyphtounicode{qubutswidehebrew}{05BB}
\pdfglyphtounicode{question}{003F}
\pdfglyphtounicode{questionarabic}{061F}
\pdfglyphtounicode{questionarmenian}{055E}
\pdfglyphtounicode{questiondown}{00BF}
\pdfglyphtounicode{questiondownsmall}{F7BF}
\pdfglyphtounicode{questiongreek}{037E}
\pdfglyphtounicode{questionmonospace}{FF1F}
\pdfglyphtounicode{questionsmall}{F73F}
\pdfglyphtounicode{quotedbl}{0022}
\pdfglyphtounicode{quotedblbase}{201E}
\pdfglyphtounicode{quotedblleft}{201C}
\pdfglyphtounicode{quotedblmonospace}{FF02}
\pdfglyphtounicode{quotedblprime}{301E}
\pdfglyphtounicode{quotedblprimereversed}{301D}
\pdfglyphtounicode{quotedblright}{201D}
\pdfglyphtounicode{quoteleft}{2018}
\pdfglyphtounicode{quoteleftreversed}{201B}
\pdfglyphtounicode{quotereversed}{201B}
\pdfglyphtounicode{quoteright}{2019}
\pdfglyphtounicode{quoterightn}{0149}
\pdfglyphtounicode{quotesinglbase}{201A}
\pdfglyphtounicode{quotesingle}{0027}
\pdfglyphtounicode{quotesinglemonospace}{FF07}
\pdfglyphtounicode{r}{0072}
\pdfglyphtounicode{raarmenian}{057C}
\pdfglyphtounicode{rabengali}{09B0}
\pdfglyphtounicode{racute}{0155}
\pdfglyphtounicode{radeva}{0930}
\pdfglyphtounicode{radical}{221A}
\pdfglyphtounicode{radicalex}{F8E5}
\pdfglyphtounicode{radoverssquare}{33AE}
\pdfglyphtounicode{radoverssquaredsquare}{33AF}
\pdfglyphtounicode{radsquare}{33AD}
\pdfglyphtounicode{rafe}{05BF}
\pdfglyphtounicode{rafehebrew}{05BF}
\pdfglyphtounicode{ragujarati}{0AB0}
\pdfglyphtounicode{ragurmukhi}{0A30}
\pdfglyphtounicode{rahiragana}{3089}
\pdfglyphtounicode{rakatakana}{30E9}
\pdfglyphtounicode{rakatakanahalfwidth}{FF97}
\pdfglyphtounicode{ralowerdiagonalbengali}{09F1}
\pdfglyphtounicode{ramiddlediagonalbengali}{09F0}
\pdfglyphtounicode{ramshorn}{0264}
\pdfglyphtounicode{ratio}{2236}
\pdfglyphtounicode{rbopomofo}{3116}
\pdfglyphtounicode{rcaron}{0159}
\pdfglyphtounicode{rcedilla}{0157}
\pdfglyphtounicode{rcircle}{24E1}
\pdfglyphtounicode{rcommaaccent}{0157}
\pdfglyphtounicode{rdblgrave}{0211}
\pdfglyphtounicode{rdotaccent}{1E59}
\pdfglyphtounicode{rdotbelow}{1E5B}
\pdfglyphtounicode{rdotbelowmacron}{1E5D}
\pdfglyphtounicode{referencemark}{203B}
\pdfglyphtounicode{reflexsubset}{2286}
\pdfglyphtounicode{reflexsuperset}{2287}
\pdfglyphtounicode{registered}{00AE}
\pdfglyphtounicode{registersans}{F8E8}
\pdfglyphtounicode{registerserif}{F6DA}
\pdfglyphtounicode{reharabic}{0631}
\pdfglyphtounicode{reharmenian}{0580}
\pdfglyphtounicode{rehfinalarabic}{FEAE}
\pdfglyphtounicode{rehiragana}{308C}
% rehyehaleflamarabic;0631 FEF3 FE8E 0644
\pdfglyphtounicode{rekatakana}{30EC}
\pdfglyphtounicode{rekatakanahalfwidth}{FF9A}
\pdfglyphtounicode{resh}{05E8}
\pdfglyphtounicode{reshdageshhebrew}{FB48}
% reshhatafpatah;05E8 05B2
% reshhatafpatahhebrew;05E8 05B2
% reshhatafsegol;05E8 05B1
% reshhatafsegolhebrew;05E8 05B1
\pdfglyphtounicode{reshhebrew}{05E8}
% reshhiriq;05E8 05B4
% reshhiriqhebrew;05E8 05B4
% reshholam;05E8 05B9
% reshholamhebrew;05E8 05B9
% reshpatah;05E8 05B7
% reshpatahhebrew;05E8 05B7
% reshqamats;05E8 05B8
% reshqamatshebrew;05E8 05B8
% reshqubuts;05E8 05BB
% reshqubutshebrew;05E8 05BB
% reshsegol;05E8 05B6
% reshsegolhebrew;05E8 05B6
% reshsheva;05E8 05B0
% reshshevahebrew;05E8 05B0
% reshtsere;05E8 05B5
% reshtserehebrew;05E8 05B5
\pdfglyphtounicode{reversedtilde}{223D}
\pdfglyphtounicode{reviahebrew}{0597}
\pdfglyphtounicode{reviamugrashhebrew}{0597}
\pdfglyphtounicode{revlogicalnot}{2310}
\pdfglyphtounicode{rfishhook}{027E}
\pdfglyphtounicode{rfishhookreversed}{027F}
\pdfglyphtounicode{rhabengali}{09DD}
\pdfglyphtounicode{rhadeva}{095D}
\pdfglyphtounicode{rho}{03C1}
\pdfglyphtounicode{rhook}{027D}
\pdfglyphtounicode{rhookturned}{027B}
\pdfglyphtounicode{rhookturnedsuperior}{02B5}
\pdfglyphtounicode{rhosymbolgreek}{03F1}
\pdfglyphtounicode{rhotichookmod}{02DE}
\pdfglyphtounicode{rieulacirclekorean}{3271}
\pdfglyphtounicode{rieulaparenkorean}{3211}
\pdfglyphtounicode{rieulcirclekorean}{3263}
\pdfglyphtounicode{rieulhieuhkorean}{3140}
\pdfglyphtounicode{rieulkiyeokkorean}{313A}
\pdfglyphtounicode{rieulkiyeoksioskorean}{3169}
\pdfglyphtounicode{rieulkorean}{3139}
\pdfglyphtounicode{rieulmieumkorean}{313B}
\pdfglyphtounicode{rieulpansioskorean}{316C}
\pdfglyphtounicode{rieulparenkorean}{3203}
\pdfglyphtounicode{rieulphieuphkorean}{313F}
\pdfglyphtounicode{rieulpieupkorean}{313C}
\pdfglyphtounicode{rieulpieupsioskorean}{316B}
\pdfglyphtounicode{rieulsioskorean}{313D}
\pdfglyphtounicode{rieulthieuthkorean}{313E}
\pdfglyphtounicode{rieultikeutkorean}{316A}
\pdfglyphtounicode{rieulyeorinhieuhkorean}{316D}
\pdfglyphtounicode{rightangle}{221F}
\pdfglyphtounicode{righttackbelowcmb}{0319}
\pdfglyphtounicode{righttriangle}{22BF}
\pdfglyphtounicode{rihiragana}{308A}
\pdfglyphtounicode{rikatakana}{30EA}
\pdfglyphtounicode{rikatakanahalfwidth}{FF98}
\pdfglyphtounicode{ring}{02DA}
\pdfglyphtounicode{ringbelowcmb}{0325}
\pdfglyphtounicode{ringcmb}{030A}
\pdfglyphtounicode{ringhalfleft}{02BF}
\pdfglyphtounicode{ringhalfleftarmenian}{0559}
\pdfglyphtounicode{ringhalfleftbelowcmb}{031C}
\pdfglyphtounicode{ringhalfleftcentered}{02D3}
\pdfglyphtounicode{ringhalfright}{02BE}
\pdfglyphtounicode{ringhalfrightbelowcmb}{0339}
\pdfglyphtounicode{ringhalfrightcentered}{02D2}
\pdfglyphtounicode{rinvertedbreve}{0213}
\pdfglyphtounicode{rittorusquare}{3351}
\pdfglyphtounicode{rlinebelow}{1E5F}
\pdfglyphtounicode{rlongleg}{027C}
\pdfglyphtounicode{rlonglegturned}{027A}
\pdfglyphtounicode{rmonospace}{FF52}
\pdfglyphtounicode{rohiragana}{308D}
\pdfglyphtounicode{rokatakana}{30ED}
\pdfglyphtounicode{rokatakanahalfwidth}{FF9B}
\pdfglyphtounicode{roruathai}{0E23}
\pdfglyphtounicode{rparen}{24AD}
\pdfglyphtounicode{rrabengali}{09DC}
\pdfglyphtounicode{rradeva}{0931}
\pdfglyphtounicode{rragurmukhi}{0A5C}
\pdfglyphtounicode{rreharabic}{0691}
\pdfglyphtounicode{rrehfinalarabic}{FB8D}
\pdfglyphtounicode{rrvocalicbengali}{09E0}
\pdfglyphtounicode{rrvocalicdeva}{0960}
\pdfglyphtounicode{rrvocalicgujarati}{0AE0}
\pdfglyphtounicode{rrvocalicvowelsignbengali}{09C4}
\pdfglyphtounicode{rrvocalicvowelsigndeva}{0944}
\pdfglyphtounicode{rrvocalicvowelsigngujarati}{0AC4}
\pdfglyphtounicode{rsuperior}{F6F1}
\pdfglyphtounicode{rtblock}{2590}
\pdfglyphtounicode{rturned}{0279}
\pdfglyphtounicode{rturnedsuperior}{02B4}
\pdfglyphtounicode{ruhiragana}{308B}
\pdfglyphtounicode{rukatakana}{30EB}
\pdfglyphtounicode{rukatakanahalfwidth}{FF99}
\pdfglyphtounicode{rupeemarkbengali}{09F2}
\pdfglyphtounicode{rupeesignbengali}{09F3}
\pdfglyphtounicode{rupiah}{F6DD}
\pdfglyphtounicode{ruthai}{0E24}
\pdfglyphtounicode{rvocalicbengali}{098B}
\pdfglyphtounicode{rvocalicdeva}{090B}
\pdfglyphtounicode{rvocalicgujarati}{0A8B}
\pdfglyphtounicode{rvocalicvowelsignbengali}{09C3}
\pdfglyphtounicode{rvocalicvowelsigndeva}{0943}
\pdfglyphtounicode{rvocalicvowelsigngujarati}{0AC3}
\pdfglyphtounicode{s}{0073}
\pdfglyphtounicode{sabengali}{09B8}
\pdfglyphtounicode{sacute}{015B}
\pdfglyphtounicode{sacutedotaccent}{1E65}
\pdfglyphtounicode{sadarabic}{0635}
\pdfglyphtounicode{sadeva}{0938}
\pdfglyphtounicode{sadfinalarabic}{FEBA}
\pdfglyphtounicode{sadinitialarabic}{FEBB}
\pdfglyphtounicode{sadmedialarabic}{FEBC}
\pdfglyphtounicode{sagujarati}{0AB8}
\pdfglyphtounicode{sagurmukhi}{0A38}
\pdfglyphtounicode{sahiragana}{3055}
\pdfglyphtounicode{sakatakana}{30B5}
\pdfglyphtounicode{sakatakanahalfwidth}{FF7B}
\pdfglyphtounicode{sallallahoualayhewasallamarabic}{FDFA}
\pdfglyphtounicode{samekh}{05E1}
\pdfglyphtounicode{samekhdagesh}{FB41}
\pdfglyphtounicode{samekhdageshhebrew}{FB41}
\pdfglyphtounicode{samekhhebrew}{05E1}
\pdfglyphtounicode{saraaathai}{0E32}
\pdfglyphtounicode{saraaethai}{0E41}
\pdfglyphtounicode{saraaimaimalaithai}{0E44}
\pdfglyphtounicode{saraaimaimuanthai}{0E43}
\pdfglyphtounicode{saraamthai}{0E33}
\pdfglyphtounicode{saraathai}{0E30}
\pdfglyphtounicode{saraethai}{0E40}
\pdfglyphtounicode{saraiileftthai}{F886}
\pdfglyphtounicode{saraiithai}{0E35}
\pdfglyphtounicode{saraileftthai}{F885}
\pdfglyphtounicode{saraithai}{0E34}
\pdfglyphtounicode{saraothai}{0E42}
\pdfglyphtounicode{saraueeleftthai}{F888}
\pdfglyphtounicode{saraueethai}{0E37}
\pdfglyphtounicode{saraueleftthai}{F887}
\pdfglyphtounicode{sarauethai}{0E36}
\pdfglyphtounicode{sarauthai}{0E38}
\pdfglyphtounicode{sarauuthai}{0E39}
\pdfglyphtounicode{sbopomofo}{3119}
\pdfglyphtounicode{scaron}{0161}
\pdfglyphtounicode{scarondotaccent}{1E67}
\pdfglyphtounicode{scedilla}{015F}
\pdfglyphtounicode{schwa}{0259}
\pdfglyphtounicode{schwacyrillic}{04D9}
\pdfglyphtounicode{schwadieresiscyrillic}{04DB}
\pdfglyphtounicode{schwahook}{025A}
\pdfglyphtounicode{scircle}{24E2}
\pdfglyphtounicode{scircumflex}{015D}
\pdfglyphtounicode{scommaaccent}{0219}
\pdfglyphtounicode{sdotaccent}{1E61}
\pdfglyphtounicode{sdotbelow}{1E63}
\pdfglyphtounicode{sdotbelowdotaccent}{1E69}
\pdfglyphtounicode{seagullbelowcmb}{033C}
\pdfglyphtounicode{second}{2033}
\pdfglyphtounicode{secondtonechinese}{02CA}
\pdfglyphtounicode{section}{00A7}
\pdfglyphtounicode{seenarabic}{0633}
\pdfglyphtounicode{seenfinalarabic}{FEB2}
\pdfglyphtounicode{seeninitialarabic}{FEB3}
\pdfglyphtounicode{seenmedialarabic}{FEB4}
\pdfglyphtounicode{segol}{05B6}
\pdfglyphtounicode{segol13}{05B6}
\pdfglyphtounicode{segol1f}{05B6}
\pdfglyphtounicode{segol2c}{05B6}
\pdfglyphtounicode{segolhebrew}{05B6}
\pdfglyphtounicode{segolnarrowhebrew}{05B6}
\pdfglyphtounicode{segolquarterhebrew}{05B6}
\pdfglyphtounicode{segoltahebrew}{0592}
\pdfglyphtounicode{segolwidehebrew}{05B6}
\pdfglyphtounicode{seharmenian}{057D}
\pdfglyphtounicode{sehiragana}{305B}
\pdfglyphtounicode{sekatakana}{30BB}
\pdfglyphtounicode{sekatakanahalfwidth}{FF7E}
\pdfglyphtounicode{semicolon}{003B}
\pdfglyphtounicode{semicolonarabic}{061B}
\pdfglyphtounicode{semicolonmonospace}{FF1B}
\pdfglyphtounicode{semicolonsmall}{FE54}
\pdfglyphtounicode{semivoicedmarkkana}{309C}
\pdfglyphtounicode{semivoicedmarkkanahalfwidth}{FF9F}
\pdfglyphtounicode{sentisquare}{3322}
\pdfglyphtounicode{sentosquare}{3323}
\pdfglyphtounicode{seven}{0037}
\pdfglyphtounicode{sevenarabic}{0667}
\pdfglyphtounicode{sevenbengali}{09ED}
\pdfglyphtounicode{sevencircle}{2466}
\pdfglyphtounicode{sevencircleinversesansserif}{2790}
\pdfglyphtounicode{sevendeva}{096D}
\pdfglyphtounicode{seveneighths}{215E}
\pdfglyphtounicode{sevengujarati}{0AED}
\pdfglyphtounicode{sevengurmukhi}{0A6D}
\pdfglyphtounicode{sevenhackarabic}{0667}
\pdfglyphtounicode{sevenhangzhou}{3027}
\pdfglyphtounicode{sevenideographicparen}{3226}
\pdfglyphtounicode{seveninferior}{2087}
\pdfglyphtounicode{sevenmonospace}{FF17}
\pdfglyphtounicode{sevenoldstyle}{F737}
\pdfglyphtounicode{sevenparen}{247A}
\pdfglyphtounicode{sevenperiod}{248E}
\pdfglyphtounicode{sevenpersian}{06F7}
\pdfglyphtounicode{sevenroman}{2176}
\pdfglyphtounicode{sevensuperior}{2077}
\pdfglyphtounicode{seventeencircle}{2470}
\pdfglyphtounicode{seventeenparen}{2484}
\pdfglyphtounicode{seventeenperiod}{2498}
\pdfglyphtounicode{seventhai}{0E57}
\pdfglyphtounicode{sfthyphen}{00AD}
\pdfglyphtounicode{shaarmenian}{0577}
\pdfglyphtounicode{shabengali}{09B6}
\pdfglyphtounicode{shacyrillic}{0448}
\pdfglyphtounicode{shaddaarabic}{0651}
\pdfglyphtounicode{shaddadammaarabic}{FC61}
\pdfglyphtounicode{shaddadammatanarabic}{FC5E}
\pdfglyphtounicode{shaddafathaarabic}{FC60}
% shaddafathatanarabic;0651 064B
\pdfglyphtounicode{shaddakasraarabic}{FC62}
\pdfglyphtounicode{shaddakasratanarabic}{FC5F}
\pdfglyphtounicode{shade}{2592}
\pdfglyphtounicode{shadedark}{2593}
\pdfglyphtounicode{shadelight}{2591}
\pdfglyphtounicode{shademedium}{2592}
\pdfglyphtounicode{shadeva}{0936}
\pdfglyphtounicode{shagujarati}{0AB6}
\pdfglyphtounicode{shagurmukhi}{0A36}
\pdfglyphtounicode{shalshelethebrew}{0593}
\pdfglyphtounicode{shbopomofo}{3115}
\pdfglyphtounicode{shchacyrillic}{0449}
\pdfglyphtounicode{sheenarabic}{0634}
\pdfglyphtounicode{sheenfinalarabic}{FEB6}
\pdfglyphtounicode{sheeninitialarabic}{FEB7}
\pdfglyphtounicode{sheenmedialarabic}{FEB8}
\pdfglyphtounicode{sheicoptic}{03E3}
\pdfglyphtounicode{sheqel}{20AA}
\pdfglyphtounicode{sheqelhebrew}{20AA}
\pdfglyphtounicode{sheva}{05B0}
\pdfglyphtounicode{sheva115}{05B0}
\pdfglyphtounicode{sheva15}{05B0}
\pdfglyphtounicode{sheva22}{05B0}
\pdfglyphtounicode{sheva2e}{05B0}
\pdfglyphtounicode{shevahebrew}{05B0}
\pdfglyphtounicode{shevanarrowhebrew}{05B0}
\pdfglyphtounicode{shevaquarterhebrew}{05B0}
\pdfglyphtounicode{shevawidehebrew}{05B0}
\pdfglyphtounicode{shhacyrillic}{04BB}
\pdfglyphtounicode{shimacoptic}{03ED}
\pdfglyphtounicode{shin}{05E9}
\pdfglyphtounicode{shindagesh}{FB49}
\pdfglyphtounicode{shindageshhebrew}{FB49}
\pdfglyphtounicode{shindageshshindot}{FB2C}
\pdfglyphtounicode{shindageshshindothebrew}{FB2C}
\pdfglyphtounicode{shindageshsindot}{FB2D}
\pdfglyphtounicode{shindageshsindothebrew}{FB2D}
\pdfglyphtounicode{shindothebrew}{05C1}
\pdfglyphtounicode{shinhebrew}{05E9}
\pdfglyphtounicode{shinshindot}{FB2A}
\pdfglyphtounicode{shinshindothebrew}{FB2A}
\pdfglyphtounicode{shinsindot}{FB2B}
\pdfglyphtounicode{shinsindothebrew}{FB2B}
\pdfglyphtounicode{shook}{0282}
\pdfglyphtounicode{sigma}{03C3}
\pdfglyphtounicode{sigma1}{03C2}
\pdfglyphtounicode{sigmafinal}{03C2}
\pdfglyphtounicode{sigmalunatesymbolgreek}{03F2}
\pdfglyphtounicode{sihiragana}{3057}
\pdfglyphtounicode{sikatakana}{30B7}
\pdfglyphtounicode{sikatakanahalfwidth}{FF7C}
\pdfglyphtounicode{siluqhebrew}{05BD}
\pdfglyphtounicode{siluqlefthebrew}{05BD}
\pdfglyphtounicode{similar}{223C}
\pdfglyphtounicode{sindothebrew}{05C2}
\pdfglyphtounicode{siosacirclekorean}{3274}
\pdfglyphtounicode{siosaparenkorean}{3214}
\pdfglyphtounicode{sioscieuckorean}{317E}
\pdfglyphtounicode{sioscirclekorean}{3266}
\pdfglyphtounicode{sioskiyeokkorean}{317A}
\pdfglyphtounicode{sioskorean}{3145}
\pdfglyphtounicode{siosnieunkorean}{317B}
\pdfglyphtounicode{siosparenkorean}{3206}
\pdfglyphtounicode{siospieupkorean}{317D}
\pdfglyphtounicode{siostikeutkorean}{317C}
\pdfglyphtounicode{six}{0036}
\pdfglyphtounicode{sixarabic}{0666}
\pdfglyphtounicode{sixbengali}{09EC}
\pdfglyphtounicode{sixcircle}{2465}
\pdfglyphtounicode{sixcircleinversesansserif}{278F}
\pdfglyphtounicode{sixdeva}{096C}
\pdfglyphtounicode{sixgujarati}{0AEC}
\pdfglyphtounicode{sixgurmukhi}{0A6C}
\pdfglyphtounicode{sixhackarabic}{0666}
\pdfglyphtounicode{sixhangzhou}{3026}
\pdfglyphtounicode{sixideographicparen}{3225}
\pdfglyphtounicode{sixinferior}{2086}
\pdfglyphtounicode{sixmonospace}{FF16}
\pdfglyphtounicode{sixoldstyle}{F736}
\pdfglyphtounicode{sixparen}{2479}
\pdfglyphtounicode{sixperiod}{248D}
\pdfglyphtounicode{sixpersian}{06F6}
\pdfglyphtounicode{sixroman}{2175}
\pdfglyphtounicode{sixsuperior}{2076}
\pdfglyphtounicode{sixteencircle}{246F}
\pdfglyphtounicode{sixteencurrencydenominatorbengali}{09F9}
\pdfglyphtounicode{sixteenparen}{2483}
\pdfglyphtounicode{sixteenperiod}{2497}
\pdfglyphtounicode{sixthai}{0E56}
\pdfglyphtounicode{slash}{002F}
\pdfglyphtounicode{slashmonospace}{FF0F}
\pdfglyphtounicode{slong}{017F}
\pdfglyphtounicode{slongdotaccent}{1E9B}
\pdfglyphtounicode{smileface}{263A}
\pdfglyphtounicode{smonospace}{FF53}
\pdfglyphtounicode{sofpasuqhebrew}{05C3}
\pdfglyphtounicode{softhyphen}{00AD}
\pdfglyphtounicode{softsigncyrillic}{044C}
\pdfglyphtounicode{sohiragana}{305D}
\pdfglyphtounicode{sokatakana}{30BD}
\pdfglyphtounicode{sokatakanahalfwidth}{FF7F}
\pdfglyphtounicode{soliduslongoverlaycmb}{0338}
\pdfglyphtounicode{solidusshortoverlaycmb}{0337}
\pdfglyphtounicode{sorusithai}{0E29}
\pdfglyphtounicode{sosalathai}{0E28}
\pdfglyphtounicode{sosothai}{0E0B}
\pdfglyphtounicode{sosuathai}{0E2A}
\pdfglyphtounicode{space}{0020}
\pdfglyphtounicode{spacehackarabic}{0020}
\pdfglyphtounicode{spade}{2660}
\pdfglyphtounicode{spadesuitblack}{2660}
\pdfglyphtounicode{spadesuitwhite}{2664}
\pdfglyphtounicode{sparen}{24AE}
\pdfglyphtounicode{squarebelowcmb}{033B}
\pdfglyphtounicode{squarecc}{33C4}
\pdfglyphtounicode{squarecm}{339D}
\pdfglyphtounicode{squarediagonalcrosshatchfill}{25A9}
\pdfglyphtounicode{squarehorizontalfill}{25A4}
\pdfglyphtounicode{squarekg}{338F}
\pdfglyphtounicode{squarekm}{339E}
\pdfglyphtounicode{squarekmcapital}{33CE}
\pdfglyphtounicode{squareln}{33D1}
\pdfglyphtounicode{squarelog}{33D2}
\pdfglyphtounicode{squaremg}{338E}
\pdfglyphtounicode{squaremil}{33D5}
\pdfglyphtounicode{squaremm}{339C}
\pdfglyphtounicode{squaremsquared}{33A1}
\pdfglyphtounicode{squareorthogonalcrosshatchfill}{25A6}
\pdfglyphtounicode{squareupperlefttolowerrightfill}{25A7}
\pdfglyphtounicode{squareupperrighttolowerleftfill}{25A8}
\pdfglyphtounicode{squareverticalfill}{25A5}
\pdfglyphtounicode{squarewhitewithsmallblack}{25A3}
\pdfglyphtounicode{srsquare}{33DB}
\pdfglyphtounicode{ssabengali}{09B7}
\pdfglyphtounicode{ssadeva}{0937}
\pdfglyphtounicode{ssagujarati}{0AB7}
\pdfglyphtounicode{ssangcieuckorean}{3149}
\pdfglyphtounicode{ssanghieuhkorean}{3185}
\pdfglyphtounicode{ssangieungkorean}{3180}
\pdfglyphtounicode{ssangkiyeokkorean}{3132}
\pdfglyphtounicode{ssangnieunkorean}{3165}
\pdfglyphtounicode{ssangpieupkorean}{3143}
\pdfglyphtounicode{ssangsioskorean}{3146}
\pdfglyphtounicode{ssangtikeutkorean}{3138}
\pdfglyphtounicode{ssuperior}{F6F2}
\pdfglyphtounicode{sterling}{00A3}
\pdfglyphtounicode{sterlingmonospace}{FFE1}
\pdfglyphtounicode{strokelongoverlaycmb}{0336}
\pdfglyphtounicode{strokeshortoverlaycmb}{0335}
\pdfglyphtounicode{subset}{2282}
\pdfglyphtounicode{subsetnotequal}{228A}
\pdfglyphtounicode{subsetorequal}{2286}
\pdfglyphtounicode{succeeds}{227B}
\pdfglyphtounicode{suchthat}{220B}
\pdfglyphtounicode{suhiragana}{3059}
\pdfglyphtounicode{sukatakana}{30B9}
\pdfglyphtounicode{sukatakanahalfwidth}{FF7D}
\pdfglyphtounicode{sukunarabic}{0652}
\pdfglyphtounicode{summation}{2211}
\pdfglyphtounicode{sun}{263C}
\pdfglyphtounicode{superset}{2283}
\pdfglyphtounicode{supersetnotequal}{228B}
\pdfglyphtounicode{supersetorequal}{2287}
\pdfglyphtounicode{svsquare}{33DC}
\pdfglyphtounicode{syouwaerasquare}{337C}
\pdfglyphtounicode{t}{0074}
\pdfglyphtounicode{tabengali}{09A4}
\pdfglyphtounicode{tackdown}{22A4}
\pdfglyphtounicode{tackleft}{22A3}
\pdfglyphtounicode{tadeva}{0924}
\pdfglyphtounicode{tagujarati}{0AA4}
\pdfglyphtounicode{tagurmukhi}{0A24}
\pdfglyphtounicode{taharabic}{0637}
\pdfglyphtounicode{tahfinalarabic}{FEC2}
\pdfglyphtounicode{tahinitialarabic}{FEC3}
\pdfglyphtounicode{tahiragana}{305F}
\pdfglyphtounicode{tahmedialarabic}{FEC4}
\pdfglyphtounicode{taisyouerasquare}{337D}
\pdfglyphtounicode{takatakana}{30BF}
\pdfglyphtounicode{takatakanahalfwidth}{FF80}
\pdfglyphtounicode{tatweelarabic}{0640}
\pdfglyphtounicode{tau}{03C4}
\pdfglyphtounicode{tav}{05EA}
\pdfglyphtounicode{tavdages}{FB4A}
\pdfglyphtounicode{tavdagesh}{FB4A}
\pdfglyphtounicode{tavdageshhebrew}{FB4A}
\pdfglyphtounicode{tavhebrew}{05EA}
\pdfglyphtounicode{tbar}{0167}
\pdfglyphtounicode{tbopomofo}{310A}
\pdfglyphtounicode{tcaron}{0165}
\pdfglyphtounicode{tccurl}{02A8}
\pdfglyphtounicode{tcedilla}{0163}
\pdfglyphtounicode{tcheharabic}{0686}
\pdfglyphtounicode{tchehfinalarabic}{FB7B}
\pdfglyphtounicode{tchehinitialarabic}{FB7C}
\pdfglyphtounicode{tchehmedialarabic}{FB7D}
% tchehmeeminitialarabic;FB7C FEE4
\pdfglyphtounicode{tcircle}{24E3}
\pdfglyphtounicode{tcircumflexbelow}{1E71}
\pdfglyphtounicode{tcommaaccent}{0163}
\pdfglyphtounicode{tdieresis}{1E97}
\pdfglyphtounicode{tdotaccent}{1E6B}
\pdfglyphtounicode{tdotbelow}{1E6D}
\pdfglyphtounicode{tecyrillic}{0442}
\pdfglyphtounicode{tedescendercyrillic}{04AD}
\pdfglyphtounicode{teharabic}{062A}
\pdfglyphtounicode{tehfinalarabic}{FE96}
\pdfglyphtounicode{tehhahinitialarabic}{FCA2}
\pdfglyphtounicode{tehhahisolatedarabic}{FC0C}
\pdfglyphtounicode{tehinitialarabic}{FE97}
\pdfglyphtounicode{tehiragana}{3066}
\pdfglyphtounicode{tehjeeminitialarabic}{FCA1}
\pdfglyphtounicode{tehjeemisolatedarabic}{FC0B}
\pdfglyphtounicode{tehmarbutaarabic}{0629}
\pdfglyphtounicode{tehmarbutafinalarabic}{FE94}
\pdfglyphtounicode{tehmedialarabic}{FE98}
\pdfglyphtounicode{tehmeeminitialarabic}{FCA4}
\pdfglyphtounicode{tehmeemisolatedarabic}{FC0E}
\pdfglyphtounicode{tehnoonfinalarabic}{FC73}
\pdfglyphtounicode{tekatakana}{30C6}
\pdfglyphtounicode{tekatakanahalfwidth}{FF83}
\pdfglyphtounicode{telephone}{2121}
\pdfglyphtounicode{telephoneblack}{260E}
\pdfglyphtounicode{telishagedolahebrew}{05A0}
\pdfglyphtounicode{telishaqetanahebrew}{05A9}
\pdfglyphtounicode{tencircle}{2469}
\pdfglyphtounicode{tenideographicparen}{3229}
\pdfglyphtounicode{tenparen}{247D}
\pdfglyphtounicode{tenperiod}{2491}
\pdfglyphtounicode{tenroman}{2179}
\pdfglyphtounicode{tesh}{02A7}
\pdfglyphtounicode{tet}{05D8}
\pdfglyphtounicode{tetdagesh}{FB38}
\pdfglyphtounicode{tetdageshhebrew}{FB38}
\pdfglyphtounicode{tethebrew}{05D8}
\pdfglyphtounicode{tetsecyrillic}{04B5}
\pdfglyphtounicode{tevirhebrew}{059B}
\pdfglyphtounicode{tevirlefthebrew}{059B}
\pdfglyphtounicode{thabengali}{09A5}
\pdfglyphtounicode{thadeva}{0925}
\pdfglyphtounicode{thagujarati}{0AA5}
\pdfglyphtounicode{thagurmukhi}{0A25}
\pdfglyphtounicode{thalarabic}{0630}
\pdfglyphtounicode{thalfinalarabic}{FEAC}
\pdfglyphtounicode{thanthakhatlowleftthai}{F898}
\pdfglyphtounicode{thanthakhatlowrightthai}{F897}
\pdfglyphtounicode{thanthakhatthai}{0E4C}
\pdfglyphtounicode{thanthakhatupperleftthai}{F896}
\pdfglyphtounicode{theharabic}{062B}
\pdfglyphtounicode{thehfinalarabic}{FE9A}
\pdfglyphtounicode{thehinitialarabic}{FE9B}
\pdfglyphtounicode{thehmedialarabic}{FE9C}
\pdfglyphtounicode{thereexists}{2203}
\pdfglyphtounicode{therefore}{2234}
\pdfglyphtounicode{theta}{03B8}
\pdfglyphtounicode{theta1}{03D1}
\pdfglyphtounicode{thetasymbolgreek}{03D1}
\pdfglyphtounicode{thieuthacirclekorean}{3279}
\pdfglyphtounicode{thieuthaparenkorean}{3219}
\pdfglyphtounicode{thieuthcirclekorean}{326B}
\pdfglyphtounicode{thieuthkorean}{314C}
\pdfglyphtounicode{thieuthparenkorean}{320B}
\pdfglyphtounicode{thirteencircle}{246C}
\pdfglyphtounicode{thirteenparen}{2480}
\pdfglyphtounicode{thirteenperiod}{2494}
\pdfglyphtounicode{thonangmonthothai}{0E11}
\pdfglyphtounicode{thook}{01AD}
\pdfglyphtounicode{thophuthaothai}{0E12}
\pdfglyphtounicode{thorn}{00FE}
\pdfglyphtounicode{thothahanthai}{0E17}
\pdfglyphtounicode{thothanthai}{0E10}
\pdfglyphtounicode{thothongthai}{0E18}
\pdfglyphtounicode{thothungthai}{0E16}
\pdfglyphtounicode{thousandcyrillic}{0482}
\pdfglyphtounicode{thousandsseparatorarabic}{066C}
\pdfglyphtounicode{thousandsseparatorpersian}{066C}
\pdfglyphtounicode{three}{0033}
\pdfglyphtounicode{threearabic}{0663}
\pdfglyphtounicode{threebengali}{09E9}
\pdfglyphtounicode{threecircle}{2462}
\pdfglyphtounicode{threecircleinversesansserif}{278C}
\pdfglyphtounicode{threedeva}{0969}
\pdfglyphtounicode{threeeighths}{215C}
\pdfglyphtounicode{threegujarati}{0AE9}
\pdfglyphtounicode{threegurmukhi}{0A69}
\pdfglyphtounicode{threehackarabic}{0663}
\pdfglyphtounicode{threehangzhou}{3023}
\pdfglyphtounicode{threeideographicparen}{3222}
\pdfglyphtounicode{threeinferior}{2083}
\pdfglyphtounicode{threemonospace}{FF13}
\pdfglyphtounicode{threenumeratorbengali}{09F6}
\pdfglyphtounicode{threeoldstyle}{F733}
\pdfglyphtounicode{threeparen}{2476}
\pdfglyphtounicode{threeperiod}{248A}
\pdfglyphtounicode{threepersian}{06F3}
\pdfglyphtounicode{threequarters}{00BE}
\pdfglyphtounicode{threequartersemdash}{F6DE}
\pdfglyphtounicode{threeroman}{2172}
\pdfglyphtounicode{threesuperior}{00B3}
\pdfglyphtounicode{threethai}{0E53}
\pdfglyphtounicode{thzsquare}{3394}
\pdfglyphtounicode{tihiragana}{3061}
\pdfglyphtounicode{tikatakana}{30C1}
\pdfglyphtounicode{tikatakanahalfwidth}{FF81}
\pdfglyphtounicode{tikeutacirclekorean}{3270}
\pdfglyphtounicode{tikeutaparenkorean}{3210}
\pdfglyphtounicode{tikeutcirclekorean}{3262}
\pdfglyphtounicode{tikeutkorean}{3137}
\pdfglyphtounicode{tikeutparenkorean}{3202}
\pdfglyphtounicode{tilde}{02DC}
\pdfglyphtounicode{tildebelowcmb}{0330}
\pdfglyphtounicode{tildecmb}{0303}
\pdfglyphtounicode{tildecomb}{0303}
\pdfglyphtounicode{tildedoublecmb}{0360}
\pdfglyphtounicode{tildeoperator}{223C}
\pdfglyphtounicode{tildeoverlaycmb}{0334}
\pdfglyphtounicode{tildeverticalcmb}{033E}
\pdfglyphtounicode{timescircle}{2297}
\pdfglyphtounicode{tipehahebrew}{0596}
\pdfglyphtounicode{tipehalefthebrew}{0596}
\pdfglyphtounicode{tippigurmukhi}{0A70}
\pdfglyphtounicode{titlocyrilliccmb}{0483}
\pdfglyphtounicode{tiwnarmenian}{057F}
\pdfglyphtounicode{tlinebelow}{1E6F}
\pdfglyphtounicode{tmonospace}{FF54}
\pdfglyphtounicode{toarmenian}{0569}
\pdfglyphtounicode{tohiragana}{3068}
\pdfglyphtounicode{tokatakana}{30C8}
\pdfglyphtounicode{tokatakanahalfwidth}{FF84}
\pdfglyphtounicode{tonebarextrahighmod}{02E5}
\pdfglyphtounicode{tonebarextralowmod}{02E9}
\pdfglyphtounicode{tonebarhighmod}{02E6}
\pdfglyphtounicode{tonebarlowmod}{02E8}
\pdfglyphtounicode{tonebarmidmod}{02E7}
\pdfglyphtounicode{tonefive}{01BD}
\pdfglyphtounicode{tonesix}{0185}
\pdfglyphtounicode{tonetwo}{01A8}
\pdfglyphtounicode{tonos}{0384}
\pdfglyphtounicode{tonsquare}{3327}
\pdfglyphtounicode{topatakthai}{0E0F}
\pdfglyphtounicode{tortoiseshellbracketleft}{3014}
\pdfglyphtounicode{tortoiseshellbracketleftsmall}{FE5D}
\pdfglyphtounicode{tortoiseshellbracketleftvertical}{FE39}
\pdfglyphtounicode{tortoiseshellbracketright}{3015}
\pdfglyphtounicode{tortoiseshellbracketrightsmall}{FE5E}
\pdfglyphtounicode{tortoiseshellbracketrightvertical}{FE3A}
\pdfglyphtounicode{totaothai}{0E15}
\pdfglyphtounicode{tpalatalhook}{01AB}
\pdfglyphtounicode{tparen}{24AF}
\pdfglyphtounicode{trademark}{2122}
\pdfglyphtounicode{trademarksans}{F8EA}
\pdfglyphtounicode{trademarkserif}{F6DB}
\pdfglyphtounicode{tretroflexhook}{0288}
\pdfglyphtounicode{triagdn}{25BC}
\pdfglyphtounicode{triaglf}{25C4}
\pdfglyphtounicode{triagrt}{25BA}
\pdfglyphtounicode{triagup}{25B2}
\pdfglyphtounicode{ts}{02A6}
\pdfglyphtounicode{tsadi}{05E6}
\pdfglyphtounicode{tsadidagesh}{FB46}
\pdfglyphtounicode{tsadidageshhebrew}{FB46}
\pdfglyphtounicode{tsadihebrew}{05E6}
\pdfglyphtounicode{tsecyrillic}{0446}
\pdfglyphtounicode{tsere}{05B5}
\pdfglyphtounicode{tsere12}{05B5}
\pdfglyphtounicode{tsere1e}{05B5}
\pdfglyphtounicode{tsere2b}{05B5}
\pdfglyphtounicode{tserehebrew}{05B5}
\pdfglyphtounicode{tserenarrowhebrew}{05B5}
\pdfglyphtounicode{tserequarterhebrew}{05B5}
\pdfglyphtounicode{tserewidehebrew}{05B5}
\pdfglyphtounicode{tshecyrillic}{045B}
\pdfglyphtounicode{tsuperior}{F6F3}
\pdfglyphtounicode{ttabengali}{099F}
\pdfglyphtounicode{ttadeva}{091F}
\pdfglyphtounicode{ttagujarati}{0A9F}
\pdfglyphtounicode{ttagurmukhi}{0A1F}
\pdfglyphtounicode{tteharabic}{0679}
\pdfglyphtounicode{ttehfinalarabic}{FB67}
\pdfglyphtounicode{ttehinitialarabic}{FB68}
\pdfglyphtounicode{ttehmedialarabic}{FB69}
\pdfglyphtounicode{tthabengali}{09A0}
\pdfglyphtounicode{tthadeva}{0920}
\pdfglyphtounicode{tthagujarati}{0AA0}
\pdfglyphtounicode{tthagurmukhi}{0A20}
\pdfglyphtounicode{tturned}{0287}
\pdfglyphtounicode{tuhiragana}{3064}
\pdfglyphtounicode{tukatakana}{30C4}
\pdfglyphtounicode{tukatakanahalfwidth}{FF82}
\pdfglyphtounicode{tusmallhiragana}{3063}
\pdfglyphtounicode{tusmallkatakana}{30C3}
\pdfglyphtounicode{tusmallkatakanahalfwidth}{FF6F}
\pdfglyphtounicode{twelvecircle}{246B}
\pdfglyphtounicode{twelveparen}{247F}
\pdfglyphtounicode{twelveperiod}{2493}
\pdfglyphtounicode{twelveroman}{217B}
\pdfglyphtounicode{twentycircle}{2473}
\pdfglyphtounicode{twentyhangzhou}{5344}
\pdfglyphtounicode{twentyparen}{2487}
\pdfglyphtounicode{twentyperiod}{249B}
\pdfglyphtounicode{two}{0032}
\pdfglyphtounicode{twoarabic}{0662}
\pdfglyphtounicode{twobengali}{09E8}
\pdfglyphtounicode{twocircle}{2461}
\pdfglyphtounicode{twocircleinversesansserif}{278B}
\pdfglyphtounicode{twodeva}{0968}
\pdfglyphtounicode{twodotenleader}{2025}
\pdfglyphtounicode{twodotleader}{2025}
\pdfglyphtounicode{twodotleadervertical}{FE30}
\pdfglyphtounicode{twogujarati}{0AE8}
\pdfglyphtounicode{twogurmukhi}{0A68}
\pdfglyphtounicode{twohackarabic}{0662}
\pdfglyphtounicode{twohangzhou}{3022}
\pdfglyphtounicode{twoideographicparen}{3221}
\pdfglyphtounicode{twoinferior}{2082}
\pdfglyphtounicode{twomonospace}{FF12}
\pdfglyphtounicode{twonumeratorbengali}{09F5}
\pdfglyphtounicode{twooldstyle}{F732}
\pdfglyphtounicode{twoparen}{2475}
\pdfglyphtounicode{twoperiod}{2489}
\pdfglyphtounicode{twopersian}{06F2}
\pdfglyphtounicode{tworoman}{2171}
\pdfglyphtounicode{twostroke}{01BB}
\pdfglyphtounicode{twosuperior}{00B2}
\pdfglyphtounicode{twothai}{0E52}
\pdfglyphtounicode{twothirds}{2154}
\pdfglyphtounicode{u}{0075}
\pdfglyphtounicode{uacute}{00FA}
\pdfglyphtounicode{ubar}{0289}
\pdfglyphtounicode{ubengali}{0989}
\pdfglyphtounicode{ubopomofo}{3128}
\pdfglyphtounicode{ubreve}{016D}
\pdfglyphtounicode{ucaron}{01D4}
\pdfglyphtounicode{ucircle}{24E4}
\pdfglyphtounicode{ucircumflex}{00FB}
\pdfglyphtounicode{ucircumflexbelow}{1E77}
\pdfglyphtounicode{ucyrillic}{0443}
\pdfglyphtounicode{udattadeva}{0951}
\pdfglyphtounicode{udblacute}{0171}
\pdfglyphtounicode{udblgrave}{0215}
\pdfglyphtounicode{udeva}{0909}
\pdfglyphtounicode{udieresis}{00FC}
\pdfglyphtounicode{udieresisacute}{01D8}
\pdfglyphtounicode{udieresisbelow}{1E73}
\pdfglyphtounicode{udieresiscaron}{01DA}
\pdfglyphtounicode{udieresiscyrillic}{04F1}
\pdfglyphtounicode{udieresisgrave}{01DC}
\pdfglyphtounicode{udieresismacron}{01D6}
\pdfglyphtounicode{udotbelow}{1EE5}
\pdfglyphtounicode{ugrave}{00F9}
\pdfglyphtounicode{ugujarati}{0A89}
\pdfglyphtounicode{ugurmukhi}{0A09}
\pdfglyphtounicode{uhiragana}{3046}
\pdfglyphtounicode{uhookabove}{1EE7}
\pdfglyphtounicode{uhorn}{01B0}
\pdfglyphtounicode{uhornacute}{1EE9}
\pdfglyphtounicode{uhorndotbelow}{1EF1}
\pdfglyphtounicode{uhorngrave}{1EEB}
\pdfglyphtounicode{uhornhookabove}{1EED}
\pdfglyphtounicode{uhorntilde}{1EEF}
\pdfglyphtounicode{uhungarumlaut}{0171}
\pdfglyphtounicode{uhungarumlautcyrillic}{04F3}
\pdfglyphtounicode{uinvertedbreve}{0217}
\pdfglyphtounicode{ukatakana}{30A6}
\pdfglyphtounicode{ukatakanahalfwidth}{FF73}
\pdfglyphtounicode{ukcyrillic}{0479}
\pdfglyphtounicode{ukorean}{315C}
\pdfglyphtounicode{umacron}{016B}
\pdfglyphtounicode{umacroncyrillic}{04EF}
\pdfglyphtounicode{umacrondieresis}{1E7B}
\pdfglyphtounicode{umatragurmukhi}{0A41}
\pdfglyphtounicode{umonospace}{FF55}
\pdfglyphtounicode{underscore}{005F}
\pdfglyphtounicode{underscoredbl}{2017}
\pdfglyphtounicode{underscoremonospace}{FF3F}
\pdfglyphtounicode{underscorevertical}{FE33}
\pdfglyphtounicode{underscorewavy}{FE4F}
\pdfglyphtounicode{union}{222A}
\pdfglyphtounicode{universal}{2200}
\pdfglyphtounicode{uogonek}{0173}
\pdfglyphtounicode{uparen}{24B0}
\pdfglyphtounicode{upblock}{2580}
\pdfglyphtounicode{upperdothebrew}{05C4}
\pdfglyphtounicode{upsilon}{03C5}
\pdfglyphtounicode{upsilondieresis}{03CB}
\pdfglyphtounicode{upsilondieresistonos}{03B0}
\pdfglyphtounicode{upsilonlatin}{028A}
\pdfglyphtounicode{upsilontonos}{03CD}
\pdfglyphtounicode{uptackbelowcmb}{031D}
\pdfglyphtounicode{uptackmod}{02D4}
\pdfglyphtounicode{uragurmukhi}{0A73}
\pdfglyphtounicode{uring}{016F}
\pdfglyphtounicode{ushortcyrillic}{045E}
\pdfglyphtounicode{usmallhiragana}{3045}
\pdfglyphtounicode{usmallkatakana}{30A5}
\pdfglyphtounicode{usmallkatakanahalfwidth}{FF69}
\pdfglyphtounicode{ustraightcyrillic}{04AF}
\pdfglyphtounicode{ustraightstrokecyrillic}{04B1}
\pdfglyphtounicode{utilde}{0169}
\pdfglyphtounicode{utildeacute}{1E79}
\pdfglyphtounicode{utildebelow}{1E75}
\pdfglyphtounicode{uubengali}{098A}
\pdfglyphtounicode{uudeva}{090A}
\pdfglyphtounicode{uugujarati}{0A8A}
\pdfglyphtounicode{uugurmukhi}{0A0A}
\pdfglyphtounicode{uumatragurmukhi}{0A42}
\pdfglyphtounicode{uuvowelsignbengali}{09C2}
\pdfglyphtounicode{uuvowelsigndeva}{0942}
\pdfglyphtounicode{uuvowelsigngujarati}{0AC2}
\pdfglyphtounicode{uvowelsignbengali}{09C1}
\pdfglyphtounicode{uvowelsigndeva}{0941}
\pdfglyphtounicode{uvowelsigngujarati}{0AC1}
\pdfglyphtounicode{v}{0076}
\pdfglyphtounicode{vadeva}{0935}
\pdfglyphtounicode{vagujarati}{0AB5}
\pdfglyphtounicode{vagurmukhi}{0A35}
\pdfglyphtounicode{vakatakana}{30F7}
\pdfglyphtounicode{vav}{05D5}
\pdfglyphtounicode{vavdagesh}{FB35}
\pdfglyphtounicode{vavdagesh65}{FB35}
\pdfglyphtounicode{vavdageshhebrew}{FB35}
\pdfglyphtounicode{vavhebrew}{05D5}
\pdfglyphtounicode{vavholam}{FB4B}
\pdfglyphtounicode{vavholamhebrew}{FB4B}
\pdfglyphtounicode{vavvavhebrew}{05F0}
\pdfglyphtounicode{vavyodhebrew}{05F1}
\pdfglyphtounicode{vcircle}{24E5}
\pdfglyphtounicode{vdotbelow}{1E7F}
\pdfglyphtounicode{vecyrillic}{0432}
\pdfglyphtounicode{veharabic}{06A4}
\pdfglyphtounicode{vehfinalarabic}{FB6B}
\pdfglyphtounicode{vehinitialarabic}{FB6C}
\pdfglyphtounicode{vehmedialarabic}{FB6D}
\pdfglyphtounicode{vekatakana}{30F9}
\pdfglyphtounicode{venus}{2640}
\pdfglyphtounicode{verticalbar}{007C}
\pdfglyphtounicode{verticallineabovecmb}{030D}
\pdfglyphtounicode{verticallinebelowcmb}{0329}
\pdfglyphtounicode{verticallinelowmod}{02CC}
\pdfglyphtounicode{verticallinemod}{02C8}
\pdfglyphtounicode{vewarmenian}{057E}
\pdfglyphtounicode{vhook}{028B}
\pdfglyphtounicode{vikatakana}{30F8}
\pdfglyphtounicode{viramabengali}{09CD}
\pdfglyphtounicode{viramadeva}{094D}
\pdfglyphtounicode{viramagujarati}{0ACD}
\pdfglyphtounicode{visargabengali}{0983}
\pdfglyphtounicode{visargadeva}{0903}
\pdfglyphtounicode{visargagujarati}{0A83}
\pdfglyphtounicode{vmonospace}{FF56}
\pdfglyphtounicode{voarmenian}{0578}
\pdfglyphtounicode{voicediterationhiragana}{309E}
\pdfglyphtounicode{voicediterationkatakana}{30FE}
\pdfglyphtounicode{voicedmarkkana}{309B}
\pdfglyphtounicode{voicedmarkkanahalfwidth}{FF9E}
\pdfglyphtounicode{vokatakana}{30FA}
\pdfglyphtounicode{vparen}{24B1}
\pdfglyphtounicode{vtilde}{1E7D}
\pdfglyphtounicode{vturned}{028C}
\pdfglyphtounicode{vuhiragana}{3094}
\pdfglyphtounicode{vukatakana}{30F4}
\pdfglyphtounicode{w}{0077}
\pdfglyphtounicode{wacute}{1E83}
\pdfglyphtounicode{waekorean}{3159}
\pdfglyphtounicode{wahiragana}{308F}
\pdfglyphtounicode{wakatakana}{30EF}
\pdfglyphtounicode{wakatakanahalfwidth}{FF9C}
\pdfglyphtounicode{wakorean}{3158}
\pdfglyphtounicode{wasmallhiragana}{308E}
\pdfglyphtounicode{wasmallkatakana}{30EE}
\pdfglyphtounicode{wattosquare}{3357}
\pdfglyphtounicode{wavedash}{301C}
\pdfglyphtounicode{wavyunderscorevertical}{FE34}
\pdfglyphtounicode{wawarabic}{0648}
\pdfglyphtounicode{wawfinalarabic}{FEEE}
\pdfglyphtounicode{wawhamzaabovearabic}{0624}
\pdfglyphtounicode{wawhamzaabovefinalarabic}{FE86}
\pdfglyphtounicode{wbsquare}{33DD}
\pdfglyphtounicode{wcircle}{24E6}
\pdfglyphtounicode{wcircumflex}{0175}
\pdfglyphtounicode{wdieresis}{1E85}
\pdfglyphtounicode{wdotaccent}{1E87}
\pdfglyphtounicode{wdotbelow}{1E89}
\pdfglyphtounicode{wehiragana}{3091}
\pdfglyphtounicode{weierstrass}{2118}
\pdfglyphtounicode{wekatakana}{30F1}
\pdfglyphtounicode{wekorean}{315E}
\pdfglyphtounicode{weokorean}{315D}
\pdfglyphtounicode{wgrave}{1E81}
\pdfglyphtounicode{whitebullet}{25E6}
\pdfglyphtounicode{whitecircle}{25CB}
\pdfglyphtounicode{whitecircleinverse}{25D9}
\pdfglyphtounicode{whitecornerbracketleft}{300E}
\pdfglyphtounicode{whitecornerbracketleftvertical}{FE43}
\pdfglyphtounicode{whitecornerbracketright}{300F}
\pdfglyphtounicode{whitecornerbracketrightvertical}{FE44}
\pdfglyphtounicode{whitediamond}{25C7}
\pdfglyphtounicode{whitediamondcontainingblacksmalldiamond}{25C8}
\pdfglyphtounicode{whitedownpointingsmalltriangle}{25BF}
\pdfglyphtounicode{whitedownpointingtriangle}{25BD}
\pdfglyphtounicode{whiteleftpointingsmalltriangle}{25C3}
\pdfglyphtounicode{whiteleftpointingtriangle}{25C1}
\pdfglyphtounicode{whitelenticularbracketleft}{3016}
\pdfglyphtounicode{whitelenticularbracketright}{3017}
\pdfglyphtounicode{whiterightpointingsmalltriangle}{25B9}
\pdfglyphtounicode{whiterightpointingtriangle}{25B7}
\pdfglyphtounicode{whitesmallsquare}{25AB}
\pdfglyphtounicode{whitesmilingface}{263A}
\pdfglyphtounicode{whitesquare}{25A1}
\pdfglyphtounicode{whitestar}{2606}
\pdfglyphtounicode{whitetelephone}{260F}
\pdfglyphtounicode{whitetortoiseshellbracketleft}{3018}
\pdfglyphtounicode{whitetortoiseshellbracketright}{3019}
\pdfglyphtounicode{whiteuppointingsmalltriangle}{25B5}
\pdfglyphtounicode{whiteuppointingtriangle}{25B3}
\pdfglyphtounicode{wihiragana}{3090}
\pdfglyphtounicode{wikatakana}{30F0}
\pdfglyphtounicode{wikorean}{315F}
\pdfglyphtounicode{wmonospace}{FF57}
\pdfglyphtounicode{wohiragana}{3092}
\pdfglyphtounicode{wokatakana}{30F2}
\pdfglyphtounicode{wokatakanahalfwidth}{FF66}
\pdfglyphtounicode{won}{20A9}
\pdfglyphtounicode{wonmonospace}{FFE6}
\pdfglyphtounicode{wowaenthai}{0E27}
\pdfglyphtounicode{wparen}{24B2}
\pdfglyphtounicode{wring}{1E98}
\pdfglyphtounicode{wsuperior}{02B7}
\pdfglyphtounicode{wturned}{028D}
\pdfglyphtounicode{wynn}{01BF}
\pdfglyphtounicode{x}{0078}
\pdfglyphtounicode{xabovecmb}{033D}
\pdfglyphtounicode{xbopomofo}{3112}
\pdfglyphtounicode{xcircle}{24E7}
\pdfglyphtounicode{xdieresis}{1E8D}
\pdfglyphtounicode{xdotaccent}{1E8B}
\pdfglyphtounicode{xeharmenian}{056D}
\pdfglyphtounicode{xi}{03BE}
\pdfglyphtounicode{xmonospace}{FF58}
\pdfglyphtounicode{xparen}{24B3}
\pdfglyphtounicode{xsuperior}{02E3}
\pdfglyphtounicode{y}{0079}
\pdfglyphtounicode{yaadosquare}{334E}
\pdfglyphtounicode{yabengali}{09AF}
\pdfglyphtounicode{yacute}{00FD}
\pdfglyphtounicode{yadeva}{092F}
\pdfglyphtounicode{yaekorean}{3152}
\pdfglyphtounicode{yagujarati}{0AAF}
\pdfglyphtounicode{yagurmukhi}{0A2F}
\pdfglyphtounicode{yahiragana}{3084}
\pdfglyphtounicode{yakatakana}{30E4}
\pdfglyphtounicode{yakatakanahalfwidth}{FF94}
\pdfglyphtounicode{yakorean}{3151}
\pdfglyphtounicode{yamakkanthai}{0E4E}
\pdfglyphtounicode{yasmallhiragana}{3083}
\pdfglyphtounicode{yasmallkatakana}{30E3}
\pdfglyphtounicode{yasmallkatakanahalfwidth}{FF6C}
\pdfglyphtounicode{yatcyrillic}{0463}
\pdfglyphtounicode{ycircle}{24E8}
\pdfglyphtounicode{ycircumflex}{0177}
\pdfglyphtounicode{ydieresis}{00FF}
\pdfglyphtounicode{ydotaccent}{1E8F}
\pdfglyphtounicode{ydotbelow}{1EF5}
\pdfglyphtounicode{yeharabic}{064A}
\pdfglyphtounicode{yehbarreearabic}{06D2}
\pdfglyphtounicode{yehbarreefinalarabic}{FBAF}
\pdfglyphtounicode{yehfinalarabic}{FEF2}
\pdfglyphtounicode{yehhamzaabovearabic}{0626}
\pdfglyphtounicode{yehhamzaabovefinalarabic}{FE8A}
\pdfglyphtounicode{yehhamzaaboveinitialarabic}{FE8B}
\pdfglyphtounicode{yehhamzaabovemedialarabic}{FE8C}
\pdfglyphtounicode{yehinitialarabic}{FEF3}
\pdfglyphtounicode{yehmedialarabic}{FEF4}
\pdfglyphtounicode{yehmeeminitialarabic}{FCDD}
\pdfglyphtounicode{yehmeemisolatedarabic}{FC58}
\pdfglyphtounicode{yehnoonfinalarabic}{FC94}
\pdfglyphtounicode{yehthreedotsbelowarabic}{06D1}
\pdfglyphtounicode{yekorean}{3156}
\pdfglyphtounicode{yen}{00A5}
\pdfglyphtounicode{yenmonospace}{FFE5}
\pdfglyphtounicode{yeokorean}{3155}
\pdfglyphtounicode{yeorinhieuhkorean}{3186}
\pdfglyphtounicode{yerahbenyomohebrew}{05AA}
\pdfglyphtounicode{yerahbenyomolefthebrew}{05AA}
\pdfglyphtounicode{yericyrillic}{044B}
\pdfglyphtounicode{yerudieresiscyrillic}{04F9}
\pdfglyphtounicode{yesieungkorean}{3181}
\pdfglyphtounicode{yesieungpansioskorean}{3183}
\pdfglyphtounicode{yesieungsioskorean}{3182}
\pdfglyphtounicode{yetivhebrew}{059A}
\pdfglyphtounicode{ygrave}{1EF3}
\pdfglyphtounicode{yhook}{01B4}
\pdfglyphtounicode{yhookabove}{1EF7}
\pdfglyphtounicode{yiarmenian}{0575}
\pdfglyphtounicode{yicyrillic}{0457}
\pdfglyphtounicode{yikorean}{3162}
\pdfglyphtounicode{yinyang}{262F}
\pdfglyphtounicode{yiwnarmenian}{0582}
\pdfglyphtounicode{ymonospace}{FF59}
\pdfglyphtounicode{yod}{05D9}
\pdfglyphtounicode{yoddagesh}{FB39}
\pdfglyphtounicode{yoddageshhebrew}{FB39}
\pdfglyphtounicode{yodhebrew}{05D9}
\pdfglyphtounicode{yodyodhebrew}{05F2}
\pdfglyphtounicode{yodyodpatahhebrew}{FB1F}
\pdfglyphtounicode{yohiragana}{3088}
\pdfglyphtounicode{yoikorean}{3189}
\pdfglyphtounicode{yokatakana}{30E8}
\pdfglyphtounicode{yokatakanahalfwidth}{FF96}
\pdfglyphtounicode{yokorean}{315B}
\pdfglyphtounicode{yosmallhiragana}{3087}
\pdfglyphtounicode{yosmallkatakana}{30E7}
\pdfglyphtounicode{yosmallkatakanahalfwidth}{FF6E}
\pdfglyphtounicode{yotgreek}{03F3}
\pdfglyphtounicode{yoyaekorean}{3188}
\pdfglyphtounicode{yoyakorean}{3187}
\pdfglyphtounicode{yoyakthai}{0E22}
\pdfglyphtounicode{yoyingthai}{0E0D}
\pdfglyphtounicode{yparen}{24B4}
\pdfglyphtounicode{ypogegrammeni}{037A}
\pdfglyphtounicode{ypogegrammenigreekcmb}{0345}
\pdfglyphtounicode{yr}{01A6}
\pdfglyphtounicode{yring}{1E99}
\pdfglyphtounicode{ysuperior}{02B8}
\pdfglyphtounicode{ytilde}{1EF9}
\pdfglyphtounicode{yturned}{028E}
\pdfglyphtounicode{yuhiragana}{3086}
\pdfglyphtounicode{yuikorean}{318C}
\pdfglyphtounicode{yukatakana}{30E6}
\pdfglyphtounicode{yukatakanahalfwidth}{FF95}
\pdfglyphtounicode{yukorean}{3160}
\pdfglyphtounicode{yusbigcyrillic}{046B}
\pdfglyphtounicode{yusbigiotifiedcyrillic}{046D}
\pdfglyphtounicode{yuslittlecyrillic}{0467}
\pdfglyphtounicode{yuslittleiotifiedcyrillic}{0469}
\pdfglyphtounicode{yusmallhiragana}{3085}
\pdfglyphtounicode{yusmallkatakana}{30E5}
\pdfglyphtounicode{yusmallkatakanahalfwidth}{FF6D}
\pdfglyphtounicode{yuyekorean}{318B}
\pdfglyphtounicode{yuyeokorean}{318A}
\pdfglyphtounicode{yyabengali}{09DF}
\pdfglyphtounicode{yyadeva}{095F}
\pdfglyphtounicode{z}{007A}
\pdfglyphtounicode{zaarmenian}{0566}
\pdfglyphtounicode{zacute}{017A}
\pdfglyphtounicode{zadeva}{095B}
\pdfglyphtounicode{zagurmukhi}{0A5B}
\pdfglyphtounicode{zaharabic}{0638}
\pdfglyphtounicode{zahfinalarabic}{FEC6}
\pdfglyphtounicode{zahinitialarabic}{FEC7}
\pdfglyphtounicode{zahiragana}{3056}
\pdfglyphtounicode{zahmedialarabic}{FEC8}
\pdfglyphtounicode{zainarabic}{0632}
\pdfglyphtounicode{zainfinalarabic}{FEB0}
\pdfglyphtounicode{zakatakana}{30B6}
\pdfglyphtounicode{zaqefgadolhebrew}{0595}
\pdfglyphtounicode{zaqefqatanhebrew}{0594}
\pdfglyphtounicode{zarqahebrew}{0598}
\pdfglyphtounicode{zayin}{05D6}
\pdfglyphtounicode{zayindagesh}{FB36}
\pdfglyphtounicode{zayindageshhebrew}{FB36}
\pdfglyphtounicode{zayinhebrew}{05D6}
\pdfglyphtounicode{zbopomofo}{3117}
\pdfglyphtounicode{zcaron}{017E}
\pdfglyphtounicode{zcircle}{24E9}
\pdfglyphtounicode{zcircumflex}{1E91}
\pdfglyphtounicode{zcurl}{0291}
\pdfglyphtounicode{zdot}{017C}
\pdfglyphtounicode{zdotaccent}{017C}
\pdfglyphtounicode{zdotbelow}{1E93}
\pdfglyphtounicode{zecyrillic}{0437}
\pdfglyphtounicode{zedescendercyrillic}{0499}
\pdfglyphtounicode{zedieresiscyrillic}{04DF}
\pdfglyphtounicode{zehiragana}{305C}
\pdfglyphtounicode{zekatakana}{30BC}
\pdfglyphtounicode{zero}{0030}
\pdfglyphtounicode{zeroarabic}{0660}
\pdfglyphtounicode{zerobengali}{09E6}
\pdfglyphtounicode{zerodeva}{0966}
\pdfglyphtounicode{zerogujarati}{0AE6}
\pdfglyphtounicode{zerogurmukhi}{0A66}
\pdfglyphtounicode{zerohackarabic}{0660}
\pdfglyphtounicode{zeroinferior}{2080}
\pdfglyphtounicode{zeromonospace}{FF10}
\pdfglyphtounicode{zerooldstyle}{F730}
\pdfglyphtounicode{zeropersian}{06F0}
\pdfglyphtounicode{zerosuperior}{2070}
\pdfglyphtounicode{zerothai}{0E50}
\pdfglyphtounicode{zerowidthjoiner}{FEFF}
\pdfglyphtounicode{zerowidthnonjoiner}{200C}
\pdfglyphtounicode{zerowidthspace}{200B}
\pdfglyphtounicode{zeta}{03B6}
\pdfglyphtounicode{zhbopomofo}{3113}
\pdfglyphtounicode{zhearmenian}{056A}
\pdfglyphtounicode{zhebrevecyrillic}{04C2}
\pdfglyphtounicode{zhecyrillic}{0436}
\pdfglyphtounicode{zhedescendercyrillic}{0497}
\pdfglyphtounicode{zhedieresiscyrillic}{04DD}
\pdfglyphtounicode{zihiragana}{3058}
\pdfglyphtounicode{zikatakana}{30B8}
\pdfglyphtounicode{zinorhebrew}{05AE}
\pdfglyphtounicode{zlinebelow}{1E95}
\pdfglyphtounicode{zmonospace}{FF5A}
\pdfglyphtounicode{zohiragana}{305E}
\pdfglyphtounicode{zokatakana}{30BE}
\pdfglyphtounicode{zparen}{24B5}
\pdfglyphtounicode{zretroflexhook}{0290}
\pdfglyphtounicode{zstroke}{01B6}
\pdfglyphtounicode{zuhiragana}{305A}
\pdfglyphtounicode{zukatakana}{30BA}

% entries from zapfdingbats.txt:
\pdfglyphtounicode{a100}{275E}
\pdfglyphtounicode{a101}{2761}
\pdfglyphtounicode{a102}{2762}
\pdfglyphtounicode{a103}{2763}
\pdfglyphtounicode{a104}{2764}
\pdfglyphtounicode{a105}{2710}
\pdfglyphtounicode{a106}{2765}
\pdfglyphtounicode{a107}{2766}
\pdfglyphtounicode{a108}{2767}
\pdfglyphtounicode{a109}{2660}
\pdfglyphtounicode{a10}{2721}
\pdfglyphtounicode{a110}{2665}
\pdfglyphtounicode{a111}{2666}
\pdfglyphtounicode{a112}{2663}
\pdfglyphtounicode{a117}{2709}
\pdfglyphtounicode{a118}{2708}
\pdfglyphtounicode{a119}{2707}
\pdfglyphtounicode{a11}{261B}
\pdfglyphtounicode{a120}{2460}
\pdfglyphtounicode{a121}{2461}
\pdfglyphtounicode{a122}{2462}
\pdfglyphtounicode{a123}{2463}
\pdfglyphtounicode{a124}{2464}
\pdfglyphtounicode{a125}{2465}
\pdfglyphtounicode{a126}{2466}
\pdfglyphtounicode{a127}{2467}
\pdfglyphtounicode{a128}{2468}
\pdfglyphtounicode{a129}{2469}
\pdfglyphtounicode{a12}{261E}
\pdfglyphtounicode{a130}{2776}
\pdfglyphtounicode{a131}{2777}
\pdfglyphtounicode{a132}{2778}
\pdfglyphtounicode{a133}{2779}
\pdfglyphtounicode{a134}{277A}
\pdfglyphtounicode{a135}{277B}
\pdfglyphtounicode{a136}{277C}
\pdfglyphtounicode{a137}{277D}
\pdfglyphtounicode{a138}{277E}
\pdfglyphtounicode{a139}{277F}
\pdfglyphtounicode{a13}{270C}
\pdfglyphtounicode{a140}{2780}
\pdfglyphtounicode{a141}{2781}
\pdfglyphtounicode{a142}{2782}
\pdfglyphtounicode{a143}{2783}
\pdfglyphtounicode{a144}{2784}
\pdfglyphtounicode{a145}{2785}
\pdfglyphtounicode{a146}{2786}
\pdfglyphtounicode{a147}{2787}
\pdfglyphtounicode{a148}{2788}
\pdfglyphtounicode{a149}{2789}
\pdfglyphtounicode{a14}{270D}
\pdfglyphtounicode{a150}{278A}
\pdfglyphtounicode{a151}{278B}
\pdfglyphtounicode{a152}{278C}
\pdfglyphtounicode{a153}{278D}
\pdfglyphtounicode{a154}{278E}
\pdfglyphtounicode{a155}{278F}
\pdfglyphtounicode{a156}{2790}
\pdfglyphtounicode{a157}{2791}
\pdfglyphtounicode{a158}{2792}
\pdfglyphtounicode{a159}{2793}
\pdfglyphtounicode{a15}{270E}
\pdfglyphtounicode{a160}{2794}
\pdfglyphtounicode{a161}{2192}
\pdfglyphtounicode{a162}{27A3}
\pdfglyphtounicode{a163}{2194}
\pdfglyphtounicode{a164}{2195}
\pdfglyphtounicode{a165}{2799}
\pdfglyphtounicode{a166}{279B}
\pdfglyphtounicode{a167}{279C}
\pdfglyphtounicode{a168}{279D}
\pdfglyphtounicode{a169}{279E}
\pdfglyphtounicode{a16}{270F}
\pdfglyphtounicode{a170}{279F}
\pdfglyphtounicode{a171}{27A0}
\pdfglyphtounicode{a172}{27A1}
\pdfglyphtounicode{a173}{27A2}
\pdfglyphtounicode{a174}{27A4}
\pdfglyphtounicode{a175}{27A5}
\pdfglyphtounicode{a176}{27A6}
\pdfglyphtounicode{a177}{27A7}
\pdfglyphtounicode{a178}{27A8}
\pdfglyphtounicode{a179}{27A9}
\pdfglyphtounicode{a17}{2711}
\pdfglyphtounicode{a180}{27AB}
\pdfglyphtounicode{a181}{27AD}
\pdfglyphtounicode{a182}{27AF}
\pdfglyphtounicode{a183}{27B2}
\pdfglyphtounicode{a184}{27B3}
\pdfglyphtounicode{a185}{27B5}
\pdfglyphtounicode{a186}{27B8}
\pdfglyphtounicode{a187}{27BA}
\pdfglyphtounicode{a188}{27BB}
\pdfglyphtounicode{a189}{27BC}
\pdfglyphtounicode{a18}{2712}
\pdfglyphtounicode{a190}{27BD}
\pdfglyphtounicode{a191}{27BE}
\pdfglyphtounicode{a192}{279A}
\pdfglyphtounicode{a193}{27AA}
\pdfglyphtounicode{a194}{27B6}
\pdfglyphtounicode{a195}{27B9}
\pdfglyphtounicode{a196}{2798}
\pdfglyphtounicode{a197}{27B4}
\pdfglyphtounicode{a198}{27B7}
\pdfglyphtounicode{a199}{27AC}
\pdfglyphtounicode{a19}{2713}
\pdfglyphtounicode{a1}{2701}
\pdfglyphtounicode{a200}{27AE}
\pdfglyphtounicode{a201}{27B1}
\pdfglyphtounicode{a202}{2703}
\pdfglyphtounicode{a203}{2750}
\pdfglyphtounicode{a204}{2752}
\pdfglyphtounicode{a205}{276E}
\pdfglyphtounicode{a206}{2770}
\pdfglyphtounicode{a20}{2714}
\pdfglyphtounicode{a21}{2715}
\pdfglyphtounicode{a22}{2716}
\pdfglyphtounicode{a23}{2717}
\pdfglyphtounicode{a24}{2718}
\pdfglyphtounicode{a25}{2719}
\pdfglyphtounicode{a26}{271A}
\pdfglyphtounicode{a27}{271B}
\pdfglyphtounicode{a28}{271C}
\pdfglyphtounicode{a29}{2722}
\pdfglyphtounicode{a2}{2702}
\pdfglyphtounicode{a30}{2723}
\pdfglyphtounicode{a31}{2724}
\pdfglyphtounicode{a32}{2725}
\pdfglyphtounicode{a33}{2726}
\pdfglyphtounicode{a34}{2727}
\pdfglyphtounicode{a35}{2605}
\pdfglyphtounicode{a36}{2729}
\pdfglyphtounicode{a37}{272A}
\pdfglyphtounicode{a38}{272B}
\pdfglyphtounicode{a39}{272C}
\pdfglyphtounicode{a3}{2704}
\pdfglyphtounicode{a40}{272D}
\pdfglyphtounicode{a41}{272E}
\pdfglyphtounicode{a42}{272F}
\pdfglyphtounicode{a43}{2730}
\pdfglyphtounicode{a44}{2731}
\pdfglyphtounicode{a45}{2732}
\pdfglyphtounicode{a46}{2733}
\pdfglyphtounicode{a47}{2734}
\pdfglyphtounicode{a48}{2735}
\pdfglyphtounicode{a49}{2736}
\pdfglyphtounicode{a4}{260E}
\pdfglyphtounicode{a50}{2737}
\pdfglyphtounicode{a51}{2738}
\pdfglyphtounicode{a52}{2739}
\pdfglyphtounicode{a53}{273A}
\pdfglyphtounicode{a54}{273B}
\pdfglyphtounicode{a55}{273C}
\pdfglyphtounicode{a56}{273D}
\pdfglyphtounicode{a57}{273E}
\pdfglyphtounicode{a58}{273F}
\pdfglyphtounicode{a59}{2740}
\pdfglyphtounicode{a5}{2706}
\pdfglyphtounicode{a60}{2741}
\pdfglyphtounicode{a61}{2742}
\pdfglyphtounicode{a62}{2743}
\pdfglyphtounicode{a63}{2744}
\pdfglyphtounicode{a64}{2745}
\pdfglyphtounicode{a65}{2746}
\pdfglyphtounicode{a66}{2747}
\pdfglyphtounicode{a67}{2748}
\pdfglyphtounicode{a68}{2749}
\pdfglyphtounicode{a69}{274A}
\pdfglyphtounicode{a6}{271D}
\pdfglyphtounicode{a70}{274B}
\pdfglyphtounicode{a71}{25CF}
\pdfglyphtounicode{a72}{274D}
\pdfglyphtounicode{a73}{25A0}
\pdfglyphtounicode{a74}{274F}
\pdfglyphtounicode{a75}{2751}
\pdfglyphtounicode{a76}{25B2}
\pdfglyphtounicode{a77}{25BC}
\pdfglyphtounicode{a78}{25C6}
\pdfglyphtounicode{a79}{2756}
\pdfglyphtounicode{a7}{271E}
\pdfglyphtounicode{a81}{25D7}
\pdfglyphtounicode{a82}{2758}
\pdfglyphtounicode{a83}{2759}
\pdfglyphtounicode{a84}{275A}
\pdfglyphtounicode{a85}{276F}
\pdfglyphtounicode{a86}{2771}
\pdfglyphtounicode{a87}{2772}
\pdfglyphtounicode{a88}{2773}
\pdfglyphtounicode{a89}{2768}
\pdfglyphtounicode{a8}{271F}
\pdfglyphtounicode{a90}{2769}
\pdfglyphtounicode{a91}{276C}
\pdfglyphtounicode{a92}{276D}
\pdfglyphtounicode{a93}{276A}
\pdfglyphtounicode{a94}{276B}
\pdfglyphtounicode{a95}{2774}
\pdfglyphtounicode{a96}{2775}
\pdfglyphtounicode{a97}{275B}
\pdfglyphtounicode{a98}{275C}
\pdfglyphtounicode{a99}{275D}
\pdfglyphtounicode{a9}{2720}

% entries from texglyphlist.txt:
% Delta;2206
\pdfglyphtounicode{Ifractur}{2111}
\pdfglyphtounicode{FFsmall}{D804}
\pdfglyphtounicode{FFIsmall}{D807}
\pdfglyphtounicode{FFLsmall}{D808}
\pdfglyphtounicode{FIsmall}{D805}
\pdfglyphtounicode{FLsmall}{D806}
\pdfglyphtounicode{Germandbls}{D800}
\pdfglyphtounicode{Germandblssmall}{D803}
\pdfglyphtounicode{Ng}{014A}
% Omega;2126
\pdfglyphtounicode{Rfractur}{211C}
\pdfglyphtounicode{SS}{D800}
\pdfglyphtounicode{SSsmall}{D803}
\pdfglyphtounicode{altselector}{D802}
\pdfglyphtounicode{angbracketleft}{27E8}
\pdfglyphtounicode{angbracketright}{27E9}
\pdfglyphtounicode{arrowbothv}{2195}
\pdfglyphtounicode{arrowdblbothv}{21D5}
\pdfglyphtounicode{arrowleftbothalf}{21BD}
\pdfglyphtounicode{arrowlefttophalf}{21BC}
\pdfglyphtounicode{arrownortheast}{2197}
\pdfglyphtounicode{arrownorthwest}{2196}
\pdfglyphtounicode{arrowrightbothalf}{21C1}
\pdfglyphtounicode{arrowrighttophalf}{21C0}
\pdfglyphtounicode{arrowsoutheast}{2198}
\pdfglyphtounicode{arrowsouthwest}{2199}
\pdfglyphtounicode{ascendercompwordmark}{D80A}
\pdfglyphtounicode{asteriskcentered}{2217}
\pdfglyphtounicode{bardbl}{2225}
\pdfglyphtounicode{capitalcompwordmark}{D809}
\pdfglyphtounicode{ceilingleft}{2308}
\pdfglyphtounicode{ceilingright}{2309}
\pdfglyphtounicode{circlecopyrt}{20DD}
\pdfglyphtounicode{circledivide}{2298}
\pdfglyphtounicode{circledot}{2299}
\pdfglyphtounicode{circleminus}{2296}
\pdfglyphtounicode{coproduct}{2A3F}
\pdfglyphtounicode{cwm}{200C}
\pdfglyphtounicode{dblbracketleft}{27E6}
\pdfglyphtounicode{dblbracketright}{27E7}
% diamond;2662
% diamondmath;22C4
% dotlessj;0237
% emptyset;2205
\pdfglyphtounicode{emptyslot}{D801}
\pdfglyphtounicode{epsilon1}{03F5}
\pdfglyphtounicode{equivasymptotic}{224D}
\pdfglyphtounicode{flat}{266D}
\pdfglyphtounicode{floorleft}{230A}
\pdfglyphtounicode{floorright}{230B}
\pdfglyphtounicode{follows}{227B}
\pdfglyphtounicode{followsequal}{227D}
\pdfglyphtounicode{greatermuch}{226B}
% heart;2661
\pdfglyphtounicode{interrobang}{203D}
\pdfglyphtounicode{interrobangdown}{D80B}
\pdfglyphtounicode{intersectionsq}{2293}
\pdfglyphtounicode{latticetop}{22A4}
\pdfglyphtounicode{lessmuch}{226A}
\pdfglyphtounicode{lscript}{2113}
\pdfglyphtounicode{natural}{266E}
\pdfglyphtounicode{negationslash}{0338}
\pdfglyphtounicode{ng}{014B}
\pdfglyphtounicode{owner}{220B}
\pdfglyphtounicode{pertenthousand}{2031}
% phi;03D5
% phi1;03C6
\pdfglyphtounicode{pi1}{03D6}
\pdfglyphtounicode{precedesequal}{227C}
\pdfglyphtounicode{prime}{2032}
\pdfglyphtounicode{rho1}{03F1}
\pdfglyphtounicode{ringfitted}{D80D}
\pdfglyphtounicode{sharp}{266F}
\pdfglyphtounicode{similarequal}{2243}
\pdfglyphtounicode{slurabove}{2322}
\pdfglyphtounicode{slurbelow}{2323}
\pdfglyphtounicode{star}{22C6}
\pdfglyphtounicode{subsetsqequal}{2291}
\pdfglyphtounicode{supersetsqequal}{2292}
\pdfglyphtounicode{triangle}{25B3}
\pdfglyphtounicode{triangleinv}{25BD}
\pdfglyphtounicode{triangleleft}{25B9}
\pdfglyphtounicode{triangleright}{25C3}
\pdfglyphtounicode{turnstileleft}{22A2}
\pdfglyphtounicode{turnstileright}{22A3}
\pdfglyphtounicode{twelveudash}{D80C}
\pdfglyphtounicode{unionmulti}{228E}
\pdfglyphtounicode{unionsq}{2294}
\pdfglyphtounicode{vector}{20D7}
\pdfglyphtounicode{visualspace}{2423}
\pdfglyphtounicode{wreathproduct}{2240}
\pdfglyphtounicode{Dbar}{0110}
\pdfglyphtounicode{compwordmark}{200C}
\pdfglyphtounicode{dbar}{0111}
\pdfglyphtounicode{rangedash}{2013}
\pdfglyphtounicode{hyphenchar}{002D}
\pdfglyphtounicode{punctdash}{2014}
\pdfglyphtounicode{visiblespace}{2423}

% entries from additional.tex:
\pdfglyphtounicode{ff}{0066 0066}
\pdfglyphtounicode{fi}{0066 0069}
\pdfglyphtounicode{fl}{0066 006C}
\pdfglyphtounicode{ffi}{0066 0066 0069}
\pdfglyphtounicode{ffl}{0066 0066 006C}
\pdfglyphtounicode{IJ}{0049 004A}
\pdfglyphtounicode{ij}{0069 006A}
\pdfglyphtounicode{longs}{0073}


% Angaben zu den Pfaden zu Grafiken, BibTeX-.Datei etc
% Pfad und Name der BibTeX-Datei,
% Pfade zu Grafik-Dateien,
% erlaubte Erweiterungen für Grafikdateien
%
%
% Pfad und Name der BibteX-Datei(en)
% Hier alle bib-Ressourcen angeben, die Literaturreferenzen enthalten,
% welche im Text der Arbeit referenziert werden bzw. im Haupt-Literaturverzeichnis
% stehen sollen.
% Achtung: die Bibliografien sollten in biblatex mit UTF8-Notation formatiert sein!
\addbibresource{./bib/Diss.bib}
\addbibresource{./bib/example.bib}

% Man könnte eigene Publikationen, Patente und betreute Arbeiten in jeweils
% eigenen Bibliografie-Dateien vorhalten, kann aber alle Angaben auch
% in einer einziger Datei verwalten (z.B Diss.bib).
% Bei Erstellung dieser Zusatz-Literaturverzeichnisse werden jeweils
% folgende Makros verwendet. Sie können auf eine der oben angegebenen Dateien
% oder auf jeweils eine andere Datei zeigen.
\newcommand{\bibpathOwnPatents}{./bib/Diss.bib}
\newcommand{\bibpathOwnPublications}{./bib/Diss.bib}
\newcommand{\bibpathStudentTheses}{./bib/Diss.bib}


%% Pfade zu Bildern:
%% Verwendete Pfade: (Reihenfolge wichtig, da Durchsuchen in dieser Reihenfolge erfolgt!!!)
%% Achtung: Es gibt keine Warnung, falls Pfade an mehreren Stellen gesetzt werden!
\graphicspath{{./images/}{../Papers/images/Diss/}{../Papers/images/}}
%
% Grafikdatei-Erweiterungen
\DeclareGraphicsExtensions{.pdf,.png,.jpg,.jpeg,.bmp,.eps}


% Übersetzungen verschiedener Überschriften etc.
%% Begriffe in deutsch oder englisch je nach Hauptsprache

%Makro zur Definition eines neuen multilingualen Bezeichner-Makros
\makeatletter
\newcommand{\newlanguagecommand}[1]{%
  \newcommand#1{%
    \@ifundefined{\string#1\languagename}
      {``No definition of \texttt{\string#1} for \languagename. Please define it in the file Translations.tex !''}
      {\@nameuse{\string#1\languagename}}%
  }%
}
%Mit diesem Makro können Varianten in verschiedenen Sprachen hinzugefügt werden.
\newcommand{\addtolanguagecommand}[3]{%
  \@namedef{\string#1#2}{#3}}
\makeatother


%Name für das Lesezeichen für die Titelseite
\newlanguagecommand{\TransTitlePageName}
\addtolanguagecommand{\TransTitlePageName}{ngerman}{Titelseite}
\addtolanguagecommand{\TransTitlePageName}{english}{Cover}

\newlanguagecommand{\TransOwnPublications}
\addtolanguagecommand{\TransOwnPublications}{ngerman}{Publikationen}
\addtolanguagecommand{\TransOwnPublications}{english}{Publications}

\newlanguagecommand{\TransOwnPatents}
\addtolanguagecommand{\TransOwnPatents}{ngerman}{Patente}
\addtolanguagecommand{\TransOwnPatents}{english}{Patents}

\newlanguagecommand{\TransSupervisedTheses}
\addtolanguagecommand{\TransSupervisedTheses}{ngerman}{Betreute studentische Arbeiten}
\addtolanguagecommand{\TransSupervisedTheses}{english}{Supervised student theses}

\newlanguagecommand{\AppendixName}
\addtolanguagecommand{\AppendixName}{ngerman}{Anhang}
\addtolanguagecommand{\AppendixName}{english}{Appendix}

\newlanguagecommand{\TodoListName}
\addtolanguagecommand{\TodoListName}{ngerman}{Todo-Liste}
\addtolanguagecommand{\TodoListName}{english}{Todo List}

\newlanguagecommand{\TransAcknowledgements}
\addtolanguagecommand{\TransAcknowledgements}{ngerman}{Danksagung}
\addtolanguagecommand{\TransAcknowledgements}{english}{Acknowledgements}


%% Befehle für korrekte Behandlung von Textteilen in anderer Sprache (je nach gewählter Hauptsprache)
\ifthenelse{\boolean{iesenglishs}}{%
	% Hauptsprache ist Englisch
	\newcommand{\textInEnglish}[1]{#1}
	\newcommand{\textInGerman}[1]{\foreignlanguage{ngerman}{#1}}
}{%
	% Hauptsprache ist Deutsch
	\newcommand{\textInEnglish}[1]{\foreignlanguage{english}{#1}}
	\newcommand{\textInGerman}[1]{#1}
}


\newcommand{\theoremname}{} % initialization
\newcommand{\examplename}{} % initialization
%\newcommand{\proofname}{} % initialization
\newcommand{\definitionname}{} % initialization
\newcommand{\lemmaname}{} % initialization
\newcommand{\corollaryname}{} % initialization
\newcommand{\propositionname}{} % initialization

\addto\captionsenglish{%
  \renewcommand{\theoremname}{Theorem}%
  \renewcommand{\examplename}{Example}%
	%\renewcommand{\proofname}{Proof}%
	\renewcommand{\definitionname}{Definition}%
	\renewcommand{\lemmaname}{Lemma}%
	\renewcommand{\corollaryname}{Corollary}%
	\renewcommand{\propositionname}{Proposition}%
}
\addto\captionsngerman{%
  \renewcommand{\theoremname}{Satz}%
  \renewcommand{\examplename}{Beispiel}%
	%\renewcommand{\proofname}{Beweis}%
	\renewcommand{\definitionname}{Definition}%
	\renewcommand{\lemmaname}{Lemma}%
	\renewcommand{\corollaryname}{Korollar}%
	\renewcommand{\propositionname}{Proposition}%
}

%\newtheorem{theorem}{Satz}[chapter]
%\newtheorem{definition}[theorem]{Definition}
%\newtheorem{lemma}[theorem]{Lemma}
%\newtheorem{corollary}[theorem]{Corollary}
%\newtheorem{proposition}[theorem]{Proposition}



%%% Wenn man das Aussehen der Tabellen ändern will, findet man die Einstellungen in dieser Datei
%% Kommandos fuer Tabellen. Entnommen aus The LateX Companion, tabsatz.ps und diversen Dokus:

%%% ---| Farben fuer Tabellen |-------------------
\IfPackageLoaded{xcolor}{
   \colorlet{tablesubheadcolor}{gray!30}
   \colorlet{tableheadcolor}{gray!25}
   \colorlet{tableblackheadcolor}{black!100}
   \colorlet{tablerowcolor}{gray!10.0}
}
%%% ---------------------------------------------


%%% -| Neue Spaltendefinitionen 'columntypes' |--
%
% Belegte Spaltentypen:
% l - links
% c - zentriert
% r - rechts
% p,m,b  - oben, mittig, unten
% X - tabularx Auto-Spalte

% um Tabellenspalten mit Flattersatz zu setzen, muss \\ vor
% (z.B.) \raggedright geschuetzt werden:
\newcommand{\PreserveBackslash}[1]{\let\temp=\\#1\let\\=\temp}

% Vorgabe vom KIT-Verlag: keine Worttrennung in einer Tabelle, daher \raggedright statt \RaggedRight
%\renewcommand\multirowsetup{\RaggedRight}
\renewcommand\multirowsetup{\raggedRight}

% Spalten mit Flattersatz und definierte Breite:
% m{} -> mittig
% p{} -> oben
% b{} -> unten
%
%\newcolumntype{L}[1]{>{\hsize=#1\hsize\RaggedRight\arraybackslash}X}%
%\newcolumntype{R}[1]{>{\hsize=#1\hsize\RaggedLeft\arraybackslash}X}%
%\newcolumntype{C}[1]{>{\hsize=#1\hsize\Centering\arraybackslash}X}%
%
% Vorgabe vom KIT-Verlag: keine Worttrennung in einer Tabelle, daher \raggedright statt \RaggedRight etc.
\newcolumntype{L}[1]{>{\raggedright\arraybackslash}p{#1}} % linksbündig mit Breitenangabe
\newcolumntype{C}[1]{>{\centering\arraybackslash}p{#1}} % zentriert mit Breitenangabe
\newcolumntype{R}[1]{>{\raggedleft\arraybackslash}p{#1}} % rechtsbündig mit Breitenangabe
%
\newcolumntype{M}{>{\begin{minipage}[t]{2cm}\raggedright}c<{\end{minipage}}}
%
\newcolumntype{G}[1]{>{\RaggedLeft\arraybackslash}p{#1}}
\newcolumntype{U}[1]{>{\RaggedRight\arraybackslash}p{#1}}
%\newcolumntype{C}[1]{>{\Centering\arraybackslash}p{#1}}

\newcolumntype{v}[1]{>{\PreserveBackslash\RaggedRight\hspace{0pt}}p{#1}}
\newcolumntype{Y}{>{\PreserveBackslash\RaggedLeft\hspace{0pt}}X}
% Tabellenspaltentyp fuer den Kopf: (Farbe + Ausrichtung)
\newcolumntype{H}[1]{>{\columncolor{tableheadcolor}}l}
% % Rechtsbuendig :
% \newcolumntype{R}[1]{>{\PreserveBackslash\RaggedLeft\hspace{0pt}}m{#1}}
% \newcolumntype{S}[1]{>{\PreserveBackslash\RaggedLeft\hspace{0pt}}p{#1}}
% % Zentriert :
% \newcolumntype{Z}[1]{>{\PreserveBackslash\Centering\hspace{0pt}}m{#1}}
% \newcolumntype{A}[1]{>{\PreserveBackslash\Centering\hspace{0pt}}p{#1}}

%%% Spalten fuer Mathematik
% serifenlose Matheschrift
\newcolumntype{s}[1]{>{\DC@{.}{,}{#1}\mathsf\bgroup}l<{\egroup\DV@end}}

% aequivalent aus typokurz (fett+grau+links)
% \newcolumntype{H}{>{\fontseries{b}\selectfont%
%     \columncolor[gray]{.8}[6pt][0pt]}l}
%%% --------------------------------------------


%%% ---|Listen in Tabellen |--------------------
\newcommand{\removeindentation}{%
	\leftmargini=\labelsep%
	\advance\leftmargini by \labelsep%
}
%
\makeatletter
\newcommand\tableitemize{
	\@minipagetrue%
	\removeindentation
}
\makeatother


%% Aufzählungen in einer Tabelle
\newenvironment{tabitemize}{%
\begin{list}{\textbullet}{%
\setlength\topsep{0pt}%
\setlength\parsep{0pt}%
\setlength\itemsep{0pt}%
\setlength\leftmargin{0em}%
\setlength\leftmargin{1em}%
\setlength\labelwidth{0.5em}%
\setlength\labelsep{0.5em}%
}
}{%
\end{list}
}
%%% --------------------------------------------

%%% ---|Layout der Tabellen |-------------------

% Neue Umgebung fuer Tabellen:

\newenvironment{Tabelle}[2][c]{%
  \tablestylecommon
  \begin{longtable}[#1]{#2}
  }
  {\end{longtable}%
  \tablerestoresettings
}


% Groesse der Schrift in Tabellen
\newcommand{\tablefontsize}{ \footnotesize}
\newcommand{\tableheadfontsize}{\footnotesize}

% Layout der Tabelle: Ausrichtung, Schrift, Zeilenabstand
\newcommand\tablestylecommon{%
  \renewcommand{\arraystretch}{1.4} % Groessere Abstaende zwischen Zeilen
  \normalfont\normalsize            %
  \sffamily\tablefontsize           % Serifenlose und kleine Schrift
  \centering%                       % Tabelle zentrieren
}

\newcommand{\tablestyle}{
	\tablestylecommon
	%\tablealtcolored
}

% Ruecksetzten der Aenderungen
\newcommand\tablerestoresettings{%
  \renewcommand{\arraystretch}{1}% Abstaende wieder zuruecksetzen
  \normalsize\rmfamily % Schrift wieder zuruecksetzen
}

% Tabellenkopf: Serifenlos+fett+schraeg+Schriftfarbe
%\newcommand\tablehead{%
  %\tableheadfontsize%
  %\sffamily\bfseries%
  %%\slshape
  %%\color{white}
%}

\newcommand\tablesubheadfont{%
  \tableheadfontsize%
  \sffamily\bfseries%
  \slshape
  %\color{white}
}


\newcommand\tableheadcolor{%
	%\rowcolor{tablesubheadcolor}
	%\rowcolor{tableblackheadcolor}
	\rowcolor{tableheadcolor}%
}

\newcommand\tablesubheadcolor{%
	\rowcolor{tablesubheadcolor}
	%\rowcolor{tableblackheadcolor}
}


\newcommand{\tableend}{\arrayrulecolor{black}\hline}

% Tabellenkopf (1=Spaltentyp, 2=Text)
% \newcommand{\tablehead}[2]{
%   \multicolumn{1}{#1@{}}{%
%     \raisebox{.1mm}{% Ausrichtung der Beschriftung
%       #2%
%     }\rule{0pt}{4mm}}% unsichtbare Linie, die die Kopfzeile hoeher macht
% }


\newcommand{\tablesubhead}[2]{%
  \multicolumn{#1}{>{\columncolor{tablesubheadcolor}}l}{\tablesubheadfont #2}%
}

% Tabellenbody (=Inhalt)
\newcommand\tablebody{%
\tablefontsize\sffamily\upshape%
}

\newcommand\tableheadshaded{%
	\rowcolor{tableheadcolor}%
}
\newcommand\tablealtcolored{%
	\rowcolors{1}{tablerowcolor}{white!100}%
}
%%% --------------------------------------------

\newlength{\mylen}
\newlength{\adjusthspace}

\newenvironment{tabularc}[2]
{%
	\setlength\mylen{#2/(#1)-\tabcolsep*2-\arrayrulewidth*(#1+1)/(#1)}%
	%\setlength{\adjusthspace}{((#2-1)/2)*\linewidth}
	%\par\noindent
	%\hspace*{-\the\adjusthspace}
	\begin{tabular}%{#2}%
		{*{#1}{v{\the\mylen}}}%
}
{\end{tabular}\par}


%%%% Aussehen von Glossaren
%%%%%%%%%%%%%%%%%%%%%%%%%%%%%%%%%%%%%%%%%%%%%%%%%%%%%%%%%%%%%%%%%%%%%%%%%%%%%%%%%%%%%%%%%%%%%%%%
%% Hack for correct glossary width in case of too long entries:
%% define an own glossary style "mylongglossstyle"
%% (see https://tex.stackexchange.com/questions/25380/glossaries-printglossaries-prints-too-wide)
%%%%%%%%%%%%%%%%%%%%%%%%%%%%%%%%%%%%%%%%%%%%%%%%%%%%%%%%%%%%%%%%%%%%%%%%%%%%%%%%%%%%%%%%%%%%%%%%
\newlength{\myglstargetwidth}
\newlength{\myglshspace}
\newlength{\myglsdescwidth}
\newglossarystyle{mylongglossstyle}{%
% put the glossary in the longtable environment:
	\renewenvironment{theglossary}{%
		\setlength{\myglstargetwidth}{0.19\textwidth}%
		\setlength{\myglshspace}{0.02\textwidth}%
		\setlength{\myglsdescwidth}{0.79\textwidth}%
		\setlength{\tabcolsep}{0pt}%
		\setlength{\extrarowheight}{12pt}%
		\begin{longtable}{p{\myglstargetwidth} @{\hspace{\myglshspace}} p{\myglsdescwidth}}
	}{%
	  \end{longtable}%
	}%
	% have nothing after \begin{theglossary}:
	\renewcommand*{\glossaryheader}{}%
	% have nothing between glossary groups:
	\renewcommand*{\glsgroupheading}[1]{}%
	\renewcommand*{\glsgroupskip}{}%
	% set how each entry should appear:
	\renewcommand*{\glossaryentryfield}[5]{%
		\raggedright\strong{\glstarget{##1}{##2}}% the entry name
		##4% the symbol 
		&##3%,% the description
		%\space%
		%##5% the number list 
		\\%
	}%
	% set how sub-entries appear:
	\renewcommand*{\glossarysubentryfield}[6]{%
		\glossaryentryfield{##2}{##3}{##4}{##5}{##6}
	}%
}

% setglossarystyle must be issued before \printglossaries.
% the longragged style can be set only after the corresponding package (i.e. glossary-longragged) has been loaded
%\setglossarystyle{longragged}
\setglossarystyle{mylongglossstyle}

%% Glossarentries sollen fett sein
\renewcommand{\glsnamefont}[1]{\textbf{#1}}
%% Glossareintäge (auch Links) schwarz
\renewcommand*{\glstextformat}[1]{\textcolor{black}{#1}}%

%% Zusätzlichen Punkt am Ende jeder Beschreibung deaktivieren
%\renewcommand*{\glspostdescription}{}

%% Deaktivieren von Hyperlinks auf das Glossar.
\glsdisablehyper

% Hinzufügen verschieder Zusatz-Formen für Glossareinträge ermöglichen (Genitiv, Dativ + Plural)
% s. https://tex.stackexchange.com/questions/178725/how-to-use-glossaries-for-different-grammatical-acronym-forms
% genitive
\glsaddkey*
 {genitive}% key
 {\acrshort{\glslabel}}% default value
 {\glsentrygenitive}% no link cs
 {\Glsentrygenitive}% no link ucfirst cs
 {\glsgenitive}% link cs
 {\Glsgenitive}% link ucfirst cs
 {\GLSgenitive}% link all caps cs

% dative
\glsaddkey*
 {dative}% key
 {\acrshort{\glslabel}}% default value
 {\glsentrydative}% no link cs
 {\Glsentrydative}% no link ucfirst cs
 {\glsdative}% link cs
 {\Glsdative}% link ucfirst cs
 {\GLSdative}% link all caps cs

% accusative
\glsaddkey*
 {accusative}% key
 {\acrshort{\glslabel}}% default value
 {\glsentryaccusative}% no link cs
 {\Glsentryaccusative}% no link ucfirst cs
 {\glsaccusative}% link cs
 {\Glsaccusative}% link ucfirst cs
 {\GLSaccusative}% link all caps cs

% short genitive
\glsaddkey*
 {shortgenitive}% key
 {\acrshort{\glslabel}}% default value
 {\glsentryshortgenitive}% no link cs
 {\Glsentryshortgenitive}% no link ucfirst cs
 {\glsshortgenitive}% link cs
 {\Glsshortgenitive}% link ucfirst cs
 {\GLSshortgenitive}% link all caps cs

% short dative
\glsaddkey*
 {shortdative}% key
 {\acrshort{\glslabel}}% default value
 {\glsentryshortdative}% no link cs
 {\Glsentryshortdative}% no link ucfirst cs
 {\glsshortdative}% link cs
 {\Glsshortdative}% link ucfirst cs
 {\GLSshortdative}% link all caps cs

% short accusative
\glsaddkey*
 {shortaccusative}% key
 {\acrshort{\glslabel}}% default value
 {\glsentryshortaccusative}% no link cs
 {\Glsentryshortaccusative}% no link ucfirst cs
 {\glsshortaccusative}% link cs
 {\Glsshortaccusative}% link ucfirst cs
 {\GLSshortaccusative}% link all caps cs

% genitive plural
\glsaddkey*
 {pluralgenitive}% key
 {\acrshort{\glslabel}}% default value
 {\glsentrypluralgenitive}% no link cs
 {\Glsentrypluralgenitive}% no link ucfirst cs
 {\glsplgenitive}% link cs
 {\Glsplgenitive}% link ucfirst cs
 {\GLSplgenitive}% link all caps cs

% dative plural
\glsaddkey*
 {pluraldative}% key
 {\acrshort{\glslabel}}% default value
 {\glsentrypluraldative}% no link cs
 {\Glsentrypluraldative}% no link ucfirst cs
 {\glspldative}% link cs
 {\Glspldative}% link ucfirst cs
 {\GLSpldative}% link all caps cs

% accusative plural
\glsaddkey*
 {pluralaccusative}% key
 {\acrshort{\glslabel}}% default value
 {\glsentrypluralaccusative}% no link cs
 {\Glsentrypluralaccusative}% no link ucfirst cs
 {\glsplaccusative}% link cs
 {\Glsplaccusative}% link ucfirst cs
 {\GLSplaccusative}% link all caps cs

% short genitive plural
\glsaddkey*
 {shortpluralgenitive}% key
 {\acrshort{\glslabel}}% default value
 {\glsentryshortpluralgenitive}% no link cs
 {\Glsentryshortpluralgenitive}% no link ucfirst cs
 {\glssplgenitive}% link cs
 {\Glssplgenitive}% link ucfirst cs
 {\GLSsplgenitive}% link all caps cs

% short dative plural
\glsaddkey*
 {shortpluraldative}% key
 {\acrshort{\glslabel}}% default value
 {\glsentryshortpluraldative}% no link cs
 {\Glsentryshortpluraldative}% no link ucfirst cs
 {\glsspldative}% link cs
 {\Glsspldative}% link ucfirst cs
 {\GLSspldative}% link all caps cs

% short accusative plural
\glsaddkey*
 {shorttpluralaccusative}% key
 {\acrshort{\glslabel}}% default value
 {\glsentryshortpluralaccusative}% no link cs
 {\Glsentryshortpluralaccusative}% no link ucfirst cs
 {\glssplaccusative}% link cs
 {\Glssplaccusative}% link ucfirst cs
 {\GLSsplaccusative}% link all caps cs


% command for usage of genitive:
\newcommand{\glsgen}[1]{%
  \glsdoifexists{#1}{% do something only if the glossary entry has been defined
	\ifthenelse{\equal{\glsentrytype{#1}}{acronym}}{% if the entry is an acronym:
       \ifglsused{#1}{% if this acronym has been used:
	     \glsshortgenitive{#1}% use only the short form
       }{% esle (acronym has not been used yet):
         \glsgenitive{#1} (\glsshortgenitive{#1})% use the long genitive form followed by the short form
         \glsunset{#1}% unset the "first use" flag
       }%
    }{% else (the entry is not an acronym):
       \glsgenitive{#1}%
    }%% fi
  }%
}

% ommand for usage of dative
\newcommand{\glsdat}[1]{%
  \glsdoifexists{#1}{% do something only if the glossary entry has been defined
	\ifthenelse{\equal{\glsentrytype{#1}}{acronym}}{% if the entry is an acronym:
       \ifglsused{#1}{% if this acronym has been used:
	     \glsshortdative{#1}% use only the short form
       }{% esle (acronym has not been used yet):
         \glsdative{#1} (\glsshortdative{#1})% use the long genitive form followed by the short form
         \glsunset{#1}% unset the "first use" flag
       }%
    }{% else (the entry is not an acronym):
       \glsdative{#1}%
    }%% fi
  }%
}

% ommand for usage of accusative
\newcommand{\glsacc}[1]{%
  \glsdoifexists{#1}{% do something only if the glossary entry has been defined
	\ifthenelse{\equal{\glsentrytype{#1}}{acronym}}{% if the entry is an acronym:
       \ifglsused{#1}{% if this acronym has been used:
	     \glsshortaccusative{#1}% use only the short form
       }{% esle (acronym has not been used yet):
         \glsaccusative{#1} (\glsshortacccusative{#1})% use the long accusative form followed by the short form
         \glsunset{#1}% unset the "first use" flag
       }%
    }{% else (the entry is not an acronym):
       \glsaccusative{#1}%
    }%% fi
  }%
}

% command for usage of plural genitive:
\newcommand{\glsplgen}[1]{%
  \glsdoifexists{#1}{% do something only if the glossary entry has been defined
	\ifthenelse{\equal{\glsentrytype{#1}}{acronym}}{% if the entry is an acronym:
       \ifglsused{#1}{% if this acronym has been used:
	     \glssplgenitive{#1}% use only the short form
       }{% esle (acronym has not been used yet):
         \glsplgenitive{#1} (\glssplgenitive{#1})% use the long form followed by the short form
         \glsunset{#1}% unset the "first use" flag
       }%
    }{% else (the entry is not an acronym):
       \glsplgenitive{#1}%
    }%% fi
  }%
}

% command for usage of plural dative
\newcommand{\glspldat}[1]{%
  \glsdoifexists{#1}{% do something only if the glossary entry has been defined
	\ifthenelse{\equal{\glsentrytype{#1}}{acronym}}{% if the entry is an acronym:
       \ifglsused{#1}{% if this acronym has been used:
	     \glsspldative{#1}% use only the short form
       }{% esle (acronym has not been used yet):
         \glsspldative{#1} (\glspldative{#1})% use the long form followed by the short form
         \glsunset{#1}% unset the "first use" flag
       }%
    }{% else (the entry is not an acronym):
       \glspldative{#1}%
    }%% fi
  }%
}

% command for usage of plural accusative
\newcommand{\glsplacc}[1]{%
  \glsdoifexists{#1}{% do something only if the glossary entry has been defined
	\ifthenelse{\equal{\glsentrytype{#1}}{acronym}}{% if the entry is an acronym:
       \ifglsused{#1}{% if this acronym has been used:
	     \glssplaccusative{#1}% use only the short form
       }{% esle (acronym has not been used yet):
         \glssplaccusative{#1} (\glsplaccusative{#1})% use the long form followed by the short form
         \glsunset{#1}% unset the "first use" flag
       }%
    }{% else (the entry is not an acronym):
       \glsplaccusative{#1}%
    }%% fi
  }%
}

%Define shortcuts similar to \ac \acl, acf, acp etc.
\newcommand{\acgen}[1]{\glsgen{#1}}
\newcommand{\acdat}[1]{\glsdat{#1}}
\newcommand{\acacc}[1]{\glsacc{#1}}
\newcommand{\acpgen}[1]{\glsplgen{#1}}
\newcommand{\acpdat}[1]{\glspldat{#1}}
\newcommand{\acpacc}[1]{\glsplacc{#1}}
\newcommand{\acsgen}[1]{\glsshortgenitive{#1}}
\newcommand{\acsdat}[1]{\glsshortdative{#1}}
\newcommand{\acsacc}[1]{\glsshortacccusative{#1}}
\newcommand{\aclgen}[1]{\glsgenitive{#1}}
\newcommand{\acldat}[1]{\glsdative{#1}}
\newcommand{\aclacc}[1]{\glsaccusative{#1}}
% weitere noch zu definieren...


%%\newglossary[alg]{acronym}{acr}{acn}{\acronymname} %Unnötig durch die Option "acronym" des glossaries-Pakets.
\newglossary[nlg]{notation}{not}{ntn}{Notation}
%\newglossary[slg]{symbols}{sls}{slo}{\glssymbolsgroupname}

%%%makeglossaries muss nach \newglossary eingebunden werden!
%\makeglossaries


%%%% Neue Befehle / Makros
% -- new commands by MiG---

\newcommand{\ie}{i.e.,\xspace}
\newcommand{\Ie}{I.e.,\xspace}
\newcommand{\eg}{e.g.,\xspace}
\newcommand{\Eg}{E.g.,\xspace}

\newcommand{\bspw}{bspw.\xspace}
\newcommand{\Bspw}{Bspw.\xspace}
\newcommand{\bzw}{bzw.\xspace}
\newcommand{\ca}{ca.\xspace}
\newcommand{\Ca}{Ca.\xspace}
\newcommand{\dhe}{d.\,h.\xspace}
\newcommand{\Dhe}{D.\,h.\xspace}
\newcommand{\etc}{etc.\xspace}
\newcommand{\eV}{e.\,V.\xspace}
\newcommand{\evtl}{evtl.\xspace}
\newcommand{\Evtl}{Evtl.\xspace}
\newcommand{\ggf}{ggf.\xspace}
\newcommand{\ia}{i.\,a.\xspace}
\newcommand{\iA}{i.\,A.\xspace}
\newcommand{\IA}{I.\,A.\xspace}
\newcommand{\insbes}{insbes.\xspace}
\newcommand{\Insbes}{Insbes.\xspace}
\newcommand{\oae}{o.\,ä.\xspace}
\newcommand{\og}{o.\,g.\xspace}
\newcommand{\sog}{sog.\xspace}
\newcommand{\Sog}{Sog.\xspace}
\newcommand{\teilw}{teilw.\xspace}
\newcommand{\Teilw}{Teilw.\xspace}
\newcommand{\ua}{u.\,a.\xspace}
\newcommand{\Ua}{U.\,a.\xspace}
\newcommand{\usw}{usw.\xspace}
\newcommand{\uU}{u.\,U.\xspace}
\newcommand{\UU}{U.\,U.\xspace}
\newcommand{\vgl}{vgl.\xspace}
\newcommand{\Vgl}{Vgl.\xspace}
\newcommand{\zB}{z.\,B.\xspace}
\newcommand{\zb}{z.\,B.\xspace}
\newcommand{\ZB}{Z.\,B.\xspace}
\newcommand{\Zb}{Z.\,B.\xspace}


%Macro for code inclusions
\newcommand{\code}[1]{\lstinline|#1|}

% macros for emphasizing text
\newcommand{\myemph}[1]{\emph{#1}}
\newcommand{\mydef}[1]{\textbf{#1}}
\newcommand{\myexcl}[1]{\textbf{#1}}

% Short command for backslash in text mode:
\newcommand{\bs}{\textbackslash}
% Short command for printing latex commands (appends backslash in front)
\newcommand{\lc}[1]{{\ttfamily\textbackslash #1}}
% Formatting keywords, menu settings, parameters
\newcommand{\printkeyword}[1]{\enquote{\ttfamily #1}}
% Formatting file names or file paths
\newcommand{\printfilepath}[1]{\enquote{\ttfamily #1}}
% Formatting name of a software (in bold)
\newcommand{\printswname}[1]{{\bfseries #1}}
% Short command for referencing a latex package in the text (adds an index)
\newcommand{\pkg}[1]{{\ttfamily\index{#1}\index{Paket!#1}#1}}

%\newcommand{\bild}[6]% Bild-Pfadname, Beschriftung, Label, Breite, Kurzbeschriftung (für Abbildungsverzeichnis), optional: Platzierung
%{%
%  \begin{figure}[#6]%
%		\Centering%
%        \includegraphics*[width=#4]{#1}%
%        \caption[#5]{\label{#3} #2}%
% \end{figure}
%}



% 1.) Seitliche Kommentare/Abschnitts-Untertitel
%   a.) Für Verwendung im Text (Float-Variante)
\newcommand{\floatmarginnote}[1]{%
\ifthenelse{\boolean{showMarginNotes}}{%if margin notes should be displayed:
%Variante mit Kapitälchen
%\marginpar[\flushleft{\textcolor{gray}{\textsc{#1}}}]{\flushleft{\textcolor{gray}{\textsc{#1}}}}%
%Variante ohne Kapitälchen
\marginpar[\flushleft{\textcolor{gray}{#1}}]{\flushleft{\textcolor{gray}{#1}}}%
}{%else
\relax%
}}

%   b.) Non-Float-Variante (für Verwendung in Gleichungen):
\newcommand{\nonfloatmarginnote}[1]{%
\ifthenelse{\boolean{showMarginNotes}}{%if margin notes should be displayed:
\marginnote[\RaggedRight#1]{\RaggedRight#1}%
%%Variante mit Kapitälchen (obsolet, s.u.)
%\marginnote[\textcolor{gray}{\textsc{#1}}]{\textcolor{gray}{\textsc{#1}}}
%%Variante mit Kapitälchen wird direkt durch die folgenden Befehle in der Präambel gesetzt:
%%\renewcommand*{\raggedleftmarginnote}{}
%%\renewcommand*{\marginfont}{\color{gray}\sffamily\scshape}
}{%else
\relax%
}}

\newlength{\marginwidth}
\setlength{\marginwidth}{\marginparwidth}
\addtolength{\marginwidth}{\marginparsep}


%Breite der Grafiken in einer fbox:
\newcommand{\linewidthwithoutfbox}{\linewidth-2\fboxsep-2\fboxrule}



\newenvironment{myNotationTable}{%
		\setlength{\tabcolsep}{0pt}% Kein Einzug bei Notation
		\renewcommand{\arraystretch}{1.3}% Etwas mehr Abstand zwuischen den Zeilen, damit sie nicht zusammenfasllen
		%% Forderung des KSP-Verlages: keine Worttrennung in einer Tabelle, daher \raggedright statt \RaggedRight
		\begin{longtable}{>{\raggedright\arraybackslash}p{0.21\linewidth-2\tabcolsep}>{\raggedright\arraybackslash}p{0.79\linewidth-2\tabcolsep}}%
	}{%
		\end{longtable}%
	}
\newenvironment{myNotationDescTable}{%
		\setlength{\tabcolsep}{0pt}% Kein Einzug bei Notation
		\renewcommand{\arraystretch}{1.3}%
		%% Forderung des KSP-Verlages: keine Zeilenumbrüche, daher \raggedright statt \RaggedRight
		\begin{longtable}{>{\raggedright\arraybackslash}p{0.3\linewidth-2\tabcolsep}>{\raggedright\arraybackslash}p{0.6\linewidth-2\tabcolsep}>{\centering\arraybackslash}p{0.1\linewidth-2\tabcolsep}}%
	}{%
		\end{longtable}%
	}

\newcommand{\myNotationDescTableEntry}[3]{{#1}&{#2}&{$#3$}\\}
\newcommand{\myNotationTableEntryText}[2]{{#1}&{#2}\\}
\newcommand{\myNotationTableEntryMath}[2]{{$#1$}&{#2}\\}



% In der Arbeit verwendete Acronyme und Glossarbegriffe
%% ----------------------------------------------------------------------------------------
%% In dieser Datei werden die verwendeten Akronyme definiert
%% ----------------------------------------------------------------------------------------
%%
%% ----------------------------------------------------------------------------------------
%% Beschreibung:
%% ----------------------------------------------------------------------------------------
%% \newacronym[]{}{}{} nimmt 4 Argumente
%% - Im ersten, optionalen Argument (in Eckigen Klammern), kann die Pluralform definiert werden (Kurzform und Langform).
%%   Außerdem kann hier Genitiv-, Dativ- und Akkusativ-Form definiert werden, sofern sich diese von der grundform unterscheiden.
%%   Falls der Akronym LaTeX-Befehle beinhaltet, kann hier eine alphanumerischer Sortierschlüssel definiert werden.
%% - Zweites Argument ist das Schlüsselwort. Konvention: zur Markierung des Schlüsselwortes als solches
%%   und zur Unterscheidung dieses von der Kurzform sollte dem Schlüsselwort ein "ac:" vorangestellt werden.
%% - Drittes Argument ist das eigentliche Akronym.
%% - Viertes Argument ist die Langform.
%% ----------------------------------------------------------------------------------------
%%
%% ----------------------------------------------------------------------------------------
%% Beispiel:
% ----------------------------------------------------------------------------------------
\newacronym[shortgenitive={MSAs},%
            genitive={meines schönen Akronyms},%
            %shortdative={MSA},%
            dative={meinem schönen Akronym},%
            %shortaccusative={MSA},%
            %accusative={mein schönes Akronym},%
            shortplural={MSAs},
            longplural={meine schönen Akronyme},%
            %shortpluralgenitive={MSAs},%
            pluralgenitive={meiner schönen Akronyme},%
            %shortpluraldative={MSAs},%
            pluraldative={meinen schönen Akronymen},%
            %shortpluralaccusative={MSAs},%
            pluralaccusative={meine schönen Akronyme},%
            description={mein schönes Akronym, ein Beispiel für eine Abkürzung}% <- optional
           ]{ac:MSA}{MSA}{mein schönes Akronym}
%% ----------------------------------------------------------------------------------------
%%
%%  Nachfolgend eigene Begriffe definieren!
%%
%% ----------------------------------------------------------------------------------------
%%
%% A
%%
%% ----------------------------------------------------------------------------------------
\newacronym{ac:ABS}{ABS}{Anti-lock Braking System}
\newacronym[shortplural={ADAS},longplural={Advanced Driver Assistance Systems}]%
           {ac:ADAS}{ADAS}{Advanced Driver Assistance System}
%% ----------------------------------------------------------------------------------------
%%
%% B
%%
%% ----------------------------------------------------------------------------------------
\newacronym{ac:BOF}{BOF}{Bayesian Occupancy Filter}
%% ----------------------------------------------------------------------------------------
%%
%% C
%%
%% ----------------------------------------------------------------------------------------
\newacronym{ac:CAN}{CAN}{Controller Area Network}
\newacronym[shortplural={CNNs},longplural={Convolutional Neural Networks}]%
           {ac:CNN}{CNN}{Convolutional Neural Network}
\newacronym{ac:CRF}{CRF}{Conditional Random Field}
\newacronym{ac:CTAN}{CTAN}{Comprehensive TeX Archive Network}
%% ----------------------------------------------------------------------------------------
%%
%% D
%%
%% ----------------------------------------------------------------------------------------
\newacronym{ac:DARPA}{DARPA}{Defense Advanced Research Projects Agency}
\newacronym{ac:DBSCAN}{DBSCAN}{Density-Based Spacial Clustering of Applications with Noise}
%% ----------------------------------------------------------------------------------------
%%
%% E
%%
%% ----------------------------------------------------------------------------------------
\newacronym[shortplural={EKFs},longplural={Extended Kalman Filters},
            genitive={Extended Kalman Filters}]%
           {ac:EKF}{EKF}{Extended Kalman Filter}
%% ----------------------------------------------------------------------------------------
%%
%% F
%%
%% ----------------------------------------------------------------------------------------
\newacronym[shortplural={FoVs},longplural={Fields of View}]%
           {ac:FoV}{FoV}{Field of view}
\newacronym[longplural={Frames per Second}]%
           {ac:fps}{FPS}{Frame per Second}
%% ----------------------------------------------------------------------------------------
%%
%% G
%%
%% ----------------------------------------------------------------------------------------
\newacronym{ac:GNN}{GNN}{Global Nearest Neighbor}
\newacronym[shortplural={GUIs},longplural={Graphical User Interfaces}]%
           {ac:GUI}{GUI}{Graphical User Interface}
%% ----------------------------------------------------------------------------------------
%%
%% H
%%
%% ----------------------------------------------------------------------------------------
\newacronym{ac:HMM}{HMM}{Hidden Markov Model}
%% ----------------------------------------------------------------------------------------
%%
%% I
%%
%% ----------------------------------------------------------------------------------------
\newacronym[shortplural={IEKFs},longplural={Iterative Extended Kalman Filters}]%
           {ac:IEKF}{IEKF}{Iterative Extended Kalman Filter}
\newacronym[genitive={Lehrstuhls für Interaktive Echtzeitsysteme}]%
           {ac:IES}{IES}{Lehrstuhl für Interaktive Echtzeitsysteme}
\newacronym[genitive={Fraunhofer-Instituts für Optronik, Systemtechnik und Bildauswertung}]%
           {ac:IOSB}{IOSB}{Fraunhofer-Institut für Optronik, Systemtechnik und Bildauswertung}
%% ----------------------------------------------------------------------------------------
%%
%% J
%%
%% ----------------------------------------------------------------------------------------
\newacronym{ac:JPDA}{JPDA}{Joint Probabilistic Data Association}
%% ----------------------------------------------------------------------------------------
%%
%% K
%%
%% ----------------------------------------------------------------------------------------
\newacronym[shortplural={KFs},longplural={Kalman Filters}]%
           {ac:KF}{KF}{Kalman Filter}
\newacronym{ac:KHYS}{KHYS}{Karlsruhe House of Young Scientists}
\newacronym[genitive={Karlsruher Instituts für Technologie}]%
           {ac:KIT}{KIT}{Karlsruher Institut für Technologie}
\newacronym[name={KSP},% Bezeichnung im Glossar
            description={\acrshort{ac:KIT} Scientific Publishing},% Beschreibung im Glossar
            genitive={\acrshort{ac:KIT} Scientific Publishing Verlages},%
            shortgenitive={KSP Verlages}%
           ]%
           {ac:KSP}{KSP Verlag}{\acrshort{ac:KIT} Scientific Publishing Verlag}
%% ----------------------------------------------------------------------------------------
%%
%% L
%%
%% ----------------------------------------------------------------------------------------
\newacronym{ac:LeastSquareError}{LSE}{Least square error}
%% ----------------------------------------------------------------------------------------
%%
%% M
%%
%% ----------------------------------------------------------------------------------------
\newacronym{ac:MDP}{MDP}{Markov Decision Process}
\newacronym{ac:MHT}{MHT}{Multi-Hypotheses Tracking}
\newacronym{ac:MOT}{MOT}{Multi-Object Tracking}
%% ----------------------------------------------------------------------------------------
%%
%% N
%%
%% ----------------------------------------------------------------------------------------
\newacronym{ac:NN}{NN}{Nearest Neighbor}
%% ----------------------------------------------------------------------------------------
%%
%% O
%%
%% ----------------------------------------------------------------------------------------
\newacronym{ac:OF}{OF}{Optical Flow}
%% ----------------------------------------------------------------------------------------
%%
%% P
%%
%% ----------------------------------------------------------------------------------------
\newacronym[shortplural={PCAs},longplural={Principal Component Analyses}]%
           {ac:PCA}{PCA}{Principal Component Analysis}
\newacronym[shortplural={PFs},longplural={Particle Filters}]%
           {ac:PF}{PF}{Particle Filter}
\newacronym{ac:PDF}{PDF}{Portable Document Format}
\newacronym[shortplural={pdfs},longplural={Probability Density Functions}]%
           {ac:ProbabilityDensityFunction}{pdf}{Probability Density Function}
\newacronym[shortplural={pmfs},longplural={Probability Mass Functions}]%
           {ac:ProbabilityMassFunction}{pmf}{Probability Mass Function}
%% ----------------------------------------------------------------------------------------
%%
%% R
%%
%% ----------------------------------------------------------------------------------------
\newacronym{ac:RANSAC}{RANSAC}{RAndom SAmple Consensus}
\newacronym{ac:RCNN}{R-CNN}{Recurrent Convolutional Neural Network}
\newacronym[shortplural={ROIs},longplural={Regions of Interest}]%
           {ac:ROI}{ROI}{Region of Interest}
%% ----------------------------------------------------------------------------------------
%%
%% S
%%
%% ----------------------------------------------------------------------------------------
\newacronym{ac:SIFT}{SIFT}{Scale-Invariant Feature Transform}
\newacronym{ac:SLAM}{SLAM}{Simultaneous Localization and Mapping}
\newacronym{ac:SURF}{SURF}{Speeded Up Robust Features}
%% ----------------------------------------------------------------------------------------
%%
%% T
%%
%% ----------------------------------------------------------------------------------------
\newacronym{ac:TOMHT}{TOMHT}{Track-Oriented MHT}
%% ----------------------------------------------------------------------------------------
%%
%% U
%%
%% ----------------------------------------------------------------------------------------
\newacronym{ac:UKF}{UKF}{Unscented Kalman Filter}
%% ----------------------------------------------------------------------------------------
%%
%% V
%%
%% ----------------------------------------------------------------------------------------
\newacronym{ac:VID}{VID}{Video Exploitation Systems}
%% ----------------------------------------------------------------------------------------
%%
%% W
%%
%% ----------------------------------------------------------------------------------------
\newacronym{ac:WAMI}{WAMI}{Wide Area Motion Imagery}
\newacronym{ac:WYSIWYG}{WYSIWYG}{What you see is what you get}
%% ----------------------------------------------------------------------------------------
%%
%% X
%%
%% ----------------------------------------------------------------------------------------
%

%% ----------------------------------------------------------------------------------------
%% In dieser Datei werden die verwendeten Glossarbegriffe definiert
%% ----------------------------------------------------------------------------------------
%%
%%
%% ----------------------------------------------------------------------------------------
%% Definition eines eigenes Macro \myglossaryentry[]{}{}{} zur Erstellung von Glossareinträgen
%% ----------------------------------------------------------------------------------------
\newcommand{\myglossaryentry}[4][]{%
	%\AtBeginDocument{%
		\ifx\relax#1\relax%
			\newglossaryentry{#2}{name={#3},description={#4}}%
		\else%
			\newglossaryentry{#2}{name={#3},description={#4},#1}%
		\fi%
	%}%
}% ----------------------------------------------------------------------------------------
%%
%% ----------------------------------------------------------------------------------------
%% Beschreibung:
%% ----------------------------------------------------------------------------------------
%% \myglossaryentry[]{}{}{} nimmt 4 Argumente
%% Im ersten, optionalen Argument (in Eckigen Klammern), kann die Pluralform definiert werden
%% Außerdem kann hier der Sortierschlüssel definiert werden, falls der Akronym LaTeX-Befehle beinhaltet
%% Zweites Argument ist das Schlüsselwort. Konvontion: zur Markierung des Schlüsselwortes als solches
%%    und zur Unterscheidung dieses von der Kurzform sollte dem Schlüsselwort ein "ac:" vorangestellt werden.
%% Drittes Argument ist das eigentliche Akronym.
%% Viertes Argument ist die Langform.
%% ----------------------------------------------------------------------------------------
%%
%% ----------------------------------------------------------------------------------------
%% Beispiel:
%% ----------------------------------------------------------------------------------------
\myglossaryentry[plural={Glossare},%
                 genitive={Glossars}]%
                {gls:Glossar}{Glossar}{alphabetisch sortierte Liste von Begriffen mit Erklärung}
%% ----------------------------------------------------------------------------------------
%%
%% ----------------------------------------------------------------------------------------
%%
%% A
%%
%% ----------------------------------------------------------------------------------------
%
%% ----------------------------------------------------------------------------------------
%%
%% B
%%
%% ----------------------------------------------------------------------------------------
\myglossaryentry{gls:biblatex}{BibLaTex}{Der Nachfolger von BibTex zum Erzeugen
                von Literaturverzeichnissen in LaTeX. Es zeichnet sich vor allem
                durch deutlich bessere Flexibilität bei der Gestaltung des
                Literaturverzeichnisses und der Art und Weise wie Zitatmarken
                gesetzt werden aus. Darüber hinaus ist es vollständig UTF-8-kompatibel.}
\myglossaryentry{gls:bibtex}{BibTeX}{Der Vorgänger von BibLaTex}
\myglossaryentry{gls:BicycleModel}{bicycle model}{Einspurmodell}
%% ----------------------------------------------------------------------------------------
%%
%% C
%%
%% ----------------------------------------------------------------------------------------
%
%% ----------------------------------------------------------------------------------------
%%
%% D
%%
%% ----------------------------------------------------------------------------------------
\myglossaryentry[plural={degrees of freedom}]%
                {gls:DegreeOfFreedom}{degree of freedom}{Freiheitsgrad}
\myglossaryentry{gls:DepthOfField}{depth of field}{Schärfebereich}
%% ----------------------------------------------------------------------------------------
%%
%% E
%%
%% ----------------------------------------------------------------------------------------
%
%% ----------------------------------------------------------------------------------------
%%
%% F
%%
%% ----------------------------------------------------------------------------------------
\myglossaryentry[plural={fields of view}]%
                {gls:FieldOfView}{field of view}{Sichtfeld}
%% ----------------------------------------------------------------------------------------
%%
%% G
%%
%% ----------------------------------------------------------------------------------------
%
%% ----------------------------------------------------------------------------------------
%%
%% H
%%
%% ----------------------------------------------------------------------------------------
%
%% ----------------------------------------------------------------------------------------
%%
%% I
%%
%% ----------------------------------------------------------------------------------------
\myglossaryentry{gls:InterceptTheorem}{intercept theorem}{Strahlensatz}
\myglossaryentry{gls:InstantaneousCenterOfRotation}{instantaneous center of rotation}{Momentanpol}
%% ----------------------------------------------------------------------------------------
%%
%% J
%%
%% ----------------------------------------------------------------------------------------
\myglossaryentry{gls:java}{Java}{Eine von Sun Microsystems 1995 veröffentlichte, objektorientierte Programmiersprache.}
%% ----------------------------------------------------------------------------------------
%%
%% K
%%
%% ----------------------------------------------------------------------------------------
%
%% ----------------------------------------------------------------------------------------
%%
%% L
%%
%% ----------------------------------------------------------------------------------------
\myglossaryentry{gls:latex}{LaTeX}{Eine von Leslie Lamport 1980 entwickelter Satz von Makros zur Erweiterung von TeX.}
\myglossaryentry{gls:LeastSquares}{least squares method}{Methode kleinster Fehlerquadrate}
%% ----------------------------------------------------------------------------------------
%%
%% M
%%
%% ----------------------------------------------------------------------------------------
%
%% ----------------------------------------------------------------------------------------
%%
%% N
%%
%% ----------------------------------------------------------------------------------------
%
%% ----------------------------------------------------------------------------------------
%%
%% O
%%
%% ----------------------------------------------------------------------------------------
%
%% ----------------------------------------------------------------------------------------
%%
%% P
%%
%% ----------------------------------------------------------------------------------------
\myglossaryentry{gls:ProbabilityDensityFunction}{probability density function}{Wahrscheinlichkeitsdichte}
\myglossaryentry{gls:ProbabilityMassFunction}{probability mass function}{Zähldichte}
\myglossaryentry{gls:pgfplots}{PGFplots}{Eine Sammlung von TikZ-Paketen, die ein direktes Erzeugen
                von Diagrammen aller Art (inkl. 3D-Diagramme) direkt aus LaTeX heraus ermöglicht.}
\myglossaryentry[plural={Pakete},%
                 genitive={Pakets}]%
								{gls:pkg}{Paket}{Ein LaTeX-Paket besteht aus einer oder mehrerer Dateien,
								die entweder vorhandene Kernfunktionen von LaTeX umdefinieren und so das Verhalten
								derselbigen bzw. das Erscheinungsbild des fertigen Dokuments verändern
								oder die zusätzliche Befehle zur Verfügung stellen.}
\myglossaryentry{gls:postscript}{PostScript}{Eine von Adobe 1984 entwickelte Seitenbeschreibungssprache.}
%% ----------------------------------------------------------------------------------------
%%
%% R
%%
%% ----------------------------------------------------------------------------------------

%% ----------------------------------------------------------------------------------------
%%
%% S
%%
%% ----------------------------------------------------------------------------------------

%% ----------------------------------------------------------------------------------------
%%
%% T
%%
%% ----------------------------------------------------------------------------------------
\myglossaryentry{gls:tikz}{TikZ}{Eine Sammlung von LaTeX-Paketen, die ein direktes Erzeugen
                von (technischen) Zeichnungen, Diagrammen, etc. in LaTeX erlaubt.}
%% ----------------------------------------------------------------------------------------
%%
%% U
%%
%% ----------------------------------------------------------------------------------------
\myglossaryentry[plural={Umgebungen}]%
                {gls:umgebung}{Umgebung}{Ein Bereich im LaTeX-Code der mit
								\texttt{begin} eingeleitet und mit \texttt{end} beendet wird.
								Umgebungen können auch verschachtelt sein.}
\myglossaryentry{gls:utf8}{UTF-8}{Ein Schema zur Kodierung von Zeichen in computerverarbeitbarer Form,
                             die Zeichen aus allen Sprachen umfasst.}
%% ----------------------------------------------------------------------------------------
%%
%% V
%%
%% ----------------------------------------------------------------------------------------

%% ----------------------------------------------------------------------------------------
%%
%% W
%%
%% ----------------------------------------------------------------------------------------

%% ----------------------------------------------------------------------------------------
%%
%% X
%%
%% ----------------------------------------------------------------------------------------
%
%% ----------------------------------------------------------------------------------------
%%
%% Achtung: Falls die Begriffe im Text nicht mit \gls{label} referenziert werden:
%% Alle Begriffe am Anfang des Dokuments zur Liste hinzufügen
%\AtBeginDocument{%
	%\glsadd{#1}%
%}%
%% Problem: erzeugt eine leere Seite am Anfang des Dokumentes
%% ----------------------------------------------------------------------------------------
%% In dieser Datei werden die verwendeten Symbole als Glossareinträge definiert
%% ----------------------------------------------------------------------------------------
%%
%% ----------------------------------------------------------------------------------------
%% Beschreibung:
%% ----------------------------------------------------------------------------------------
%% \neglossaryentry[]{}{}{} nimmt 4 Argumente
%% - Im ersten, optionalen Argument (in Eckigen Klammern) wird der Sortierschlüssel definiert.
%% - Zweites Argument ist die Makre (Label)
%%   Konvention: der Marke wird ein Prefix "symb:" vorangestellt.
%% - Drittes Argument ist das eigentliche Symbol.
%% - Viertes Argument ist die Beschreibung.
%% ----------------------------------------------------------------------------------------
%%
%% ----------------------------------------------------------------------------------------
%% Beispiel:
%% ----------------------------------------------------------------------------------------
\newglossaryentry{symb:pi}{%
									name={\ensuremath{\pi}},%
									sort={pi},%
									type=symbols,%
									%parent=greekletter,%
									description={Kreiszahl, Verhältnis des Umfangs eines Kreises zu seinem Durchmesser}%
									}
%% ----------------------------------------------------------------------------------------
%% Achtung: Falls die Symbole im Text nicht mit \gls{<label>} referenziert werden:
%% Alle Begriffe am Anfang des Dokuments zur Liste hinzufügen
%\AtBeginDocument{%
%\glsaddall[type=symbols]%
%}%
%% Problem: erzeugt eine leere Seite am Anfang des Dokumentes



%%% Silbentrennung
%% Hier Sonderfälle der Silbentrennung setzen
%
%% English:
\babelhyphenation[english]{%
Bayes-ian
char-ac-terised
data-set
here-by
Kal-man
Mar-kov
over-view
par-a-digm
pa-ra-me-tri-za-ti-on
sur-veil-lance
track-ing
}
%
%% Deutsch:
\babelhyphenation[ngerman]{%
di-men-sio-nale
Kal-man
Kor-res-pon-denz-su-che
kor-res-pon-di-er-en-de
Multi-hypo-thesen
Tra-cking
Zu-ge-hö-rig-keits-wahr-schein-lich-keit
Zu-ge-hö-rig-keits-wahr-schein-lich-keit-en
Zu-stand-schät-zung
}
%


%remove padding in fboxes
\setlength{\fboxsep}{0pt}
% make a wider description in the glossary
%\setlength{\glsdescwidth}{0.7\linewidth}

%% Optionen für das Stichwortverzeichnis:
%% Create the file IndexStyle.ist on the fly:
%\usepackage{filecontents}
%\begin{filecontents}{IndexStyle.ist}
%headings_flag 1 % we use headings for letters
%heading_prefix "{\\textbf{\\hspace{4.5em}" % and they should be bold
%heading_suffix "}}\\nopagebreak\n"
%\end{filecontents}

% Auszufuehrende Befehle  ------------------------------------------------
%%\makeindex[options=-s ../../macros/IndexStyle.ist]
\IfDefined{makeindex}{\makeindex[options=-s preambel/IndexStyle.tex]}
\IfDefined{makenomenclature}{\makenomenclature}
%%makeglossaries muss nach \newglossary eingebunden werden!
\makeglossaries
%% Add all acronyms, glossary entries, etc.
%\glsaddall
\IfPackageLoaded{minitoc}{\IfElseUnDefined{chapter}{\dosecttoc}{\dominitoc}}
\listfiles
%------------------------------------------------------------------------