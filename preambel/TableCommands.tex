%% Kommandos fuer Tabellen. Entnommen aus The LateX Companion, tabsatz.ps und diversen Dokus:

%%% ---| Farben fuer Tabellen |-------------------
\IfPackageLoaded{xcolor}{
   \colorlet{tablesubheadcolor}{gray!30}
   \colorlet{tableheadcolor}{gray!25}
   \colorlet{tableblackheadcolor}{black!100}
   \colorlet{tablerowcolor}{gray!10.0}
}
%%% ---------------------------------------------


%%% -| Neue Spaltendefinitionen 'columntypes' |--
%
% Belegte Spaltentypen:
% l - links
% c - zentriert
% r - rechts
% p,m,b  - oben, mittig, unten
% X - tabularx Auto-Spalte

% um Tabellenspalten mit Flattersatz zu setzen, muss \\ vor
% (z.B.) \raggedright geschuetzt werden:
\newcommand{\PreserveBackslash}[1]{\let\temp=\\#1\let\\=\temp}

%% Forderung des KSP-Verlages: keine Worttrennung in einer Tabelle, daher \raggedright statt \RaggedRight
%\renewcommand\multirowsetup{\RaggedRight}
\renewcommand\multirowsetup{\raggedright}

% Spalten mit Flattersatz und definierte Breite:
% m{} -> mittig
% p{} -> oben
% b{} -> unten
%
%\newcolumntype{L}[1]{>{\hsize=#1\hsize\RaggedRight\arraybackslash}X}%
%\newcolumntype{R}[1]{>{\hsize=#1\hsize\RaggedLeft\arraybackslash}X}%
%\newcolumntype{C}[1]{>{\hsize=#1\hsize\Centering\arraybackslash}X}%
%
%% Forderung des KSP-Verlages: keine Worttrennung in einer Tabelle, daher \raggedright statt \RaggedRight etc.
\newcolumntype{L}[1]{>{\raggedright\arraybackslash}p{#1}} % linksbündig mit Breitenangabe
\newcolumntype{C}[1]{>{\centering\arraybackslash}p{#1}} % zentriert mit Breitenangabe
\newcolumntype{R}[1]{>{\raggedleft\arraybackslash}p{#1}} % rechtsbündig mit Breitenangabe
%
\newcolumntype{M}{>{\begin{minipage}[t]{2cm}\raggedright}c<{\end{minipage}}}
%
\newcolumntype{G}[1]{>{\RaggedLeft\arraybackslash}p{#1}}
\newcolumntype{U}[1]{>{\RaggedRight\arraybackslash}p{#1}}
%\newcolumntype{C}[1]{>{\Centering\arraybackslash}p{#1}}

\newcolumntype{v}[1]{>{\PreserveBackslash\RaggedRight\hspace{0pt}}p{#1}}
\newcolumntype{Y}{>{\PreserveBackslash\RaggedLeft\hspace{0pt}}X}
% Tabellenspaltentyp fuer den Kopf: (Farbe + Ausrichtung)
\newcolumntype{H}[1]{>{\columncolor{tableheadcolor}}l}
% % Rechtsbuendig :
% \newcolumntype{R}[1]{>{\PreserveBackslash\RaggedLeft\hspace{0pt}}m{#1}}
% \newcolumntype{S}[1]{>{\PreserveBackslash\RaggedLeft\hspace{0pt}}p{#1}}
% % Zentriert :
% \newcolumntype{Z}[1]{>{\PreserveBackslash\Centering\hspace{0pt}}m{#1}}
% \newcolumntype{A}[1]{>{\PreserveBackslash\Centering\hspace{0pt}}p{#1}}

%%% Spalten fuer Mathematik
% serifenlose Matheschrift
\newcolumntype{s}[1]{>{\DC@{.}{,}{#1}\mathsf\bgroup}l<{\egroup\DV@end}}

% aequivalent aus typokurz (fett+grau+links)
% \newcolumntype{H}{>{\fontseries{b}\selectfont%
%     \columncolor[gray]{.8}[6pt][0pt]}l}
%%% --------------------------------------------


%%% ---|Listen in Tabellen |--------------------
\newcommand{\removeindentation}{%
	\leftmargini=\labelsep%
	\advance\leftmargini by \labelsep%
}
%
\makeatletter
\newcommand\tableitemize{
	\@minipagetrue%
	\removeindentation
}
\makeatother


%% Aufzählungen in einer Tabelle
\newenvironment{tabitemize}{%
\begin{list}{\textbullet}{%
\setlength\topsep{0pt}%
\setlength\parsep{0pt}%
\setlength\itemsep{0pt}%
\setlength\leftmargin{0em}%
\setlength\leftmargin{1em}%
\setlength\labelwidth{0.5em}%
\setlength\labelsep{0.5em}%
}
}{%
\end{list}
}
%%% --------------------------------------------

%%% ---|Layout der Tabellen |-------------------

% Neue Umgebung fuer Tabellen:

\newenvironment{Tabelle}[2][c]{%
  \tablestylecommon
  \begin{longtable}[#1]{#2}
  }
  {\end{longtable}%
  \tablerestoresettings
}


% Groesse der Schrift in Tabellen
\newcommand{\tablefontsize}{ \footnotesize}
\newcommand{\tableheadfontsize}{\footnotesize}

% Layout der Tabelle: Ausrichtung, Schrift, Zeilenabstand
\newcommand\tablestylecommon{%
  \renewcommand{\arraystretch}{1.4} % Groessere Abstaende zwischen Zeilen
  \normalfont\normalsize            %
  \sffamily\tablefontsize           % Serifenlose und kleine Schrift
  \centering%                       % Tabelle zentrieren
}

\newcommand{\tablestyle}{
	\tablestylecommon
	%\tablealtcolored
}

% Ruecksetzten der Aenderungen
\newcommand\tablerestoresettings{%
  \renewcommand{\arraystretch}{1}% Abstaende wieder zuruecksetzen
  \normalsize\rmfamily % Schrift wieder zuruecksetzen
}

% Tabellenkopf: Serifenlos+fett+schraeg+Schriftfarbe
%\newcommand\tablehead{%
  %\tableheadfontsize%
  %\sffamily\bfseries%
  %%\slshape
  %%\color{white}
%}

\newcommand\tablesubheadfont{%
  \tableheadfontsize%
  \sffamily\bfseries%
  \slshape
  %\color{white}
}


\newcommand\tableheadcolor{%
	%\rowcolor{tablesubheadcolor}
	%\rowcolor{tableblackheadcolor}
	\rowcolor{tableheadcolor}%
}

\newcommand\tablesubheadcolor{%
	\rowcolor{tablesubheadcolor}
	%\rowcolor{tableblackheadcolor}
}


\newcommand{\tableend}{\arrayrulecolor{black}\hline}

% Tabellenkopf (1=Spaltentyp, 2=Text)
% \newcommand{\tablehead}[2]{
%   \multicolumn{1}{#1@{}}{%
%     \raisebox{.1mm}{% Ausrichtung der Beschriftung
%       #2%
%     }\rule{0pt}{4mm}}% unsichtbare Linie, die die Kopfzeile hoeher macht
% }


\newcommand{\tablesubhead}[2]{%
  \multicolumn{#1}{>{\columncolor{tablesubheadcolor}}l}{\tablesubheadfont #2}%
}

% Tabellenbody (=Inhalt)
\newcommand\tablebody{%
\tablefontsize\sffamily\upshape%
}

\newcommand\tableheadshaded{%
	\rowcolor{tableheadcolor}%
}
\newcommand\tablealtcolored{%
	\rowcolors{1}{tablerowcolor}{white!100}%
}
%%% --------------------------------------------

\newlength{\mylen}
\newlength{\adjusthspace}

\newenvironment{tabularc}[2]
{%
	\setlength\mylen{#2/(#1)-\tabcolsep*2-\arrayrulewidth*(#1+1)/(#1)}%
	%\setlength{\adjusthspace}{((#2-1)/2)*\linewidth}
	%\par\noindent
	%\hspace*{-\the\adjusthspace}
	\begin{tabular}%{#2}%
		{*{#1}{v{\the\mylen}}}%
}
{\end{tabular}\par}
