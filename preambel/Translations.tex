%% Begriffe in deutsch oder englisch je nach Hauptsprache

%Makro zur Definition eines neuen multilingualen Bezeichner-Makros
\makeatletter
\newcommand{\newlanguagecommand}[1]{%
  \newcommand#1{%
    \@ifundefined{\string#1\languagename}
      {``No definition of \texttt{\string#1} for \languagename. Please define it in the file Translations.tex !''}
      {\@nameuse{\string#1\languagename}}%
  }%
}
%Mit diesem Makro können Varianten in verschiedenen Sprachen hinzugefügt werden.
\newcommand{\addtolanguagecommand}[3]{%
  \@namedef{\string#1#2}{#3}}
\makeatother


%Name für das Lesezeichen für die Titelseite
\newlanguagecommand{\TransTitlePageName}
\addtolanguagecommand{\TransTitlePageName}{ngerman}{Titelseite}
\addtolanguagecommand{\TransTitlePageName}{english}{Cover}

\newlanguagecommand{\TransOwnPublications}
\addtolanguagecommand{\TransOwnPublications}{ngerman}{Publikationen}
\addtolanguagecommand{\TransOwnPublications}{english}{Publications}

\newlanguagecommand{\TransOwnPatents}
\addtolanguagecommand{\TransOwnPatents}{ngerman}{Patente}
\addtolanguagecommand{\TransOwnPatents}{english}{Patents}

\newlanguagecommand{\TransSupervisedTheses}
\addtolanguagecommand{\TransSupervisedTheses}{ngerman}{Betreute studentische Arbeiten}
\addtolanguagecommand{\TransSupervisedTheses}{english}{Supervised student theses}

\newlanguagecommand{\AppendixName}
\addtolanguagecommand{\AppendixName}{ngerman}{Anhang}
\addtolanguagecommand{\AppendixName}{english}{Appendix}

\newlanguagecommand{\TodoListName}
\addtolanguagecommand{\TodoListName}{ngerman}{Todo-Liste}
\addtolanguagecommand{\TodoListName}{english}{Todo List}


%% Befehle für korrekte Behandlung von Textteilen in anderer Sprache (je nach gewählter Hauptsprache)
\ifthenelse{\boolean{iesenglishs}}{%
	% Hauptsprache ist Englisch
	\newcommand{\textInEnglish}[1]{#1}
	\newcommand{\textInGerman}[1]{\foreignlanguage{ngerman}{#1}}
}{%
	% Hauptsprache ist Deutsch
	\newcommand{\textInEnglish}[1]{\foreignlanguage{english}{#1}}
	\newcommand{\textInGerman}[1]{#1}
}


\newcommand{\theoremname}{} % initialization
\newcommand{\examplename}{} % initialization
%\newcommand{\proofname}{} % initialization
\newcommand{\definitionname}{} % initialization
\newcommand{\lemmaname}{} % initialization
\newcommand{\corollaryname}{} % initialization
\newcommand{\propositionname}{} % initialization

\addto\captionsenglish{%
  \renewcommand{\theoremname}{Theorem}%
  \renewcommand{\examplename}{Example}%
	%\renewcommand{\proofname}{Proof}%
	\renewcommand{\definitionname}{Definition}%
	\renewcommand{\lemmaname}{Lemma}%
	\renewcommand{\corollaryname}{Corollary}%
	\renewcommand{\propositionname}{Proposition}%
}
\addto\captionsngerman{%
  \renewcommand{\theoremname}{Satz}%
  \renewcommand{\examplename}{Beispiel}%
	%\renewcommand{\proofname}{Beweis}%
	\renewcommand{\definitionname}{Definition}%
	\renewcommand{\lemmaname}{Lemma}%
	\renewcommand{\corollaryname}{Korollar}%
	\renewcommand{\propositionname}{Proposition}%
}

%\newtheorem{theorem}{Satz}[chapter]
%\newtheorem{definition}[theorem]{Definition}
%\newtheorem{lemma}[theorem]{Lemma}
%\newtheorem{corollary}[theorem]{Corollary}
%\newtheorem{proposition}[theorem]{Proposition}

