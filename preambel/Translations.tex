%% Begriffe in deutsch oder englisch je nach Hauptsprache

%Makro zur Definition eines neuen multilingualen Bezeichner-Makros
\makeatletter
\newcommand{\newlanguagecommand}[1]{%
  \newcommand#1{%
    \@ifundefined{\string#1\languagename}
      {``No definition of \texttt{\string#1} for \languagename. Please define it in the file Translations.tex !''}
      {\@nameuse{\string#1\languagename}}%
  }%
}
%Mit diesem Makro können Varianten in verschiedenen Sprachen hinzugefügt werden.
\newcommand{\addtolanguagecommand}[3]{%
  \@namedef{\string#1#2}{#3}}
\makeatother


%Name für das Lesezeichen für die Titelseite
\newlanguagecommand{\TransTitlePageName}
\addtolanguagecommand{\TransTitlePageName}{ngerman}{Titelseite}
\addtolanguagecommand{\TransTitlePageName}{english}{Cover}

\newlanguagecommand{\TransOwnPublications}
\addtolanguagecommand{\TransOwnPublications}{ngerman}{Publikationen}
\addtolanguagecommand{\TransOwnPublications}{english}{Publications}

\newlanguagecommand{\TransOwnPatents}
\addtolanguagecommand{\TransOwnPatents}{ngerman}{Patente}
\addtolanguagecommand{\TransOwnPatents}{english}{Patents}

\newlanguagecommand{\TransSupervisedTheses}
\addtolanguagecommand{\TransSupervisedTheses}{ngerman}{Betreute studentische Arbeiten}
\addtolanguagecommand{\TransSupervisedTheses}{english}{Supervised student theses}

\newlanguagecommand{\AppendixName}
\addtolanguagecommand{\AppendixName}{ngerman}{Anhang}
\addtolanguagecommand{\AppendixName}{english}{Appendix}

\newlanguagecommand{\TodoListName}
\addtolanguagecommand{\TodoListName}{ngerman}{Todo-Liste}
\addtolanguagecommand{\TodoListName}{english}{Todo List}


%% Befehle für korrekte Behandlung von Textteilen in anderer Sprache (je nach gewählter Hauptsprache)
\ifthenelse{\boolean{iesenglishs}}{%
	% Hauptsprache ist Englisch
	\newcommand{\textInEnglish}[1]{#1}
	\newcommand{\textInGerman}[1]{\foreignlanguage{ngerman}{#1}}
}{%
	% Hauptsprache ist Deutsch
	\newcommand{\textInEnglish}[1]{\foreignlanguage{english}{#1}}
	\newcommand{\textInGerman}[1]{#1}
}


\newtheorem{theorem}{Satz}[chapter]
\newtheorem{definition}[theorem]{Definition}
\newtheorem{lemma}[theorem]{Lemma}
\newtheorem{corollary}[theorem]{Corollary}
\newtheorem{proposition}[theorem]{Proposition}

\crefname{chapter}{Kapitel}{Kapitel}
\crefname{section}{Abschnitt}{Abschnitte}
\crefname{subsection}{Abschnitt}{Abschnitte}
\crefname{figure}{Abbildung}{Abbildungen}
\crefname{table}{Tabelle}{Tabellen}
\crefname{appendix}{Anhang}{Anhänge}

\crefname{theorem}{Satz}{Sätze}
\crefname{definition}{Definition}{Definitionen}
\crefname{lemma}{Lemma}{Lemmata}
\crefname{corollary}{Korollar}{Korollare}
\crefname{proposition}{Proposition}{Propositionen}

\newcommand{\crefpairconjunction}{ und }
\newcommand{\crefmiddleconjunction}{, }
\newcommand{\creflastconjunction}{ und }
\newcommand{\crefpairgroupconjunction}{ sowie }
\newcommand{\crefmiddlegroupconjunction}{, }
\newcommand{\creflastgroupconjunction}{ sowie }
