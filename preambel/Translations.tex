%%%%%%%%%%%%%%%%%%%%%%%%%%%%%%%%%%%%%%%%%%%%%%%%%%%%%%%%%%%%%
%% Multilinguale Begriffsdefinitionen
%% Begriffe in Deutsch oder Englisch je nach Hauptsprache
%% Begriffe in weiteren Sprachen können bei Bedarf hinzugefügt werden
%%%%%%%%%%%%%%%%%%%%%%%%%%%%%%%%%%%%%%%%%%%%%%%%%%%%%%%%%%%%%

%Makro zur Definition eines neuen multilingualen Bezeichner-Makros
\makeatletter
% Mit diesem Makro können Varianten in verschiedenen Sprachen hinzugefügt werden.
% Sie werden auch zu der Liste der Babel-Übersetzungen hinzugefügt
%Sofern ein gleichbenannter String existiert, wird er umbenannt.
\newcommand{\addtolanguagecommand}[3]{%
	% create a command with such name if it does not exist yet
	% generate a command that prints a warning if the command is used with a language,
	% for which no translation has been defined
	\ifdef{#1}{}{%
	  \newcommand#1{%
			\@ifundefined{\string#1\languagename}%
				{%
					\GenericError{}{No definition of \string#1 for \languagename. Please define it in the file Translations.tex !}%
					\texorpdfstring{%
						``Placeholder for \texttt{\string#1} in \languagename. Please define the missing translation in the file Translations.tex !''%
					}{%
						Placeholder. Please define the missing translation in the file Translations.tex !%
					}%
				}{%
					\@nameuse{\string#1\languagename}%
				}%
		}%
  }%
	%define own translation for the required language
	\@namedef{\string#1#2}{#3}%
	%add or override babel translation in the required language
	\expandafter\addto\csname captions#2\endcsname{\renewcommand{#1}{#3}}%
}
\makeatother

%-----------------------------------------------------------
%% Define own multilingual macros:
%-----------------------------------------------------------

%Name für das Lesezeichen für die Titelseite
\addtolanguagecommand{\TransTitlePageName}{ngerman}{Titelseite}
\addtolanguagecommand{\TransTitlePageName}{english}{Cover}

\addtolanguagecommand{\TransOwnPublications}{ngerman}{Publikationen}
\addtolanguagecommand{\TransOwnPublications}{english}{Publications}

\addtolanguagecommand{\TransOwnPatents}{ngerman}{Patente}
\addtolanguagecommand{\TransOwnPatents}{english}{Patents}

\addtolanguagecommand{\TransSupervisedTheses}{ngerman}{Betreute studentische Arbeiten}
\addtolanguagecommand{\TransSupervisedTheses}{english}{Supervised student theses}

\addtolanguagecommand{\AppendixName}{ngerman}{Anhang}
\addtolanguagecommand{\AppendixName}{english}{Appendix}

\addtolanguagecommand{\TodoListName}{ngerman}{Todo-Liste}
\addtolanguagecommand{\TodoListName}{english}{Todo List}

\addtolanguagecommand{\TransAcknowledgements}{ngerman}{Danksagung}
\addtolanguagecommand{\TransAcknowledgements}{english}{Acknowledgements}

\addtolanguagecommand{\TransTheorem}{ngerman}{Satz}
\addtolanguagecommand{\TransTheorem}{english}{Theorem}

\addtolanguagecommand{\TransExample}{ngerman}{Beispiel}
\addtolanguagecommand{\TransExample}{english}{Example}

\addtolanguagecommand{\TransLemma}{ngerman}{Lemma}
\addtolanguagecommand{\TransLemma}{english}{Lemma}

\addtolanguagecommand{\TransCorollary}{ngerman}{Korollar}
\addtolanguagecommand{\TransCorollary}{english}{Corollary}

\addtolanguagecommand{\TransProposition}{ngerman}{Proposition}
\addtolanguagecommand{\TransProposition}{english}{Proposition}


\addtolanguagecommand{\TransDefinition}{ngerman}{Definition}
\addtolanguagecommand{\TransDefinition}{english}{Definition}

\addtolanguagecommand{\TransProof}{ngerman}{Beweis}
\addtolanguagecommand{\TransProof}{english}{Proof}

%-----------------------------------------------------------
%% Redefine existing translations from the babel package:
%-----------------------------------------------------------

\addtolanguagecommand{\indexname}{ngerman}{Stichwortverzeichnis}
\addtolanguagecommand{\indexname}{english}{Index}

\addtolanguagecommand{\glssymbolsgroupname}{ngerman}{Symbolverzeichnis}
\addtolanguagecommand{\glssymbolsgroupname}{english}{Symbols}

\addtolanguagecommand{\acronymname}{ngerman}{Abkürzungsverzeichnis}
\addtolanguagecommand{\acronymname}{english}{Acronyms}



%-----------------------------------------------------------
%% Befehle für korrekte Behandlung von Textteilen in anderer Sprache
%% (je nach gewählter Hauptsprache)
%-----------------------------------------------------------
\ifthenelse{\boolean{iesenglishs}}{%
	% Hauptsprache ist Englisch
	\newcommand{\textInEnglish}[1]{#1}
	\newcommand{\textInGerman}[1]{\foreignlanguage{ngerman}{#1}}
}{%
	% Hauptsprache ist Deutsch
	\newcommand{\textInEnglish}[1]{\foreignlanguage{english}{#1}}
	\newcommand{\textInGerman}[1]{#1}
}

