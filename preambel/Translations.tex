%% Begriffe in deutsch oder englisch je nach Hauptsprache

%Makro zur Definition eines neuen multilingualen Bezeichner-Makros
\makeatletter
\newcommand{\newlanguagecommand}[1]{%
  \newcommand#1{%
    \@ifundefined{\string#1\languagename}
      {``No definition of \texttt{\string#1} for \languagename. Please define it in the file Translations.tex !''}
      {\@nameuse{\string#1\languagename}}%
  }%
}
%Mit diesem Makro können Varianten in verschiedenen Sprachen hinzugefügt werden.
\newcommand{\addtolanguagecommand}[3]{%
  \@namedef{\string#1#2}{#3}}
\makeatother


%Name für das Lesezeichen für die Titelseite
\newlanguagecommand{\TransTitlePageName}
\addtolanguagecommand{\TransTitlePageName}{ngerman}{Titelseite}
\addtolanguagecommand{\TransTitlePageName}{english}{Cover}

\newlanguagecommand{\TransOwnPublications}
\addtolanguagecommand{\TransOwnPublications}{ngerman}{Publikationen}
\addtolanguagecommand{\TransOwnPublications}{english}{Publications}

\newlanguagecommand{\TransOwnPatents}
\addtolanguagecommand{\TransOwnPatents}{ngerman}{Patente}
\addtolanguagecommand{\TransOwnPatents}{english}{Patents}

\newlanguagecommand{\TransSupervisedTheses}
\addtolanguagecommand{\TransSupervisedTheses}{ngerman}{Betreute studentische Arbeiten}
\addtolanguagecommand{\TransSupervisedTheses}{english}{Supervised student theses}
