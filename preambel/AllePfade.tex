% Pfad und Name der BibTeX-Datei,
% Pfade zu Grafik-Dateien,
% erlaubte Erweiterungen für Grafikdateien
%
%
% Pfad und Name der BibteX-Datei(en)
% Hier alle bib-Ressourcen angeben, die Literaturreferenzen enthalten,
% welche im Text der Arbeit referenziert werden bzw. im Haupt-Literaturverzeichnis
% stehen sollen.
% Achtung: die Bibliografien sollten in biblatex mit UTF8-Notation formatiert sein!
\addbibresource{./bib/Diss.bib}
\addbibresource{./bib/example.bib}

% Man könnte eigene Publikationen, Patente und betreute Arbeiten in jeweils
% eigenen Bibliografie-Dateien vorhalten, kann aber alle Angaben auch
% in einer einziger Datei verwalten (z.B Diss.bib).
% Bei Erstellung dieser Zusatz-Literaturverzeichnisse werden jeweils
% folgende Makros verwendet. Sie können auf eine der oben angegebenen Dateien
% oder auf jeweils eine andere Datei zeigen.
\newcommand{\bibpathOwnPatents}{./bib/Diss.bib}
\newcommand{\bibpathOwnPublications}{./bib/Diss.bib}
\newcommand{\bibpathStudentTheses}{./bib/Diss.bib}


%% Pfade zu Bildern:
%% Verwendete Pfade: (Reihenfolge wichtig, da Durchsuchen in dieser Reihenfolge erfolgt!!!)
%% Achtung: Es gibt keine Warnung, falls Pfade an mehreren Stellen gesetzt werden!
\graphicspath{{./images/}{../Papers/images/Diss/}{../Papers/images/}}
%
% Grafikdatei-Erweiterungen
\DeclareGraphicsExtensions{.pdf,.png,.jpg,.jpeg,.bmp,.eps}
