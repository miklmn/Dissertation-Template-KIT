% -- new commands by MiG---

% 1.) Seitliche Kommentare/Abschnitts-Untertitel
%   a.) Für Verwendung im Text (Float-Variante)
\newcommand{\partitle}[1]{%
\ifthenelse{\boolean{showMarginNotes}}{%if margin notes should be displayed:
%Variante mit Kapitälchen
\marginpar[\flushleft{\textcolor{gray}{\textsc{#1}}}]{\flushleft{\textcolor{gray}{\textsc{#1}}}}%
%Variante ohne Kapitälchen
%\marginpar[\flushleft{\textcolor{gray}{#1}}]{\flushleft{\textcolor{gray}{#1}}}%
}{%else
\relax%
}}

%   b.) Non-Float-Variante (für Verwendung in Gleichungen):
\newcommand{\partitlenonfloat}[1]{%
\ifthenelse{\boolean{showMarginNotes}}{%if margin notes should be displayed:
\marginnote[\RaggedRight#1]{\RaggedRight#1}%
%%Variante mit Kapitälchen (obsolet, s.u.)
%\marginnote[\textcolor{gray}{\textsc{#1}}]{\textcolor{gray}{\textsc{#1}}}
%%Variante mit Kapitälchen wird direkt durch die folgenden Befehle in der Präambel gesetzt:
%%\renewcommand*{\raggedleftmarginnote}{}
%%\renewcommand*{\marginfont}{\color{gray}\sffamily\scshape}
}{%else
\relax%
}}

%2.) Befehle zum Rumspielen mit Layout:
%Zusätzliche Seitenumbrüche, die bei Bedarf deaktiviert werden können
%\newcommand{\mynewpage}{\newpage}
\newcommand{\mynewpage}{\relax}
\newcommand{\myspecialpagebreak}{\clearpage}
%\newcommand{\myspecialpagebreak}{\relax}
%%
%

%Breite der Grafiken in einer fbox:
\newcommand{\linewidthwithoutfbox}{\linewidth-2\fboxsep-2\fboxrule}

\newcommand{\code}[1]{\lstinline|#1|}
\newcommand{\myemph}[1]{\emph{#1}}
\newcommand{\mydef}[1]{\textbf{#1}}


%Hilfsmakros:

% Definition von Hilfsmacros \showif und \hideif, die
% das Verstecken bzw. Einblenden von Textteilen ermöglichen werden:
% Es wird eine Boolean-Variable mit dem angegebenen Namen angelegt
% und entsprechend auf true (showif) oder false (hideif) gesetzt.
% Außerdem werden entsprechende includeversion/excludeversion-Macros aufgerufen.
\newcommand{\showif}[1]{%
	\ifcsname#1\endcsname%
		%falls Boolean bereits existiert: nichts tun
	\else%
		\newboolean{#1}%
	\fi%
	\setboolean{#1}{true}%
	\includeversion{#1}%
}
\newcommand{\hideif}[1]{%
	\ifcsname#1\endcsname%
		%falls Boolean bereits existiert: nichts tun
	\else%
		\newboolean{#1}%
	\fi%
	\setboolean{#1}{false}%
	\excludeversion{#1}%
}


%Ein TikZ-TODO-Bild:
\newcommand{\myTodoImg}{%
  \begin{tikzpicture}
   \draw[
     use as bounding box,
     black,
     thick,
     double, 
     double distance = 4pt, 
     rounded corners = 6pt] 
    (-4,-3)rectangle(4,3);    
   \node[rotate=45,scale = 5,opacity = 0.5] at (0,0) {TODO!};
  \end{tikzpicture}%
}

%% Erzeugung einer ToDo-Liste
%% newlistof bedarf das Paket tocloft (\usepackage{tocloft})
\newlistof{TODOs}{tdo}{\TodoListName}%tdo ist die Erweiterung der zu bildenden Hilfsdatei


%% Befehl zum Eintragen der Inhalte in die ToDo-Liste
\newcommand{\addToTodoList}[1]{%
	\refstepcounter{TODOs}%
	\addcontentsline{tdo}{TODOs}{%
		%Füge den Eintrag als eine neue Zeile zur ToDo-Liste hinzu
		\protect{\numberline{\theTODOs:}} % (\thechapter.\thesection)}}
		\ifx\relax#1\relax% Falls Argument #1 leer:
				\textcolor{red}{ToDo!}%
		\else% Falls #1 mit Inhalt:
				\textcolor{red}{#1}%
		\fi%
	}%
}


\newcounter{imgtdoCounter}


%TODO-Gestaltungsmakro für Textmodus mit einer Unterscheidung,
%ob etwas übergeben worden ist oder nicht
%und mit einem optionalen Argument für die Nachfrage-Liste
\newcommand{\myTodo}[2][]{%
	\ifthenelse{\boolean{showTODOs}}{%if TODOs should be displayed:
		\ifx\relax#2\relax% Falls der erforderliche Argument #2 leer:
			\hyperlink{TODO-List}{\textcolor{red}{\textbf{ToDo!}}}%
				\ifx\relax#2\relax% Falls der optionale Argument #1 ebenfalls leer:
					\addToTodoList{\textcolor{red}{ToDo!}}%
				\else% #2 leer aber #1 mit Inhalt
					\addToTodoList{\textcolor{red}{#1}}%
				\fi%
		\else% Falls der erforderliche Argument #2 mit Inhalt:
			\hyperlink{TODO-List}{\textcolor{red}{\textbf{ToDo: }}}\textcolor{red}{ #2}%
				\ifx\relax#1\relax% Falls der optionale Argument #1 leer:
					\addToTodoList{\textcolor{red}{#2}}
				\else% sowohl #2 als auch #1 mit Inhalt
					\addToTodoList{\textcolor{red}{#1}}
				\fi%
		\fi%
	}{%if no TODOs should be displayed: don't do anything
	\ }%
}

%% TODO-Gestaltungsmakro für Mathe-Modus mit einer Unterscheidung,
%% ob etwas übergeben worden ist oder nicht
%% und mit einem optionalen Argument für die TODO-Liste
%% Verwendung, z.B. \myMathTodo[Alternativtext für die Übersicht]{x+y=z}
\newcommand{\myMathTodo}[2][]{%
	\ifthenelse{\boolean{showTODOs}}{%if TODOs should be displayed:
		\ifx\relax#2\relax% Falls der erforderliche Argument #2 leer:
			\hyperlink{TODO-List}{\textcolor{red}{\symbf{\textnormal{ ToDo! }}}}%
			\ifx\relax#1\relax% Falls der optionale Argument #1 auch leer:
				\addToTodoList{TODO in an equation}%
			\else% Argument #2 leer, aber Argument #1 nicht leer
				\addToTodoList{#1}%
			\fi%
		\else% Falls der erforderliche Argument #2 mit Inhalt:
			\hyperlink{TODO-List}{\textcolor{red}{\symbf{\textcolor{red}{#2}}}}%
			\ifx\relax#1\relax% Falls der optionale Argument #1 leer:
				\addToTodoList{$#2$}%
			\else% sowohl #1 als auch #2 mit Inhalt
				\addToTodoList{#1}%
			\fi%
		\fi%
	}{%if no TODOs should be displayed: don't do anything
	\. }%
}


%TODO-Makro für Zitate
%mit einem optionalen Argument für die TODO-Liste
\newcommand{\myCiteTodo}[1]{%
	\ifx\relax#1\relax%
		\myTodo[Literaturverweis!]{Referenz!}%
	\else%
			\myTodo[Bib-Eintrag für #1!]{[#1]}%
	\fi%
}

%TODO-Makro für Querverweise
%mit einem optionalen Argument für die TODO-Liste
\newcommand{\myRefTodo}[1]{%
	\ifx\relax#1\relax%
		\myTodo[Verweis!]{Querverweis!}%
	\else%
		\myTodo[Verweis auf #1!]{(#1)}%
	\fi%
}


%TODO-Makro für Fußnoten
%mit dem optionalen ersten Argument für die TODO-Liste
\newcommand{\myFootTodo}[2][]{%
	\ifx\relax#1\relax%
		\myTodo[#2]{\footnote{#2}}%
	\else%
		\myTodo[#1]{\footnote{#2}}%
	\fi%
}


%Makro für Qualltext-TODOs
%mit einem optionalen Argument für die TODO-Liste
\newcommand{\myRemTodo}[1]{\myTodo[#1]{s. Quelltext}}

%#1: [optional] short caption
%#2: caption
%#3: label, z.B. fig:Blockschaltbild
\newcommand{\myImgTodo}[3][]
{
	% increase image-todo counter
	\stepcounter{imgtdoCounter}
	\ifx\relax#1\relax%
		\addToTodoList{Bild: #2}%
	\else%
		\addToTodoList{Bild: #1}%
	\fi%
 % new figure floating environment
 \begin{figure}[!htb]
  \centering
  % TODO-image
	\myTodoImg
	%\includegraphics[width=0.8\textwidth]{\myTodoImg}
  \caption[#1]{\hyperlink{TODO-List}{\color{red}{ToDo:}} \color{red}{#2}}
  \label{#3}
 \end{figure}
}



%TODO-Befehle für den Textmodus
%optional argument #1 for alternative text in the TODO list
\newcommand{\todo}[2][]{\myTodo[#1]{#2}}
\newcommand{\ToDo}[2][]{\myTodo[#1]{#2}}
\newcommand{\Todo}[2][]{\myTodo[#1]{#2}}
\newcommand{\TODO}[2][]{\myTodo[#1]{#2}}


%Add a ToDo for a citation
\newcommand{\citetodo}[1]{\myCiteTodo{#1}}
\newcommand{\citeToDo}[1]{\myCiteTodo{#1}}
\newcommand{\citeTodo}[1]{\myCiteTodo{#1}}
\newcommand{\citeTODO}[1]{\myCiteTodo{#1}}

%add a ToDo for a reference
\newcommand{\reftodo}[1]{\myRefTodo{#1}}
\newcommand{\refToDo}[1]{\myRefTodo{#1}}
\newcommand{\refTodo}[1]{\myRefTodo{#1}}
\newcommand{\refTODO}[1]{\myRefTodo{#1}}

%add a ToDo as a footnote
\newcommand{\foottodo}[2][]{\myFootTodo[#1]{#2}}
\newcommand{\footToDo}[2][]{\myFootTodo[#1]{#2}}
\newcommand{\footTodo}[2][]{\myFootTodo[#1]{#2}}
\newcommand{\footTODO}[2][]{\myFootTodo[#1]{#2}}

%TODO-Befehle für den Mathe-Modus
% Verwendung, z.B.
% \mathTodo[Alternativtext für die Übersicht]{x+y=z}
\newcommand{\mathtodo}[2][]{\myMathTodo[#1]{#2}}
\newcommand{\mathToDo}[2][]{\myMathTodo[#1]{#2}}
\newcommand{\mathTodo}[2][]{\myMathTodo[#1]{#2}}
\newcommand{\mathTODO}[2][]{\myMathTodo[#1]{#2}}

%Erzeugen eines TODO-Bildes:
%[#1]: optionale Kurzbeschriftung (in eckigen Klammern)
%{#2}: Beschriftung
%{#3}: Label, z.B. fig:Blockschaltbild
\newcommand{\imgtodo}[3][]{\myImgTodo[#1]{#2}{#3}}
\newcommand{\imgToDo}[3][]{\myImgTodo[#1]{#2}{#3}}
\newcommand{\imgTodo}[3][]{\myImgTodo[#1]{#2}{#3}}
\newcommand{\imgTODO}[3][]{\myImgTodo[#1]{#2}{#3}}


\newenvironment{myNotationTable}{%
		\setlength{\tabcolsep}{0pt}% Kein Einzug bei Notation
		\renewcommand{\arraystretch}{1.3}% Etwas mehr Abstand zwuischen den Zeilen, damit sie nicht zusammenfasllen
		% Vorgabe vom KIT-Verlag: keine Worttrennung in einer Tabelle, daher \raggedright statt \RaggedRight
		\begin{longtable}{>{\raggedright\arraybackslash}p{0.21\linewidth-2\tabcolsep}>{\raggedright\arraybackslash}p{0.79\linewidth-2\tabcolsep}}%
	}{%
		\end{longtable}%
	}
\newenvironment{myNotationDescTable}{%
		\setlength{\tabcolsep}{0pt}% Kein Einzug bei Notation
		\renewcommand{\arraystretch}{1.3}%
		% Vorgabe vom KIT-Verlag: keine Zeilenumbrüche, daher \raggedright statt \RaggedRight
		\begin{longtable}{>{\raggedright\arraybackslash}p{0.3\linewidth-2\tabcolsep}>{\raggedright\arraybackslash}p{0.6\linewidth-2\tabcolsep}>{\raggedright\arraybackslash}p{0.1\linewidth-2\tabcolsep}}%
	}{%
		\end{longtable}%
	}

\newcommand{\myNotationDescTableEntry}[3]{{#1}&{#2}&{$#3$}\\}
\newcommand{\myNotationTableEntryText}[2]{{#1}&{#2}\\}
\newcommand{\myNotationTableEntryMath}[2]{{$#1$}&{#2}\\}
\newcommand{\mySymbolListEntry}[4]{%
	%\AtBeginDocument{%
		\newglossaryentry{#1}{type=notation,name={\ensuremath{#3}},description={#4},sort={#2}}%
	%}%
	%% Falls die Begriffe im Text nicht referenziert werden:
	%% alle Begriffe am Anfang des Dokuments zur Liste hinzufügen
	%% Problem: erzeuge eine leere Seite am Anfang des Dokumentes
	%\AtBeginDocument{%
	%	\glsadd{#1}%
	%}%
}%

\newcommand{\ie}{i.e.,\ }
\newcommand{\eg}{e.g.,\ }
\newcommand{\dhe}{d.\,h.\ }
\newcommand{\Dhe}{D.\,h.\ }
\newcommand{\eV}{e.\,V.\ }
\newcommand{\zB}{z.\,B.\ }
\newcommand{\zb}{z.\,B.\ }
\newcommand{\oae}{o.\,\"{a}.\ }
\newcommand{\uU}{u.\,U.\ }
\newcommand{\ua}{u.\,a.\ }
\newcommand{\ia}{i.\,a.\ }
\newcommand{\og}{o.\,g.\ }
\newcommand{\bzw}{bzw.\ }
\newcommand{\usw}{usw.\ }
\newcommand{\vgl}{vgl.\ }
\newcommand{\etc}{etc.\ }
\newcommand{\evtl}{evtl.\ }
\newcommand{\ca}{ca.\ }
\newcommand{\sog}{sog.\ }
\newcommand{\insbes}{insbes.\ }

\newcommand{\mb}[1]{\ensuremath{\symbfit{#1}}}

\DeclareMathOperator*{\argmax}{arg\,max}
\DeclareMathOperator*{\argmin}{arg\,min}


% Die Definition eigener mathematischer Befehle ist besonders sinnvoll, wenn diese im Dokument oft verwendet werden.
% Man kann dann hier an zentraler Stelle z.B. Vektoren mit Pfeil statt fett formatieren.


\newcommand{\bild}[6]      % Bild-Pfadname, Beschriftung, Label, Breite, Kurzbeschriftung (für Abbildungsverzeichnis), optional: Platzierung
{
     \begin{figure}[#6]
				\Centering
        \includegraphics*[width=#4]{#1}
        \caption[#5]{\label{#3} #2}
     \end{figure}
}


\makeatletter
\newcommand{\figureref}[1]{(Abbildung \ref{#1})}%
\newcommand{\eqnref}[1]{(\ref{#1})}%

% Command for margin text with usefull style
%\newcommand{\marginlabel}[1]{\mbox{}\marginline{\hspace{0pt}\footnotesize\sffamily #1}}%
\newcommand{\marginlabel}[1]{\marginnote{#1}}%

\newlength{\marginwidth}
\setlength{\marginwidth}{\marginparwidth}
\addtolength{\marginwidth}{\marginparsep}
