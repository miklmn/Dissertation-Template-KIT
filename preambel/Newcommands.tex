% -- new commands by MiG---

\newcommand{\ie}{i.e.,\ }
\newcommand{\eg}{e.g.,\ }
\newcommand{\dhe}{d.\,h.\ }
\newcommand{\Dhe}{D.\,h.\ }
\newcommand{\eV}{e.\,V.\ }
\newcommand{\zB}{z.\,B.\ }
\newcommand{\zb}{z.\,B.\ }
\newcommand{\oae}{o.\,ä.\ }
\newcommand{\uU}{u.\,U.\ }
\newcommand{\ua}{u.\,a.\ }
\newcommand{\ia}{i.\,a.\ }
\newcommand{\og}{o.\,g.\ }
\newcommand{\bzw}{bzw.\ }
\newcommand{\usw}{usw.\ }
\newcommand{\vgl}{vgl.\ }
\newcommand{\etc}{etc.\ }
\newcommand{\evtl}{evtl.\ }
\newcommand{\ca}{ca.\ }
\newcommand{\sog}{sog.\ }
\newcommand{\teilw}{teilw.\ }
\newcommand{\insbes}{insbes.\ }
\newcommand{\bspw}{bspw.\ }


%Macro for code inclusions
\newcommand{\code}[1]{\lstinline|#1|}

% macros for emphasizing text
\newcommand{\myemph}[1]{\emph{#1}}
\newcommand{\mydef}[1]{\textbf{#1}}
\newcommand{\myexcl}[1]{\textbf{#1}}

% Short command for backslash in text mode:
\newcommand{\bs}{\textbackslash}
% Short command for printing latex commands (appends backslash in front)
\newcommand{\lc}[1]{{\ttfamily\textbackslash #1}}
% Formatting keywords, menu settings, parameters
\newcommand{\printkeyword}[1]{\enquote{\ttfamily #1}}
% Formatting file names or file paths
\newcommand{\printfilepath}[1]{\enquote{\ttfamily #1}}
% Formatting name of a software (in bold)
\newcommand{\printswname}[1]{{\bfseries #1}}
% Short command for referencing a latex package in the text (adds an index)
\newcommand{\pkg}[1]{{\ttfamily\index{#1}\index{Paket!#1}#1}}

%\newcommand{\bild}[6]% Bild-Pfadname, Beschriftung, Label, Breite, Kurzbeschriftung (für Abbildungsverzeichnis), optional: Platzierung
%{%
%  \begin{figure}[#6]%
%		\Centering%
%        \includegraphics*[width=#4]{#1}%
%        \caption[#5]{\label{#3} #2}%
% \end{figure}
%}



% 1.) Seitliche Kommentare/Abschnitts-Untertitel
%   a.) Für Verwendung im Text (Float-Variante)
\newcommand{\floatmarginnote}[1]{%
\ifthenelse{\boolean{showMarginNotes}}{%if margin notes should be displayed:
%Variante mit Kapitälchen
%\marginpar[\flushleft{\textcolor{gray}{\textsc{#1}}}]{\flushleft{\textcolor{gray}{\textsc{#1}}}}%
%Variante ohne Kapitälchen
\marginpar[\flushleft{\textcolor{gray}{#1}}]{\flushleft{\textcolor{gray}{#1}}}%
}{%else
\relax%
}}

%   b.) Non-Float-Variante (für Verwendung in Gleichungen):
\newcommand{\nonfloatmarginnote}[1]{%
\ifthenelse{\boolean{showMarginNotes}}{%if margin notes should be displayed:
\marginnote[\RaggedRight#1]{\RaggedRight#1}%
%%Variante mit Kapitälchen (obsolet, s.u.)
%\marginnote[\textcolor{gray}{\textsc{#1}}]{\textcolor{gray}{\textsc{#1}}}
%%Variante mit Kapitälchen wird direkt durch die folgenden Befehle in der Präambel gesetzt:
%%\renewcommand*{\raggedleftmarginnote}{}
%%\renewcommand*{\marginfont}{\color{gray}\sffamily\scshape}
}{%else
\relax%
}}

\newlength{\marginwidth}
\setlength{\marginwidth}{\marginparwidth}
\addtolength{\marginwidth}{\marginparsep}


%Breite der Grafiken in einer fbox:
\newcommand{\linewidthwithoutfbox}{\linewidth-2\fboxsep-2\fboxrule}



\newenvironment{myNotationTable}{%
		\setlength{\tabcolsep}{0pt}% Kein Einzug bei Notation
		\renewcommand{\arraystretch}{1.3}% Etwas mehr Abstand zwuischen den Zeilen, damit sie nicht zusammenfasllen
		%% Forderung des KSP-Verlages: keine Worttrennung in einer Tabelle, daher \raggedright statt \RaggedRight
		\begin{longtable}{>{\raggedright\arraybackslash}p{0.21\linewidth-2\tabcolsep}>{\raggedright\arraybackslash}p{0.79\linewidth-2\tabcolsep}}%
	}{%
		\end{longtable}%
	}
\newenvironment{myNotationDescTable}{%
		\setlength{\tabcolsep}{0pt}% Kein Einzug bei Notation
		\renewcommand{\arraystretch}{1.3}%
		%% Forderung des KSP-Verlages: keine Zeilenumbrüche, daher \raggedright statt \RaggedRight
		\begin{longtable}{>{\raggedright\arraybackslash}p{0.3\linewidth-2\tabcolsep}>{\raggedright\arraybackslash}p{0.6\linewidth-2\tabcolsep}>{\centering\arraybackslash}p{0.1\linewidth-2\tabcolsep}}%
	}{%
		\end{longtable}%
	}

\newcommand{\myNotationDescTableEntry}[3]{{#1}&{#2}&{$#3$}\\}
\newcommand{\myNotationTableEntryText}[2]{{#1}&{#2}\\}
\newcommand{\myNotationTableEntryMath}[2]{{$#1$}&{#2}\\}

