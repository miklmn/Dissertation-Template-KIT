% *** Mathematik **************************************
%
% amsmath schon vorher geladen da es vor pst-pdf geladen werden muss

\ifxetex
	% hier nichts tun
\else
%%% Doc: ftp://tug.ctan.org/pub/tex-archive/macros/latex/contrib/mh/doc/mathtools.pdf
% Erweitert amsmath und behebt einige Bugs.
% Muss vor ntheorem geladen werden!
%\usepackage{mathtools}
%\usepackage[disallowspaces]{mathtools}
%\usepackage[fixamsmath,disallowspaces]{mathtools}


%%% Doc: http://www.ctan.org/info?id=fixmath
% LaTeX's default style of typesetting mathematics does not comply
% with the International Standards ISO31-0:1992 to ISO31-13:1992
% which indicate that uppercase Greek letters always be typset
% upright, as opposed to italic (even though they usually
% represent variables) and allow for typsetting of variables in a
% boldface italic style (even though the required fonts are
% available). This package ensures that uppercase Greek be typeset
% in italic style, that upright $\Delta$ and $\Omega$ symbols are
% available through the commands \upDelta and \upOmega; and
% provides a new math alphabet \mathbold for boldface
% italic letters, including Greek.


%\usepackage{fixmath}
%%%%%PW: fixmath ist schuld, dass mathbold mit eulervm nicht mehr funktioniert


	%bei Verwendung von xelatex folgendes nicht einbinden:
	
	%%% Doc: ftp://tug.ctan.org/pub/tex-archive/macros/latex/contrib/onlyamsmath/onlyamsmath.dvi
	% Warnt bei Benutzung von Befehlen die mit amsmath inkompatibel sind.
	% verursacht Probleme mit XeLaTeX
	\usepackage[%
		all,%
		warning%
	]{onlyamsmath}


%
%%PW:
% Ohne (!) bm-Paket gehen Formeln in Überschriften nicht mehr --> too many math alphabets used in version normal
\newcommand\hmmax{0}
\newcommand\bmmax{1}  %%2 ist schon zuviel
\usepackage{bm}
% bm an sich gibt aber auch schnell "too many math alphabets" Fehler. Insg. kann TeX 16 Mathe-Alphabete
%% Kann man wegbekommen, indem man die heavy families reduziert (hmmax) und die bold families (bmmax)
%% http://www.tex.ac.uk/cgi-bin/texfaq2html?label=manymathalph
%\renewcommand{\hmmax}{0}
%\renewcommand{\bmmax}{3}
%Boldmath mit \bm Befehl. Empfohlen im LaTeX-Begleiter. Laden nach allen Pakten, die Mathefont-Einstellungen machen.
\fi

%Alternative for XeLaTeX-Befehlen zur Textauszeichnung \symbf und \mathbfcal
\makeatletter
\@ifpackageloaded{unicode-math}{%
%nichts tun
}{%ansonsten fehlende Makros definieren
%
%\usepackage{amssymb}   %for using \mathbb{}, \nexists etc.
%
%Folgende Makros, die in unicode-math definiert sind, brauchen das bm-Paket
\newcommand{\symbf}[1]{\bm{#1}}
\newcommand{\symbfit}[1]{\bm{#1}}
\newcommand{\symbfcal}[1]{\bm{\mathcal{#1}}}
\newcommand{\mathbfcal}[1]{\bm{\mathcal{#1}}}
}
\makeatother




%%PW: Für Beispiele, Sätze, Lemmata: ntheorem
%%PW: amsthm gibt es auch noch, ist aber älter
% Muss nach mathtools geladen werden!
%\usepackage[thmmarks,standard,hyperref]{ntheorem} 
%\usepackage[amsmath,thmmarks,standard,hyperref]{ntheorem}   %hyperref-Option macht Probleme  %im Minimaltest aber nicht
%\usepackage{thmtools}   %Fuer \listoftheorems  %Nicht in MikTeX 2.8 %Wird nicht benutzt
%\usepackage{theoremref}  %Für bessere Referenzierung

%\fi


%% PW: Asymptote Unterstützung. Asymptote braucht auch ein TeX-\write von denen es nur 16 gibt. Bin schon am Limit, deswegen Asymptote XOR comment (braucht auch eins).
%\usepackage[inline]{asymptote}  %%gibts erst nach Installation von Hand
%\usepackage[% %%gehört zu MikTeX, macht aber nicht das, was man sich erhofft.
%process=all,%
%%process=none,
%]{asyfig}

%------------------------------------------------------

% -- Vektor fett darstellen -----------------
% \let\oldvec\vec
% \def\vec#1{{\boldsymbol{#1}}} %Fetter Vektor
% \newcommand{\ve}{\vec} %
% -------------------------------------------


%%% Doc: ftp://tug.ctan.org/pub/tex-archive/macros/latex/contrib/misc/braket.sty
%\usepackage{braket}  % Quantenmechanik Bracket Schreibweise

%%% Doc: ftp://tug.ctan.org/pub/tex-archive/macros/latex/contrib/misc/cancel.sty
\usepackage{cancel}  % Durchstreichen

%%% Doc: ftp://tug.ctan.org/pub/tex-archive/macros/latex/contrib/mh/doc/empheq.pdf
\usepackage{empheq}  % Hervorheben

%%% Doc: ftp://tug.ctan.org/pub/tex-archive/info/math/voss/mathmode/Mathmode.pdf
%\usepackage{exscale} % Skaliert Mathe-Modus Ausgaben in allen Umgebungen richtig.

%%% Doc: ftp://tug.ctan.org/pub/tex-archive/macros/latex/contrib/was/icomma.dtx
% Erlaubt die Benutzung von Kommas im Mathematikmodus
\usepackage{icomma}


%%PW stackrel
\usepackage{stackrel} %provides enhanced \stackrel and \stackbin
%\stackrel [subscript] {superscript} {rel}
%\stackbin [subscript] {superscript} {bin}
%Example:
%A \stackbin[\text{and}]{}{+} B \stackrel[x]{!}{=} C

%%% Doc: http://www.ctex.org/documents/packages/special/units.pdf
%\usepackage[nice]{nicefrac}
\usepackage{xfrac}

%%PW commath
%commath -- A LaTeX class which provides some commands which help you to format formulas flexibly.
%Differentiale aufrecht setzen mit vorgefertigten Operatoren, Intervalle, Senkrechter Evaluationsstrich, Norm
%%% PW: Gibt aber leider Probleme mit figureref-Befehl und wenn das draußen ist mit "`too many math alphabets"'
%\usepackage{commath}


%%% Tauschen von Epsilon und andere:
% \let\ORGvarrho=\varrho
% \let\varrho=\rho
% \let\rho=\ORGvarrho
%
\let\ORGvarepsilon=\varepsilon
\let\varepsilon=\epsilon
\let\epsilon=\ORGvarepsilon
%
% \let\ORGvartheta=\vartheta
% \let\vartheta=\theta
% \let\theta=\ORGvartheta
%
% \let\ORGvarphi=\varphi
% \let\varphi=\phi
% \let\phi=\ORGvarphi

% Vier Indizes an beliebigen Zeichen (2 links, 2 rechts)
\usepackage{fouridx}

% Matrizen mit Randbeschriftung
\usepackage{blkarray}


% ~~~~~~~~~~~~~~~~~~~~~~~~~~~~~~~~~~~~~~~~~~~~~~~~~~~~~~~~~~~~~~~~~~~~~~~~
% Symbole
% ~~~~~~~~~~~~~~~~~~~~~~~~~~~~~~~~~~~~~~~~~~~~~~~~~~~~~~~~~~~~~~~~~~~~~~~~
%
%%% General Doc: http://www.ctan.org/tex-archive/info/symbols/comprehensive/symbols-a4.pdf
%
%% Symbole für Mathematiksatz
%\usepackage{mathrsfs} %% Schreibschriftbuchstaben für den Mathematiksatz (nur Großbuchstaben)   %%%%% PW: Auch einfach mal reingemacht
%\usepackage{dsfont}   %% Double Stroke Fonts   %%%% PW: Für Mengensymbole R Q Z N
%\usepackage[mathcal]{euscript} %% adds euler mathcal font    %%%% PW: Mathcal ist zwar bei Eulervm schon irgendwie drin. Ist zuviel. Gibt dann "`too many math alphabets"' Fehler.
%\usepackage{amssymb}      %%%% knallt mit eufrak, was das gleiche macht
%\usepackage[Symbolsmallscale]{upgreek} % upright symbols from euler package [Euler] or Adobe Symbols [Symbol]
%\usepackage[upmu]{gensymb}             % Option upmu  %% bekomme Fehler: command \upmu is undefined

%% Allgemeine Symbole
%\usepackage{wasysym}  %% Doc: http://www.ctan.org/tex-archive/macros/latex/contrib/wasysym/wasysym.pdf
%\usepackage{marvosym} %% Symbole aus der marvosym Schrift
%\usepackage{pifont}   %% ZapfDingbats