% *** Mathematik **************************************
%
% Eine Definition eigener mathematischer Befehle ist besonders sinnvoll, wenn diese im Dokument oft verwendet werden.
% Man kann dann hier an zentraler Stelle z.B. alle Vektoren mit Pfeil statt fett formatieren.
%
%




% amsmath schon vorher geladen da es vor pst-pdf geladen werden muss

\ifxetex
	% hier nichts tun
\else
%%% Doc: ftp://tug.ctan.org/pub/tex-archive/macros/latex/contrib/mh/doc/mathtools.pdf
% Erweitert amsmath und behebt einige Bugs.
% Muss vor ntheorem geladen werden!
%\usepackage{mathtools}
%\usepackage[disallowspaces]{mathtools}
%\usepackage[fixamsmath,disallowspaces]{mathtools}


%%% Doc: http://www.ctan.org/info?id=fixmath
% LaTeX's default style of typesetting mathematics does not comply
% with the International Standards ISO31-0:1992 to ISO31-13:1992
% which indicate that uppercase Greek letters always be typset
% upright, as opposed to italic (even though they usually
% represent variables) and allow for typsetting of variables in a
% boldface italic style (even though the required fonts are
% available). This package ensures that uppercase Greek be typeset
% in italic style, that upright $\Delta$ and $\Omega$ symbols are
% available through the commands \upDelta and \upOmega; and
% provides a new math alphabet \mathbold for boldface
% italic letters, including Greek.


%\usepackage{fixmath}
%%%%%PW: fixmath ist schuld, dass mathbold mit eulervm nicht mehr funktioniert


	%bei Verwendung von xelatex folgendes nicht einbinden:
	
	%%% Doc: ftp://tug.ctan.org/pub/tex-archive/macros/latex/contrib/onlyamsmath/onlyamsmath.dvi
	% Warnt bei Benutzung von Befehlen die mit amsmath inkompatibel sind.
	% verursacht Probleme mit XeLaTeX
	\usepackage[%
		all,%
		warning%
	]{onlyamsmath}


%
%%PW:
% Ohne (!) bm-Paket gehen Formeln in Überschriften nicht mehr --> too many math alphabets used in version normal
\newcommand\hmmax{0}
\newcommand\bmmax{1}  %%2 ist schon zuviel
\usepackage{bm}
% bm an sich gibt aber auch schnell "too many math alphabets" Fehler. Insg. kann TeX 16 Mathe-Alphabete
%% Kann man wegbekommen, indem man die heavy families reduziert (hmmax) und die bold families (bmmax)
%% http://www.tex.ac.uk/cgi-bin/texfaq2html?label=manymathalph
%\renewcommand{\hmmax}{0}
%\renewcommand{\bmmax}{3}
%Boldmath mit \bm Befehl. Empfohlen im LaTeX-Begleiter. Laden nach allen Pakten, die Mathefont-Einstellungen machen.
\fi

%Alternative for XeLaTeX-Befehlen zur Textauszeichnung \symbf und \mathbfcal
\makeatletter
\@ifpackageloaded{unicode-math}{%
%nichts tun
}{%ansonsten fehlende Makros definieren
%
%\usepackage{amssymb}   %for using \mathbb{}, \nexists etc.
%
%Folgende Makros, die in unicode-math definiert sind, brauchen das bm-Paket
\newcommand{\symbf}[1]{\bm{#1}}
\newcommand{\symbfit}[1]{\bm{#1}}
\newcommand{\symbfcal}[1]{\bm{\mathcal{#1}}}
\newcommand{\mathbfcal}[1]{\bm{\mathcal{#1}}}
}
\makeatother




%%PW: Für Beispiele, Sätze, Lemmata: ntheorem
\usepackage{amsthm}
%%PW: amsthm gibt es auch noch, ist aber älter
% Muss nach mathtools geladen werden!
%\usepackage{ntheorem}
%\usepackage[standard]{ntheorem}  %funktioniert nicht, da mit der Option "Standard" asmsymb eingebunden wird, das Probleme ergibt
%\usepackage[thmmarks,standard,hyperref]{ntheorem} 
%\usepackage[amsmath,thmmarks,standard,hyperref]{ntheorem}   %hyperref-Option macht Probleme  %im Minimaltest aber nicht
%\usepackage{thmtools}   %Fuer \listoftheorems  %Nicht in MikTeX 2.8 %Wird nicht benutzt
%\usepackage{theoremref}  %Für bessere Referenzierung
\newtheorem{theorem}{\protect\TransTheorem}[chapter]
\newtheorem{definition}[theorem]{\protect\TransDefinition}
\newtheorem{proposition}[theorem]{\protect\TransProposition}
\newtheorem{lemma}[theorem]{\protect\TransLemma}
\newtheorem{example}{\protect\TransExample}
%\newtheorem{proof}{\protect\TransProof}
\newtheorem{cor}{\protect\TransCorollary}

%% Changing the qed symbol from white square to something else:
%\renewcommand\qedsymbol{QED}% just "QED"
%\renewcommand\qedsymbol{$\blacksquare$}% black square, needs amssymb package!


%\fi


%% PW: Asymptote Unterstützung. Asymptote braucht auch ein TeX-\write von denen es nur 16 gibt. Bin schon am Limit, deswegen Asymptote XOR comment (braucht auch eins).
%\usepackage[inline]{asymptote}  %%gibts erst nach Installation von Hand
%\usepackage[% %%gehört zu MikTeX, macht aber nicht das, was man sich erhofft.
%process=all,%
%%process=none,
%]{asyfig}

%------------------------------------------------------

% -- Vektor fett darstellen -----------------
% \let\oldvec\vec
% \def\vec#1{{\boldsymbol{#1}}} %Fetter Vektor
% \newcommand{\ve}{\vec} %
% -------------------------------------------


%%% Doc: ftp://tug.ctan.org/pub/tex-archive/macros/latex/contrib/misc/braket.sty
%\usepackage{braket}  % Quantenmechanik Bracket Schreibweise

%%% Doc: ftp://tug.ctan.org/pub/tex-archive/macros/latex/contrib/misc/cancel.sty
\usepackage{cancel}  % Durchstreichen

%%% Doc: ftp://tug.ctan.org/pub/tex-archive/macros/latex/contrib/mh/doc/empheq.pdf
\usepackage{empheq}  % Hervorheben

%%% Doc: ftp://tug.ctan.org/pub/tex-archive/info/math/voss/mathmode/Mathmode.pdf
%\usepackage{exscale} % Skaliert Mathe-Modus Ausgaben in allen Umgebungen richtig.

%%% Doc: ftp://tug.ctan.org/pub/tex-archive/macros/latex/contrib/was/icomma.dtx
% Erlaubt die Benutzung von Kommas im Mathematikmodus
\usepackage{icomma}


%%PW stackrel
\usepackage{stackrel} %provides enhanced \stackrel and \stackbin
%\stackrel [subscript] {superscript} {rel}
%\stackbin [subscript] {superscript} {bin}
%Example:
%A \stackbin[\text{and}]{}{+} B \stackrel[x]{!}{=} C

%%% Doc: http://www.ctex.org/documents/packages/special/units.pdf
%\usepackage[nice]{nicefrac}
\usepackage{xfrac}

%%PW commath
%commath -- A LaTeX class which provides some commands which help you to format formulas flexibly.
%Differentiale aufrecht setzen mit vorgefertigten Operatoren, Intervalle, Senkrechter Evaluationsstrich, Norm
%%% PW: Gibt aber leider Probleme mit figureref-Befehl und wenn das draußen ist mit "`too many math alphabets"'
%\usepackage{commath}


%%% Tauschen von Epsilon und andere:
% \let\ORGvarrho=\varrho
% \let\varrho=\rho
% \let\rho=\ORGvarrho
%
\let\ORGvarepsilon=\varepsilon
\let\varepsilon=\epsilon
\let\epsilon=\ORGvarepsilon
%
% \let\ORGvartheta=\vartheta
% \let\vartheta=\theta
% \let\theta=\ORGvartheta
%
% \let\ORGvarphi=\varphi
% \let\varphi=\phi
% \let\phi=\ORGvarphi

% Vier Indizes an beliebigen Zeichen (2 links, 2 rechts)
\usepackage{fouridx}

% Matrizen mit Randbeschriftung
\usepackage{blkarray}


% ~~~~~~~~~~~~~~~~~~~~~~~~~~~~~~~~~~~~~~~~~~~~~~~~~~~~~~~~~~~~~~~~~~~~~~~~
% Symbole
% ~~~~~~~~~~~~~~~~~~~~~~~~~~~~~~~~~~~~~~~~~~~~~~~~~~~~~~~~~~~~~~~~~~~~~~~~
%
%%% General Doc: http://www.ctan.org/tex-archive/info/symbols/comprehensive/symbols-a4.pdf
%
%% Symbole für Mathematiksatz
%\usepackage{mathrsfs} %% Schreibschriftbuchstaben für den Mathematiksatz (nur Großbuchstaben)   %%%%% PW: Auch einfach mal reingemacht
%\usepackage{dsfont}   %% Double Stroke Fonts   %%%% PW: Für Mengensymbole R Q Z N
%\usepackage[mathcal]{euscript} %% adds euler mathcal font    %%%% PW: Mathcal ist zwar bei Eulervm schon irgendwie drin. Ist zuviel. Gibt dann "`too many math alphabets"' Fehler.
%\usepackage{amssymb}      %%%% knallt mit eufrak, was das gleiche macht
%\usepackage[Symbolsmallscale]{upgreek} % upright symbols from euler package [Euler] or Adobe Symbols [Symbol]
%\usepackage[upmu]{gensymb}             % Option upmu  %% bekomme Fehler: command \upmu is undefined

%% Allgemeine Symbole
%\usepackage{wasysym}  %% Doc: http://www.ctan.org/tex-archive/macros/latex/contrib/wasysym/wasysym.pdf
%\usepackage{marvosym} %% Symbole aus der marvosym Schrift
%\usepackage{pifont}   %% ZapfDingbats


\newcommand{\mb}[1]{\ensuremath{\symbfit{#1}}}



\newcommand{\Ast}{\ensuremath{\mathord{\ast}}}
\newcommand{\Sim}{\ensuremath{\mathord{\sim}}}
\newcommand{\Cdot}{\ensuremath{\mathord{\,\cdot\,}}}
\newcommand{\Tr}{\ensuremath{\mathsf{T}}}
\newcommand{\const}{\ensuremath{\mathord{\mathrm{const}}}}

\newcommand{\SkalProd}[2]{\langle #1,#2 \rangle}
\newcommand{\SkalProdD}[2]{\left\langle #1,#2 \right\rangle}

\DeclareMathOperator{\supp}{supp}
\DeclareMathOperator{\rect}{rect}
\DeclareMathOperator{\ld}{ld}
\DeclareMathOperator{\SO}{SO}
\DeclareMathOperator{\Var}{Var}
\DeclareMathOperator{\Cov}{Cov}
\DeclareMathOperator{\vol}{vol}
\DeclareMathOperator{\tr}{tr}
\DeclareMathOperator*{\argmin}{arg\,min}
\DeclareMathOperator*{\argmax}{arg\,max}
\DeclareMathOperator{\grad}{grad}
\DeclareMathOperator{\Arg}{Arg}
\DeclareMathOperator{\col}{col}
\DeclareMathOperator{\spn}{span}
\DeclareMathOperator{\aff}{aff}
% http://tex.stackexchange.com/questions/84302/what-is-the-difference-of-mathop-operatorname-and-declaremathoperator
\newcommand{\diff}{\mathop{}\!\mathrm{d}}
\newcommand{\ceq}{\mathrel{\mathop:}=}


% ------------------------------------------------------------------------------
% Expectation value:
% Erwartungswert:
% ------------------------------------------------------------------------------
%\DeclareMathOperator{\E}{E}
\DeclareMathOperator{\E}{\mathbb{E}}
%\newcommand{\E}{\mathbb{E}}
\newcommand{\Erw}[1]{ \E[#1] }
\newcommand{\ERW}[1]{ \E\bigl[#1\bigr] } %Mit großen Klammern
% ------------------------------------------------------------------------------

% ------------------------------------------------------------------------------
% Random variables, command is small "rv" followed by letter
% ------------------------------------------------------------------------------
\newcommand{\randomvar}[1]{#1}

\newcommand{\rvA}{\randomvar{A}}
\newcommand{\rvB}{\randomvar{B}}
\newcommand{\rvC}{\randomvar{C}}
\newcommand{\rvD}{\randomvar{D}}
\newcommand{\rvE}{\randomvar{E}}
\newcommand{\rvF}{\randomvar{F}}
\newcommand{\rvG}{\randomvar{G}}
\newcommand{\rvH}{\randomvar{H}}
\newcommand{\rvI}{\randomvar{I}}
\newcommand{\rvJ}{\randomvar{J}}
\newcommand{\rvK}{\randomvar{K}}
\newcommand{\rvL}{\randomvar{L}}
\newcommand{\rvM}{\randomvar{M}}
\newcommand{\rvN}{\randomvar{N}}
\newcommand{\rvO}{\randomvar{O}}
\newcommand{\rvP}{\randomvar{P}}
\newcommand{\rvQ}{\randomvar{Q}}
\newcommand{\rvR}{\randomvar{R}}
\newcommand{\rvS}{\randomvar{S}}
\newcommand{\rvT}{\randomvar{T}}
\newcommand{\rvU}{\randomvar{U}}
\newcommand{\rvV}{\randomvar{V}}
\newcommand{\rvW}{\randomvar{W}}
\newcommand{\rvX}{\randomvar{X}}
\newcommand{\rvY}{\randomvar{Y}}
\newcommand{\rvZ}{\randomvar{Z}}
\newcommand{\rvGamma}{\randomvar{\Gamma}}
% ------------------------------------------------------------------------------


% ------------------------------------------------------------------------------
% Multidimensionale Zufallsvariablen:
% Multidimensional random variables, command is small "mrv" followed by letter
% ------------------------------------------------------------------------------
\newcommand{\mdrv}[1]{\symbfit{#1}}

\newcommand{\mdrvA}{\mdrv{A}}
\newcommand{\mdrvB}{\mdrv{B}}
\newcommand{\mdrvC}{\mdrv{C}}
\newcommand{\mdrvD}{\mdrv{D}}
\newcommand{\mdrvE}{\mdrv{E}}
\newcommand{\mdrvF}{\mdrv{F}}
\newcommand{\mdrvG}{\mdrv{G}}
\newcommand{\mdrvH}{\mdrv{H}}
\newcommand{\mdrvI}{\mdrv{I}}
\newcommand{\mdrvJ}{\mdrv{J}}
\newcommand{\mdrvK}{\mdrv{K}}
\newcommand{\mdrvL}{\mdrv{L}}
\newcommand{\mdrvM}{\mdrv{M}}
\newcommand{\mdrvN}{\mdrv{N}}
\newcommand{\mdrvO}{\mdrv{O}}
\newcommand{\mdrvP}{\mdrv{P}}
\newcommand{\mdrvQ}{\mdrv{Q}}
\newcommand{\mdrvR}{\mdrv{R}}
\newcommand{\mdrvS}{\mdrv{S}}
\newcommand{\mdrvT}{\mdrv{T}}
\newcommand{\mdrvU}{\mdrv{U}}
\newcommand{\mdrvV}{\mdrv{V}}
\newcommand{\mdrvW}{\mdrv{W}}
\newcommand{\mdrvX}{\mdrv{X}}
\newcommand{\mdrvY}{\mdrv{Y}}
\newcommand{\mdrvZ}{\mdrv{Z}}
\newcommand{\mdrvGamma}{\mdrv{\Gamma}}
% ------------------------------------------------------------------------------


% ------------------------------------------------------------------------------
% Estimation, command is mall "es" followed by letter
% ------------------------------------------------------------------------------
\newcommand{\esti}[1]{\hat{#1}}

\newcommand{\esa}{\esti{a}}
\newcommand{\esb}{\esti{b}}
\newcommand{\esc}{\esti{c}}
\newcommand{\esd}{\esti{d}}
\newcommand{\ese}{\esti{e}}
\newcommand{\esf}{\esti{f}}
\newcommand{\esg}{\esti{g}}
\newcommand{\esh}{\esti{h}}
\newcommand{\esi}{\esti{i}}
\newcommand{\esj}{\esti{j}}
\newcommand{\esk}{\esti{k}}
\newcommand{\esl}{\esti{l}}
\newcommand{\esm}{\esti{m}}
\newcommand{\esn}{\esti{n}}
\newcommand{\eso}{\esti{o}}
\newcommand{\esp}{\esti{p}}
\newcommand{\esq}{\esti{q}}
\newcommand{\esr}{\esti{r}}
\newcommand{\ess}{\esti{s}}
\newcommand{\est}{\esti{t}}
\newcommand{\esu}{\esti{u}}
\newcommand{\esv}{\esti{v}}
\newcommand{\esw}{\esti{w}}
\newcommand{\esx}{\esti{x}}
\newcommand{\esy}{\esti{y}}
\newcommand{\esz}{\esti{z}}
\newcommand{\esgamma}{\esti{\gamma}}
\newcommand{\essigma}{\esti{\sigma}}
\newcommand{\esomega}{\esti{\omega}}
\newcommand{\eskappa}{\esti{\kappa}}
\newcommand{\esmu}{\esti{\mu}}
\newcommand{\esSigma}{\esti{\Sigma}}
% ------------------------------------------------------------------------------



%-------------------------------------------------------------------------------
% Matrices and vectors:
% Matrizen und Vektoren:
% 
% Vectors, command is "v" followed by lowercase letter
% Matrices, command is "m" followed by capital letter
%-------------------------------------------------------------------------------
\newcommand{\vecfont}[1]{\symbf{#1}}
\newcommand{\matfont}[1]{\symbf{#1}}

\newcommand{\Vektor}[1]{\vecfont{#1}}
\newcommand{\Matrix}[1]{\matfont{#1}}
\newcommand{\Vek}[1]{\vecfont{#1}}
\newcommand{\Mat}[1]{\matfont{#1}}

\newcommand{\va}{\vecfont{a}}
\newcommand{\vb}{\vecfont{b}}
\newcommand{\vc}{\vecfont{c}}
\newcommand{\vd}{\vecfont{d}}
\newcommand{\ve}{\vecfont{e}}
\newcommand{\vf}{\vecfont{f}}
\newcommand{\vg}{\vecfont{g}}
\newcommand{\vh}{\vecfont{h}}
\newcommand{\vi}{\vecfont{i}}
\newcommand{\vj}{\vecfont{j}}
\newcommand{\vk}{\vecfont{k}}
\newcommand{\vl}{\vecfont{l}}
\newcommand{\vm}{\vecfont{m}}
\newcommand{\vn}{\vecfont{n}}
\newcommand{\vo}{\vecfont{o}}
\newcommand{\vp}{\vecfont{p}}
\newcommand{\vq}{\vecfont{q}}
\newcommand{\vr}{\vecfont{r}}
\newcommand{\vs}{\vecfont{s}}
\newcommand{\vt}{\vecfont{t}}
\newcommand{\vu}{\vecfont{u}}
\newcommand{\vv}{\vecfont{v}}
\newcommand{\vw}{\vecfont{w}}
\newcommand{\vx}{\vecfont{x}}
\newcommand{\vy}{\vecfont{y}}
\newcommand{\vz}{\vecfont{z}}

\newcommand{\valpha}{\vecfont{\alpha}}
\newcommand{\vepsilon}{\vecfont{\varepsilon}}
\newcommand{\vgamma}{\vecfont{\gamma}}
\newcommand{\veta}{\vecfont{\eta}}
\newcommand{\vmu}{\vecfont{\mu}}

\newcommand{\mA}{\matfont{A}}
\newcommand{\mB}{\matfont{B}}
\newcommand{\mC}{\matfont{C}}
\newcommand{\mD}{\matfont{D}}
\newcommand{\mE}{\matfont{E}}
\newcommand{\mF}{\matfont{F}}
\newcommand{\mG}{\matfont{G}}
\newcommand{\mH}{\matfont{H}}
\newcommand{\mI}{\matfont{I}}
\newcommand{\mJ}{\matfont{J}}
\newcommand{\mK}{\matfont{K}}
\newcommand{\mL}{\matfont{L}}
\newcommand{\mM}{\matfont{M}}
\newcommand{\mN}{\matfont{N}}
\newcommand{\mO}{\matfont{O}}
\newcommand{\mP}{\matfont{P}}
\newcommand{\mQ}{\matfont{Q}}
\newcommand{\mR}{\matfont{R}}
\newcommand{\mS}{\matfont{S}}
\newcommand{\mT}{\matfont{T}}
\newcommand{\mU}{\matfont{U}}
\newcommand{\mV}{\matfont{V}}
\newcommand{\mW}{\matfont{W}}
\newcommand{\mX}{\matfont{X}}
\newcommand{\mY}{\matfont{Y}}
\newcommand{\mZ}{\matfont{Z}}




%-------------------------------------------------------------------------------
% Spaces, command is "sp" followed by capital letter
%-------------------------------------------------------------------------------
\newcommand{\spacefont}[1]{\symbfcal{#1}}

\newcommand{\spA}{\spacefont{A}}
\newcommand{\spB}{\spacefont{B}}
\newcommand{\spC}{\spacefont{C}}
\newcommand{\spD}{\spacefont{D}}
\newcommand{\spE}{\spacefont{E}}
\newcommand{\spF}{\spacefont{F}}
\newcommand{\spG}{\spacefont{G}}
\newcommand{\spH}{\spacefont{H}}
\newcommand{\spI}{\spacefont{I}}
\newcommand{\spJ}{\spacefont{J}}
\newcommand{\spK}{\spacefont{K}}
\newcommand{\spL}{\spacefont{L}}
\newcommand{\spM}{\spacefont{M}}
\newcommand{\spN}{\spacefont{N}}
\newcommand{\spO}{\spacefont{O}}
\newcommand{\spP}{\spacefont{P}}
\newcommand{\spQ}{\spacefont{Q}}
\newcommand{\spR}{\spacefont{R}}
\newcommand{\spS}{\spacefont{S}}
\newcommand{\spT}{\spacefont{T}}
\newcommand{\spU}{\spacefont{U}}
\newcommand{\spV}{\spacefont{V}}
\newcommand{\spW}{\spacefont{W}}
\newcommand{\spX}{\spacefont{X}}
\newcommand{\spY}{\spacefont{Y}}
\newcommand{\spZ}{\spacefont{Z}}
%-------------------------------------------------------------------------------
% State space and measurement space:
% Zustandsraum und Messraum:
%-------------------------------------------------------------------------------
\newcommand{\statespace}{\spX}
\newcommand{\measurementspace}{\spZ}
%-------------------------------------------------------------------------------




%-------------------------------------------------------------------------------
% Sets, command is "s" followed by capital letter
%-------------------------------------------------------------------------------
\newcommand{\setfont}[1]{\mathcal{#1}}

\newcommand{\sA}{\setfont{A}}
\newcommand{\sB}{\setfont{B}}
\newcommand{\sC}{\setfont{C}}
\newcommand{\sD}{\setfont{D}}
\newcommand{\sE}{\setfont{E}}
\newcommand{\sF}{\setfont{F}}
\newcommand{\sG}{\setfont{G}}
\newcommand{\sH}{\setfont{H}}
\newcommand{\sI}{\setfont{I}}
\newcommand{\sJ}{\setfont{J}}
\newcommand{\sK}{\setfont{K}}
\newcommand{\sL}{\setfont{L}}
\newcommand{\sM}{\setfont{M}}
\newcommand{\sN}{\setfont{N}}
\newcommand{\sO}{\setfont{O}}
\newcommand{\sP}{\setfont{P}}
\newcommand{\sQ}{\setfont{Q}}
\newcommand{\sR}{\setfont{R}}
\newcommand{\sS}{\setfont{S}}
\newcommand{\sT}{\setfont{T}}
\newcommand{\sU}{\setfont{U}}
\newcommand{\sV}{\setfont{V}}
\newcommand{\sW}{\setfont{W}}
\newcommand{\sX}{\setfont{X}}
\newcommand{\sY}{\setfont{Y}}
\newcommand{\sZ}{\setfont{Z}}

%-------------------------------------
% Special sets:
%-------------------------------------
% Short forms for well-known sets ("ds" = double stroke, \mathbb is correct
% because it is redefined by libertine to print nice double stroke letters)
\newcommand{\dsA}{\mathbb{A}}
\newcommand{\dsB}{\mathbb{B}}
\newcommand{\dsC}{\mathbb{C}}
\newcommand{\dsD}{\mathbb{D}}
\newcommand{\dsE}{\mathbb{E}}
\newcommand{\dsF}{\mathbb{F}}
\newcommand{\dsG}{\mathbb{G}}
\newcommand{\dsH}{\mathbb{H}}
\newcommand{\dsI}{\mathbb{I}}
\newcommand{\dsJ}{\mathbb{J}}
\newcommand{\dsK}{\mathbb{K}}
\newcommand{\dsL}{\mathbb{L}}
\newcommand{\dsM}{\mathbb{M}}
\newcommand{\dsN}{\mathbb{N}}
\newcommand{\dsO}{\mathbb{O}}
\newcommand{\dsP}{\mathbb{P}}
\newcommand{\dsQ}{\mathbb{Q}}
\newcommand{\dsR}{\mathbb{R}}
\newcommand{\dsS}{\mathbb{S}}
\newcommand{\dsT}{\mathbb{T}}
\newcommand{\dsU}{\mathbb{U}}
\newcommand{\dsV}{\mathbb{V}}
\newcommand{\dsW}{\mathbb{W}}
\newcommand{\dsX}{\mathbb{X}}
\newcommand{\dsY}{\mathbb{Y}}
\newcommand{\dsZ}{\mathbb{Z}}
%
% Set of natural numbers, set of real numbers, set of rational numbers:
% Menge natürlicher Zahlen, Menge reeller Zahlen, menge rationaler Zahlen:
%-------------------------------------
\newcommand{\NatNum}{\mathbb{N}}
\newcommand{\RealNum}{\mathbb{R}}
\newcommand{\RatNum}{\mathbb{Q}}
%-------------------------------------------------------------------------------




%-------------------------------------------------------------------------------
% Distributions
%-------------------------------------------------------------------------------
% Gauss-Verteilung + Gauss-Verteilung an einer Stelle:
\newcommand{\GaussDist}{\mathcal{N}}
\newcommand{\NormDist}{\mathcal{N}}
\newcommand{\GaussDistValue}[3]{\mathcal{N}({#1}; {#2}, {#3})}
\newcommand{\NormDistValue}[3]{\mathcal{N}({#1}; {#2}, {#3})}
% Chi^2-Verteilung:
\newcommand{\ChiSqDist}{\chi^2}
%Dirac-Verteilung
\newcommand{\DiracDist}{\delta}
%-------------------------------------------------------------------------------



% ~~~~~~~~~~~~~~~~~~~~~~~~~~~~~~~~~~~~~~~~~~~~~~~~~~~~~~~~~~~~~~~~~~~~~~~~
% Special symbols:
% Sondersymbole:
% ~~~~~~~~~~~~~~~~~~~~~~~~~~~~~~~~~~~~~~~~~~~~~~~~~~~~~~~~~~~~~~~~~~~~~~~~
%Rotation matrix
\newcommand{\RotMat}{\mR}
%Translation vector
\newcommand{\transVec}{\vt}
%
\newcommand{\uVec}{\vu} % Control vector (vector with control parameters)
\newcommand{\wVec}{\vw} % 
\newcommand{\vVec}{\vv} % 
%-------------------------------------------------------------------------------
\newcommand{\Imat}{\mI} % Einheitsmatrix
\newcommand{\ZeroMat}{\Mat{0}} % zero matrix
\newcommand{\Fmat}{\mF} % system matrix of the Kalman Filter
\newcommand{\Gmat}{\mG} % control matrix of the Kalman Filter
\newcommand{\Hmat}{\mH} % measurement matrix of the Kalman Filter
\newcommand{\Qmat}{\mQ} % system noise covariance matrix of the Kalman Filter
\newcommand{\Rmat}{\mR} % measurement noise covariance matrix of the Kalman Filter
\newcommand{\Wmat}{\mW} % Jacobian matrix in the Extended Kalman Filter
\newcommand{\Vmat}{\mV} % Jacobian matrix in the Extended Kalman Filter
\newcommand{\KG}{\mK} % Kalman Gain
%-------------------------------------------------------------------------------
\newcommand{\uk}{\vu_k} % Control vector (vector with control parameters) at time $k$
\newcommand{\Qk}{\mQ_k} % System noise covariance matrix at time $k$
\newcommand{\Rk}{\mR_k} % Measurement noise covariance matrix at time $k$
\newcommand{\KGk}{\mK_k} % Kalman gain at time $k$ at time $k$
%
%
%
% point in image coordinates
\newcommand{\pkt}{\vp}
\newcommand{\pImage}{\pkt}
\newcommand{\pImageUntransf}{\vecfont{p'}}
% points in camera coordinates
\newcommand{\PCam}{\pkt_{\textnormal{C}}}
\newcommand{\pLeft}{\pkt_{\textnormal{L}}}
\newcommand{\pRight}{\pkt_{\textnormal{R}}}
% point in world coordinates
\newcommand{\PWorld}{\pkt_{\textnormal{W}}}
% homogenious coordinates
\newcommand{\pHomogen}{\check{\pkt}}


