%%% Doc: ftp://tug.ctan.org/pub/tex-archive/macros/latex/required/amslatex/math/amsldoc.pdf
\usepackage{ifxetex}
%
%%% Doc: ftp://tug.ctan.org/pub/tex-archive/macros/latex/contrib/oberdiek/ifpdf.sty
% command for testing for pdf-creation
\usepackage{ifpdf} %\ifpdf \else \fi
%
% Amsmath - Mathematik Basispaket
%
% fuer pst-pdf displaymath Modus vor pst-pdf benoetigt.
\usepackage[%
   centertags, % (default) center tags vertically
   %tbtags,    % 'Top-or-bottom tags': For a split equation, place equation numbers level
               % with the last (resp. first) line, if numbers are on the right (resp. left).
   sumlimits,  %(default) Place the subscripts and superscripts of summation
               % symbols above and below
   %nosumlimits, % Always place the subscripts and superscripts of summation-type
               % symbols to the side, even in displayed equations.
   intlimits,  % Like sumlimits, but for integral symbols.
   %nointlimits, % (default) Opposite of intlimits.
   namelimits, % (default) Like sumlimits, but for certain 'operator names' such as
               % det, inf, lim, max, min, that traditionally have subscripts placed underneath
               % when they occur in a displayed equation.
   %nonamelimits, % Opposite of namelimits.
   %leqno,     % Place equation numbers on the left.
   %reqno,     % Place equation numbers on the right.
   fleqn,     % Position equations at a fixed indent from the left margin
   			  % rather than centered in the text column.
]{amsmath} %
%
\usepackage{amsfonts}
%\usepackage{amssymb} %bei Kompilierung mit LaTeX verursacht probleme mit tikz
\usepackage[fixamsmath,disallowspaces]{mathtools}
\usepackage{fixmath}
% eqnarray nicht zusammen mit amsmath benutzen, siehe l2tabu.pdf für
% Hintergruende.
%
%--------------------------------
% Hier die Schriftart auswählen:
%--------------------------------
%\def\UseLibertine{true} %Problem bei Bibliografie: fett und kursiv statt nur kursiv
%\def\UseLibertinus{true}
%\def\UseSTIX{true}
%\def\UseXITS{true}
\def\UseLibertinusSTIXMix{true}
%\def\UseLibertinusFrutigerSTIXMix{true}
%--------------------------------
%
%
\makeatletter
\ifxetex %Nutzung von XeLaTeX anstatt nur LaTeX:
  \usepackage{xltxtra}
	\usepackage{unicode-math}
	%--------------------------------------------------
	%  Libertine Fonts
	%--------------------------------------------------
	\ifdefined\UseLibertine
		\setmainfont[
								%Scale = 0.95,
								UprightFont = *_R_G,
								ItalicFont = *_RI_G,
								BoldFont = *_RB_G,
								BoldItalicFont = *_RBI_G,
								SmallCapsFeatures={Letters=SmallCaps},
								Path=./fonts/,
								Extension = .ttf
								]{LinLibertine}
		\setsansfont[
								%Scale = 0.85,
								UprightFont = *-regular,
								ItalicFont = *-italic,
								BoldFont = *-bold,
								%BoldItalicFont = *-bolditalic,
								SmallCapsFeatures={Letters=SmallCaps},
								Path=./fonts/,
								Extension = .otf
								]{libertinussans}
		\setmonofont[
								%Scale = 0.95,
								FakeStretch = 0.8,
								UprightFont = *-regular,
								%ItalicFont = *-italic,
								%BoldFont = *-bold,
								% AutoFakeBold = 1.5,
								Path=./fonts/,
								Extension = .otf
								]{libertinusmono}
		\setmathfont[
								%Scale = 0.95,
								Path=./fonts/,
								Extension = .otf
								]{libertinusmath-regular}
	\fi
	%--------------------------------------------------
	%  Libertinus Fonts
	%--------------------------------------------------
	\ifdefined\UseLibertinus
		\setmainfont[
								%Scale = 0.95,
								UprightFont = *-regular,
								ItalicFont = *-italic,
								BoldFont = *-bold,
								BoldItalicFont = *-bolditalic,
								SmallCapsFeatures={Letters=SmallCaps},
								Path=./fonts/,
								Extension = .otf
								]{libertinusserif}
		\setsansfont[
								%Scale = 0.85,
								UprightFont = *-regular,
								ItalicFont = *-italic,
								BoldFont = *-bold,
								%BoldItalicFont = *-bolditalic,
								SmallCapsFeatures={Letters=SmallCaps},
								Path=./fonts/,
								Extension = .otf
								]{libertinussans}
		\setmonofont[
								%Scale = 0.95,
								FakeStretch = 0.8,
								UprightFont = *-regular,
								%ItalicFont = *-italic,
								%BoldFont = *-bold,
								% AutoFakeBold = 1.5,
								Path=./fonts/,
								Extension = .otf
								]{libertinusmono}
		\setmathfont[
								%Scale = 0.95,
								Path=./fonts/,
								Extension = .otf
								]{libertinusmath-regular}
	\fi
	%--------------------------------------------------
	%  STIX Fonts; use STIX-Regular as SANS font
	%--------------------------------------------------
	\ifdefined\UseSTIX
		\setmainfont[
								UprightFont = *-Regular,
								ItalicFont = *-Italic,
								BoldFont = *-Bold,
								BoldItalicFont = *-BoldItalic,
								SmallCapsFeatures={Letters=SmallCaps},
								Extension = .otf
								]{STIX2Text}
		\setsansfont[
								Path=./fonts/,
								Extension = .otf
								]{STIX2Text-Regular}
		\setmathfont[
								Path=./fonts/,
								Extension = .otf
								]{STIX2Math}
	\fi
	%--------------------------------------------------
	%  XITS Fonts; use xits-regular as SANS font
	%--------------------------------------------------
	\ifdefined\UseXITS
		\setmainfont[
								UprightFont = *-regular,
								ItalicFont = *-italic,
								BoldFont = *-bold,
								BoldItalicFont = *-bolditalic,
								SmallCapsFeatures={Letters=SmallCaps},
								Path=./fonts/,
								Extension = .otf
								]{xits}
		\setsansfont[
								Path=./fonts/,
								Extension = .otf
								]{xits-regular}
		\setmathfont[
								Path=./fonts/,
								Extension = .otf
								]{xits-math}
	\fi
	%--------------------------------------------------
	%  Libertinus as the main font, STIX as math font
	%--------------------------------------------------
	\ifdefined\UseLibertinusSTIXMix
		\setmainfont[
								%Scale = 0.95,
								UprightFont = *-regular,
								ItalicFont = *-italic,
								BoldFont = *-bold,
								BoldItalicFont = *-bolditalic,
								SmallCapsFeatures={Letters=SmallCaps},
								Path=./fonts/,
								Extension = .otf
								]{libertinusserif}
		\setsansfont[
								%Scale = 0.85,
								UprightFont = *-regular,
								ItalicFont = *-italic,
								BoldFont = *-bold,
								%BoldItalicFont = *-bolditalic,
								SmallCapsFeatures={Letters=SmallCaps},
								Path=./fonts/,
								Extension = .otf
								]{libertinussans}
		\setmonofont[
								Scale = 0.8,
								FakeStretch = 0.8,
								UprightFont = *-regular,
								Path=./fonts/,
								Extension = .otf
								]{libertinusmono}
		\setmathfont[
								%Scale = 0.85,
								Path=./fonts/,
								Extension = .otf
								]{STIX2Math}
	\fi
	\ifdefined\UseLibertinusFrutigerSTIXMix
		\setmainfont[
								%Scale = 0.95,
								UprightFont = *-regular,
								ItalicFont = *-italic,
								BoldFont = *-bold,
								BoldItalicFont = *-bolditalic,
								SmallCapsFeatures={Letters=SmallCaps},
								Path=./fonts/,
								Extension = .otf
								]{libertinusserif}
		\setsansfont[
								%Scale = 0.90,
								UprightFont = *-Roman,
								ItalicFont = *-Italic,
								BoldFont = *-Bold,
								BoldItalicFont = *-BoldItalic,
								SmallCapsFeatures={Letters=SmallCaps},
								%Path=./fonts/,
								Extension = .ttf
								]{FrutigerLTCom}
		\setmonofont[
								Scale = 0.875,
								FakeStretch = 0.85,
								UprightFont = *-regular,
								Path=./fonts/,
								Extension = .otf
								]{libertinusmono}
		\setmathfont[
								%Scale = 0.85,
								Path=./fonts/,
								Extension = .otf
								]{STIX2Math}
	\fi
\else %falls doch LaTeX anstatt XeLaTeX verwendet wird soll:
  \GenericError{}{This template should be compiled with xelatex! Please use xelatex instead of pdflatex as compiler}{}{}%
	\usepackage[utf8]{inputenc} %Die Dateien liegen in der UTF-8-Kodierung vor
	\usepackage[T1]{fontenc} %Der Text enthält Umlaute
	\usepackage{sanitize-umlaut} %behebt Probleme mit Umlauten im Stichwortverzeichnis
	\usepackage[ttscale=.875]{libertine}
	\usepackage{libertinust1math}
	%% Alternative:
	%\usepackage{mathptmx}
	%\usepackage[scaled=.92]{helvet}
	%\usepackage{courier}
	%
	%% Workaround for a TikZ-related error :
	%% https://tex.stackexchange.com/questions/165929/semiverbatim-with-tikz-in-beamer/165937#165937
	%\global\let\tikz@ensure@dollar@catcode=\relax
\fi
%}
\makeatother