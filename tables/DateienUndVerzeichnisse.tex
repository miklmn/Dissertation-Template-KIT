%%%%%%%%%%%%%%%%%%%%%%%%%%%%%%%%%%%%%%%%%%%%%%%%%%%%%%%%%%%%%%%%%
%% Übersichtstabelle für Dateien und Verzeichnisse der Vorlage %%
%%%%%%%%%%%%%%%%%%%%%%%%%%%%%%%%%%%%%%%%%%%%%%%%%%%%%%%%%%%%%%%%%
%% Breite der ersten Spalte auf Inhalt anpassen,
%% Breite der zweiten Spalte automatisch bestimmen,
%% Spacing zwischen den Spalten auf 8pt setzen:
\begin{longtable}{l@{\extracolsep{8pt}}X}%
%-----------------------------------------------------------------------------------------------------
\caption[Dateien und Verzeichnisse der Vorlage]{Dateien und Verzeichnisse der Vorlage}%
\label{tab:StrukturDerVorlage}%
\endlastfoot%
%-----------------------------------------------------------------------------------------------------
\toprule%
%-----------------------------------------------------------------------------------------------------
\bfseries Datei/Verzeichnis               & \bfseries Bedeutung und Benutzerinteraktion
\tabularnewline%
%-----------------------------------------------------------------------------------------------------
\midrule%
\endhead%
%-----------------------------------------------------------------------------------------------------
\texttt{./Diss.tcp}                       & TeXnicCenter-Projektdatei. Aufruf im TeXnicCenter.
                                          Indirekte Änderung durch Einstellungen im Programm.\\
%-----------------------------------------------------------------------------------------------------
\texttt{./Diss.tex}                       & \texttt{TeX}-Hauptdatei. Einbindung der Inhalte.
                                          Aus- und Wieder"=Einblenden der einzelnen Manuskript"=Teile
                                          durch Ersetzen von \lc{showif} durch \lc{hideif} und vice versa.\\
%-----------------------------------------------------------------------------------------------------
\texttt{./bib/Diss.bib}                   & Literaturdatenbank im \texttt{BibLaTeX}-Format. Verwendete Referenzen einfügen.\\
%-----------------------------------------------------------------------------------------------------
\texttt{./content/*}                      & Inhalte der Arbeit. Hier Inhalte der einzelnen \texttt{LaTeX}-Kapitel einfügen.\\
%-----------------------------------------------------------------------------------------------------
\texttt{./figures-src/*}                  & \texttt{TikZ}-Zeichnungen. %(Quellcode).
                                          Ggf. weitere hinzufügen.\\
%-----------------------------------------------------------------------------------------------------
\texttt{./figures-compiled/*}             & Temporäre Kompilate der \texttt{TikZ}-Zeichnungen.
                                          Bei Aktualisierung der \texttt{TikZ}-Zeichnungen löschen.\\
%-----------------------------------------------------------------------------------------------------
\texttt{./fonts/*}                        & Verwendete Schriften. Keine.\\
%-----------------------------------------------------------------------------------------------------
\texttt{./images/*}                       & Bilder im Binärformat. %(jpg, png, pdf etc).
                                          Ggf. weitere hinzufügen. \\
%-----------------------------------------------------------------------------------------------------
\texttt{./preambel/*}                     & Konfigurationsdateien. S.u.\\
%-----------------------------------------------------------------------------------------------------
\texttt{./preambel/Acronyms.tex}          & Definition der Akronyme.
                                          Ggf. weitere hinzufügen.\\
%-----------------------------------------------------------------------------------------------------
\texttt{./preambel/AlleAngaben.tex}       & Wichtige Angaben und Einstellungen.
                                          Hier Angaben zum Typ der Arbeit, zum Autor und zu den Gutachtern ändern.
                                          Bei Bedarf Hauptsprache auf Englisch umstellen.
                                          Anpassungen für die Druckversion vornehmen.\\
%-----------------------------------------------------------------------------------------------------
\texttt{./preambel/AllePfade.tex}         & Definition von Suchpfaden für Bildverzeichnisse und Bibliografie"=Dateien.
                                          Ggf. ergänzen.\\
%-----------------------------------------------------------------------------------------------------
%\texttt{./preambel/BibSettings.tex}       & Konfiguration der Bibliografie
%                                          Normalerweise keine, es sei denn man möchte den Stil der Literaturverzeichnisse ändern.\\
%-----------------------------------------------------------------------------------------------------
\texttt{./preambel/EncodingAndFont.tex}   & Schriftarteinstellungen.
                                          Keine, sofern man die vorgegeben Schriftarten nutzen möchte.\\
%-----------------------------------------------------------------------------------------------------
\texttt{./preambel/Glossary.tex}          & Definition der Glossar-Einträge.
                                          Ggf. weitere hinzufügen.\\
%-----------------------------------------------------------------------------------------------------
\texttt{./preambel/GlossarySymbols.tex}   & Glossar-Einträge für automatisches Symbolverzeichnis.
                                          Bei Verwendung ergänzen.\\
%-----------------------------------------------------------------------------------------------------
\texttt{./preambel/Header.tex}            & Alle Präambel-Definitionen (\teilw in weiteren Dateien).
                                          Aktivierung der \printkeyword{draft}-Option und des A4-Layouts.\\
%-----------------------------------------------------------------------------------------------------
\texttt{./preambel/Hyphenation.tex}       & Silbentrennung für unbekannte Wörter.
                                          Ggf. Regeln für die Silbentrennung weiterer Begriffe hinzufügen.\\
%-----------------------------------------------------------------------------------------------------
\texttt{./preambel/IndexStyle.tex}        & Layout des Stichwortverzeichnisses.
                                          Keine.\\
%-----------------------------------------------------------------------------------------------------
\texttt{./preambel/KomaOptions.tex}       & KomaScript-Optionen.
                                          Keine.\\
%-----------------------------------------------------------------------------------------------------
\texttt{./preambel/Math.tex}              & Mathe-Einstellungen und Makros.
                                          Bei Bedarf eigene Mathe-Makros definieren.\\
%-----------------------------------------------------------------------------------------------------
\texttt{./preambel/MyPackages.tex}        & Zusatzpakete
                                          Ggf. Einbindung von Zusatzpaketen.\\
%-----------------------------------------------------------------------------------------------------
\texttt{./preambel/Newcommands.tex}       & Eigene \LaTeX{}-Makros.
                                          Ggf. weitere Makros hinzufügen.\\
%-----------------------------------------------------------------------------------------------------
\texttt{./preambel/preambel-commands.tex} & Interne Befehle aus der alten Vorlage.
                                          Normalerweise keine.\\
%-----------------------------------------------------------------------------------------------------
\texttt{./preambel/settings.tex}          & Einstellungen zu Längen, Breiten, Verzeichnistiefen etc.
                                          Normalerweise keine, da diese Einstellungen mit dem \gls{ac:KSP} abgestimmt worden sind.\\
%-----------------------------------------------------------------------------------------------------
\texttt{./preambel/TableCommands.tex}     & Tabelleneinstellungen.
                                          Normalerweise keine.\\
%-----------------------------------------------------------------------------------------------------
\texttt{./preamble/Translations.tex}      & Multilinguale Begriffsdefinitionen für Beschriftungen etc.
                                          Normalerweise keine.\\
%-----------------------------------------------------------------------------------------------------
\bottomrule%
%-----------------------------------------------------------------------------------------------------
\end{longtable}%