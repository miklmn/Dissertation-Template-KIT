%%%%%%%%%%%%%%%%%%%%%%%%%%%%%%%%%%%%%%%%%%%%%%%%%%%%%%%%%%%%
\section{Tabellen}
\label{sec:Tabellen}
%%%%%%%%%%%%%%%%%%%%%%%%%%%%%%%%%%%%%%%%%%%%%%%%%%%%%%%%%%%%

Typografisch gute Tabellen haben \emph{niemals} vertikale Trennlinien,
sondern nur wenige horizontale Linien.
Ferner haben sie eine trennende Linie ganz oben und ganz unten.
Hierfür stellt das Paket \pkg{booktabs} die Befehle
\begin{itemize*}
  \item \lc{toprule}
	\item \lc{midrule}
	\item \lc{bottomrule}
\end{itemize*}
zur Verfügung.
Der Befehl \lc{hline} ist tabu.
Für eine ausführliche Erläuterung auch über gute und schlechte Tabellen
siehe die Dokumentation des \pkg{booktabs}-Pakets \cite{Fear2005}.

Eine einfache Tabelle hat den folgenden Code:
\begin{latex}[caption={Einfache Tabelle in \LaTeX},label={lst:tables}]
\begin{table}%
	\centering%
	\begin{tabularx}{\columnwidth}{l l X}%
		\toprule%
		Datei       &  Bedeutung    &  Benutzerinteraktion \\%
		\midrule%
		\endheader%
		main.tex  &  Hauptdatei   &  nein     \\%
		figures/  &  Zeichnungen  &  ja       \\%
		content/  &  Kapitel      &  ja       \\%
		logos/    &  Logos        &  nein     \\%
		\bottomrule%
	\end{tabularx}%
	\caption{Dateien der Vorlage}%
	\label{tab:files-dirs-of-template}%
\end{table}
\end{latex}
%und sieht dann so aus wie \cref{tab:files-dirs-of-template}.


Etwas komplizierter wird es, wenn man eine Tabelle mit alternierender Farbe einfügen möchte
%
%% -- Einstellungen für Tabellen  ----------
%% -- Können auch in die Präambel ----------
%\colorlet{tablesubheadcolor}{gray!40}
%\colorlet{tableheadcolor}{gray!25}
%\colorlet{tableblackheadcolor}{black!60}
%\colorlet{tablerowcolor}{gray!15.0}
%
%
%\renewcommand\tablehead{%
  %\tableheadfontsize%
  %\sffamily\bfseries%
  %\slshape
%}
%
%%\renewcommand\tableheadcolor{
   %%\rowcolor{tableblackheadcolor}
%%}
%
%
%%---------------------------------------
%%
\begin{table}
	\tablestyle%
	\tablealtcolored%
	\begin{tabular}{*{2}{v{0.45\textwidth}}}
		\toprule%
		\tableheadcolor%
		\tableheadformat Tabellenkopf &	\tableheadformat Tabellenkopf
		\tabularnewline%
		\midrule%
		%% Zwischenkopf ---------------------------------------------
		\multicolumn{2}{>{\columncolor{tablesubheadcolor}}l}{\bfseries\color{KITblue} Zwischenkopf}%
		\tabularnewline%
		%%-----------------------------------------------------------
		Inhalt  & Inhalt \tabularnewline
		Inhalt  & Inhalt \tabularnewline
		Inhalt  & Inhalt \tabularnewline
		%% Zwischenkopf ---------------------------------------------
		\multicolumn{2}{>{\columncolor{tablesubheadcolor}}l}{\bfseries\color{KITgreen} Zwischenkopf}%
		\tabularnewline
		%%-----------------------------------------------------------
		Inhalt  & Inhalt \tabularnewline
		Inhalt  & Inhalt \tabularnewline
		\bottomrule%
	\end{tabular}%
\captionof{table}{Superteil 4}%
\end{table}
%%
%% ------------------------------------------------------------
%\IfDefined{LTXtable}{%
%
%Und hier ein Beispiel für eine sehr lange Tabelle, die umbrochen werden soll.
%
%%--Einstellungen für Tabellen ----------
%\colorlet{tablerowcolor}{gray!10.0}
%
%\renewcommand\tableheadcolor{
   %\rowcolor{tableheadcolor}
%}
%
%\renewcommand\tablehead{%
  %\tableheadfontsize%
  %\sffamily\bfseries%
  %\slshape
  %\color{black}
%}
%%---------------------------------------
%{
   %\tablestyle
   %\tablealtcolored
   %\LTXtable{\textwidth}{tables/LongTableExample.tex}
%}
%} % End If 
%
%
%
Sollten eine Tabelle einmal so breit sein, dass sie nicht mehr horizontal auf
eine Seite passt, so ist es natürlich möglich, diese mithilfe des Pakets
\printkeyword{rotfloat} \parencite{Sommerfeldt2004} in eine
\printkeyword{sidewaystable} statt in eine \printkeyword{table}-Umgebung zu setzen.
Also so:
\begin{latex}[caption={Gedrehte Tabelle},label={lst:rotated-table}]
\begin{sidewaystable}
  \centering%
  \begin{tabular}{...}%
    ...
  \end{tabular}%
  \caption{Bezeichnung}%
  \label{Referenzmarke}%
\end{sidewaystable}%
\end{latex}

Ein Ergebnis sieht man in \cref{tab:ex-sideways}.


\begin{sidewaystable}[p]
\scriptsize%
%\tiny
\centering%
\begin{tabular}{l M M M M M}%
\toprule%
\addlinespace[0pt]%
 & \multicolumn{5}{c}{\bfseries Level} \tabularnewline
 & \multicolumn{2}{c}{\bfseries Qualitative} & \multicolumn{3}{c}{\bfseries Quantitative} \tabularnewline
 & \bfseries\centering Nominal & \bfseries\centering Ordinal & \bfseries\centering Interval & \bfseries\centering Ratio & \bfseries\centering Absolute \tabularnewline \addlinespace[0pt]\midrule\addlinespace[0pt]
Empirical relation &
\begin{tabitemize}\item[$\sim$] Equivalence\end{tabitemize} &
\begin{tabitemize}\item[$\sim$] Equivalence\item[$\prec$] Ordering\end{tabitemize} &
\begin{tabitemize}\item[$\sim$] Equivalence\item[$\prec$] Ordering\end{tabitemize} &
\begin{tabitemize}\item[$\sim$] Equivalence\item[$\prec$] Ordering\strut\end{tabitemize} &
\begin{tabitemize}\item[$\sim$] Equivalence\item[$\prec$] Ordering\strut\end{tabitemize} \tabularnewline \midrule
Empirical operation &
 &
 &
\begin{tabitemize}\item[$\oplus$] Addition\end{tabitemize} &
\begin{tabitemize}\item[$\oplus$] Addition\item[$\otimes$] Multiplication\strut\end{tabitemize} &
\begin{tabitemize}\item[$\oplus$] Addition\item[$\otimes$] Multiplication\strut\end{tabitemize} \tabularnewline \midrule
Feasable transformation &
$m' = f( m )$ for $f$ bij.\strut &
$m' = f( m )$ for $f$ mon.\strut &
$m' = am + b$ for $a>0$\strut &
$m' = am$ for $a>0$\strut &
$m' = m$\strut \tabularnewline \midrule
Examples of features &
\begin{tabitemize}\item Telephone numbers\item Postal codes\item Gender\strut\end{tabitemize} &
\begin{tabitemize}\item Grades\item Degree of hardness\item Wind intensity\strut\end{tabitemize} &
\begin{tabitemize}\item Temperatur in F\textdegree\item Calendric time\item Geographic altitude\strut\end{tabitemize} &
\begin{tabitemize}\item Temperatur in K\item Mass\item Length\item Electric current\strut\end{tabitemize} &
\begin{tabitemize}\item Quantum numbers\item Error number\strut\end{tabitemize} \tabularnewline \midrule
Range of features &
\begin{tabitemize}\item Numbers\item Names\item Symbols\strut\end{tabitemize} &
Natural numbers &
Real numbers &
Real, positive numbers &
Natural numbers \tabularnewline \midrule
Expressiveness & low & \dots & \dots & \dots & high\strut \tabularnewline \addlinespace[0pt]
\bottomrule%
\end{tabular}%
\caption{Beispiel für eine breite, gedrehte Tabelle (hier: Taxonomie der Maßskalen)}%
\label{tab:ex-sideways}%
\end{sidewaystable}