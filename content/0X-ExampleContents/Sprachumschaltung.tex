%%%%%%%%%%%%%%%%%%%%%%%%%%%%%%%%%%%%%%%%%%%%%%%%%%%%%%%%%%%%
\section{Sprachumstellung und temporäre Sprachumschaltung bei fremdsprachlichen Begriffen und Textabschnitten}%
\index{Fremdsprachen}%
\index{Sprache!Fremdsprache}%
\index{Sprache!Umstellung}%
\index{Sprache!Temporäre Umschaltung}%
\label{sec:Sprache}
%%%%%%%%%%%%%%%%%%%%%%%%%%%%%%%%%%%%%%%%%%%%%%%%%%%%%%%%%%%%
%
Die Hauptsprache der Arbeit wird in der Datei \printfilepath{./preambel/AlleSchalter.tex} festgelegt.
Aktuell werden nur Deutsch und Englisch als Hauptsprachen unterstützt.
Die Auswahl geschieht mit der Angabe des Wertes \printkeyword{true} oder \printkeyword{false}
in der Zeile \lc{setboolean\{iesenglishs\}\{<Wert>\}}.

Bei Verwendung von fremdsprachlichen Begriffen oder Textabschnitten
(\zB bei englischen oder französischen Zitaten in einer deutschsprachigen Arbeit
oder bei deutschen Begriffen in einer englischsprachigen Arbeit),
sollte man dies entsprechend markieren,
damit \LaTeX{} die richtigen Regeln für die \index{Silbentrennung}Silbentrennung
und die passenden Anführungszeichen bei Verwendung des Befehls
\lc{enquote\{...\}} ansetzt.
Für die einzelnen Begriffe und kürzere Texte gibt es den Befehl
\verb+\foreignlanguage{Sprache}{...}+.
Dann wird für den Text in den geschweiften Klammern die in den eckigen Klammern angegebene Sprache verwendet.
Um die Sprache bis zum nächsten Aufruf des gleichen Kommandos dauerhaft umstellen,
gibt es den Befehl \verb+\selectlanguage{Sprache}+.
Es gilt eine Liste der Sprachen aus dem Paket \pkg{babel}.
Für Deustch sollte \printkeyword{ngerman} verwendet werden, was für die
\index{Rechtschreibung!neue deutsche}neue deutsche Rechtschreibung steht.

Für eine korrekte Behandlung der deutschen Kurzfassung und des englischen Abstracts
unabhängig von der gewählten Hauptsprache ist durch die Verwendung der Befehle
\lc{textInGerman\{...\}} und \lc{textInEnglish\{...\}} bereits gesorgt.