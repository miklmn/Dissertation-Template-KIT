%%%%%%%%%%%%%%%%%%%%%%%%%%%%%%%%%%%%%%%%%%%%%%%%%%%%%%%%%%%%
\section[Anpassungen für den Druck]{Anpassungen für den Druck}%
\index{Druck}%
\label{sec:Druckanpassungen}
%%%%%%%%%%%%%%%%%%%%%%%%%%%%%%%%%%%%%%%%%%%%%%%%%%%%%%%%%%%%
%
Vor der Abgabe einer Druckversion an den KSP-Verlag sollte die korrekte Schaltersetzung in der Datei \printfilepath{./preambel/AlleSchalter.tex} nochmals geprüft werden.
Insbesondere sollte mit der Einstellung
\lstinline|\setUserDefinedBoolean{showFrame}{true}|
die Satzspiegel-Ränder angezeigt werden und so kontrolliert werden,
ob nichts hinausragt und ob Tabellen und Bilder die komplett verfügbare Breite ausfüllen.
Besonders sollte man dabei auch die Overfullbox- und Underfullbox"=Warnungen achten und diese beheben.
Eine Overfullbox"=Warnung tritt dann auf, wenn ein Objekt (Text, Bild, Tabelle)
über den dafür vorgesehenen Bereich (normalerweise über den Satzspiegelrand) hinausragt.
Eine Underfullbox"=Warnung tritt auf, wenn durch fehlende Möglichkeit der Zeilenumbrüche der Text zu stark gedehnt werden müsste und die Wortzwischenräume zu groß sind.
Hierzu sollte man insbesondere auch Hinweise zur Silbentrennung aus
\cref{sec:Hyphenation,sec:Titles} beachten.
Danach sollte der Schalter wieder auf \printkeyword{false} umgestellt werden.

Bei Abbildungen, die am Anfang einer Seite stehen, ist dabei zu achten,
dass es keinen leeren Zwischenraum zwischen dem oberen Rand des Satzspiegels und der Abbildung gibt.
Normalerweise passiert dies durch die Einstellungen der Vorlage automatisch.
Falls doch noch ein leeres Raum entsteht, liegt dies zumeist daran, dass die Abbildung selbst weiße Ränder hat.
Bei Binärbildern können diese durch Setzen der Optionen
\printkeyword{clip=true} und \printkeyword{trim=<left> <lower> <right> <upper>}
im \printkeyword{includegraphics}-Befehl beschnitten werden, z.B. so:
\lstinline|\includegraphics[clip=true, trim={0cm 1cm 0cm 1cm}]{Bildpfad\Dateiname}|.

Mit den Einstellungen 
\lstinline|\setUserDefinedBoolean{coloredlistings}{false}| und 
\lstinline|\setUserDefinedBoolean{coloredlinks}{false}|
sollte dafür gesorgt werden, dass alle Quellcode-Listings sowie Querverweise und URL-Adressen,
die normalerweise farbig sind, für die Druckversion nicht farbig gesetzt werden
und so die Anzahl der farbig zu druckenden Seiten reduziert wird.

Schließlich kann mit der Einstellung
\lstinline|\setUserDefinedBoolean{useCMYKcolors}{true}|
eine Farbkonvertierung aller Farben in den CMYK-Farbraum für den Offset-Druck vorgenommen werden.
So kann man die Farbgebung der Druckversion begutachten und sicherstellen, dass alle Farben beim Druck gut aussehen.%
\footnote{Da manche RGB-Farben bei der Konvertierung in den CMYK-Farbraum blass aussehen,
ist es sinnvoll, eine Alternativversion der betroffenen Farben im CYMK-Farbraum zu definieren,
die gut aussieht.
Beispiele dafür gibt es in der Datei \printfilepath{./preambel/ColorSettings.tex}.
Vorzugsweise sollten aber die KIT-Corporate-Identity-Farben verwendet werden,
welche in der Datei \printfilepath{KAcolors.sty} definiert sind.
Für diese Farben wurde sowohl eine RGB- als auch eine CMYK-Definition erstellt.}