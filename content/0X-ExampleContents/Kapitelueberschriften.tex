%%%%%%%%%%%%%%%%%%%%%%%%%%%%%%%%%%%%%%%%%%%%%%%%%%%%%%%%%%%%
\section[% Kurzversion für das Inhaltsverz., Kolumnentitel und PDF-Lesezeichen:
         Wichtiges zu Umbrüchen bei Überschriften
         \mbox{(Kurzversion für das Inhaltsverzeichnis etc.)}%
				]{% Langversion, die im Text gedruckt wird:
         Wichtiges zu Umbrüchen bei Überschriften
         (und ein Beispiel \newline für eine lange Überschrift,
         welche \newline für das Inhaltsverzeichnis und 
         \newline die Kolumnentiteln zu lang ist).%
				}%
\index{Zeilenumbruch}\index{Titel}\index{Ueberschrift@Überschrift}\index{PDF!Lesezeichen}%
\label{chap:Titles}
%%%%%%%%%%%%%%%%%%%%%%%%%%%%%%%%%%%%%%%%%%%%%%%%%%%%%%%%%%%%
%
%
Bei den Kapitelüberschriften kann man zwei Versionen definieren:
eine lange Überschrift in geschweiften Klammern, welche in der Arbeit selbst angezeigt wird, 
und optional eine Kurzversion in eckigen Klammern, welche im Inhaltsverzeichnis und in den 
\href{https://de.wikipedia.org/wiki/Kolumnentitel}{Kolumnentiteln}%
\footnote{Kolumnentitel sind Überschriften der einzelnen Seiten. Meist stehen sie in der Kopfzeile.}
angezeigt wird:
\begin{latex}
\section[Kurzversion]{Langer Titel}
\end{latex}
Dasselbe gilt für Bild- und Tabellenunterschriften. Hier kann man dem
\lc{caption}-Befehl ebenfalls einen optionalen Parameter übergeben.

Manchmal sind dem \glsdat{ac:KSP} die von \LaTeX{} automatisch eingefügten Zeilenumbrüche in den Kapitelüberschriften im Inhaltsverzeichnis nicht \enquote{schön} genug.
Ein manuelles Einfügen der Zeilenumbrüche etwa mit \verb+\newline+ in der Kurzversion des Titels funktioniert leider nicht,
da diese dann nicht nur im Inhaltsverzeichnis, sondern auch in den Kolumnentiteln und PDF"=Lesezeichen zur Geltung kommen, 
was normalerweise nicht erwünscht ist.

Abhilfe schafft der folgende Trick:
man schließt den letzten, umzubrechenden Teil der Kurzversion des Titels in eine \verb+\mbox{}+.
Der Text, der in eine \verb+\mbox{}+ eingeschlossen wird, darf nicht umbrochen werden.
Im Kolumnentiteln und in den PDF"=Lesezeichen hat dies keine besondere Wirkung; im Inhaltsverzeichnis führt dies jedoch dazu, dass \LaTeX{} den Zeilenumbruch vor der \verb+\mbox{}+ einfügt.
Dasselbe gilt für die ungünstig umbrochene Wörter (so will 
Ein entsprechendes Beispiel stellt die Überschrift dieses Abschnitts dar.