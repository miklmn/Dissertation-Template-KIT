%%%%%%%%%%%%%%%%%%%%%%%%%%%%%%%%%%%%%%%%%%%%%%%%%%%%%%%%%%%%
\section{Speichererweiterung}%
\index{Speichererweiterung}%
\label{sec:Speichererweiterung}
%%%%%%%%%%%%%%%%%%%%%%%%%%%%%%%%%%%%%%%%%%%%%%%%%%%%%%%%%%%%
%
Sollte bei der Kompilierung die \index{Fehler!TeX capacity exceeded}Fehler-Meldung \enquote{\texttt{TeX capacity exceeded}} kommen,
bedeutet dies, dass die von der \LaTeX-Distribution vorgesehene Arbeitsspeicherkapazität
nicht ausreicht, um die Datenmenge zu verarbeiten.
Wenn im Code kein Fehler vorliegt (z.B. eine vergessene Klammer,
die ebenfalls für eine solche Fehlermeldung sorgen kann),
kann dies auch daran liegen, dass die Verarbeitung umfangreicher TiKZ-Bilder oder Bibliographien 
Zusatzspeicher benötigen.
Um den Speicher zu erweitern, gibt es zwei Möglichkeiten.
Zum einen kann man bei jedem Aufruf von \texttt{xelatex} die Optionen zur Speichererweiterung,
z.B. \printkeyword{-main-memory=500000000}, \printkeyword{-extra-mem-top=500000000}, \printkeyword{-extra-mem-bot=500000000} und ggf. weitere setzen.
Zum anderen kann man diese Einstellungen ein für alle Male direkt bei den LaTeX-Distributionen setzen.

Unter TeXLive kann die Einstellung in der Datei
\printfilepath{/Pfad-zur-TeXLive-Installation/texmf.cnf} vorgenommen werden.
Der Aufruf erfolgt am besten, indem man im Terminal den Befehl \printkeyword{kpsewhich -a texmf.cnf} eintippt.
In der Datei kann dann beispielsweise folgendes gesetzt werden:
\begin{lstlisting}[caption={[Einstellungen zur erweiterten Speichernutzung]Einstellungen zur erweiterten Speichernutzung in der Datei \printfilepath{texmf.cnf} bei TeXLive bzw. \printfilepath{xelatex.ini} bei MiKTeX},label={lst:Speichererweiterung}]
main_memory=5000000
extra_mem_top=5000000
extra_mem_bot=5000000
pool_size=5000000
buf_size=5000000
save_size=79999
\end{lstlisting}
Der maximal mögliche Wert für \printkeyword{main\_memory} etc. ist \printkeyword{79999999}.

Unter MiKTeX unterscheidet sich die Vorgehensweise je nach dem,
ob man die Einstellung nur für den aktuellen Benutzer oder
global für alle Benutzer des Computers machen möchte.
Im zweiten Fall braucht man Admin-Rechte.

Um die Einstellungen nur für den aktuellen Benutzer zu setzen,
öffnet man die Windows-Eingabeaufforderung
(Auf den \printkeyword{Start}-Knopf gehen und \printkeyword{cmd} eintippen)
und tippt dort \printkeyword{initexmf --edit-config-file=xelatex} ein.
Um die Einstellungen nur für alle Benutzer zu setzen,
öffnet man die Windows"=Eingabeaufforderung im Administrator"=Modus
(dazu auf den Menüeintrag \printkeyword{Eingabeaufforderung} mit der rechten Maustaste anklicken und \printkeyword{Als Administrator ausführen} wählen)
und tippt dort \printkeyword{initexmf \textbf{--admin} --edit-config-file=xelatex} ein.

Dabei öffnet sich die Datei \printfilepath{xelatex.ini}, in welcher die \og Einstellungen (s. \cref{lst:Speichererweiterung}) gesetzt werden sollten.%
\footnote{Die Datei \printfilepath{xelatex.ini} mit benutzerübergreifend gültigen Einstellungen befindet sich unter
\printfilepath{C:\bs Program Files\bs MiKTeX 2.9\bs miktex\bs config}.
Die dort gesetzten Einstellungen gelten, sofern der einzelne Benutzer keine eigenen Einstellungen definiert hat.
Die benutzerspezifische Datei, deren Einstellungen die globalen Einstellungen überrufen können, befindet sich unter
\printfilepath{C:\bs Users\bs <Benutzername>\bs AppData\bs Roaming\bs MiKTeX\bs 2.9\bs miktex\bs config}.}

Nach Speicherung der Datei ist bei der benutzerspezifischen Anpassung in der Eingabeaufforderung der Befehl
\printkeyword{initexmf --dump=xelatex} aufzurufen.
Im Falle einer benutzerübergreifenden Anpassung ist der Befehl
\printkeyword{initexmf \textbf{--admin} --dump=xelatex}
aufzurufen.