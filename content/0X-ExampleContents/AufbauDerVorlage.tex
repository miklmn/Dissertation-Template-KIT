%%%%%%%%%%%%%%%%%%%%%%%%%%%%%%%%%%%%%%%%%%%%%%%%%%%%%%%%%%%%
\section{Aufbau der Vorlage}%
\label{sec:AufbauDerVorlage}
%%%%%%%%%%%%%%%%%%%%%%%%%%%%%%%%%%%%%%%%%%%%%%%%%%%%%%%%%%%%
%
Die Vorlage besteht aus mehreren Dateien und Verzeichnissen.
Ihre Bedeutung ist in \cref{tab:StrukturDerVorlage} zusammengefasst.

{%%% Übersicht über die Dateien und Verzechnisse der Vorlage %%%
% kleinere Schrift
\footnotesize%
% Abstand zwischen den Zeilen auf 1,5faches setzen
\renewcommand{\arraystretch}{2.0}%
% Seitliche Abstände links und rechts eliminieren
%\setlength\LTleft{0pt}%
%\setlength\LTright{0pt}%
\setlength{\tabcolsep}{0pt}%
% Tabelle aus einer Extra-Datei einlesen und auf die Seitenbreite strecken
\LTXtable{\columnwidth}{tables/DateienUndVerzeichnisse.tex}%
}%


%%%%%%%%%%%%%%%%%%%%%%%%%%%%%%%%%%%%%%%%%%%%%%%%%%%%%%%%%%%%
\subsection{Projektdatei \printfilepath{Diss.tcp}}%
\index{Projektdatei}%
\label{sec:Projektdatei}
%%%%%%%%%%%%%%%%%%%%%%%%%%%%%%%%%%%%%%%%%%%%%%%%%%%%%%%%%%%%
%
Bei der Datei \printfilepath{Diss.tcp} handelt es sich um die Projektdatei für den \LaTeX-Editor TeXnicCenter.
In ihr werden die projektbezogenen Einstellungen des TeXnicCenter festgehalten.
Das sind u.a. Angaben zur Hauptdatei des Projektes und zur Projektsprache.
Die korrekte Angabe der Projektsprache ist insofern wichtig, als dass diese in TeXnicCenter ab Version 2.0 Beta 1 zur Bestimmung der Sprache für die Rechtschreibprüfung verwendet wird.
Die entsprechenden Einstellungen können im TeXnicCenter über den Menüeintrag \printkeyword{Projekt} $\rightarrow$ \printkeyword{Eigenschaften} vorgenommen werden.


%%%%%%%%%%%%%%%%%%%%%%%%%%%%%%%%%%%%%%%%%%%%%%%%%%%%%%%%%%%%
\subsection[Hauptdatei \printfilepath{Diss.tex}]{Hauptdatei \printfilepath{Diss.tex}}%
\index{Hauptdatei}%
\label{sec:Hauptdatei}
%%%%%%%%%%%%%%%%%%%%%%%%%%%%%%%%%%%%%%%%%%%%%%%%%%%%%%%%%%%%
%
Die Hauptdatei ist die Datei \printfilepath{Diss.tex}.
Sie ist verhältnismäßig kurz, da die Hauptinhalte in andere Dateien ausgelagert sind, welche mit Hilfe des \lc{input\{\}} \bzw des \lc{include\{\}}-Befehls eingebunden werden.
Die Hauptdatei besteht im Wesentlichen aus drei Abschnitten.
Im ersten stehen die sogenannten \index{Magic Comments}\enquote{Magic comments}, mit deren Hilfe manche \LaTeX-IDEs (wie \zB TeXStudio) sich selbst vorkonfigurieren können.
Sie fangen mit \enquote{\texttt{\%~!TeX}} an und geben an, welche Kodierung für die Dateien verwendet wird und welche Programme für die Kompilierung des Quelltextes und der Bibliografie verwendet werden sollen.
Außerdem kann damit angegeben werden, welche Sprache für die Rechtschreibprüfung innerhalb der IDE verwendet werden soll.
Da es aktuell keine dokumentierte Möglichkeit gibt, Aufruf für \printkeyword{makeglossaries} miteinzubinden,
führt eine Übernahme der durch die \enquote{Magic Comments} vordefinierten Reihenfolge dazu, dass keine Glossare generiert werden.
Daher sind die Aktuell sind \enquote{Magic Comments} aktuell durch ein zusätzliches Prozentzeichen deaktiviert.

Im zweiten Abschnitt wird die Header-Datei eingebunden.
In dieser wird die verwendeten Dokumentklasse (inklusive Papierformat und Schriftgröße) definiert, sowie weitere Dateien eingebunden.
In diesen werden die zu landenden Pakete, Layout"=Parameter sowie alle weiteren Einstellungen und Makros definiert und konfiguriert.

Im dritten Abschnitt werden nun die einzelnen Inhalte der Arbeit eingebunden.

Die einzelnen \index{Kapitel}Kapiteln der Arbeit werden im Verzeichnis \printfilepath{./content/} als separate Dateien gespeichert.
Es empfiehlt sich als Dateiname das Schema \printfilepath{nn-name.tex} zu verwenden, wobei \printkeyword{nn} die Nummer des Kapitels ist,
sodass die Dateien in der semantisch richtigen Reihenfolge sortiert angezeigt werden.
Die einzelnen Dateien werden per \verb+include{}+-Direktive in der Datei \printfilepath{Diss.tex} eingebunden.
Theoretisch wäre es an dieser Stelle auch möglich mit \verb+\input{}+ zu arbeiten, was jedoch seine Nachteile hätte.
Der Unterschied zwischen den beiden Befehlen wird \href{https://texwelt.de/wissen/fragen/32/was-ist-der-unterschied-zwischen-include-and-input}{auf texwelt.de} erklärt:

\begin{quote}
{\small
\verb+\input{file}+ lädt die Datei an Ort und Stelle in die Ziel-Datei und ist äquivalent
als ob man den Text in \printkeyword{file} direkt in die Ziel-Datei geschrieben hätte.
\verb+\input+ kann letztlich überall für jede Art Datei verwendet werden und kann auch verschachtelt angewendet werden,
d.h. eine eingebundene Datei kann ihrerseits Dateien mit \verb+\input+ einbinden.

\verb+\include{file}+ hingegen führt zunächst einmal ein \verb+\clearpage+ aus bevor es \verb+\input{file}+ ausführt.
Im Gegensatz zu \verb+\input+ kann eine Datei, die mit \verb+\include+ eingebunden wird,
kein weiteres \verb+\include+ enthalten, es ist also keine verschachtelte Anwendung möglich.
Eine mit \verb+\include+ eingebundene Datei kann aber natürlich \verb+\input+ enthalten.
\verb+\include+ erzeugt eine neue \printfilepath{aux}-Datei für die eingebundene Datei.
Das erlaubt es beispielsweise, ein Dokument in mehrere logische Einheiten zu zerlegen (etwa einzelne Kapitel),
die jede einer Datei entsprechen, die mit \verb+\include+ in die Hauptdatei eingebunden wird.
\verb+\includeonly{file1,file3}+ würde dann erlauben, nur gerade bearbeitete Dateien für die Kompilation einzubinden
und durch die separaten \printfilepath{aux}-Dateien dennoch korrekte Seitenzahlen und Querverweise zu erhalten.
Es gibt auch das \pkg{excludeonly} Paket, dessen Befehl \verb+\excludeonly+ das gegensätzliche Verhalten bietet.%
}%
\footnote{\url{https://texwelt.de/wissen/fragen/32/was-ist-der-unterschied-zwischen-include-and-input}}
\end{quote}

Es empfiehlt sich, die bestehende Struktur (zumindest jedoch die beiden Dateien
\printfilepath{Inhalt-FrontMatter.tex} und \printfilepath{Inhalt-Backmatter.tex})
als Vorlage zu übernehmen.

Das entstehende PDF heißt genauso wie die Hauptdatei.

Zur besseren Übersicht und zur Vereinfachung der Fehlersuche wird empfohlen,
die einzelnen Unterkapitel ebenfalls als separate Dateien in Unterverzeichnissen von \printfilepath{./content/} anzulegen
und sie mit den \verb+\input{}+-Direktiven in die jeweiligen Kapitel-Dateien einzubinden.



%%%%%%%%%%%%%%%%%%%%%%%%%%%%%%%%%%%%%%%%%%%%%%%%%%%%%%%%%%%%
\subsection[Bibliografie-Dateien]{Bibliografie-Dateien}%
\index{Bibliografie}%
\label{sec:Bibliografie}
%%%%%%%%%%%%%%%%%%%%%%%%%%%%%%%%%%%%%%%%%%%%%%%%%%%%%%%%%%%%
%
Es wird davon ausgegangen, dass sich sämtliche Bibliografie-Angaben in der Datei
\printfilepath{./bib/Diss.bib} befinden.
Sollten mehrere Bibliografie-Dateien verwendet werden, können diese in der Datei
\printfilepath{./preamble/AllePfade.tex} gesetzt werden.


%%%%%%%%%%%%%%%%%%%%%%%%%%%%%%%%%%%%%%%%%%%%%%%%%%%%%%%%%%%%
\subsection[Bilder und Zeichnungen]{Bilder und Zeichnungen}%
\index{Bilder!Binär-}%
\index{Bild!Vektor-}%
\label{sec:BilderUndZeichnungen}
%%%%%%%%%%%%%%%%%%%%%%%%%%%%%%%%%%%%%%%%%%%%%%%%%%%%%%%%%%%%
%
Bilder bzw. \index{Zeichnung|see{Bild}}Zeichnungen werden auf zwei Arten eingebunden.
Bilder im \index{Bild!Binär-}Binärformat (PNG, JPEG, TIFF, PDF, etc.)
werden mit \lc{includegraphics}-Befehl eingebunden. 
Bei den \index{Bild!TikZ}\gls{gls:tikz}-Zeichnungen handelt es sich um reguläre TeX-Quellcode-Dateien,
die mit dem \verb+\input+-Befehl eingebunden werden.
Für eine einfache Verwaltung wird empfohlen, Binärbilder im Verzeichnis \printfilepath{./images/} abzulegen.
Zusätzliche Pfade können in der Datei \printfilepath{./preambel/AllePfade.tex} definiert werden.
Die \gls{gls:tikz}-Quellcode-Dateien sollten im Verzeichnis \printfilepath{./figures-src/} abgelegt werden.
Während des Kompilierens werden für jede \gls{gls:tikz}-Zeichnung im Verzeichnis \printfilepath{./figures-compiled/} mehrere Dateien erzeugt.
Der Inhalt dieses Verzeichnisses kann gefahrlos gelöscht werden.
Weitere Hinweise und Beispiele zur Einbindung von Grafiken finden sich in \cref{sec:Bilder}.

%%%%%%%%%%%%%%%%%%%%%%%%%%%%%%%%%%%%%%%%%%%%%%%%%%%%%%%%%%%%
\subsection[Datei \printfilepath{AlleAngaben.tex}: Angaben zur Arbeit]{Datei \printfilepath{AlleAngaben.tex}: Angaben zur Arbeit}%
\index{Autor}%
\index{Titel}%
\label{sec:Angaben}
%%%%%%%%%%%%%%%%%%%%%%%%%%%%%%%%%%%%%%%%%%%%%%%%%%%%%%%%%%%%
%
Die wichtigsten Einstellungen, die auf jeden Fall geändert werden müssen,
finden sich in der Datei \printfilepath{./preambel/AlleAngaben.tex}.
Hier werden \ua Angaben zum Verfasser, Art und Titel der Arbeit sowie zu den Gutachtern gemacht.

%%%%%%%%%%%%%%%%%%%%%%%%%%%%%%%%%%%%%%%%%%%%%%%%%%%%%%%%%%%%
\subsection[Datei \printfilepath{AlleSchalter.tex}: Anzeigeeinstellungen]{Datei \printfilepath{AlleSchalter.tex}: Anzeigeeinstellungen}%
\index{Schalter}%
\label{sec:Schalter}
%%%%%%%%%%%%%%%%%%%%%%%%%%%%%%%%%%%%%%%%%%%%%%%%%%%%%%%%%%%%
%
Eine weitere wichtige Datei ist \printfilepath{./preambel/AlleSchalter.tex}.
Darin wird u.a. die Hauptsprache \index{Sprache!Hauptsprache}Hauptsprache des Manuskriptes gesetzt, was sich an mehreren Stellen auswirkt.
So wird beispielsweise bei \index{Sprache!globale Umstellung}Umstellung auf Englisch als Hauptsprache
\enquote{Danksagung} durch \enquote{Acknowledgments},
\enquote{Inhaltsverzeichnis} durch \enquote{Contents}
\usw ersetzt.
Auch die Regeln der \index{Silbentrennung!Allgemeine Einstellung}Silbentrennung werden entsprechend angepasst.

Durch das Umsetzen der Schalter 
\lstinline|\setUserDefinedBoolean{SchalterName}{Bool-Wert}| 
können hier Teile der Arbeit aus- und wieder eingeblendet werden, ohne dass sie auskommentiert werden müssen.
Neben diesen Einstellungen befinden sich in der Datei \printfilepath{./preambel/AlleSchalter.tex} auch Einstellungen,
die für die Vorbereitung des Manuskriptes zum Druck beim \gls{ac:KSP} wichtig sind.
Insbesondere können hier
\begin{itemize}
\item mit der Einstellung
\lstinline|\setUserDefinedBoolean{showFrame}{true}|
die Satzspiegel-Ränder angezeigt werden und so kontrolliert werden,
ob nichts hinausragt und ob Tabellen und Bilder die komplett verfügbare Breite ausfüllen,
\item mit den Einstellungen
\lstinline|\setUserDefinedBoolean{coloredlistings}{false}| und 
\lstinline|\setUserDefinedBoolean{coloredlinks}{false}|
dafür gesorgt werden, dass alle Quellcode-Listings sowie Querverweise und URL-Adressen,
die normalerweise farbig sind, für die Druckversion nicht farbig gesetzt werden
und so die Anzahl der farbig zu druckenden Seiten reduziert wird,
\item mit der Einstellung
\lstinline|\setUserDefinedBoolean{useCMYKcolors}{true}|
eine Farbkonvertierung aller Farben in den CMYK-Farbraum für den Offset-Druck vorgenommen werden.
Diese Einstellung sollte nur für die Druckversion vorgenommen werden.%
\footnote{Da manche RGB-Farben bei der Konvertierung in den CMYK-Farbraum blass aussehen,
ist es sinnvoll, eine Alternativversion der betroffenen Farben im CYMK-Farbraum zu definieren,
die gut aussieht.
Beispiele dafür gibt es in der Datei \printfilepath{./preambel/ColorSettings.tex}.
Vorzugsweise sollten aber die KIT-Corporate-Identity-Farben verwendet werden,
welche in der Datei \printfilepath{KAcolors.sty} definiert sind.
Für diese Farben wurde sowohl eine RGB- als auch eine CMYK-Definition erstellt.}
\end{itemize}

%%%%%%%%%%%%%%%%%%%%%%%%%%%%%%%%%%%%%%%%%%%%%%%%%%%%%%%%%%%%
\subsection[Datei \printfilepath{Hyphenation.tex}: Silbentrennung von Spezialwörtern]{Datei \printfilepath{Hyphenation.tex}: Silbentrennung von Spezialwörtern}%
\index{Silbentrennung}%
\label{sec:Hyphenation}
%%%%%%%%%%%%%%%%%%%%%%%%%%%%%%%%%%%%%%%%%%%%%%%%%%%%%%%%%%%%
%
Regeln zur Silbentrennung unbekannter Wörter (\zB Fachbegriffe)
können in der Datei \printfilepath{./preambel/Hyphenation.tex} festgelegt werden.
Zu beachten ist, dass zusammengesetzte Wörter mit einem Bindestrich
ausschließlich am Bindestrich getrennt werden,
wogegen auch ein Eintrag in die Datei \printfilepath{Hyphenation.tex} nicht hilft.
Um Silbentrennung an anderen Stellen eines zusammengesetzten Wortes zu erlauben,
muss man bei Verwendung zusammengesetzter Wörter mit Bindestrich diesen durch „\verb+"=+“ ersetzen.
Dies gilt jedoch nur für deutschsprachige Texte.
Alternative Trennstellen in bestimmten Wörtern kann man Hilfe der Hinweise auf 
\href{https://de.wikibooks.org/wiki/LaTeX-W%C3%B6rterbuch:_Silbentrennung}{wikibooks.org}%
\footnote{\url{https://de.wikibooks.org/wiki/LaTeX-W%C3%B6rterbuch:_Silbentrennung}}
erreichen.


%%%%%%%%%%%%%%%%%%%%%%%%%%%%%%%%%%%%%%%%%%%%%%%%%%%%%%%%%%%%
\subsection[Abkürzungen und Fachbegriff-Definitionen]{Abkürzungen und Fachbegriff-Definitionen}%
\index{Abkürzungen}%
\index{Akronyme}%
\index{Glossar}%
\index{Symbole}%
\label{sec:Abkuerzungen}
%%%%%%%%%%%%%%%%%%%%%%%%%%%%%%%%%%%%%%%%%%%%%%%%%%%%%%%%%%%%
%
In den Dateien
\printfilepath{./preambel/Acronyms.tex},
\printfilepath{./preambel/Glossary.tex} und
\printfilepath{./preambel/GlossarySymbols.tex}
kann eine Liste der Abkürzungen, Fachbegriff-Definitionen und Symbole angelegt werden.
Details hierzu finden sich im \cref{sec:Glossare}.


%%%%%%%%%%%%%%%%%%%%%%%%%%%%%%%%%%%%%%%%%%%%%%%%%%%%%%%%%%%%
\subsection[Frontmatter und Backmatter]{Frontmatter und Backmatter}%
\index{Frontmatter}%
\index{Backmatter}%
\index{Notation}%
\index{Symbolverzeichnis}%
\index{Literaturverzeichnis}%
\index{Abkürzungsverzeichnis}%
\index{Abbildungsverzeichnis}%
\index{Tabellenverzeichnis}%
\index{Listingsverzeichnis}%
\index{Glossar}%
\index{Stichwortverzeichnis}%
\index{Index|see{Stichwortverzeichnis}}
\label{sec:FrontmatterBackmatter}
%%%%%%%%%%%%%%%%%%%%%%%%%%%%%%%%%%%%%%%%%%%%%%%%%%%%%%%%%%%%
%
Die Einbindung von der Titelseite, Danksagung, dem englischen und deutschen Abstract,
einer Notationsübersicht und ggf. einem Symbolverzeichnis
erfolgt in der Datei \printfilepath{./content/Inhalt-FrontMatter.tex}.
Die jeweilige Inhalte sind im Verzeichnis \printfilepath{./content/00-FrontMatter/} zu finden.
Die Einbindung von Literatur-, Abbildungs, Tabellen, Quellcodelistings-, Abkürzungsverzeichnissen sowie ggf. einem Glossar, Stichwortverzeichnis (Index) und den Anhängen erfolgt in der Datei \printfilepath{./content/Inhalt-BackMatter.tex}.
