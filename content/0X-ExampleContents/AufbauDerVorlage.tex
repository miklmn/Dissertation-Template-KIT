%%%%%%%%%%%%%%%%%%%%%%%%%%%%%%%%%%%%%%%%%%%%%%%%%%%%%%%%%%%%
\section{Aufbau der Vorlage}%
\label{sec:AufbauDerVorlage}
%%%%%%%%%%%%%%%%%%%%%%%%%%%%%%%%%%%%%%%%%%%%%%%%%%%%%%%%%%%%
%
Die Vorlage besteht aus mehreren Dateien und Verzeichnissen.
Ihre Bedeutung ist in \cref{tab:StrukturDerVorlage} zusammengefasst.
%
%{\footnotesize
\begin{table}[htbp]
%\scriptsize% kleinere Schrift
\footnotesize% kleinere Schrift
\centering%% Tabelle zentrieren (falls nicht volle Seitenbreite)
\renewcommand{\arraystretch}{1.5}% Abstand zwischen den Zeilen auf 1,5faches
\setlength{\tabcolsep}{0pt}% Seitliche Abstände eliminieren
% Tabelle auf die Seitenbreite strecken
\begin{tabularx}{\columnwidth}%
% Breite der ersten Spalte auf Inhalt anpassen,
% Breite der zweiten Spalte auf 25% der Seitenbreite,
% Breite der dritten Spalte automatisch bestimmen,
%Spacing zwischen den Spalten auf 10pt setzen:
{l@{\extracolsep{10pt}}p{0.25\columnwidth}X}%
%-----------------------------------------------------------------------------------------------------
\toprule%
%-----------------------------------------------------------------------------------------------------
\bfseries Datei/Verzeichnis               & \bfseries Bedeutung
                                          & \bfseries Benutzerinteraktion\\
%-----------------------------------------------------------------------------------------------------
\midrule%
%-----------------------------------------------------------------------------------------------------
\texttt{./Diss.tcp}                       & \texttt{TeXnicCenter}-Projektdatei
                                          & Aufruf im TeXnicCenter. Indirekte Änderung durch Einstellungen im TeXnicCenter.\\
%-----------------------------------------------------------------------------------------------------
\texttt{./Diss.tex}                       & \texttt{TeX}-Hauptdatei
                                          & Aus- und Einblenden der einzelnen Teile durch Ersetzen von \texttt{showif} durch \texttt{hideif} und vice versa.\\
%-----------------------------------------------------------------------------------------------------
\texttt{./bib/Diss.bib}                   & \texttt{BibLaTeX}-Quelldatei
                                          & Verwendete Referenzen einfügen.\\
%-----------------------------------------------------------------------------------------------------
\texttt{./content/*}                      & \texttt{LaTeX}-Kapitel
                                          & Kapitelinhalte einfügen.\\
%-----------------------------------------------------------------------------------------------------
\texttt{./figures-src/*}                  & \texttt{TikZ}-Zeichnungen
                                          & Ggf. \texttt{TikZ}-Zeichnungen hinzufügen.\\
%-----------------------------------------------------------------------------------------------------
\texttt{./figures-compiled/*}             & Temporäre Kompilate
                                          & nur löschen \\
%-----------------------------------------------------------------------------------------------------
\texttt{./fonts/*}                        & Schriften
                                          & keine \\
%-----------------------------------------------------------------------------------------------------
\texttt{./images/*}                       & Binärformat-Bilder
                                          & Ggf. Bilder im Binärformat hinzufügen. \\
%-----------------------------------------------------------------------------------------------------
\texttt{./preamble/*}                     & Konfigurationsdateien
                                          & s.u.\\
%-----------------------------------------------------------------------------------------------------
\texttt{./preamble/Acronyms.tex}          & Akronyme
                                          & Ggf. Definition von Akronymen.\\
%-----------------------------------------------------------------------------------------------------
\texttt{./preamble/AlleAngaben.tex}       & Daten der Arbeit
                                          & Hier werden die Angaben zum Typ der Arbeit, zum Autor und zu den Gutachtern gemacht. Auch lässt sich hier die Sprache der Arbeit einstellen.\\
%-----------------------------------------------------------------------------------------------------
\texttt{./preamble/AllePfade.tex}         & Definition von Zusatzpfaden
                                          & Ggf. Pfade zu weiteren Bildverzeichnissen und Bibliografie"=Dateien ergänzen.\\
%-----------------------------------------------------------------------------------------------------
%\texttt{./preamble/BibSettings.tex}       & Konfiguration der Bibliografie
%                                          & Normalerweise keine, es sei denn man möchte den Stil der Literaturverzeichnisse ändern. \\
%-----------------------------------------------------------------------------------------------------
%\texttt{./preamble/EncodingAndFont.tex}   & Schriftarteinstellungen
%                                          & Keine.\\
%-----------------------------------------------------------------------------------------------------
\texttt{./preamble/Glossary.tex}          & Glossareinträge
                                          & Ggf. Glossareinträge definieren.\\
%-----------------------------------------------------------------------------------------------------
\texttt{./preamble/Header.tex}            & Präambel-Definitionen
                                          & Hier könnte das A4-Layout und die Draft-Option aktiviert werden.\\
%-----------------------------------------------------------------------------------------------------
\texttt{./preamble/Hyphenation.tex}       & Silbentrennung
                                          & Ggf. Regeln für die Silbentrennung unbekannter Wörter hinzufügen.\\
%-----------------------------------------------------------------------------------------------------
%\texttt{./preamble/IndexStyle.tex}        & Layout des Stichwortverzeichnisses
%                                          & Keine.\\
%-----------------------------------------------------------------------------------------------------
%\texttt{./preamble/KomaOptions.tex}       & KomaScript-Optionen
%                                          & Keine.\\
%-----------------------------------------------------------------------------------------------------
\texttt{./preamble/Math.tex}              & Mathe-Einstellungen
                                          & Bei Bedarf eigene Mathe-Makros definieren.\\
%-----------------------------------------------------------------------------------------------------
\texttt{./preamble/MyPackages.tex}        & Zusatzpakete
                                          & Ggf. Einbindung von Zusatzpaketen.\\
%-----------------------------------------------------------------------------------------------------
\texttt{./preamble/Newcommands.tex}       & Eigene Macros
                                          & Ggf. eigene \LaTeX{}-Macros hinzufügen.\\
%-----------------------------------------------------------------------------------------------------
%\texttt{./preamble/preambel.tex}          & Paketkonfigurationen
%                                          & normalerweise keine \\
%-----------------------------------------------------------------------------------------------------
%\texttt{./preamble/preambel-commands.tex} & Paketkonfigurationen
%                                          & normalerweise keine \\
%-----------------------------------------------------------------------------------------------------
%\texttt{./preamble/settings.tex}          & Einstellungen zu Längen, Breiten, Verzeichnistiefen etc.
%                                          & normalerweise keine \\
%-----------------------------------------------------------------------------------------------------
%\texttt{./preamble/TableCommands.tex}     & Tabelleneinstellungen
%                                          & normalerweise keine \\
%-----------------------------------------------------------------------------------------------------
%\texttt{./preamble/Translations.tex}      & Multilinguale Begriffsdefinitionen
%                                          & normalerweise keine \\
%-----------------------------------------------------------------------------------------------------
\bottomrule%
%-----------------------------------------------------------------------------------------------------
\end{tabularx}%
\caption[Dateien und Verzeichnisse der Vorlage]{Dateien und Verzeichnisse der Vorlage}%
\label{tab:StrukturDerVorlage}%
\end{table}

Bei der Datei \texttt{Diss.tcp} handelt es sich um die Projektdatei für den \LaTeX-Editor TeXnicCenter.
In ihr werden die projektbezogenen Einstellungen des TeXnicCenter festgehalten.
Das sind u.a. Angaben zur Hauptdatei des Projektes und zur Projektsprache.
Die korrekte Angabe der Projektsprache ist insofern wichtig, als dass diese in TeXnicCenter ab Version 2.0 Beta 1 zur Bestimmung der Sprache für die Rechtschreibprüfung verwendet wird.
Die entsprechenden Einstellungen können im TeXnicCenter über den Menüeintrag \texttt{Projekt} $\rightarrow$ \texttt{Eigenschaften} vorgenommen werden.


Die Hauptdatei ist die Datei \texttt{Diss.tex}.
Sie ist verhältnismäßig kurz, da die Hauptinhalte in andere Dateien ausgelagert sind, welche mit Hilfe des \verb+\input{}+ \bzw des \verb+\include{}+-Befehls eingebunden werden.
Die Hauptdatei besteht im Wesentlichen aus vier Abschnitten.
Im ersten stehen die sogenannten \enquote{Magic comments}, mit deren Hilfe manche \LaTeX-IDEs sich selbst vorkonfigurieren können.
Sie fangen mit \verb+\% !TeX+ an und geben an, welche Kodierung für die Dateien verwendet wird und welche Programme für die Kompilierung des Quelltextes und der Bibliografie verwendet werden sollen.
Außerdem kann hier angegeben werden, welche Sprache für die Rechtschreibprüfung innerhalb der IDE verwendet werden soll.
Im zweiten Abschnitt wird die Header-Datei eingebunden.
In dieser wird die verwendeten Dokumentklasse (inklusive Papierformat und Schriftgröße) definiert, sowie weitere Dateien eingebunden,
in welchen die zu landenden Pakete, Layout"=Parameter sowie alle weiteren Einstellungen definiert und konfiguriert werden.
Im dritten Teil können mit Hilfe von Schaltern einzelne Teile der Arbeit aus- und wieder eingeblendet werden, ohne dass sie auskommentiert werden müssen.
Im vierten Teil werden nun die einzelnen Inhalte der Arbeit eingebunden.

Das entstehende PDF heißt genauso wie die Hauptdatei.

Die einzelnen \index{Kapitel}Kapiteln der Arbeit werden im Verzeichnis \texttt{./content/} als separate Dateien gespeichert.
Es empfiehlt sich als Dateiname das Schema \texttt{nn-name.tex} zu verwenden, wobei \texttt{nn} die Nummer des Kapitels ist,
sodass die Dateien in der semantisch richtigen Reihenfolge sortiert angezeigt werden.
Die einzelnen Dateien werden per \verb+include{}+-Direktive in der Datei \texttt{Diss.tex} eingebunden.
Theoretisch wäre es an dieser Stelle auch möglich mit \verb+\input{}+ zu arbeiten, was jedoch seine Nachteile hätte.
Der Unterschied zwischen den beiden Befehlen wird \href{https://texwelt.de/wissen/fragen/32/was-ist-der-unterschied-zwischen-include-and-input}{auf texwelt.de} erklärt:

\begin{quote}
{\small
\verb+\input{file}+ lädt die Datei an Ort und Stelle in die Ziel-Datei und ist äquivalent
als ob man den Text in \texttt{file} direkt in die Ziel-Datei geschrieben hätte.
\verb+\input+ kann letztlich überall für jede Art Datei verwendet werden und kann auch verschachtelt angewendet werden,
d.h. eine eingebundene Datei kann ihrerseits Dateien mit \verb+\input+ einbinden.

\verb+\include{file}+ hingegen führt zunächst einmal ein \verb+\clearpage+ aus bevor es \verb+\input{file}+ ausführt.
Im Gegensatz zu \verb+\input+ kann eine Datei, die mit \verb+\include+ eingebunden wird,
kein weiteres \verb+\include+ enthalten, es ist also keine verschachtelte Anwendung möglich.
Eine mit \verb+\include+ eingebundene Datei kann aber natürlich \verb+\input+ enthalten.
\verb+\include+ erzeugt eine neue \texttt{aux}-Datei für die eingebundene Datei.
Das erlaubt es beispielsweise, ein Dokument in mehrere logische Einheiten zu zerlegen (etwa einzelne Kapitel),
die jede einer Datei entsprechen, die mit \verb+\include+ in die Hauptdatei eingebunden wird.
\verb+\includeonly{file1,file3}+ würde dann erlauben, nur gerade bearbeitete Dateien für die Kompilation einzubinden
und durch die separaten \texttt{aux}-Dateien dennoch korrekte Seitenzahlen und Querverweise zu erhalten.
Es gibt auch das \texttt{excludeonly} Paket, dessen Befehl \verb+\excludeonly+ das gegensätzliche Verhalten bietet.%
}%
\footnote{\url{https://texwelt.de/wissen/fragen/32/was-ist-der-unterschied-zwischen-include-and-input}}
\end{quote}

Zur besseren Übersicht und zur Vereinfachung der Fehlersuche wird empfohlen,
die einzelnen Unterkapitel ebenfalls als separate Dateien in Unterverzeichnissen von \texttt{./content/} anzulegen
und sie mit den \verb+\input{}+-Direktiven in die jeweiligen Kapitel-Dateien einzubinden.

Es wird davon ausgegangen, dass sich sämtliche Bibliografie-Angaben in der Datei
\texttt{./bib/Diss.bib} befinden.
Sollten mehrere Bibliografie-Dateien verwendet werden, können diese in der Datei
\texttt{./preamble/AllePfade.tex} gesetzt werden.

\index{Bild}Bilder bzw. \index{Zeichnung|see{Bild}}Zeichnungen werden auf zwei Arten eingebunden.
Bilder im \index{Bild!Binär-}Binärformat (PNG, JPEG, TIFF, PDF, etc.)
werden mit \verb+\includegraphics+-Befehl eingebunden. 
Bei den \index{Bild!TikZ}\gls{gls:tikz}-Zeichnungen handelt es sich um reguläre TeX-Quellcode-Dateien,
die mit dem \verb+\input+-Befehl eingebunden werden.
Für eine einfache Verwaltung wird empfohlen, Binärbilder im Verzeichnis \texttt{./images/} abzulegen.
Zusätzliche Pfade können in der Datei \texttt{./preamble/AllePfade.tex} definiert werden.
Die \gls{gls:tikz}-Quellcode-Dateien sollten im Verzeichnis \texttt{./figures-src/} abgelegt werden.
Während des Kompilierens werden für jede \gls{gls:tikz}-Zeichnung im Verzeichnis \texttt{./figures-compiled/} mehrere Dateien erzeugt.
Der Inhalt dieses Verzeichnisses kann gefahrlos gelöscht werden.
Weitere Hinweise und Beispiele zur Einbindung von Grafiken finden sich in \cref{sec:Bilder}.

Die wichtigsten Einstellungen, die auf jeden Fall geändert werden müssen,
finden sich in der Datei \verb+./preamble/AlleAngaben.tex+.
Hier werden \ua Angaben zum Verfasser, Art und Titel der Arbeit sowie zu den Gutachtern gemacht.
Außerdem wird hier die Hauptsprache der Arbeit gesetzt, was sich an mehreren Stellen auswirkt.
So wird beispielsweise bei Umstellung auf Englisch als Hauptsprache
\enquote{Danksagung} durch \enquote{Acknowledgments},
\enquote{Inhaltsverzeichnis} durch \enquote{Contents}
\usw ersetzt.
Auch die Regeln der Silbentrennung werden entsprechend angepasst.