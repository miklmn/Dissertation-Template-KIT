%%%%%%%%%%%%%%%%%%%%%%%%%%%%%%%%%%%%%%%%%%%%%%%%%%%%%%%%%%%%
\section{Mathematische Sätze, Lemmas, Definitionen etc.}%
\label{sec:Theoreme}
%%%%%%%%%%%%%%%%%%%%%%%%%%%%%%%%%%%%%%%%%%%%%%%%%%%%%%%%%%%%
%
Für eine mathematische Ausarbeitung gibt es LaTeX-\glspl{gls:umgebung}, um
\index{Satz|see{Theorem}}\index{Theorem}Sätze (Theoreme), \index{Lemma|see{Theorem}}Lemma,
\index{Beispiel|see{Theorem}}Beispiele etc. im üblichen Stil von
Mathematik-Büchern zu setzen und zu referenzieren. Vordefiniert sind die
\glspl{gls:umgebung}
\begin{itemize*}
  \item \texttt{theorem} für Sätze
  \item \texttt{definition} für Definitionen
  \item \texttt{lemma} für Lemma
  \item \texttt{corollary} für Korollare
  \item \texttt{proposition} für Propositionen
\end{itemize*}
Die übliche Verwendung ist
\begin{latex}[caption={Beispiel für Theorem-Umgebungen},label={lst:ntheorem}]
\begin{theorem}[Optionaler Name]\label{thm:my-theorem}
...
\end{theorem}
\end{latex}
Weitere Informationen findet man in der Dokumentation zum \texttt{ntheorem}-Paket
\parencite{May2011}. Das Ganze sieht dann beispielsweise wie folgt aus.

\begin{theorem}[Theorem von Arthur Dent]\label{thm:arthur-dent} Die Antwort auf
die Frage nach dem Leben, dem Universum und den ganzen Rest ist 42.
\end{theorem}

\begin{proposition}[Zweifelhafte Folgerung] LaTeX ist schön. Beweis folgt
unmittelbar aus \cref{thm:arthur-dent}.
\end{proposition}