%%%%%%%%%%%%%%%%%%%%%%%%%%%%%%%%%%%%%%%%%%%%%%%%%%%%%%%%%%%%
\section{Aufzählungen}%
\index{Aufzählungen}%
\label{sec:Aufzählungen}
%%%%%%%%%%%%%%%%%%%%%%%%%%%%%%%%%%%%%%%%%%%%%%%%%%%%%%%%%%%%
%
Nachfolgend gibt es Beispiele für unterschiedliche Aufzählungen.
%
\begin{enumerate}
\item Aufzählungen mit längeren \enquote{Items}, die ganze Sätze enthalten, sollten mit \printkeyword{itemize} (unnummerierte Liste) bzw. \printkeyword{enumerate} (nummerierte Liste) eingefügt werden.
\item Dabei ist darauf zu achten, dass im Quellcode keine leeren Zeilen und damit kein zusätzlicher Absatz zwischen der Aufzählung und dem vorhergehenden Textabschnitt eingefügt wird.
\item Um die Aufzählung im Quellcode optisch vom vorangegangenem Textabschnitt zu trennen, können leere Zeilen mit einem Prozentzeichen verwendet werden, die von \LaTeX{} ignoriert werden.
\end{enumerate}
%
Bei Verwendung von \printkeyword{itemize} bzw. \printkeyword{enumerate} ist der vertikale Abstand zwischen den einzelnen \enquote{Items} ziemlich groß.
Dies ist gut für längere Punkte, sieht aber bei Aufzählungen mit kürzeren \enquote{Items} unschön aus.
Hierfür bieten die beiden Umgebungen \printkeyword{itemize*} und \printkeyword{enumerate*} mit einem Stern eine Abhilfe.
Das Ergebnis sind kompaktere Auflistungen mit kleinerem Abstand:
%
\begin{itemize*}
\item Punkt 1
\item Punkt 2
\item Punkt 3
\end{itemize*}
