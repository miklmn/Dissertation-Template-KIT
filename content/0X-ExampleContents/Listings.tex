%%%%%%%%%%%%%%%%%%%%%%%%%%%%%%%%%%%%%%%%%%%%%%%%%%%%%%%%%%%%
\section{Listings}%
\label{sec:Listings}
%%%%%%%%%%%%%%%%%%%%%%%%%%%%%%%%%%%%%%%%%%%%%%%%%%%%%%%%%%%%
%
Zum Einbinden und formatieren von \index{Code|see{Listing}}Quellcode"=Beispielen
-- sog. \index{Listing}Listings -- wird das Paket \texttt{listings}
\parencite{Hoffmann2014} verwendet.
Das Hervorheben von \index{Schlusselwort@Schlüsselwort}Schlüsselwörtern
wird von LaTeX automatisch erledigt,
wenn die korrekte Sprache des Listings angegeben ist.
Vordefininiert sind die Umgebungen \index{Java}\texttt{java} für \gls{gls:java},
\index{C++}\texttt{C++} für C++ und \texttt{latex} für \gls{gls:latex}.

So bewirkt
%
\begin{latex}[%
  caption={Listing-Beispiel},%
  label={lst:listing}]
\begin{java}[caption={A Java Hello-World example},label={lst:hello-world}]
public class HelloWorld {
  public static void main( String[] args ) {
    System.out.println( "HelloWorld" );
  }
}
\end{java}
\end{latex}
%
das folgende Ergebnis:
%
\begin{java}[caption={A Java Hello-World example},label={lst:hello-world}]
public class HelloWorld {
  public static void main( String[] args ) {
    System.out.println( "HelloWorld" );
  }
}
\end{java}

Man beachte, dass anders als bei anderen Umgebungen
die Bezeichnung (\texttt{caption}) und die Referenzmarke (\texttt{label})
nicht als gesonderte Befehle sondern als optionale Argumente übergeben werden.
Dies liegt daran, dass ein Listing in der Regel keine Fließumgebung ist,
sondern an der Stelle im Text erscheint, an der sie im Code auch steht.
Ferner folgt ein Listing den ganz normalen Seitenumbruchsregeln.
Das heißt, überlanger Code wird einfach umgebrochen. 
m ein Listing zu einem Fließobjekt zu machen, muss das optionale Argument
\texttt{float=<tbp>} angegeben werden.
Die \index{Platzierung}Plazierungsangabe \enquote{\texttt{h}} für \enquote{hier}
ist nicht erlaubt. Denn dies ist das Standardverhalten ohne \texttt{float}.