\chapter[Nicht ganz so lange Einleitung]{Wirklich sehr extrem und fast nicht auszuhalten lange Einleitung mit einleitenden Worten zur Thematik}

Das war ein Beispiel für eine sehr lange Überschrift, die im Inhaltsverzeichnis (oder anderen Verzeichnissen) zu lange erscheinen würde. In eckigen Klammern kann man einen Kurztitel angeben. Das kostet keinen %€
Euro.

\section{Schriftgrößen}
Jetzt kommen verschiedene Schriftgrößen für den normalen Text zum Einsatz. Das war \texttt{\textbackslash normalsize}:

{\small Das ist \texttt{small}. Das hier ist kleinere Schrift. Das hier ist kleinere Schrift. Das hier ist kleinere Schrift. Das hier ist kleinere Schrift. Das hier ist kleinere Schrift. Das hier ist kleinere Schrift. Das hier ist kleinere Schrift. Das hier ist kleinere Schrift. Das hier ist kleinere Schrift. Das hier ist kleinere Schrift. Das hier ist kleinere Schrift. Das hier ist kleinere Schrift.}

{\footnotesize Das ist \texttt{footnotesize}. Das hier ist kleinere Schrift. Das hier ist kleinere Schrift. Das hier ist kleinere Schrift. Das hier ist kleinere Schrift. Das hier ist kleinere Schrift. Das hier ist kleinere Schrift. Das hier ist kleinere Schrift. Das hier ist kleinere Schrift. Das hier ist kleinere Schrift. Das hier ist kleinere Schrift. Das hier ist kleinere Schrift. Das hier ist kleinere Schrift.}

{\scriptsize Das ist \texttt{scriptsize}. Das hier ist kleinere Schrift. Das hier ist kleinere Schrift. Das hier ist kleinere Schrift. Das hier ist kleinere Schrift. Das hier ist kleinere Schrift. Das hier ist kleinere Schrift. Das hier ist kleinere Schrift. Das hier ist kleinere Schrift. Das hier ist kleinere Schrift. Das hier ist kleinere Schrift. Das hier ist kleinere Schrift. Das hier ist kleinere Schrift. Das hier ist kleinere Schrift.}

{\tiny Das ist \texttt{tiny}. Das hier ist kleinere Schrift. Das hier ist kleinere Schrift. Das hier ist kleinere Schrift. Das hier ist kleinere Schrift. Das hier ist kleinere Schrift. Das hier ist kleinere Schrift. Das hier ist kleinere Schrift. Das hier ist kleinere Schrift. Das hier ist kleinere Schrift. Das hier ist kleinere Schrift. Das hier ist kleinere Schrift. Das hier ist kleinere Schrift. Das hier ist kleinere Schrift. Das hier ist kleinere Schrift. Das hier ist kleinere Schrift. Das hier ist kleinere Schrift. Das hier ist kleinere Schrift. Das hier ist kleinere Schrift. Das hier ist kleinere Schrift. }


{\large Das ist \texttt{large}.Aber auch größere Schrift ist möglich. Aber auch größere Schrift ist möglich. Aber auch größere Schrift ist möglich. Aber auch größere Schrift ist möglich. Aber auch größere Schrift ist möglich. Aber auch größere Schrift ist möglich.}

{\Large Das ist \texttt{Large} mit großem L. Aber auch größere Schrift ist möglich. Aber auch größere Schrift ist möglich. Aber auch größere Schrift ist möglich.}

{\LARGE Das ist \texttt{LARGE} komplett großgeschrieben. Aber auch größere Schrift ist möglich. Aber auch größere Schrift ist möglich.}

{\huge Das ist \texttt{huge}. Aber auch noch größere Schrift ist möglich.}

{\Huge Und das ist \texttt{Huge}, geschrieben mit großem H.}

\section{Schriftschnitte}

Es gibt auch \textbf{fette Schrift}. Die ist dann weder \textsl{schräggestellt} noch \textit{kursiv}. Das ist übrigens ein großer Unterschied!

Und das ist was anderes als \textsf{serifenlose Schrift} oder \texttt{Schreib"-maschinen"-schrift fester Breite}. Was es nicht in jeder Schrift gibt sind \textsc{Kapitälchen}. Dann bekommt man einfach eine \LaTeX-Warnung und hat keine Kapitälchen.

Zur normalen \emph{Hervorhebung} wird \texttt{\textbackslash emph} genommen, weil in kursivem Text die \textbf{\textit{Aufrechtstellung}} hervorhebt. Das wird einfach umgeschaltet. Man kann die Schnitte natürlich auch \textsl{\texttt{kom"-bi"-nieren}}.

Alternative Befehle für {\bfseries Fettdruck} sind im Code zu sehen. Da gibt es auch {\itshape kursiv} und {\slshape schräggestellt}. Auch {\ttfamily Schreib"-maschinen"-schrift bzw. Teletype} oder {\sffamily Sans-serif} ist möglich. Man darf da die geschweiften Klammern drumrum aber nicht vergessen.

\section{Akzentzeichen für Fremdsprachen}

Französische Wörter: Citroën. Français. Café. Ampère. Fenêtre
Spanische Wörter: Mañana.
Polnische Wörter: Łódź.
Nordische Wörter: Ångström, Hølm.
Türkische Wörter: Saltoǧlu.
Russische Wörter: Хорошо.
Unterscheide ă und ǎ.
Ogonek: ą

\section{Anführungszeichen}

Über die verschiedenen Anführungszeichen gibt es immer wieder Diskussionen. Am einfachsten ist es mit dem Paket \texttt{enquote}, das den gleichnamigen Befehl bereitstellt. So wird einmal festgelegt, wie die Zeichen zu setzen sind und dann hat man keine \enquote{Probleme} mehr damit. Wer sie häufig verwendet, definiert sich am geschicktesten einen kürzeren Befehl dafür.

\section{Silbentrennung}
Wenn man feststellt, dass ein Wort, das oft vorkommt, immer wieder falsch getrennt wird, kann man es in die Hyphenation-Datei eintragen. Dort gibt man die möglichen Trennstellen eines Wortes mit einem normalen Bindestrich an. Es wird dann ausschließlich an den genannten Stellen getrennt. Alle Wortformen müssen einzeln eingetragen werden, es gibt keine automatische Erweiterung auf Plural oder andere Kasus/Konjugationen.

Wörter mit Bindestrich werden oft nicht richtig getrennt weil \TeX nur am Bindestrich trennt. Im Quelltext kann man aber abhelfen, wenn es nur um ein einzelnes Wort geht, für das man keinen Hyphenation-Eintrag erstellen will:\\
\verb+"-+ und \verb+\-+ sagt, dass an dieser Stelle getrennt werden darf, ohne dass weitere Trennstellen unterdrückt werden.\\
\verb+"=+ macht einen expliziten Trennstrich an dem umbrochen werden darf, die Einzelteile bleiben weiterhin separat trennbar. Das ist wohl die am meisten gesuchte Funktion beim Thema Silbentrennung\\
\verb+""+ ist wie \verb+"-+ nur dass kein Trennstrich ausgegeben wird.\\
\verb+"~+ fügt einen geschützten Trennstrich ein, an dem nicht umbrochen werden darf. Der Rest scheint auch nicht getrennt zu werden.\\
Verhindern einer Ligatur mit \verb+"|+ bei Auflage und Auf"|lage (hier zwischen f und l).\\
%Soll eine Trennung verhindert werden... hm, ja wie macht man das :-) ?

Feinheiten sind der Unterschied zwischen Auflage und Auf"|lage. Bei ersterem werden f und l zu einer Ligatur zusammengefasst, was hier aber falsch ist, da es sich um eine Vorsilbe handelt. Darüber werden sich vermutlich die wenigsten Gedanken machen. Aber vielleicht interessiert es ein paar Perfektionisten. In eine ähnliche Kategorie fällt die Kursivkorrektur per \verb+\/+, mit der man umschalten kann, wenn \emph{kursiv\/} geschriebene Wörter \emph{direkt\/} an normale anschließen (gegenüber: \emph{direkt} an). Da kann es passieren, dass der Zwischenraum zu gering ist und das t von direkt zu nahe ans a von an herankommt.

\section{Unterstreichen}

Das Paket \texttt{ulem} erlaubt verschiedene Unterstreichungen:\\
\uline{important} underlined text, \uuline{urgent} double-underlined text, \uwave{wavy} underline, \sout{wrong} line drawn through word, \xout{removed} marked over, \dashuline{dashing} dashed underline, \dotuline{dotty} dotted underline\\

\section{Weitere Funktionen}

Die meisten der folgenden Beispiele sind aus dem Beispieldokument von Matthias Pospiech (\texttt{demo.tex}) geklaut.

\begin{multicols}{3}[Text mit 3 Spalten. Erstellt mit dem Paket \texttt{multicol}]
Suspendisse ac nibh vitae nunc iaculis accumsan. Vivamus venenatis, orci vitae interdum tristique, nisl lectus fermentum arcu, sed vehicula pede orci et nunc. Cras tempus ultrices leo. Nulla at tortor. Morbi nisl tellus, lobortis nec, nonummy a, vulputate at, felis. In interdum varius sem. Fusce pellentesque, eros vitae consectetuer dignissim, ipsum urna tincidunt urna, ut aliquet libero lectus vel purus. In commodo iaculis justo. Sed euismod. Praesent molestie leo ac erat. Etiam a felis.
\end{multicols} 

\begin{center}
Das ist Zentrierter Text.
Aliquam ultrices libero hendrerit diam. Vestibulum ultrices sapien sit amet elit. Quisque tempor nisl eu sem. Nam lorem lectus, viverra nec, rutrum quis, lobortis nec, magna. Praesent hendrerit tortor vitae elit. Vivamus sed leo at mi elementum semper. Lorem ipsum dolor sit amet, consectetuer adipiscing elit. Aliquam eu nisi. Nam eget dui a tortor congue imperdiet. Etiam mattis. Nam tristique. Sed malesuada neque ut leo. Aenean est. In id augue.
\end{center}

\begin{flushright}
Rechtsbündiger Text.
Aliquam ultrices libero hendrerit diam. Vestibulum ultrices sapien sit amet elit. Quisque tempor nisl eu sem. Nam lorem lectus, viverra nec, rutrum quis, lobortis nec, magna. Praesent hendrerit tortor vitae elit. Vivamus sed leo at mi elementum semper. Lorem ipsum dolor sit amet, consectetuer adipiscing elit. Aliquam eu nisi. Nam eget dui a tortor congue imperdiet. Etiam mattis. Nam tristique. Sed malesuada neque ut leo. Aenean est. In id augue.
\end{flushright}

URLs werden mit dem \verb+\url+ Befehl eingebunden: \url{http://www.kit.edu}

\subsection{Tabellen}


\section{Matheformeln in Überschriften wie \texorpdfstring{$V^*_{B}$}{zum Beispiel V\_B} können Ärger machen}

Wenn man ohne besondere Behandlung Matheformeln (und andere Spezialformatierungen) in Überschriften packt, dann knallt es, sobald pdftex versucht, das in PDF-Lesezeichen umzupacken. Abhilfe schafft der Befehl \texttt{\textbackslash texorpdfstring\{\TeX-Code\}\{PDF-Lesezeichentext\}} mit dem man einen Alternativtext angeben kann, der in PDF-Lesezeichen keine Probleme macht.

Damit Matheformeln in Überschriften möglich sind, musste ein Weg gefunden werden, die Anzahl Mathealphabete besser zu nutzen (sonst \LaTeX-Fehler \texttt{Too many math alphabets in version normal}), denn wie mit den \TeX-writes gibt es auch hier eine Beschränkung auf 16 Stück. Dies wird durch das \texttt{bm}-Paket erreicht, weil dieses einen Mechanismus besitzt, Mathe-Alphabete einzubinden, ohne diese Begrenzung zu sprengen. Nachteil dabei ist eine verlängerte Kompilierdauer (\zb bei der Datei ueuf.fd, eine Font Definitions Datei). Verwendet man keine Formeln in Überschriften, kann man die Kompilierung durch Entfernen des \texttt{bm}-Pakets beschleunigen.

\section{ToDo-Liste}

\todo{Erklärung Todo-Paket.} Mit dem ToDo-Paket kann man sich Platzhalter schaffen, damit man nicht vergisst, etwas hinzuzufügen. Das geht auch mit Bildern (siehe im Abschnitt \ref{sec:figures}). Dummerweise braucht das Paket auch eines der kostbaren 16 \TeX-writes (\texttt{Fehlermeldung: No room for a new \textbackslash write}). Da die Vorlage aber schon so viele Verzeichnisse besitzt sind die Register alle belegt und es wurde die backref-Funktion von Hyperref ausgeschaltet, was ein \TeX-write freigibt. \todo[fancyline]{Geht auch in schick.} Das kann man dann bei der endgültigen Fassung wieder einschalten und das ToDo-Paket dann ausschalten (man ist dann ja auch fertig).
\todo[inline]{Hier können noch viele Infos rein, wie man das \texttt{todonote}-Paket konfigurieren kann.}


\section{Codekommentare und anderes}

Codezeilen kann man mit dem \%-Zeichen auskommentieren, aber manchmal möchte man auch eine Kommentarumgebung verwenden. Für ein Beispiel bitte im nachfolgenden Quellcode nachschauen.
Das Comment-Paket braucht aber ein weiteres \TeX-write, mit der gleichen Problematik wie eben.
%
% Das Folgende geht nur wenn das Comment-Paket aktiv ist. Das braucht aber ein TeX-write, das man irgendwoher nehmen muss
%\begin{showcomment}
%Sichtbarer Kommentar des comment-Pakets.\\
%You should see me as \verb+\includeversion{showcomment}+ is set.\\
%Im Code kommt danach ein versteckter Kommentar.
%\end{showcomment}
%
%\begin{hidecomment}
%Hilfreich um Teile ein- und wieder auszukommentieren.
%Das hier wird völlig ignoriert. Es kann auch ungültiger \LaTeX-Code hier rein: Unbekannte Befehle \bleagraawewrg{ fehlende und falsche Klammern ] und backslashes \ & \\\
%You should not see me as \verb+\excludeversion{hidecomment}+ is not set.
%\end{hidecomment}
