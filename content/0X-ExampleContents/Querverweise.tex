%%%%%%%%%%%%%%%%%%%%%%%%%%%%%%%%%%%%%%%%%%%%%%%%%%%%%%%%%%%%
\section{Querverweise und Hyperlinks}%
\label{sec:Querverweise}
%%%%%%%%%%%%%%%%%%%%%%%%%%%%%%%%%%%%%%%%%%%%%%%%%%%%%%%%%%%%
%
Querverweise sollten nicht mit dem Befehl \verb#\ref{...}# gesetzt werden,
sondern mit \verb#\cref{...}# und verwandten Befehlen aus dem Paket
\pkg{cleveref} \cite{Cubitt2013}.
Diese Befehle haben den Vorteil nicht nur die Nummer zu referenzieren,
sondern auch den Typ mit anzugeben.
Hinzu kommt eine intelligente Verwendung der Pluralform und
\index{Sortierung}Sortierung bei Mehrfachaufzählungen auch unterschiedlichen Typs.
Will man \bspw auf zwei Abbildungen und eine Tabelle mit den Marken (\enquote{Labels})
%
\begin{itemize*}
\item \texttt{fig:subfloat-example}
\item \texttt{tab:files-dirs-of-template}
\item \texttt{fig:kit-colors}
\end{itemize*}
%
verweisen, so schreibt man einfach per Komma getrennt
%
\begin{latex}[caption={Cleveres Referenzieren mit \bs cref},label={lst:cref}]
\cref{fig:subfloat-example,
      tab:ex-sideways,
      fig:kit-colors}
\end{latex}
%
und erhält als Resultat
\enquote{\cref{fig:subfloat-example,tab:ex-sideways,fig:kit-colors}}.

\index{URL}\index{Internetadresse|see{URL}}Internetadressen werden in das Kommando \lstinline|\url{...}| eingefasst.
Außerdem besteht die Möglichkeit mit dem Befehl \lstinline|\href{<URL>}{Text}| eine Textstelle mit einem Hyperlink zu versehen.