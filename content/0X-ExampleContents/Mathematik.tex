%%%%%%%%%%%%%%%%%%%%%%%%%%%%%%%%%%%%%%%%%%%%%%%%%%%%%%%%%%%%
\section{Mathematik}%
\label{sec:Mathe}
%%%%%%%%%%%%%%%%%%%%%%%%%%%%%%%%%%%%%%%%%%%%%%%%%%%%%%%%%%%%
%
Grundsätzlich gilt, was in \parencites{ams1999a}{ams1999b} steht. In der Datei
\printfilepath{preambel/05-math.tex} sind eine Menge Kurzkommandos definiert, um eine
einheitliche Typografie von \index{Skalare}Skalaren, \index{Vektoren}Vektoren,
\index{Matrizen}Matrizen, \index{Zufallsvariablen}Zufallsvariablen etc.
zur vereinfachen. In diese Dateien einfach mal reinschauen, welche Kurzkommandos
es gibt.

Auf zwei besondere Kommandos wird näher eingegangen, weil dies häufig falsch
gemacht wird.
\begin{itemize}
  \item Für die Matrixtransponierte gibt es das Kommando \verb#\Tr#, also
	\verb#$A^{\Tr}$# liefert $A^{\Tr}$
	
	\item Bei \index{Integral}Integralen muss das \enquote{Differential-d} gemäß
	ISO in aufrechter Schrift als Operator gesetzt sein mit einem kleinen Abstand
	zum Integranden. Hierfür gibt es das spezielle Kommando \verb#\diff#. Also
	\begin{equation}
	 \int^1_0 x^2 d x = \frac{1}{3} \qquad \text{(falsche Typografie!)}
	\end{equation}
	ist falsch, während \verb#\int^1_0 x^2 \diff x = \frac{1}{3}# das Richtige
	liefert
	\begin{equation}
	 \int^1_0 x^2 \diff x = \frac{1}{3} \qquad \text{(richtige Typografie!)}
	\end{equation}
\end{itemize}

Auch sollte man darauf achten, dass zwischen einer Zahl und der physikalischen Einheit ein umbruchgeschütztes Leerzeichen verwendet wird.
Allerdings sollte statt dem üblichen (dehnbaren) umbruchgeschützen Wortzwischenraum, 
welches mit einer Tilde erzeugt wird, ein schmälerer Abstand verwendet werden.
Dieser wird mit \verb+\,+  erzeugt und beträgt standardmäßig 0,1667\,em. Dies geht so:
\begin{latex}
Der eingefügte Abstand beträgt 0,1667\,em.
\end{latex}
