%%%%%%%%%%%%%%%%%%%%%%%%%%%%%%%%%%%%%%%%%%%%%%%%%%%%%%%%%%%%
\section{Mathematik}%
\label{sec:Mathe}
%%%%%%%%%%%%%%%%%%%%%%%%%%%%%%%%%%%%%%%%%%%%%%%%%%%%%%%%%%%%
%
Grundsätzlich gilt, was in \parencites{ams1999a}{ams1999b} steht. In der Datei
\printfilepath{preambel/Math.tex} sind eine Menge Kurzkommandos definiert,
um eine einheitliche Typografie von
\index{Skalare}Skalaren, \index{Vektoren}Vektoren, \index{Matrizen}Matrizen,
\index{Zufallsvariablen}Zufallsvariablen etc. zur vereinfachen.
In diese Dateien einfach mal reinschauen, welche Kurzkommandos es gibt.

Auf zwei besondere Kommandos wird näher eingegangen,
weil dies häufig falsch gemacht wird.
\begin{itemize}
  \item Für die Matrixtransponierte gibt es das Kommando \verb#\Tr#, also
	\verb#$A^{\Tr}$# liefert $A^{\Tr}$
	
	\item Bei \index{Integral}Integralen muss das \enquote{Differential-d} gemäß
	ISO in aufrechter Schrift als Operator gesetzt sein mit einem kleinen Abstand
	zum Integranden. Hierfür gibt es das spezielle Kommando \verb#\diff#. Also
	\begin{equation}
	 \int^1_0 x^2 d x = \frac{1}{3} \qquad \text{(falsche Typografie!)}
	\end{equation}
	ist falsch, während \verb#\int^1_0 x^2 \diff x = \frac{1}{3}# das Richtige
	liefert
	\begin{equation}
	 \int^1_0 x^2 \diff x = \frac{1}{3} \qquad \text{(richtige Typografie!)}
	\end{equation}
\end{itemize}
Grundsätzlich empfiehlt es sich eigene Makros zu definieren und durchgehend zu verwenden.
Damit kann man bei Bedarf die Formatierung der Variablen an einer einzigen Stelle anpassen
ohne den gesamten Code des Arbeit durchgehen zu müssen.