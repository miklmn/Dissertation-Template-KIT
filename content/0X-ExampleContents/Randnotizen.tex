%%%%%%%%%%%%%%%%%%%%%%%%%%%%%%%%%%%%%%%%%%%%%%%%%%%%%%%%%%%%
\section{Randnotizen und TODO-Notizen}%
\index{Randnotizen}%
\index{TODO-Notizen}%
\label{sec:Randnotizen}
%%%%%%%%%%%%%%%%%%%%%%%%%%%%%%%%%%%%%%%%%%%%%%%%%%%%%%%%%%%%
%
Randnotizen 
\floatmarginnote{Ich bin eine überflüssige Randnotiz}%
werden mit dem Kommando \lc{floatmarginnote} gesetzt.
Diese eignet sich zum Beispiel um wichtige Begriffe oder Aussagen zu verdeutlichen
oder durch eine Kurzzusammenfassung jedes einzelnen Textabschnitts
den roten Faden zu verdeutlichen.
Da die Vorgaben des \glsgen{ac:KSP} für den Seitenlayout einen sehr kleinen Randbereich vorsehen, der zudem nicht bedruckt werden darf,
werden keine Randnotizen in der endgültigen Druckversion des Manuskriptes akzeptiert.
Um \ggf vorhandene Randnotizen auszublenden,
muss man in der Datei \printfilepath{preambel/AlleSchalter.tex} den Wert des Schalters
\printkeyword{showMarginNotes} auf \printkeyword{false} setzen.

TODOs im Text lassen sich mit Hilfe des Befehls \lc{todo\{<Hinweistext>\}} aus dem Paket \pkg{todonotes} setzen.
\todo{Features des Pakets todonotes beschreiben!}

Mit dem Befehl \lc{missingfigure\{<Hinweistext>\}} lässt sich auf eine fehlende Grafik hinweisen.
\missingfigure{Hier fehlt eine Grafik!}