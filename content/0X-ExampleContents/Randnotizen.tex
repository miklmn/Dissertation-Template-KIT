%%%%%%%%%%%%%%%%%%%%%%%%%%%%%%%%%%%%%%%%%%%%%%%%%%%%%%%%%%%%
\section{Randnotizen}%
\label{sec:Randnotizen}
%%%%%%%%%%%%%%%%%%%%%%%%%%%%%%%%%%%%%%%%%%%%%%%%%%%%%%%%%%%%
%
Randnotizen 
\marginnote{Ich bin eine überflüssige Randnotiz}%
werden mit dem Kommando \lc{partitle} gesetzt.
Diese eignet sich zum Beispiel um im Text Stellen zu kennzeichnen,
an denen man noch arbeiten sollte.
Da die Vorgaben des \glsgen{ac:KSP} für den Seitenlayout
einen sehr kleinen Randbereich vorsehen, der zudem nicht bedruckt werden darf,
werden keine Randnotizen in der endgültigen Version des Manuskriptes akzeptiert.
Die Randnotizen lassen sich bequem in der Hauptdatei ausschalten,
indem man \texttt{\bs showif\{showMarginNotes\}}
zu \texttt{\bs hideif\{showMarginNotes\}} ändert.