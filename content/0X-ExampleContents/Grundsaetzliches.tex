%%%%%%%%%%%%%%%%%%%%%%%%%%%%%%%%%%%%%%%%%%%%%%%%%%%%%%%%%%%%
\section[Grundsätzliches]{Grundsätzliches}%
\label{sec:Grundsätzliches}
%%%%%%%%%%%%%%%%%%%%%%%%%%%%%%%%%%%%%%%%%%%%%%%%%%%%%%%%%%%%
%
%
Beim Erstellen neuer Dateien bzw. Öffnen und Speichern bereits vorhandener Dateien ist darauf zu achten,
dass stets UTF-8 als Zeichenkodierung verwendet wird.
Dies gilt insbesondere auch für Quellen des Literaturverzeichnisses (Bib-Dateien).
Umlaute und Zeichen mit Akzent werden in den Quelldateien direkt als solche eingegeben,
also direkt mit
\texttt{Ä},
\texttt{ä},
\texttt{ß},
\texttt{é},
usw. und nicht etwa mit
\verb+"A+,
\verb+"a+,
\verb+\ss+,
\verb+'e+.
Die Zeiten, in welchen man sich bei der Eingabe deutscher Buchstaben verkünsteln musste,
sind zum Glück endgültig vorbei.

Es empfiehlt sich, die einzelnen Sätze jeweils in einer neuen Zeile anzufangen.
Ein einfaches Zeilenumbruch wird von LaTeX wie ein Leerzeichen gehandhabt
und hat somit keinen Einfluss auf die Zeilenumbrüche im Ergebnisdokument.
Beim Rückwärtsspringen aus der PDF-Datei zum Quellcode wird dadurch jedoch
eine bessere Lokalisierung der betroffenen Textstelle ermöglicht.
Weitere nützliche \LaTeX-Tipps finden sich in \cref{sec:DOsAndDONTs}.