%%%%%%%%%%%%%%%%%%%%%%%%%%%%%%%%%%%%%%%%%%%%%%%%%%%%%%%%%%%%
\section{Einige DOs und DON'Ts}%
\label{sec:DOsAndDONTs}
%%%%%%%%%%%%%%%%%%%%%%%%%%%%%%%%%%%%%%%%%%%%%%%%%%%%%%%%%%%%

In diesem Kapitel sind einige LaTeX-Dinge zusammengesammelt, die immer wieder
auch von Personen verkehrt gemacht werden, die bereits längere Zeit mit LaTeX
arbeiten. Dies soll hier also keine Einführung in LaTeX werden, sondern spiegelt
nur meine Erfahrung der häufigsten Fehler wieder. Außerdem soll gezeigt werden,
wie bestimmte Dinge innerhalb dieser Vorlage gemacht werden.

Grundsätzlich sollte man bei \index{Problem}Problemen nie, nimmer, niemals einfach nach
einer Lösung im Internet suchen. 95\,\% der Lösungen im Internet sind
bestenfalls falsch, aber eigentlich der größte Mist für den die jeweiligen
Autoren angespitzt in den Boden gerammt, im eigen Saft gegart, gevierteilt und
anschließend in Beton gegossen gehören. Aber eigentlich ist es schade um den
guten Beton.

\subsection{Dokumentationsquellen}

Statt wahllos im Internet nach \index{Losung@Lösung}Lösungen zu suchen, sucht man direkt auf
\url{http://www.ctan.org/tex-archive/} nach dem Paketnamen und verwendet die
dortige Originaldokumentation des Paketautors selbst. Dort zu findende \index{Warnung}Warnungen
sollte man ernst nehmen und nicht machen, auch wenn irgendwo anders behauptet
wird, es würde so funktionieren. Es funktioniert NICHT oder nur scheinbar.

Im Rahmen dieser Vorlage sind insbesondere die \index{Dokumentation}Dokumentationen aus dem
Literaturverzeichnis wärmstens zu empfehlen.

\subsection{Fließumgebungen (Floats)}

Fließumgebungen sind bei LaTeX blockbildende \index{Element!blockbildend}Elemente,
die nicht an der Stelle erscheinen, an der sie im Quellcode definiert sind sondern aus optischen
Gründen an umhergeschoben werden können. Typische Beispiele sind Tabellen,
Bilder, längere Codeausschnitte und ähnliche Dinge. Später wird noch auf
Bilder und Tabellen im Detail eingegangen, aber an diese Stelle sollen vier
Todsünden in Bezug auf Fließumgebungen abgehandelt werden.

Todsünde Nummer eins ist die Verwendung der \index{Platzierung}Platzierungsangabe \texttt{H}, also
bspw.
\begin{latex}[caption={Verbot von \texttt{H} als Platzierungsangabe},label={lst:prohibited-h}]
\begin{figure}[H]
\end{figure}
\end{latex}
um zu erzwingen, dass ein Fließobjekt an dieser Stelle (engl. \enquote{here})
passiert, wenn \texttt{h} nicht genügt. Wenn \texttt{h} nicht genügt, dann liegt
der Fehler bereits woanders und man sollte in das Log schauen, warum LaTeX
die Umgebung nicht platzieren kann und das originäre Problem lösen. Alles andere
macht es nur schlimmer.

Todsünde Nummer zwei ist die Verwendung von Leerzeilen innerhalb der
Fließumgebung oder auch das Abrücken der Fließumgebung mit einer Leerzeile von
dem Text der die Fließumgebung referenziert. Leerzeilen sind bei LaTeX Absätze
und damit potentielle Stellen für Seitenumbrüche. Korrekt ist also folgendes:
\begin{latex}[caption={Verbot von Leerzeilen},label={lst:prohibited-blank-lines}]
Ein Text der auf die \cref{fig:my-fig} verweist
\begin{figure}[h]%
  \centering%
  \includegraphics[width=\linewidth]{Bildpfad/Dateiname}%
  \caption[Kurzversion]{Lange Beschriftung}%
  \label{fig:my-fig}%
\end{figure}
und ohne Leerzeile an der figure-Umgebung dransteht.

Dies ist nun ein neuer Absatz.
\end{latex}
Wie man erkennen kann, steht die \texttt{figure}-Umgebung sogar mitten im Satz
zu der sie gehört. Der häufigste Grund, warum die Platzierungsoptionen
\texttt{t}, \texttt{b}, \texttt{h} und  \texttt{p} nicht so verhalten, wie
man erwartet, ist, dass die \texttt{figure}-Umgebung aus Gründen der Übersicht
mit Leerzeilen abgesetzt wird, sodass diese dann für LaTeX einen eigenen Block
bildet.

Todsünde Nummer drei ist die Verwendung von \verb#\begin{center}# und
\verb#\end{center}# statt von \verb#\centering# innerhalb der Fließumgebung.
Ersteres erzeugt wieder einen internen Absatz und damit einen eigenen Block.
Dies ist also genauso schlimm wie Leerzeilen.

Todsünde Nummer vier ist die falsche Reihenfolge von \verb#\caption{...}# und
\verb#\label{...}#. Die Reihenfolge ist \emph{immer} das Objekt selbst (also
\verb#\includegraphics# oder \verb#tabular#, usw.), dann folgt \verb#\caption{...}#
und zum Schluss \verb#\label{...}#.

\subsection{URLs}

\index{Internetadresse|see{URL}}Internetadressen werden in das Kommando \verb#\url{...}# eingefasst

\subsection{Anführungszeichen}

Um irgendwas in Anführungszeichen einzufassen, wird das Kommando 
\verb#\enquote{...}# verwendet. Dies hat den Vorteil, dass man sich nicht um die
korrekte typografische Variation der Anführungszeichen in Abhängigkeit der
\index{Sprache}Sprache kümmern muss und auch verschachtelte Anführungszeichen korrekt behandelt
werden. Also aus
\begin{latex}[caption={Behandlung von Anführungszeichen},label={lst:quotes},escapechar=\#]
 \enquote{Beim Erreichen der K#ü#ste sprach Hamlet: \enquote{Es ist etwas faul im Staate D#ä#nemark}}.
\end{latex}
wird \enquote{Beim Erreichen der Küste sprach Hamlet: \enquote{Es ist etwas faul
im Staate Dänemark}} mit korrekt verschachtelten einfachen Anführungszeichen.