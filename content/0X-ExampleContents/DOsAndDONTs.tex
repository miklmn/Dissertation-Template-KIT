%%%%%%%%%%%%%%%%%%%%%%%%%%%%%%%%%%%%%%%%%%%%%%%%%%%%%%%%%%%%
\section{Einige DOs und DON'Ts}%
\label{sec:DOsAndDONTs}
%%%%%%%%%%%%%%%%%%%%%%%%%%%%%%%%%%%%%%%%%%%%%%%%%%%%%%%%%%%%


%%%%%%%%%%%%%%%%%%%%%%%%%%%%%%%%%%%%%%%%%%%%%%%%%%%%%%%%%%%%
\subsection{URLs}
%%%%%%%%%%%%%%%%%%%%%%%%%%%%%%%%%%%%%%%%%%%%%%%%%%%%%%%%%%%%
%
\index{Internetadresse|see{URL}}Internetadressen werden in das Kommando \verb#\url{...}# eingefasst

%%%%%%%%%%%%%%%%%%%%%%%%%%%%%%%%%%%%%%%%%%%%%%%%%%%%%%%%%%%%
\subsection{Anführungszeichen}%
\index{Anführungszeichen}%
\label{sec:Anfuehrungszeichen}
%%%%%%%%%%%%%%%%%%%%%%%%%%%%%%%%%%%%%%%%%%%%%%%%%%%%%%%%%%%%
Um irgendwas in Anführungszeichen einzufassen, wird das Kommando 
\verb#\enquote{...}# verwendet. Seine Verwendung hat den Vorteil,
dass man sich nicht um die korrekte typografische Variation der
Anführungszeichen in Abhängigkeit der verwendeten
\index{Sprache!unterschiedliche Anführungszeichen}Sprache kümmern muss.
Außerdem werden so auch \enquote{verschachtelte \enquote{Anführungszeichen}}
korrekt behandelt werden.