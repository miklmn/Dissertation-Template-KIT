%%%%%%%%%%%%%%%%%%%%%%%%%%%%%%%%%%%%%%%%%%%%%%%%%%%%%%%%%%%%
\subsection{Literaturangaben}%
\label{sec:Literaturangaben}
%%%%%%%%%%%%%%%%%%%%%%%%%%%%%%%%%%%%%%%%%%%%%%%%%%%%%%%%%%%%
%
Die Erstellung eines Literaturverzeichnisses geschieht mithilfe von \gls{gls:biblatex}.
Hier soll an dieser Stelle nicht viel Erklärung dazugegeben werden, denn die
offizielle Dokumentation ist eine der besten Dokumentation, die ich überhaupt
kenne \parencite{Lehman2013}. Insbesondere seien die Abschnitte \enquote{3.7
Citation Commands} und das gesamte Kapitel \enquote{2 Database Guide} ans Herz
gelegt. Ersteres erklärt, wie im Text korrekt auf eine \index{Zitat}Zitatstelle verwiesen
wird, letzteres erklärt wie die BIB-Datei auszusehen hat.

Wer nicht die Originaldokumentation liest sondern nach irgendwelche (stets total
falschen) Lösungen im Internet sucht, weil er zu faul ist einfach nur mal lesen,
der hat mit Fug und Recht jede nur denkbare schlechte Note auf seine Arbeit
verdient.
