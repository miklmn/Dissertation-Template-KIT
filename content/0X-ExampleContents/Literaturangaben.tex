%%%%%%%%%%%%%%%%%%%%%%%%%%%%%%%%%%%%%%%%%%%%%%%%%%%%%%%%%%%%
\section{Literaturverzeichnisse}%
\label{sec:Literaturverzeichnisse}%
\index{Bibliographie}\index{Literaturverzeichnis}
%%%%%%%%%%%%%%%%%%%%%%%%%%%%%%%%%%%%%%%%%%%%%%%%%%%%%%%%%%%%
%
Defaultmäßig befinden sich alle Literaturangaben in der Datei
\printfilepath{bib/Diss.bib}
Falls ein davon abweichendes Pfad oder weitere Literaturdatenbanken
verwendet werden sollen, können entsprechende Pfade in der Datei
\printfilepath{preambel/AllePfade.tex} mit
\lstinline|\addbibresource{...}|
festgelegt werden.
Bitte beachten, dass bei den Literaturdatenbanken kein \gls{gls:bibtex}-,
sondern \gls{gls:biblatex}-Format verwendet werden soll,
also Umlaute direkt eingegeben werden können).

Die Vorlage ermöglicht es, neben dem Haupt"=Literaturverzeichnis
jeweils separat durchnummerierte Listen der eigenen Publikationen und Patente
sowie der betreuten studentischen Arbeiten zu erstellen.
Diese Auflistungen sind von den Referenzierungen im Manuskript unabhängig.
Standardmäßig werden die entsprechenden Angaben aus der Hauptliteraturdatenbank
\printfilepath{bib/Diss.bib} extrahiert.
Durch Änderung der Definitionen der Pfade
\printkeyword{bibpathOwnPatents},
\printkeyword{bibpathOwnPublications} und
\printkeyword{bibpathStudentTheses}
in der Datei
\printfilepath{preambel/AllePfade.tex}
können hierfür jedoch zusätzliche, speziell dafür angelegten Datenbanken
verwendet werden.

Die Einbindung aller Literaturverzeichnisse erfolgt in der Datei
\printfilepath{content/Inhalt-BackMatter.tex}.
Die Zusatz"=Literaturverzeichnisse werden separat von Haupt"=Literaturverzeichnis erstellt.
Dazu wird im jeweiligen Abschnitt,
welche jeweils mit Hilfe der \printkeyword{refsection}"=Umgebung abgegrenzt sind,
die kompletten Inhalte der jeweiligen Datenbank unsichtbar referenziert.
Dies geschieht mit dem \lstinline|\nocite{*}|"=Befehl.

Die Filterung der eigenen Publikationen, Patente und betreuten Arbeiten
für das jeweilige Verzeichnis geschieht jeweils mit der Angabe
\lstinline|keyword=ownpubl|,
\lstinline|keyword=ownpatent|
und
\lstinline|keyword=supervised|
in der Optionenliste des
\lstinline|printbibliography|"=Befehls.

Dazu müssen die Literatureinträge in der Literaturdatenbank freilich
entsprechend jeweils mit der Angabe
\lstinline|keywords = {ownpubl}|,
\lstinline|keywords = {ownpatent}|
oder
\lstinline|keywords = {supervised}|
versehen werden.

Bei Verwendung separater Literaturdatenbanken für die einzelnen Zusatzverzeichnisse
oder bei einer expliziten Aufzählung der einzelnen Veröffentlichungen mit dem
\lstinline|\nocite{Quelle1,Quelle2,...,QuelleN}|"=Befehl,
ist eine Filterung der Literaturdatenbank nach einem Keyword nicht notwendig.
Damit könnte die Zusatzangabe in der Literaturdatenbank entfallen.
Allerdings muss dann auch die entsprechende Filter-Option in den jeweiligen
\lstinline|printbibliography|"=Befehlen deaktiviert (sprich auskommentiert) werden.
