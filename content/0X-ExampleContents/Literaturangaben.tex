%%%%%%%%%%%%%%%%%%%%%%%%%%%%%%%%%%%%%%%%%%%%%%%%%%%%%%%%%%%%
\section{Literaturverzeichnisse}%
\label{sec:Literaturverzeichnisse}%
\index{Bibliographie}\index{Literaturverzeichnis}
%%%%%%%%%%%%%%%%%%%%%%%%%%%%%%%%%%%%%%%%%%%%%%%%%%%%%%%%%%%%
%
Defaultmäßig befinden sich alle Literaturangaben in der Datei
\printfilepath{bib/Diss.bib}
Falls ein davon abweichendes Pfad oder weitere Literaturdatenbanken
verwendet werden sollen, können entsprechende Pfade in der Datei
\printfilepath{preambel/AllePfade.tex} mit
\lstinline|\addbibresource{...}|
festgelegt werden.
Bitte beachten, dass bei den Literaturdatenbanken kein \gls{gls:bibtex}-,
sondern \gls{gls:biblatex}-Format verwendet werden soll,
also Umlaute direkt eingegeben werden können.

Die Vorlage ermöglicht es, neben dem Haupt"=Literaturverzeichnis
jeweils separat durchnummerierte Listen der eigenen Publikationen und Patente
sowie der betreuten studentischen Arbeiten zu erstellen.
Standardmäßig werden alle Literaturangaben aus der Hauptliteraturdatenbank
\printfilepath{bib/Diss.bib} extrahiert.
Durch Änderung der Definitionen der Pfade
\printkeyword{bibpathOwnPatents},
\printkeyword{bibpathOwnPublications} und
\printkeyword{bibpathStudentTheses}
in der Datei
\printfilepath{preambel/AllePfade.tex}
können jedoch für die einzelnen Auflistungen 
jeweils eigene, speziell dafür angelegten Datenbanken verwendet werden.

Die Einbindung der Literaturverzeichnisse in das Manuskript erfolgt in der Datei
\printfilepath{content/Inhalt-BackMatter.tex}.
Die Zusatz"=Literaturverzeichnisse werden separat von Haupt"=Literaturverzeichnis erstellt.
Dazu wird in den einzelnen Abschnitten,
welche jeweils mit Hilfe der \printkeyword{refsection}"=Umgebung abgegrenzt sind,
die kompletten Inhalte der jeweiligen Literaturdatenbank unsichtbar referenziert.
Dies geschieht mit dem \lstinline|\nocite{*}|"=Befehl.
Alternativ dazu können die einzelnen Werke mit
\lstinline|\nocite{Quelle1,Quelle2,...,QuelleN}|
explizit referenziert werden.

Sofern man eine einzige Datenbank für sämtliche Literaturverzeichnisse verwendet
und die unsichtbare Referenzierung jeweils mit dem \lstinline|\nocite{*}|"=Befehl erfolgt,
müssen die relevanten Referenzen für das jeweilige Zusatzverzeichnis
gefiltert werden.
Die Filterung der eigenen Publikationen, Patente und betreuten Arbeiten
für die einzelnen Verzeichnisse geschieht jeweils mit der Angabe
\lstinline|keyword=ownpubl|,
\lstinline|keyword=ownpatent|
oder
\lstinline|keyword=supervised|
in der Optionenliste des jeweiligen
\lstinline|printbibliography|"=Befehls.
Dazu müssen die Literatureinträge in der Literaturdatenbank freilich
entsprechend jeweils mit der Angabe
\lstinline|keywords = {ownpubl}|,
\lstinline|keywords = {ownpatent}|
oder
\lstinline|keywords = {supervised}|
versehen werden.

Bei Verwendung separater Literaturdatenbanken für die einzelnen Zusatzverzeichnisse
oder bei einer expliziten Aufzählung der einzelnen Veröffentlichungen mit dem
\lstinline|\nocite{Quelle1,Quelle2,...,QuelleN}|"=Befehl,
ist eine Filterung der Literaturdatenbank nach einem Keyword nicht notwendig.
Damit könnte die Notwendigkeit von Zusatzangaben in der Literaturdatenbank entfallen.
Allerdings muss dann auch die standardmäßig gesetzte Filter-Option in den jeweiligen
\lstinline|printbibliography|"=Befehlen
in der Datei  \printfilepath{content/Inhalt-BackMatter.tex}
deaktiviert (sprich auskommentiert) werden.

Die Zusatzauflistungen sind von den Zitierungen im Text unabhängig.
Damit ergibt sich sich für eigene Veröffentlichungen,
die im Text explizit zitiert werden, eine doppelte Listung --
einmal in der Hauptliteraturliste (mit einer alphanumerischen Referenz)
und einmal in der durchnummerierten Liste eigener Publikationen.
Dies ist kein Bug, sondern ein Feature der Vorlage.