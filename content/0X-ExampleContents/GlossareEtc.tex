%%%%%%%%%%%%%%%%%%%%%%%%%%%%%%%%%%%%%%%%%%%%%%%%%%%%%%%%%%%%
\section{Abkürzungsverzeichnis, Stichwortverzeichnis (Index) und Glossar}%
\label{sec:Glossare}
%%%%%%%%%%%%%%%%%%%%%%%%%%%%%%%%%%%%%%%%%%%%%%%%%%%%%%%%%%%%
%
Die Vorlage unterstützt auch ein Stichwortverzeichnis und ein Glossar. Ein
Stichwortverzeichnis (oder Index) ist einfach nur eine alphabetisch sortierte
Liste von Begriffen mit einer Auflistung der \index{Fundstelle}Fundstellen
im Dokument. Ein Glossar ist eine alphabetisch sortierte Liste von Begriffen mit
\index{Erklarung@Erklärung}Erklärung.

Der Index wird erzeugt, indem im Quellcode der Befehl \verb#\index{Begriff}#
eingefügt wird. Wichtig, der Begriff selbst wird dadurch nicht gedruckt, sondern
muss noch einmal wiederholt werden, um auch gedruckt zu werden. Dieses Verhalten
ist beabsichtigt, sodass im Index immer nur die Grundform des Wortes verwendet
wird, aber im Text natürlich die richtige Deklination. Also:
\begin{latex}[caption={Beispiel für Index},label={lst:index}]
Die meisten Funktionen dieser \index{Vorlage}Vorlage, werden durch
\index{Paket}Standardpakete bereitgestellt.
\end{latex}
Obiges Beispiel erzeugt einen Indexeintrag für \enquote{Vorlage} der auf
\enquote{Vorlage} verweist und einen Eintrag \enquote{Paket} der auf 
\enquote{Standardpaket} verweist.

Um ein Glossar zu erzeugen, müssen die Glossarbegriffe zunächst definiert werden.
Dies geschieht in der Datei \texttt{./content/00-glossary-definitions.tex}.
Es gibt zwei Haupttypen von Glossarbegriffen: Abkürzungen (Akronyme) und allg.
Einträge (z.\,B. Fachtermini).

Abkürzungen werden mit
\begin{latex}[caption={Definition von Abkürzungen},label={lst:acro}]
\newacronym[%
  shortplural={AUen},%
  longplural={Abgasuntersuchungen}%
] {au}{AU}{Abgasuntersuchung}
\end{latex}
Die drei Hauptargumente sind in dieser Reihenfolge Marke, Abkürzung und
Ausschreibung. Optionale Argumente sind die kurze und lange Pluralform.

Allgemeine Glossarbegriffe werden mit
\begin{latex}[caption={Definition von allg. Glossareinträgen},label={lst:gls}]
\longnewglossaryentry{pkg}{%
  name={Paket},%
  plural={Pakete}}%
{%
  Hier folgt eine lange Definition, die auch mehr als einen
	Absatz beinhalten darf.
}
\end{latex}
erzeugt.

Im Text werden die Einträge durch den Befehl \verb#\gls{gls:Marke}# verwendet.
Der wesentliche Unterschied zwischen einer Abkürzung und einem allg.
Glossareintrag ist, dass bei Abkürzungen bei erstmaliger Verwendung die
Abkürzung gedruckt und die Langform in Klammer dahinter gesetzt wird.
Bei allg. Glossareinträgen wird einfach nur der Name gesetzt. Statt
\verb#\gls{gls:Marke}# gibt es noch viele weitere Befehle, um im Kontext des
umgebenen Textes die korrekte Pluralform, Großschreibung am Satzanfang, etc.
zu gewährleisten. Hierfür konsultiere man das Handbuch zum Paket \texttt{glossaries}
(Pluralform, sic!) \parencite{talbot2014}.
