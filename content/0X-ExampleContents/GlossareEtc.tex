%%%%%%%%%%%%%%%%%%%%%%%%%%%%%%%%%%%%%%%%%%%%%%%%%%%%%%%%%%%%
\section{Abkürzungsverzeichnis, Stichwortverzeichnis (Index) und Glossar}%
\label{sec:Glossare}
%%%%%%%%%%%%%%%%%%%%%%%%%%%%%%%%%%%%%%%%%%%%%%%%%%%%%%%%%%%%
%
Die Vorlage unterstützt auch ein Abkürzungsverzeichnis, ein Stichwortverzeichnis,
ein Symbolverzeichnis sowie ein allgemeines \index{Glossar}\gls{gls:Glossar},
das Definitionen von \index{Fachbegriff!Definition}Fachtermini oder
\index{Uebersetzung@Übersetzung}Übersetzungen von fremdsprachlichen Begriffen
(d.h. eine Art Lexikons) enthalten kann.


%%%%%%%%%%%%%%%%%%%%%%%%%%%%%%%%%%%%%%%%%%%%%%%%%%%%%%%%%%%%
\subsection{Abkürzungen und Abkürzungsverzeichnis}%
\label{sec:Akronyme}
%%%%%%%%%%%%%%%%%%%%%%%%%%%%%%%%%%%%%%%%%%%%%%%%%%%%%%%%%%%%
Zur Erzeugung des Abkürzungsverzeichnis und des Glossar wird intern das
\pkg{glossaries}-\gls{gls:pkg} verwendet \cite{talbot2014}.
Zudem wurden einige Makros definiert, welche die Erfassung der Begriffe erleichtern sollen.

Um ein Abkürzungsverzeichnis zu erzeugen, muss zuerst eine Liste der Abkürzungen angelegt werden.
Dies geschieht in der Datei \printfilepath{./preambel/Acronyms.tex}.
Hierfür wird das Makro \lc{newacronym} verwendet.
Dieses Makro hat drei obligatorische Argumente, nämlich das
\begin{itemize*}
\item die Marke,
\item das Akronym und
\item die Langform.
\end{itemize*}

Als Konvention wird der Marke eines Akronyms ein \printkeyword{ac:} als Präfix vorangestellt.

\begin{latex}[caption={Definition einer Abkürzung},label={lst:SimpleAcronymEntry}]
\newacronym{ac:MSA}{MSA}{mein schönes Akronym}
\end{latex}

Optional können Pluralformen, sowie Genitiv-, Dativ- und Akkusativ-Formen 
der Abkürzung und des eigentlichen Begriffs angegeben werden,
sofern sie im Quelltext verwendet werden und sich von der Grundform unterscheiden.
Außerdem kann mit dem Schlüsselwort \printkeyword{description}
eine abweichende Version der Langform für die Verwendung im Abkürzungsverzeichnis definiert werden:

\begin{latex}[caption={Definition einer Abkürzung mit Zusatzangaben},label={lst:AdvancedAcronymEntry}]
\newacronym[shortgenitive={MSAs},%
            genitive={meines schönen Akronyms},%
            %shortdative={MSA},% <- entspriht der Hauptform 
            %            % ==> keine Zusatzdefinition nötig!
            dative={meinem schönen Akronym},%
            %shortaccusative={MSA},%
            %accusative={mein schönes Akronym},%
            shortplural={MSAs},
            longplural={meine schönen Akronyme},%
            %shortpluralgenitive={MSAs},%
            pluralgenitive={meiner schönen Akronyme},%
            %shortpluraldative={MSAs},%
            pluraldative={meinen schönen Akronymen},%
            %shortpluralaccusative={MSAs},%
            pluralaccusative={meine schönen Akronyme},%
            description={mein schönes Akronym, %   <- optional!
                         ein Beispiel für eine Abkürzung}%
           ]{ac:MSA}{MSA}{mein schönes Akronym}
\end{latex}

Im Text des Dokumentes werden die Einträge durch den Befehl \lc{ac\{<Marke>\}} verwendet.
Bei der erstmaliger Verwendung wird die Langform gedruckt, gefolgt von der Abkürzung, welche in Klammern gesetzt wird.
Beim zweiten Vorkommen wird nur noch die Abkürzung gedruckt.
Zusätzlich definiert die Vorlage die Befehle
\lc{acgen\{...\}}, \lc{acdat\{...\}} und \lc{acacc\{...\}}, sowie
\lc{acplgen\{...\}}, \lc{acpldat\{...\}} und \lc{acplacc\{...\}},
die jeweils die Genitiv-, Dativ- und Akkusativ-Form (singular und Plural) drucken.

Im \gls{gls:pkg} \pkg{glossaries} stellt der Befehl \lc{ac\{...\}} ein Shortcut für den Befehl \lc{gls\{...\}} dar.
Mit diesem kann ein allgemeines Glossar-Eintrag im Text referenziert werden.

Mit den Befehlen
\lc{acrshort\{...\}}, \lc{acrlong\{...\}}, \lc{acrfull\{...\}}
und ihren Abwandlungen sowie den Shortcuts
\lc{acs\{...\}}, \lc{acl\{...\}}, \lc{acf\{...\}},
kann jeweils nur die Abkürzung, nur die Langform oder beides explizit angefordert werden.
Allerdings wird eine solche Verwendung ggf. nicht als \enquote{erstmalige Verwendung} zählen.

Damit resultiert der folgende Quellcode
\begin{latex}[caption={Verwendung von Abkürzungen},label={lst:AcronymUsage}]
    \Acf{ac:MSA} ist ein Beispiel für die Verwendung einer
    Abkürzung am Anfang des Satzes. Man beachte, dass
    der Aufruf der Marke mit dem Makro \lc{acf} bzw. \lc{Acf}
    nicht als die erste Erwähnung im Text zählt.
    Bei der ersten Nennung des \acgen{ac:MSA} unter Verwendung
    der Makros \lc{ac}, \lc{acgen} \oae erscheint die Langform,
    gefolgt von der Kurzform. Bei der zweiten Nennung des
    \acgen{ac:MSA} erscheint nur noch die Kurzform.
\end{latex}
%
in der Ausgabe
%
\begin{quote}
\Acf{ac:MSA} ist ein Beispiel für die Verwendung einer
Abkürzung am Anfang des Satzes. Man beachte, dass
der Aufruf der Marke mit dem Makro \lc{acf} bzw. \lc{Acf}
nicht als die erste Erwähnung im Text zählt.
Bei der ersten Nennung des \acgen{ac:MSA} unter Verwendung
der Makros \lc{ac}, \lc{acgen} \oae erscheint die Langform,
gefolgt von der Kurzform. Bei der zweiten Nennung des
\acgen{ac:MSA} erscheint nur noch die Kurzform.
\end{quote}


%%%%%%%%%%%%%%%%%%%%%%%%%%%%%%%%%%%%%%%%%%%%%%%%%%%%%%%%%%%%
\subsection{Glossar}%
\label{sec:Glossar}
%%%%%%%%%%%%%%%%%%%%%%%%%%%%%%%%%%%%%%%%%%%%%%%%%%%%%%%%%%%%

Neben einem Abkürzungsverzeichnis kann man auch ein
\index{Glossar}\gls{gls:Glossar} erstellen lassen.
In diesem können Definitionen von \index{Fachbegriff!Definition}Fachtermini oder
\index{Uebersetzung@Übersetzung}Übersetzungen von fremdsprachlichen Begriffen stehen.

Der wesentliche Unterschied zwischen einer Abkürzung und einem allgemeinen Glossar-Eintrag ist,
dass bei Abkürzungen bei erstmaliger Verwendung die Abkürzung gedruckt und die Langform in Klammer dahinter gesetzt wird.
Bei allgemeinen Glossar-Einträgen wird normalerweise nur der Name gesetzt.
Durch eine Option des \pkg{glossaries}-\glsgen{gls:pkg} kann man sicher stellen,
dass alle Glossar-Einträge auf das Glossar am Ende des Manuskripts verlinkt werden.
Standardmäßig ist diese Option jedoch deaktiviert.

Die Glossar-Einträge werden in der Datei  \printfilepath{./preambel/Glossary.tex} definiert.

Die Definition der Glossar-Einträge geschieht mit dem Makro \lc{myglossaryentry},
welches drei obligatorische Argumente hat, nämlich
\begin{itemize*}
\item die Marke,
\item den Begriff und
\item die Erklärung / Definition / Übersetzung.
\end{itemize*}
Als Konvention wird der Marke eines Glossar-Eintrages ein \enquote{gls:} als Präfix vorangestellt.
Optional kann die Plural-, sowie Genitiv-, Dativ- und Akkusativ"=Form des Begriffs angegeben werden,
sofern sie im Quelltext verwendet werden und sich von der Grundform unterscheiden.

Ein Glossar-Eintrag kann beispielsweise folgendermaßen definiert werden:
\begin{latex}[caption={Definition eines Glossar-Eintrages},label={lst:GlossEntry}]
\myglossaryentry[plural={Glossare},%
                 genitive={Glossars}]%
                {gls:Glossare}{Glossar}{alphabetisch sortierte Liste von Begriffen mit Erklärung}
\end{latex}

Normalerweise werden im Glossar nur diejenigen Begriffe angezeigt,
die im Text des Dokumentes erwähnt und entsprechend referenziert worden sind.
Eine Referenzierung der Glossar-Einträge im Text geschieht normalerweise mit dem
\lc{gls\{<Marke>\}}-Befehl,
welcher den Begriff im Text druckt und für seine Aufnahme ins Glossar sorgt.
Weitere mögliche Befehle sind \lc{Gls\{...\}} und \lc{GLS\{...\}},
die den ersten bzw. alle Buchstaben in Großbuchstaben umwandeln,
\lc{glpl\{...\}}, \lc{Glspl\{...\}}, \lc{GLSpl\{...\}} für die Pluralform \usw.
Zusätzlich definiert die Vorlage die Befehle
\lc{glsgen\{...\}}, \lc{glsdat\{...\}} und \lc{glsacc\{...\}}, sowie
\lc{glsplgen\{...\}}, \lc{glspldat\{...\}} und \lc{glsplacc\{...\}},
die jeweils die Genitiv- Dativ- und Akkusativ-Form drucken.

Außerdem gibt es mit dem Befehl \lc{glsadd\{...\}} die Möglichkeit,
eine Stelle im Text mit einem Glossar-Begriff zu verlinken, ohne diesen explizit zu drucken.
Mit \lc{glsaddall} kann man alle definierte Glossar-Einträge ins Glossar aufnehmen,
ohne sie im Text des Dokumentes referenziert zu haben.

Die Verwendung des oben definierten Eintrages im Text mit dem Befehl
\lc{glsgen\{gls:Glossar\}} mündet im Text in einer Erwähnung des \glsgen{gls:Glossar}.


%%%%%%%%%%%%%%%%%%%%%%%%%%%%%%%%%%%%%%%%%%%%%%%%%%%%%%%%%%%%
\subsection{Stichwortverzeichnis (Index)}%
\label{sec:Index}
%%%%%%%%%%%%%%%%%%%%%%%%%%%%%%%%%%%%%%%%%%%%%%%%%%%%%%%%%%%%
%
Ein \index{Stichwortverzeichnis}Stichwortverzeichnis (oder
\index{Index|see{Stichwortverzeichnis}}Index)
ist einfach nur eine alphabetisch sortierte Liste von Begriffen mit einer Auflistung der Fundstellen im Dokument.
Diese ist nützlich, wenn sich der Leser zu einem Begriff alle Vorkommnisse anschauen möchte.
Der Index wird erzeugt, indem im Quellcode der Befehl \verb+\index{Begriff}+
eingefügt wird. Der Begriff selbst wird dadurch nicht gedruckt und muss daher
noch einmal wiederholt werden, um auch im Text gedruckt zu werden. Dieses Verhalten
ist beabsichtigt, sodass im Index immer nur die Grundform des Wortes verwendet
wird, aber im Text natürlich die richtige Deklination.


%%%%%%%%%%%%%%%%%%%%%%%%%%%%%%%%%%%%%%%%%%%%%%%%%%%%%%%%%%%%
\subsection{Symbolverzeichnis}%
\label{sec:Symbolverz}
%%%%%%%%%%%%%%%%%%%%%%%%%%%%%%%%%%%%%%%%%%%%%%%%%%%%%%%%%%%%
%
Ein Symbolverzeichnis kann auf zwei Arten angelegt werden.
Normalerweise reicht eine manuell erstellte Übersicht über die Notation,
so wie sie in der Datei \texttt{./00-Front-Matter/Notation.tex}
mit Hilfe von Befehlen
\texttt{\bs myNotationTableEntryMath\{<Mathe-Ausdruck>\}\{<Beschreibung>\}}
und
\texttt{\bs myNotationTableEntryText\{<Text-Ausdruck>\}\{<Beschreibung>\}}
definiert wird.
Diese ist recht einfach und lässt sich bei Bedarf beliebig ergänzen.

Allerdings gibt es auch die Möglichkeit zur automatischen Erzeugung eines
Symbolverzeichnisses mit Hilfe des \pkg{glossaries}-\glsgen{gls:pkg}.
Um dieses, am Ende des Manuskriptes eingebundene Symbolverzeichnis zu erzeugen,
müssen Symbole in Form von Glossar-Einträgen angelegt und im Text des Dokumentes
zumindest einmal entsprechend mit dem Befehl \lc{gls\{<Marke>\}} referenziert werden.
Dafür müsste ein Symboleintrag folgendermaßen angelegt werden:
\begin{latex}[caption={Definition eines Symboleintrages},label={lst:SymbEntry}]
\newglossaryentry{symb:pi}{%
                  name={\ensuremath{\pi}},%
                  sort={pi},%
                  type=symbols,%
                  description={Kreiszahl, Verhältnis des Umfangs eines Kreises zu seinem Durchmesser}%
                 }
\end{latex}
Die Referenzierung des Symbols \gls{symb:pi} im Text geschieht dann mit \verb+\gls{symb:pi}+.

Die Einbindung eines automatisch erzeugten Symbolverzeichnisses ist am Ende des Manuskripts vorgesehen.
Es passiert in der Datei
\texttt{./content/Inhalt-BackMatter.tex}.
Bei Bedarf lässt sich diese Einbindung auch an eine geeignete Stelle in den Frontmatter verschieben.
Dabei ist darauf zu achten, dass in der Datei \texttt{preambel/AlleSchalter.tex}
die Einblendung durch \lc{showif\{showListOfSymbols\}} aktiviert ist.
