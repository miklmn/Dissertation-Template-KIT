\begin{showBackmatter}%
%%%%%%%%%%%%%%%%%%%%%%%%%%%%%%%%%%%%%%%%
%%%%             Anhang             %%%%
%%%%%%%%%%%%%%%%%%%%%%%%%%%%%%%%%%%%%%%%
%\backmatter%
%
%
%%%%%%%%%%%%%%%%%%%%%%%%%%%%%%%%%%%%%%%%
%%%%          Bibliographie         %%%%
%%% (Achtung! Bibliographie darf im  %%%
%%%  deutschen nicht in den Anhang!) %%%
%%%%%%%%%%%%%%%%%%%%%%%%%%%%%%%%%%%%%%%%
\begin{showBibliography}%
%% mit BibLaTeX:
\printbibliography
\end{showBibliography}%
%
%
%%%%%%%%%%%%%%%%%%%%%%%%%%%%%%%%%%%%%%%%
%%        Eigene Publikationen        %%
%%%%%%%%%%%%%%%%%%%%%%%%%%%%%%%%%%%%%%%%
\begin{showOwnPublications}%
%
%% Hier eigene Publikationen auflisten (aus der bib-Datei, die in Der Datei
%% .preambel/AllePfade.tex in \bibpathOwnPublications definiert wird).
\begin{refsection}[\bibpathOwnPublications]
\newrefcontext[sorting=ynt]
%%
%% Aufruf der Quellen ohne Verweis im Text geschieht mit \nocite{...}:
%%
%%  -  entweder nur die benötigten Quellen einzeln mit \nocite{X,Y,Z} aufrufen, z.B.
%\nocite{GrinbergOhrBeyerer-ITSC2009,TeutschGrinberg-RobustDetectionInWAMI-2016}
%%
%%  -  oder alle Quellen auf einmal mit \nocite{*} aufrufen und (sofern keine
%%     eigenständige, sondern eine gemeinsame Literaturdatenbank verwendet wird)
%%     ein Keyword (z.B. "ownpubl") als Filter verwenden
\nocite{*}
%
\printbibliography[env=bibliographyNUM,%
									 keyword=ownpubl,%
									 %heading=subbibliography,%
									 %nottype=patent,%
									 title={\TransOwnPublications}]
\end{refsection}
\end{showOwnPublications}%
%
%
%%%%%%%%%%%%%%%%%%%%%%%%%%%%%%%%%%%%%%%%
%%           Eigene Patente           %%
%%%%%%%%%%%%%%%%%%%%%%%%%%%%%%%%%%%%%%%%
\begin{showOwnPatents}%
%
%% Hier eigene Patente auflisten (aus der bib-Datei, die in Der Datei
%% .preambel/AllePfade.tex in \bibpathOwnPatents definiert wird).
\begin{refsection}[\bibpathOwnPatents]
\newrefcontext[sorting=ynt]
%%
%% Aufruf der Quellen ohne Verweis im Text geschieht mit \nocite{...}:
%%
%%  -  entweder nur die benötigten Quellen einzeln mit \nocite{X,Y,Z} aufrufen, z.B.
%\nocite{PatentValeoUltraschall2016DE,PatentValeoUltraschall2016WO}
%%
%%  -  oder alle Quellen auf einmal mit \nocite{*} aufrufen und (sofern keine
%%     eigenständige, sondern eine gemeinsame Literaturdatenbank verwendet wird)
%%     ein Keyword (z.B. "ownpatent") als Filter verwenden
\nocite{*}
%%
\printbibliography[env=bibliographyNUM,%
									 keyword=ownpatent,%
									 %heading=subbibliography,%
									 type=patent,%
									 title={\TransOwnPatents}]
\end{refsection}
\end{showOwnPatents}%
%
%
%%%%%%%%%%%%%%%%%%%%%%%%%%%%%%%%%%%%%%%%
%%   Betreute studentische Arbeiten   %%
%%%%%%%%%%%%%%%%%%%%%%%%%%%%%%%%%%%%%%%%
\begin{showSupervisedTheses}%
%%
\begin{refsection}[\bibpathStudentTheses]
\newrefcontext[sorting=ynt]
%
%% Aufruf der Quellen ohne Verweis im Text:
%
%% entweder nur die benötigten Quellen einzeln mit \nocite{X,Y,Z} aufrufen
%\nocite{%
%Mustermann-2016,
%Mustermann-2017}
%
%% oder alle Quellen auf einmal mit \nocite{*} aufrufen
%% und ein Keyword (z.B. "supervised") als Filter verwenden
\nocite{*}
%%
\printbibliography[env=bibliographyNUM,%
									 keyword=supervised,%
									 %heading=subbibliography,%
									 %type=thesis,%
									 title={\TransSupervisedTheses}]
\end{refsection}
\end{showSupervisedTheses}%
%
%
%%%%%%%%%%%%%%%%%%%%%%%%%%%%%%%%%%%%%%%%
% Abbildungs- und Tabellenverzeichnis  %
%%%%%%%%%%%%%%%%%%%%%%%%%%%%%%%%%%%%%%%%
\begin{showListOfFigures}
\listoffigures
\cleardoublepage
\end{showListOfFigures}%
\begin{showListOfTables}
\listoftables
\cleardoublepage
\end{showListOfTables}%
\begin{showListOfListings}
\lstlistoflistings
\cleardoublepage
\end{showListOfListings}%
%
%
%%%%%%%%%%%%%%%%%%%%%%%%%%%%%%%%%%%%%%%%
%%%%  Alle Glossare etc. ausgeben:  %%%%
%%%%%%%%%%%%%%%%%%%%%%%%%%%%%%%%%%%%%%%%
%\begin{showGlossaries}%
%\IfDefined{printglossaries}{%
%	\printglossaries[style=mylongglossstyle]
%}
%\end{showGlossaries}%
%%%%%%%%%%%%%%%%%%%%%%%%%%%%%%%%%%%%%%%%
%
%
%
% Oder einzelne Glossartypen ausgeben:
%
%%%%%%%%%%%%%%%%%%%%%%%%%%%%%%%%%%%%%%%%
%%%%          Abkürzungen           %%%%
%%%%%%%%%%%%%%%%%%%%%%%%%%%%%%%%%%%%%%%%
\begin{showListOfAcronyms}%
\IfDefined{printglossary}{%
	\setlength{\LTleft}{0pt}% Einzug entfernen (Forderung des KSP-Verlages)
	%\setlength{\LTleft}{-5pt}% Bei Verwendung von style=long-booktabs
	%\printacronyms
	%\printacronyms[sytle=mylongglossstyle]
	\printglossary[type=\acronymtype,style=mylongglossstyle]
	%\printglossary[type=\acronymtype,style=mylongglossstyle]
}%
\cleardoublepage
\end{showListOfAcronyms}%

%%%%%%%%%%%%%%%%%%%%%%%%%%%%%%%%%%%%%%%%
%%%%            Symbole             %%%%
%%%%%%%%%%%%%%%%%%%%%%%%%%%%%%%%%%%%%%%%
\begin{showListOfSymbols}%
\IfDefined{printglossary}{%
	\setlength{\LTleft}{0pt}% Einzug entfernen (Forderung des KSP-Verlages)
	%\setlength{\LTleft}{-5pt}% Bei Verwendung von style=long-booktabs
	%printsymbols
	\printglossary[type=symbols,style=mylongglossstyle]
	%\printglossary[type=notation,style=longheader,sort=def] %,style=long-booktabs]
}%
\cleardoublepage
\end{showListOfSymbols}%
%
%%%%%%%%%%%%%%%%%%%%%%%%%%%%%%%%%%%%%%%%
%%%%      Begriffsverzeichnis       %%%%
%%%%%%%%%%%%%%%%%%%%%%%%%%%%%%%%%%%%%%%%
\begin{showNomenclature}%
\IfDefined{printnomenclature}{%
	\setlength{\LTleft}{0pt}% Einzug entfernen (Forderung des KSP-Verlages)
	%\setlength{\LTleft}{-5pt}% Bei Verwendung von style=long-booktabs
	\printnomenclature
}%
\cleardoublepage
\end{showNomenclature}%
%
%%%%%%%%%%%%%%%%%%%%%%%%%%%%%%%%%%%%%%%%
%%%%            Glossar             %%%%
%%%%%%%%%%%%%%%%%%%%%%%%%%%%%%%%%%%%%%%%
\begin{showGlossary}%
\IfDefined{printglossary}{%
	\setlength{\LTleft}{0pt}% Einzug entfernen (Forderung des KSP-Verlages)
	%\setlength{\LTleft}{-5pt}% Bei Verwendung von style=long-booktabs
	%\printglossary
	%\printglossary[style=super]
	%\printglossary[style=longheader]
	%\printglossary[style=long]
	\printglossary[style=mylongglossstyle]
}%
\cleardoublepage
\end{showGlossary}%
%
%
%
%
%
%%%%%%%%%%%%%%%%%%%%%%%%%%%%%%%%%%%%%%%%
%%%%             Anhang             %%%%
%%%%%%%%%%%%%%%%%%%%%%%%%%%%%%%%%%%%%%%%
\begin{showAppendix}
\appendix
% Platziere einen Verweis auf 'Anhang'
% ins Inhaltsverzeichnis
\phantomsection
\addcontentsline{toc}{part}{\AppendixName}
%
%%%%%%%%%%%%%%%%%%%%%%%%%%%%%%%%%%%%%
\chapter[Herleitungen]{Herleitungen}%
\label{chap:AppendixDerivations}
%%%%%%%%%%%%%%%%%%%%%%%%%%%%%%%%%%%%%
%
Hier kommen die Herleitungen rein.
%
% Reset the PDF bookmark level
\bookmarksetup{startatroot}
\cleardoublepage
\end{showAppendix}%
%
%
%
%%%%%%%%%%%%%%%%%%%%%%%%%%%%%%%%%%%%%%%%
%%%%     Stichwortverzeichnis       %%%%
%%%%%%%%%%%%%%%%%%%%%%%%%%%%%%%%%%%%%%%%
\begin{showIndex}%
% Platziere einen Verweis auf das Stichwortverzeichnis
% ins Inhaltsverzeichnis
\phantomsection
\IfDefined{printindex}{%
	\addcontentsline{toc}{chapter}{\indexname}%
	\setlength{\LTleft}{-5pt}% Einzug entfernen (Forderung des KSP-Verlages)
	%\renewcommand{\indexname}{\TransIndexName}%
	\printindex
	\cleardoublepage
}%
\end{showIndex}%
%
%
%
%%%%%%%%%%%%%%%%%%%%%%%%%%%%%%%%%%%%%%%%
%%%%         Liste der TODOs        %%%%
%%%%%%%%%%%%%%%%%%%%%%%%%%%%%%%%%%%%%%%%
\begin{showTODOs}
\phantomsection%
\listofTODOs%
\hypertarget{TODO-List}{}%
% Zum Inhaltsverzeichnis hinzufügen
\addcontentsline{toc}{chapter}{\TodoListName}%
\end{showTODOs}%
%
\end{showBackmatter}%