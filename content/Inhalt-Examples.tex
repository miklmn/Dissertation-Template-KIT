%%%%%%%%%%%%%%%%%%%%%%%%%%%%%%%%%%%%%%%%%%%%%%%%%%%%%%
%% "Magic comments" for the LaTeX editor -------------
% !TEX root = ../Diss.tex
%%%%%%%%%%%%%%%%%%%%%%%%%%%%%%%%%%%%%%%%%%%%%%%%%%%%%%
%
\begin{showExamples}%
%
%%%%%%%%%%%%%%%%%%%%%%%%%%%%%%%%%%%%%%%%%%%%%%%%%%%%%%%%%%%%
\chapter{Anleitung zur Nutzung dieser Vorlage}%
\label{chap:Examples}
%%%%%%%%%%%%%%%%%%%%%%%%%%%%%%%%%%%%%%%%%%%%%%%%%%%%%%%%%%%%
%
%
Dieses Kapitel, welches in anderen Kapiteln als \cref{chap:Examples} referenziert werden könnte, zeigt den grundlegenden Aufbau eines einfachen Kapitels.
Die einzelnen Abschnitte beschreiben die Struktur der Vorlage und geben wichtige allgemeine Tipps.
Die Verwendung einzelner Features dieser Vorlage wird in \cref{chap:Examples} detailliert beschrieben und demonstriert.

Es wird empfohlen, die einzelnen Unterkapiteln jedes Kapitels als eigene Dateien anzulegen und sie mit dem \verb+\input{}+-Befehl einzubinden (s. Quelltext).
Dies erlaubt die einzelnen Unterkapitel bei Bedarf leichter zu verschieben oder mit einem einzigen \%-Zeichen temporär auszukommentieren und erleichtert so die Fehlersuche.

Zur Nutzung dieser Vorlage für die eigene Arbeit empfiehlt es sich, die Anleitungskapiteln auszublenden und ansonsten die bestehende Struktur zu nutzen.
Die Anleitungskapiteln lassen sich ausblenden, indem man in der Datei
\printfilepath{preambel/AlleSchalter.tex}
\verb+\showif{showExamples}+ durch \verb+\hideif{showExamples}+ ersetzt.
Ähnlich lassen sich auch andere Teile des Manuskriptes ausblenden ohne sie auskommentieren zu müssen.%
%
\footnote{Die Befehle \verb+\showif{IrgendeinName}+ bzw. \verb+\hideif{IrgendeinName}+ sind spezielle Makros, die jeweils eine LaTeX"=Umgebung \printkeyword{IrgendeinName} definieren,
deren Inhalt von LaTeX{} angezeigt bzw. ausgeblendet wird.
Dies betrifft alles, was im weiteren \LaTeX-Quellcode zwischen \verb+\begin{IrgendeinName}+ und \verb+\end{IrgendeinName}+ steht.
Der Vorteil dieser Vorgehensweise gegenüber einem einfachen Auskommentieren liegt bei Einbindung von Dateien darin,
dass die zwischen \verb+\begin{IrgendeinName}+ und \verb+\end{IrgendeinName}+ eingebundenen Dateien
weiterhin im Verzeichnisbaum vom TeXnicCenter etc. sichtbar bleiben und bei Bedarf schnell als Referenz aufgerufen werden können.}
%
%%%%%%%%%%%%%%%%%%%%%%%%%%%%%%%%%%%%%%%%%%%%%%%%%%%%%%%%%%%%
\section{Kompilierung der Vorlage}%
\label{sec:Kompilierung}
%%%%%%%%%%%%%%%%%%%%%%%%%%%%%%%%%%%%%%%%%%%%%%%%%%%%%%%%%%%%
%
Um diese Vorlage nutzen zu können, benötigt man eine \LaTeX-Distribution (z.B. \printswname{MiKTeX}\footnote{\url{http://www.miktex.org/}} oder \printswname{TeXLive}\footnote{\url{https://tug.org/texlive/acquire.html}}).
Sofern man nicht die riesengroße Komplettinstallation wählt, wird beim ersten Kompilieren eine Internet"=Verbindung benötigt, um Zusatz"=\glspl{gls:pkg} dynamisch nachladen zu können.
Zur Erstellung des Glossars und des Abkürzungsverzeichnisses wird zusätzlich Perl benötigt.
Unter Windows müsste hierzu zusätzlich beispielsweise \printswname{ActivePerl}\footnote{\url{https://www.activestate.com/products/activeperl/}}
oder \printswname{StrawberryPerl}\footnote{\url{http://strawberryperl.com/}} installiert werden.
Bei den Linux-Distributionen ist Perl automatisch mit dabei.

Zur bequemen Bearbeitung der \LaTeX"=Quellcode"=Dateien empfiehlt sich Verwendung einer guten \LaTeX"=Entwicklungsumgebung in Kombination mit einem geeigneten, sprich SyncTeX"=fähigen, PDF"=Betrachter.
Manche Entwicklungsumgebungen bieten eine integrierte PDF"=Vorschau, welche Vor- und Nachteile haben kann.
Ein wichtiges Kriterium bei der Wahl der Entwicklungsumgebung ist die Möglichkeit, zwischen den einzelnen Stellen im Quellcode und im PDF hin- und her springen zu können.
Unter Windows war lange Zeit \textbf{TeXnicCenter}\footnote{\url{http://www.texniccenter.org/}} in Kombination mit \printswname{SumatraPDF}\footnote{\url{https://www.sumatrapdfreader.org}} eine gute Wahl gewesen.
Mittlerweile tendieren die meisten dazu, \printswname{TeXstudio}\footnote{\url{https://www.texstudio.org/}} zu verwenden.
Diese bietet eine eingebaute PDF"=Vorschau und viele nützliche Features und ist sowohl unter Windows als auch unter Linux verfügbar.

%%%%%%%%%%%%%%%%%%%%%%%%%%%%%%%%%%%%%%%%%%%%%%%%%%%%%%%%%%%%
\subsection{MiKTeX-Einstellungen}
\label{sec:MiKTeX}
%%%%%%%%%%%%%%%%%%%%%%%%%%%%%%%%%%%%%%%%%%%%%%%%%%%%%%%%%%%%
Sofern MiKTeX als \LaTeX"=Distribution verwendet wird, sollte man darauf achten, dass bei Bedarf Zusatzpakete vom Internet dynamisch nachgeladen werden können.
Sofern diese Option nicht bei der Installation gesetzt worden ist, kann sie nachträglich in der MiKTeX-Console aktiviert werden.
Zu finden ist diese im Startmenü unter \enquote{MiKTeX}, \enquote{MiKTeX Console (Admin)}.
Unter \enquote{Settings} findet sich ein Reiter \enquote{General}, wo im Bereich \enquote{Package installation} entweder die Option \enquote{Always install missing packages on the fly} oder \enquote{Ask me} ausgewählt werden soll.
Sofern sich der Rechner im Netzwerk des Fraunhofer IOSB befindet, muss zusätzlich noch die Proxy-Option korrekt gesetzt werden.
Hierfür muss man im Bereich \enquote{Package installation} der MiKTeX Console auf \enquote{Change} gehen und bei \enquote{Connection Settings} die Verwendung des Proxy-Servers \printkeyword{proxy-ka.iosb.fraunhofer.de} mit Port \printkeyword{80} aktivieren (s. \cref{fig:MiKTeX-Proxy}).
Wird dies nicht gemacht, können benötigte Pakete nicht nachgeladen werden.

\begin{figure}[htb]%
\centering%
\includegraphics[width=\linewidth]{images/examples/MiKTeX-Proxy.png}%
\caption{MiKTeX-Einstellungen zum Nachladen der Zusatzpakete}%
\label{fig:MiKTeX-Proxy}%
\end{figure}

Nach der MiKTeX-Installation sollte man im Startmenü gleich die MiKTeX-Console aufrufen, den Proxy eintragen und das Update durchführen um die aktuellste Version der vorinstallierten Pakete zu erhalten.
So vermeidet man beim automatischen Nachladen weiterer \glspl{gls:pkg} aus dem \acrshort{ac:CTAN}-Repository etwaige Inkompatibilitäten aufgrund veralteter Stammpakete.


%%%%%%%%%%%%%%%%%%%%%%%%%%%%%%%%%%%%%%%%%%%%%%%%%%%%%%%%%%%%
\subsection{TeXLive-Einstellungen}
\label{sec:TeXLive}
%%%%%%%%%%%%%%%%%%%%%%%%%%%%%%%%%%%%%%%%%%%%%%%%%%%%%%%%%%%%
Bei Verwendung von TeXLive unter Linux muss darauf geachtet werden, dass alle notwendigen Linux-\glspl{gls:pkg} installiert sind.
Unter Ubuntu 17.04 sollte es funktionieren, wenn folgende Linux-\glspl{gls:pkg} installiert wurden:
\begin{itemize*}
	\item \printkeyword{texlive}
	\item \printkeyword{texlive-lang-german}
	\item \printkeyword{texlive-fonts-extra}
	\item \printkeyword{texlive-bibtex-extra}
	\item \printkeyword{fonts-linuxlibertine}
	\item \printkeyword{biber}
	\item \printkeyword{xindy}
	\item \printkeyword{texmaker}
\end{itemize*}
Texmaker ist eine IDE für \LaTeX, die aber vermutlich über Dependencies schon einige Pakete mitbringt.


%%%%%%%%%%%%%%%%%%%%%%%%%%%%%%%%%%%%%%%%%%%%%%%%%%%%%%%%%%%%
\subsection{Kompilieraufrufe}
\label{sec:Aufruf}
%%%%%%%%%%%%%%%%%%%%%%%%%%%%%%%%%%%%%%%%%%%%%%%%%%%%%%%%%%%%
Zur erfolgreichen Kompilierung des Dokumentes müssen mehrere
Kommandozeilenprogramme aufgerufen werden.
Bei Verwendung einer integrierten Entwicklungsumgebung (IDE),
wird diese so konfiguriert, dass die Aufrufe aus der IDE heraus erfolgen und
die etwaigen Warnungen, Erfolgs- und Fehlermeldungen in der IDE sichtbar werden.
Die entsprechenden Einstellungen für TeXnicCenter und TeXstudio finden sich
in den nachfolgenden Kapiteln.

Die einzelnen Aufrufe sind:
\begin{itemize*}
\item \index{xelatex}\printkeyword{xelatex} zur eigentlichen Kompilierung von \LaTeX-Quelltext,
\item \index{biber}\printkeyword{biber} zur Kompilierung von Bibliografien,
\item \index{makeglossaries}\printkeyword{makeglossaries}, welches intern \index{xindy}\printkeyword{xindy} aufruft,
zur Erstellung einer Zwischenausgabe für das Abkürzungsverzeichnis und das Glossar
\index{makexindex}\printkeyword{makeindex}, welches durch die Verwendung des Pakets \pkg{imakeidx} implizit aufgerufen wird, zur Erstellung des Stichwortverzeichnisses
\end{itemize*}
Bei einem Durchlauf von \printkeyword{xelatex}, \printkeyword{biber}, \printkeyword{makeglossaries} und \printkeyword{makeindex}
werden die einzelnen Inhalte sowie die entsprechenden Querverweise
zuerst jeweils in eine oder mehrere Zwischendateien hinausgeschrieben,
die sodann wieder eingelesen und verarbeitet werden müssen.
Manche Inhalte werden daher erst jeweils beim zweiten Aufruf von \printkeyword{xelatex} generiert.
Für die korrekte Generierung eines Dokumentes mit allen Verzeichnissen
(Inhalts-, Abbildungs-, Tabellen-, Literatur, Abkürzungs-, Begriffs und Stichwortverzeichnis),
PDF"=Lesezeichen und korrekt gesetzten Querverweisen,
muss \printkeyword{xelatex} daher mehrmals aufgerufen werden.

Sollte man ausnahmsweise das Dokument doch noch aus der Kommandozeile oder von einem Script heraus aufrufen wollen,
so ist die Reihenfolge der Aufrufe wie folgt:
\begin{verbatim}
#> xelatex -synctex=1 -interaction=nonstopmode Diss.tex
#> biber Diss
#> makeglossaries Diss
#> xelatex -synctex=1 -interaction=nonstopmode Diss.tex
#> xelatex -synctex=1 -interaction=nonstopmode Diss.tex
#> xelatex -synctex=1 -interaction=nonstopmode Diss.tex
\end{verbatim}

Wenn kein \index{Kompilierfehler}Kompilierfehler aufgetreten ist,
sollte nach dem vierten Durchlauf von \printkeyword{xelatex} die
\index{Warnung!Please re-run latex}Warnung \enquote{Please re-run latex}
verschwunden sein.
%%%%%%%%%%%%%%%%%%%%%%%%%%%%%%%%%%%%%%%%%%%%%%%%%%%%%%%%%%%%
\section{Speichererweiterung}%
\index{Speichererweiterung}%
\label{sec:Speichererweiterung}
%%%%%%%%%%%%%%%%%%%%%%%%%%%%%%%%%%%%%%%%%%%%%%%%%%%%%%%%%%%%
%
Sollte bei der Kompilierung die \index{Fehler!TeX capacity exceeded}Fehler-Meldung \enquote{\texttt{TeX capacity exceeded}} kommen,
bedeutet dies, dass die von der \LaTeX-Distribution vorgesehene Arbeitsspeicherkapazität
nicht ausreicht, um die Datenmenge zu verarbeiten.
Wenn im Code kein Fehler vorliegt (z.B. eine vergessene Klammer,
die ebenfalls für eine solche Fehlermeldung sorgen kann),
kann dies auch daran liegen, dass die Verarbeitung umfangreicher TiKZ-Bilder oder Bibliographien 
Zusatzspeicher benötigen.
Um den Speicher zu erweitern, gibt es zwei Möglichkeiten.
Zum einen kann man bei jedem Aufruf von \texttt{xelatex} die Optionen zur Speichererweiterung,
z.B. \printkeyword{-main-memory=500000000}, \printkeyword{-extra-mem-top=500000000}, \printkeyword{-extra-mem-bot=500000000} und ggf. weitere setzen.
Zum anderen kann man diese Einstellungen ein für alle Male direkt bei den LaTeX-Distributionen setzen.

Unter TeXLive kann die Einstellung in der Datei
\printfilepath{/Pfad-zur-TeXLive-Installation/texmf.cnf} vorgenommen werden.
Der Aufruf erfolgt am besten, indem man im Terminal den Befehl \printkeyword{kpsewhich -a texmf.cnf} eintippt.
In der Datei kann dann beispielsweise folgendes gesetzt werden:
\begin{lstlisting}[caption={[Einstellungen zur erweiterten Speichernutzung]Einstellungen zur erweiterten Speichernutzung in der Datei \printfilepath{texmf.cnf} bei TeXLive bzw. \printfilepath{xelatex.ini} bei MiKTeX},label={lst:Speichererweiterung}]
main_memory=5000000
extra_mem_top=5000000
extra_mem_bot=5000000
pool_size=5000000
buf_size=5000000
save_size=79999
\end{lstlisting}
Der maximal mögliche Wert für \printkeyword{main\_memory} etc. ist \printkeyword{79999999}.

Unter MiKTeX unterscheidet sich die Vorgehensweise je nach dem,
ob man die Einstellung nur für den aktuellen Benutzer oder
global für alle Benutzer des Computers machen möchte.
Im zweiten Fall braucht man Admin-Rechte.

Um die Einstellungen nur für den aktuellen Benutzer zu setzen,
öffnet man die Windows-Eingabeaufforderung
(Auf den \printkeyword{Start}-Knopf gehen und \printkeyword{cmd} eintippen)
und tippt dort \printkeyword{initexmf --edit-config-file=xelatex} ein.
Um die Einstellungen nur für alle Benutzer zu setzen,
öffnet man die Windows"=Eingabeaufforderung im Administrator"=Modus
(dazu auf den Menüeintrag \printkeyword{Eingabeaufforderung} mit der rechten Maustaste anklicken und \printkeyword{Als Administrator ausführen} wählen)
und tippt dort \printkeyword{initexmf \textbf{--admin} --edit-config-file=xelatex} ein.

Dabei öffnet sich die Datei \printfilepath{xelatex.ini}, in welcher die \og Einstellungen (s. \cref{lst:Speichererweiterung}) gesetzt werden sollten.%
\footnote{Die Datei \printfilepath{xelatex.ini} mit benutzerübergreifend gültigen Einstellungen befindet sich unter
\printfilepath{C:\bs Program Files\bs MiKTeX 2.9\bs miktex\bs config}.
Die dort gesetzten Einstellungen gelten, sofern der einzelne Benutzer keine eigenen Einstellungen definiert hat.
Die benutzerspezifische Datei, deren Einstellungen die globalen Einstellungen überrufen können, befindet sich unter
\printfilepath{C:\bs Users\bs <Benutzername>\bs AppData\bs Roaming\bs MiKTeX\bs 2.9\bs miktex\bs config}.}

Nach Speicherung der Datei ist bei der benutzerspezifischen Anpassung in der Eingabeaufforderung der Befehl
\printkeyword{initexmf --dump=xelatex} aufzurufen.
Im Falle einer benutzerübergreifenden Anpassung ist der Befehl
\printkeyword{initexmf \textbf{--admin} --dump=xelatex}
aufzurufen.
%%%%%%%%%%%%%%%%%%%%%%%%%%%%%%%%%%%%%%%%%%%%%%%%%%%%%%%%%%%%
\section{Aufbau der Vorlage}%
\label{sec:AufbauDerVorlage}
%%%%%%%%%%%%%%%%%%%%%%%%%%%%%%%%%%%%%%%%%%%%%%%%%%%%%%%%%%%%
%
Die Vorlage besteht aus mehreren Dateien und Verzeichnissen.
Ihre Bedeutung ist in \cref{tab:StrukturDerVorlage} zusammengefasst.

{%%% Übersicht über die Dateien und Verzechnisse der Vorlage %%%
% kleinere Schrift
\footnotesize%
% Abstand zwischen den Zeilen auf 1,5faches setzen
\renewcommand{\arraystretch}{1.5}%
% Seitliche Abstände links und rechts eliminieren
%\setlength\LTleft{0pt}%
%\setlength\LTright{0pt}%
\setlength{\tabcolsep}{0pt}%
% Tabelle aus einer Extra-Datei einlesen und auf die Seitenbreite strecken
\LTXtable{\columnwidth}{tables/DateienUndVerzeichnisse.tex}%
}%
%
Bei der Datei \printfilepath{Diss.tcp} handelt es sich um die Projektdatei für den \LaTeX-Editor TeXnicCenter.
In ihr werden die projektbezogenen Einstellungen des TeXnicCenter festgehalten.
Das sind u.a. Angaben zur Hauptdatei des Projektes und zur Projektsprache.
Die korrekte Angabe der Projektsprache ist insofern wichtig, als dass diese in TeXnicCenter ab Version 2.0 Beta 1 zur Bestimmung der Sprache für die Rechtschreibprüfung verwendet wird.
Die entsprechenden Einstellungen können im TeXnicCenter über den Menüeintrag \printkeyword{Projekt} $\rightarrow$ \printkeyword{Eigenschaften} vorgenommen werden.


Die Hauptdatei ist die Datei \printfilepath{Diss.tex}.
Sie ist verhältnismäßig kurz, da die Hauptinhalte in andere Dateien ausgelagert sind, welche mit Hilfe des \lc{input\{\}} \bzw des \lc{include\{\}}-Befehls eingebunden werden.
Die Hauptdatei besteht im Wesentlichen aus drei Abschnitten.
Im ersten stehen die sogenannten \enquote{Magic comments}, mit deren Hilfe manche \LaTeX-IDEs sich selbst vorkonfigurieren können.
Sie fangen mit \enquote{\texttt{\%~!TeX}} an und geben an, welche Kodierung für die Dateien verwendet wird und welche Programme für die Kompilierung des Quelltextes und der Bibliografie verwendet werden sollen.
Außerdem kann hier angegeben werden, welche Sprache für die Rechtschreibprüfung innerhalb der IDE verwendet werden soll.
Aktuell sind sie durch ein weiteres Prozentzeichen deaktiviert, da es keine dokumentierte Möglichkeit gibt, Aufruf für Glossar-Erzeugung miteinzubinden.

Im zweiten Abschnitt wird die Header-Datei eingebunden.
In dieser wird die verwendeten Dokumentklasse (inklusive Papierformat und Schriftgröße) definiert, sowie weitere Dateien eingebunden,
in welchen die zu landenden Pakete, Layout"=Parameter sowie alle weiteren Einstellungen und Makros definiert und konfiguriert werden.
Im dritten Teil werden nun die einzelnen Inhalte der Arbeit eingebunden.

Das entstehende PDF heißt genauso wie die Hauptdatei.

Die einzelnen \index{Kapitel}Kapiteln der Arbeit werden im Verzeichnis \printfilepath{./content/} als separate Dateien gespeichert.
Es empfiehlt sich als Dateiname das Schema \printfilepath{nn-name.tex} zu verwenden, wobei \printkeyword{nn} die Nummer des Kapitels ist,
sodass die Dateien in der semantisch richtigen Reihenfolge sortiert angezeigt werden.
Die einzelnen Dateien werden per \verb+include{}+-Direktive in der Datei \printfilepath{Diss.tex} eingebunden.
Theoretisch wäre es an dieser Stelle auch möglich mit \verb+\input{}+ zu arbeiten, was jedoch seine Nachteile hätte.
Der Unterschied zwischen den beiden Befehlen wird \href{https://texwelt.de/wissen/fragen/32/was-ist-der-unterschied-zwischen-include-and-input}{auf texwelt.de} erklärt:

\begin{quote}
{\small
\verb+\input{file}+ lädt die Datei an Ort und Stelle in die Ziel-Datei und ist äquivalent
als ob man den Text in \printkeyword{file} direkt in die Ziel-Datei geschrieben hätte.
\verb+\input+ kann letztlich überall für jede Art Datei verwendet werden und kann auch verschachtelt angewendet werden,
d.h. eine eingebundene Datei kann ihrerseits Dateien mit \verb+\input+ einbinden.

\verb+\include{file}+ hingegen führt zunächst einmal ein \verb+\clearpage+ aus bevor es \verb+\input{file}+ ausführt.
Im Gegensatz zu \verb+\input+ kann eine Datei, die mit \verb+\include+ eingebunden wird,
kein weiteres \verb+\include+ enthalten, es ist also keine verschachtelte Anwendung möglich.
Eine mit \verb+\include+ eingebundene Datei kann aber natürlich \verb+\input+ enthalten.
\verb+\include+ erzeugt eine neue \printfilepath{aux}-Datei für die eingebundene Datei.
Das erlaubt es beispielsweise, ein Dokument in mehrere logische Einheiten zu zerlegen (etwa einzelne Kapitel),
die jede einer Datei entsprechen, die mit \verb+\include+ in die Hauptdatei eingebunden wird.
\verb+\includeonly{file1,file3}+ würde dann erlauben, nur gerade bearbeitete Dateien für die Kompilation einzubinden
und durch die separaten \printfilepath{aux}-Dateien dennoch korrekte Seitenzahlen und Querverweise zu erhalten.
Es gibt auch das \pkg{excludeonly} Paket, dessen Befehl \verb+\excludeonly+ das gegensätzliche Verhalten bietet.%
}%
\footnote{\url{https://texwelt.de/wissen/fragen/32/was-ist-der-unterschied-zwischen-include-and-input}}
\end{quote}

Zur besseren Übersicht und zur Vereinfachung der Fehlersuche wird empfohlen,
die einzelnen Unterkapitel ebenfalls als separate Dateien in Unterverzeichnissen von \printfilepath{./content/} anzulegen
und sie mit den \verb+\input{}+-Direktiven in die jeweiligen Kapitel-Dateien einzubinden.

Es wird davon ausgegangen, dass sich sämtliche Bibliografie-Angaben in der Datei
\printfilepath{./bib/Diss.bib} befinden.
Sollten mehrere Bibliografie-Dateien verwendet werden, können diese in der Datei
\printfilepath{./preamble/AllePfade.tex} gesetzt werden.

Bilder bzw. \index{Zeichnung|see{Bild}}Zeichnungen werden auf zwei Arten eingebunden.
Bilder im \index{Bild!Binär-}Binärformat (PNG, JPEG, TIFF, PDF, etc.)
werden mit \lc{includegraphics}-Befehl eingebunden. 
Bei den \index{Bild!TikZ}\gls{gls:tikz}-Zeichnungen handelt es sich um reguläre TeX-Quellcode-Dateien,
die mit dem \verb+\input+-Befehl eingebunden werden.
Für eine einfache Verwaltung wird empfohlen, Binärbilder im Verzeichnis \printfilepath{./images/} abzulegen.
Zusätzliche Pfade können in der Datei \printfilepath{./preambel/AllePfade.tex} definiert werden.
Die \gls{gls:tikz}-Quellcode-Dateien sollten im Verzeichnis \printfilepath{./figures-src/} abgelegt werden.
Während des Kompilierens werden für jede \gls{gls:tikz}-Zeichnung im Verzeichnis \printfilepath{./figures-compiled/} mehrere Dateien erzeugt.
Der Inhalt dieses Verzeichnisses kann gefahrlos gelöscht werden.
Weitere Hinweise und Beispiele zur Einbindung von Grafiken finden sich in \cref{sec:Bilder}.

Die wichtigsten Einstellungen, die auf jeden Fall geändert werden müssen,
finden sich in der Datei \printfilepath{./preambel/AlleAngaben.tex}.
Hier werden \ua Angaben zum Verfasser, Art und Titel der Arbeit sowie zu den Gutachtern gemacht.

Eine weitere wichtige Datei ist \printfilepath{./preambel/AlleSchalter.tex}.
Darin wird u.a. die Hauptsprache \index{Sprache!Hauptsprache}Hauptsprache des Manuskriptes gesetzt, was sich an mehreren Stellen auswirkt.
So wird beispielsweise bei \index{Sprache!Umstellung}Umstellung auf Englisch als Hauptsprache
\enquote{Danksagung} durch \enquote{Acknowledgments},
\enquote{Inhaltsverzeichnis} durch \enquote{Contents}
\usw ersetzt.
Auch die Regeln der \index{Silbentrennung!Allgemeine Einstellung}Silbentrennung werden entsprechend angepasst.
Durch das Umsetzen der Schalter \lc{hideif} \bzw. \lc{showif} können hier Teile der Arbeit aus- und wieder eingeblendet werden, ohne dass sie auskommentiert werden müssen.
Neben diesen Einstellungen befinden sich in der Datei \printfilepath{./preambel/AlleSchalter.tex} auch Einstellungen,
die für die Vorbereitung des Manuskriptes zum Druck beim \gls{ac:KSP} wichtig sind.
Insbesondere können hier
\begin{itemize*}
\item mit der Einstellung \printkeyword{showFrame}
die Satzspiegel-Ränder angezeigt werden und so kontrolliert werden,
ob nichts hinausragt und ob Tabellen und Bilder die komplett verfügbare Breite ausfüllen,
\item mit den Einstellungen \printkeyword{coloredlistings} und \printkeyword{coloredlinks}
dafür gesorgt werden, dass alle Quellcode-Listings sowie Querverweise und URL-Adressen,
die normalerweise farbig sind, für die Druckversion nicht farbig gesetzt werden
und so die Anzahl der farbig zu druckenden Seiten reduziert wird,
\item mit der Einstellung \printkeyword{useCMYKcolors} eine Farbkonvertierung
aller Farben in den CMYK-Farbraum für den Offset-Druck vorgenommen werden.
Diese Einstellung sollte nur für die Druckversion vorgenommen werden.%
\footnote{Da manche RGB-Farben bei der Konvertierung in den CMYK-Farbraum blass aussehen,
ist es sinnvoll, eine Alternativversion der betroffenen Farben im CYMK-Farbraum zu definieren,
die gut aussieht.
Beispiele dafür gibt es in der Datei \printfilepath{./preambel/ColorSettings.tex}.
Vorzugsweise sollten aber die KIT-Corporate-Identity-Farben verwendet werden,
welche in der Datei \printfilepath{KAcolors.sty} definiert sind.
Für diese Farben wurde sowohl eine RGB- als auch eine CMYK-Definition erstellt.}
\end{itemize*}

Regeln zur \index{Silbentrennung}Silbentrennung unbekannter Wörter (\zB Fachbegriffe)
können in der Datei \printfilepath{./preambel/Hyphenation.tex} festgelegt werden.
Zu beachten ist, dass zusammengesetzte Wörter mit einem Bindestrich
ausschließlich am Bindestrich getrennt werden,
wogegen auch ein Eintrag in die Datei \printfilepath{Hyphenation.tex} nicht hilft.
Um Silbentrennung an anderen Stellen eines zusammengesetzten Wortes zu erlauben,
muss man den Bindestrich durch „\verb+"=+“ ersetzen. Dies gilt jedoch nur für deutschsprachige Texte.%
\footnote{\url{https://de.wikibooks.org/wiki/LaTeX-W%C3%B6rterbuch:_Silbentrennung}}

In den Dateien
\printfilepath{./preambel/Acronyms.tex},
\printfilepath{./preambel/Glossary.tex} und
\printfilepath{./preambel/GlossarySymbols.tex}
kann eine Liste der Abkürzungen, Fachbegriff-Definitionen und Symbole angelegt werden.
Details hierzu finden sich im \cref{sec:Glossare}.

%%%%%%%%%%%%%%%%%%%%%%%%%%%%%%%%%%%%%%%%%%%%%%%%%%%%%%%%%%%%
\section[Grundsätzliches]{Grundsätzliches}%
\label{sec:Grundsätzliches}
%%%%%%%%%%%%%%%%%%%%%%%%%%%%%%%%%%%%%%%%%%%%%%%%%%%%%%%%%%%%
%
%
Beim Erstellen neuer Dateien bzw. Öffnen und Speichern bereits vorhandener Dateien ist darauf zu achten,
dass stets UTF-8 als Zeichenkodierung verwendet wird.
Dies gilt insbesondere auch für Quellen des Literaturverzeichnisses (Bib-Dateien).
Umlaute und Zeichen mit Akzent werden in den Quelldateien direkt als solche eingegeben,
also direkt mit
\texttt{Ä},
\texttt{ä},
\texttt{ß},
\texttt{é},
usw. und nicht etwa mit
\verb+"A+,
\verb+"a+,
\verb+\ss+,
\verb+'e+.
Die Zeiten, in welchen man sich bei der Eingabe deutscher Buchstaben verkünsteln musste,
sind zum Glück endgültig vorbei.

Es empfiehlt sich, die einzelnen Sätze jeweils in einer neuen Zeile anzufangen.
Ein einfaches Zeilenumbruch wird von LaTeX wie ein Leerzeichen gehandhabt
und hat somit keinen Einfluss auf die Zeilenumbrüche im Ergebnisdokument.
Beim Rückwärtsspringen aus der PDF-Datei zum Quellcode wird dadurch jedoch
eine bessere Lokalisierung der betroffenen Textstelle ermöglicht.
Weitere nützliche \LaTeX-Tipps finden sich in \cref{sec:DOsAndDONTs}.
%%%%%%%%%%%%%%%%%%%%%%%%%%%%%%%%%%%%%%%%%%%%%%%%%%%%%%%%%%%%
\section{Sprachumschaltung (Deutsch, Englisch, etc.)}%
\label{sec:Sprache}
%%%%%%%%%%%%%%%%%%%%%%%%%%%%%%%%%%%%%%%%%%%%%%%%%%%%%%%%%%%%
%
Um für einen Teil des Textes die \index{Sprache}Sprache zu wechseln, damit die
\index{Silbentrennung}Silbentrennung und die Auswahl der Anführungszeichen
korrekt funktioniert, gibt es zwei Kommandos. Für einen kürzeren Text gibt es
\verb#\foreignlanguage[Sprache]{...}#. Dann wird für den Text in den
geschweiften Klammern die angegebene Sprache verwendet. Um die Sprache bis zum
nächsten Aufruf des gleichen Kommandos dauerhaft umstellen, gibt es
\verb#\selectlanguage[Sprache]#. Gegenwärtig unterstütze Sprachen sind
\texttt{ngerman} für Deutsch nach neuer
\index{Rechtschreibung!neue deutsche}Rechtschreibung und \texttt{american} für
Englisch nach amerikanischer \index{Rechtschreibung!amerikanisch}Rechtschreibung.
%%%%%%%%%%%%%%%%%%%%%%%%%%%%%%%%%%%%%%%%%%%%%%%%%%%%%%%%%%%%
\section[% Kurzversion für das Inhaltsverz., Kolumnentitel und PDF-Lesezeichen:
         Wichtiges zu Umbrüchen bei Überschriften
         \mbox{(Kurzversion für das Inhaltsverzeichnis etc.)}%
				]{% Langversion, die im Text gedruckt wird:
         Wichtiges zu Umbrüchen bei Überschriften
         (und ein Beispiel \newline für eine lange Überschrift,
         welche \newline für das Inhaltsverzeichnis und 
         \newline die Kolumnentiteln zu lang ist).%
				}%
\index{Zeilenumbruch}\index{Titel}\index{Überschrift}\index{PDF!Lesezeichen}%
\label{chap:Titles}
%%%%%%%%%%%%%%%%%%%%%%%%%%%%%%%%%%%%%%%%%%%%%%%%%%%%%%%%%%%%
%
%
Bei den Kapitelüberschriften kann man zwei Versionen definieren:
eine lange Überschrift in geschweiften Klammern, welche in der Arbeit selbst angezeigt wird, 
und optional eine Kurzversion in eckigen Klammern, welche im Inhaltsverzeichnis und in den 
\href{https://de.wikipedia.org/wiki/Kolumnentitel}{Kolumnentiteln}%
\footnote{Kolumnentitel sind Überschriften der einzelnen Seiten. Meist stehen sie in der Kopfzeile.}
angezeigt wird:
\begin{latex}
\section[Kurzversion]{Langer Titel}
\end{latex}
Dasselbe gilt für Bild- und Tabellenunterschriften. Hier kann man dem
\lc{caption}-Befehl ebenfalls einen optionalen Parameter übergeben.

Manchmal sind dem \glsdat{ac:KSP} die von \LaTeX{} automatisch eingefügten Zeilenumbrüche in den Kapitelüberschriften im Inhaltsverzeichnis nicht \enquote{schön} genug.
Ein manuelles Einfügen der Zeilenumbrüche etwa mit \verb+\newline+ in der Kurzversion des Titels funktioniert leider nicht,
da diese dann nicht nur im Inhaltsverzeichnis, sondern auch in den Kolumnentiteln und PDF"=Lesezeichen zur Geltung kommen, 
was normalerweise nicht erwünscht ist.

Abhilfe schafft der folgende Trick:
man schließt den letzten, umzubrechenden Teil der Kurzversion des Titels in eine \verb+\mbox{}+.
Der Text, der in eine \verb+\mbox{}+ eingeschlossen wird, darf nicht umbrochen werden.
Im Kolumnentiteln und in den PDF"=Lesezeichen hat dies keine besondere Wirkung; im Inhaltsverzeichnis führt dies jedoch dazu, dass \LaTeX{} den Zeilenumbruch vor der \verb+\mbox{}+ einfügt.
Dasselbe gilt für die ungünstig umbrochene Wörter (so will 
Ein entsprechendes Beispiel stellt die Überschrift dieses Abschnitts dar.
%%%%%%%%%%%%%%%%%%%%%%%%%%%%%%%%%%%%%%%%%%%%%%%%%%%%%%%%%%%%
\section[Bilder, Grafiken und Diagramme]{Bilder, Grafiken und Diagramme}
\label{sec:Bilder}
%%%%%%%%%%%%%%%%%%%%%%%%%%%%%%%%%%%%%%%%%%%%%%%%%%%%%%%%%%%%
%
Bei Einbindung von Grafiken sind zwei Fälle zu unterscheiden:
\begin{itemize*}
  \item reguläre \index{Bild!Binär-}Bilder in einem Binärformat
	      (\texttt{PNG}, \texttt{TIFF}, \texttt{JPG}, \texttt{PDF}, etc.)
	\item \index{Bild!Vektor-}Grafiken, die im \gls{gls:tikz}-Quellcode vorliegen 
\end{itemize*}

Grundlegender Unterschied bei der Einbindung \enquote{regulärer} Bilder
und \gls{gls:tikz}-Bilder ist, dass Binärformatgrafiken mit \lc{includegraphics}
eingebunden werden, während \gls{gls:tikz}-Grafiken mit \lc{input} eingebunden
und von \texttt{latex} mitkompiliert werden.


%%%%%%%%%%%%%%%%%%%%%%%%%%%%%%%%%%%%%%%%%%%%%%%%%%%%%%%%%%%%
\subsection[Floats]{\index{Float}\index{Bild!Float}Floats}%
\label{sec:Floats}
%%%%%%%%%%%%%%%%%%%%%%%%%%%%%%%%%%%%%%%%%%%%%%%%%%%%%%%%%%%%
%
Üblicherweise werden Bilder und Tabellen in Fließumgebungen (floats) gesetzt,
damit \LaTeX\ sie geschickt positionieren kann.
Bei Bildern heißt die entsprechende Float-Umgebung \printkeyword{figure}.
Die Positionierung kann durch Angabe von Buchstaben
\printkeyword{h}, \printkeyword{t}, \printkeyword{b} und \printkeyword{p}
beeinflusst werden, die Auswirkung ist allerdings nicht immer intuitiv.
Um diese zu verstehen, empfiehlt sich die Lektüre der Beschreibung
von Frank Mittelbach
\href{https://tex.stackexchange.com/questions/39017/how-to-influence-the-position-of-float-environments-like-figure-and-table-in-lat/39020#39020}{auf stackexchange.com}.%
\footnote{\url{https://tex.stackexchange.com/questions/39017/how-to-influence-the-position-of-float-environments-like-figure-and-table-in-lat/39020#39020}}

\myexcl{Wichtig!}
Seitens des \glsgen{ac:KSP} wird bezüglich Einbindung von Floats gefordert,
dass diese die einzelnen Sätze nicht zerreißen.
Dies bedeutet, dass eine Platzierung am Anfang oder am Ende einer Seite
unerwünscht ist, wenn dadurch ein Satz aufgeteilt wird.
Somit bleibt eigentlich nur noch die Verwendung der \printkeyword{h}-Option,
die das Bild allerdings nicht immer an der gewünschten Stelle einfügen kann.
Manchmal wird das Bild dadurch ans Ende des Kapitels
(genauer gesagt, an die Stelle, wo die nächste \lc{clearpage}-Anweisung kommt)
verschoben.
Und auch wenn die Platzierung mit der \printkeyword{h}-Option klappt,
können unerwünschte Effekte auftreten.
So können beispielsweise auf der vorherigen Seite riesige leere Flächen entstehen.
Daher empfiehlt sich eine endgültige Platzierung der Bilder erst ganz am Schluss,
nachdem alle anderen Korrekturen durchgeführt sind.
Ggf. müssen die Bilderdefinitionen manuell im Quellcode herumgeschoben werden,
bis sie von \LaTeX\ optimal gesetzt werden.
Dafür empfiehlt es sich, die einzelnen Bilddefinitionen in Extra-Dateien auszulagern.

%%%%%%%%%%%%%%%%%%%%%%%%%%%%%%%%%%%%%%%%%%%%%%%%%%%%%%%%%%%%
\subsection[Binärbilder]{\index{Bild!Binär-}Binärbilder}%
\label{sec:Binaerbilder}
%%%%%%%%%%%%%%%%%%%%%%%%%%%%%%%%%%%%%%%%%%%%%%%%%%%%%%%%%%%%
%
Ein Beispiel für die Einbindung eines Bildes im Binärformat ist in \cref{lst:binary-image} angeführt:

\begin{latex}[caption={Einbindung einer Binärgrafik in LaTeX},label={lst:binary-image}]
\begin{figure}[h]%
  \centering%
  \includegraphics[width=\linewidth]{Bildpfad/Dateiname}%
  \caption[Kurzversion für das Abbildungsverzeichnis]{%
           Eine tolle sehr lange Abbildungsunterschrift}%
  \label{fig:my-binary-image}%
\end{figure}
\end{latex}

Die Angabe des Pfades kann sowohl absolut als auch relativ
zum Verzeichnis der Hauptdatei oder zu einem der Pfade angegeben werden,
die in der Datei \printfilepath{./preambel/AllePfade.tex} definiert sind.
Diese Pfade werden in angegebenen Reihenfolge durchsucht.
Dasselbe gilt für die Dateierweiterung.
Ist keine Erweiterung definiert und liegen mehrere Bilder mit gleichem Namen jedoch unterschiedlicher Dateierweiterung vor,
wird die Reihenfolge, die in der Datei texttt{AllePfade.tex} definiert ist, verwendet.

Zu beachten ist dabei, dass der \gls{ac:KSP} Skalierung der Bildern auf die Seitenbreite fordert,
was hier durch die Option \printkeyword{width=\bs linewidth} verwirklicht wurde.

Wichtig anzumerken ist, dass alle Zeilen innerhalb der \printkeyword{Figure}"=Umgebung
mit einem Prozentzeichen abzuschließen sind.
Ansonsten werden überflüssige Leerzeichen eingefügt,
was zu unerwünschten Nebenwirkungen führen kann.

Mit Hilfe von Paket \pkg{subfig} \cite{Cochran2005} können Bilder auch in
\index{Abbildung|see{Bild}}\index{Bild!Unterabbildung}Unterabbildungen gesetzt
und sowohl als ganzes (vgl. \cref{fig:subfloat-example}) als auch einzeln (vgl.
\cref{fig:subfloat-example-01,fig:subfloat-example-02,fig:subfloat-example-03,fig:subfloat-example-04})
referenziert werden.

\begin{figure}[h]%
	\centering%
	\subfloat[La Savoureuse, Lepuix, Frankreich (\copyright\ Thomas Bresson)]{%
		\label{fig:subfloat-example-01}%
		\includegraphics[width=0.49\linewidth]{./images/examples/subfloat-example-01.jpg}%
	}%
	\hfill%
	\subfloat[Bangkok, Thailand (\copyright\ Prachanart Viriyaraks)]{%
		\label{fig:subfloat-example-02}%
		\includegraphics[width=0.49\linewidth]{./images/examples/subfloat-example-02.jpg}%
	}%
	\\%
	\subfloat[Wahkeena Falls, Lincoln Park, USA (\copyright\ srslyguys)]{%
		\label{fig:subfloat-example-03}%
		\includegraphics[width=0.49\linewidth]{./images/examples/subfloat-example-03.jpg}%
	}%
	\hfill%
	\subfloat[Nacionalni park Plitvička jezer, Kroatien (\copyright\ Roman Bonnefoy)]{%
		\label{fig:subfloat-example-04}%
		\includegraphics[width=0.49\linewidth]{./images/examples/subfloat-example-04.jpg}%
	}%
	\caption[Bild mit Unterabbildungen]{Wasserfälle der Welt als Beispiel für Unterabbildungen}%
	\label{fig:subfloat-example}%
\end{figure}

Der Beispielcode dafür ist in \cref{lst:subfigures} dargestellt.

\begin{latex}[caption={Unterabbildungen in LaTeX},label={lst:subfigures}]
\begin{figure}[h]%
	\centering%
	\subfloat[Unterbezeichnung 1)]{%
		\label{fig:UnterAbb1}%
		\includegraphics[width=0.49\linewidth]{Bildpfad/Bild1}%
	}%
	\hfill%
	\subfloat[Unterbezeichnung 2]{%
		\label{fig:UnterAbb2}%
		\includegraphics[width=0.49\linewidth]{Bildpfad/Bild2}%
	}%
	\\%
	\subfloat[Unterbezeichnung 3)]{%
		\label{fig:UnterAbb3}%
		\includegraphics[width=0.49\linewidth]{Bildpfad/Bild3}%
	}%
	\hfill%
	\subfloat[Unterbezeichnung 4]{%
		\label{UnterAbb4}%
		\includegraphics[width=0.49\linewidth]{Bildpfad/Bild4}%
	}%
\caption[Kurzversion]{Langversion der Bildunterschrift}%
\label{fig:MeinGanzesBild}%
\end{figure}
\end{latex}

Man beachte die abschließenden Prozent-Zeichen am Ende jeder Zeile!

%%%%%%%%%%%%%%%%%%%%%%%%%%%%%%%%%%%%%%%%%%%%%%%%%%%%%%%%%%%%
\subsection[TikZ-Grafiken]{\index{TikZ}\index{Bild!TikZ}\gls{gls:tikz}-Grafiken}%
\label{sec:TikZ}
%%%%%%%%%%%%%%%%%%%%%%%%%%%%%%%%%%%%%%%%%%%%%%%%%%%%%%%%%%%%
%
\pkg{\Gls{gls:tikz}} eignet sich hervorragend, um wissenschaftliche Zeichnungen,
Vektorgrafiken und \index{Diagramm}Diagramme direkt mithilfe von LaTeX
zu setzen, sodass die Schrift direkt zum restlichen Dokument passt.
Zu \gls{gls:tikz} und dem darauf aufsetzenden \pkg{\gls{gls:pgfplots}} gibt
es hervorragende Dokumentation \parencites{Tantau2013}{Feuersaenger2014}.
Mit TikZ lassen sich viele gute Sachen machen.


Der Code für die Einbindung einer \gls{gls:tikz}-Grafik sieht folgendermaßen aus:
\begin{latex}[caption={Einbindung einer TikZ-Zeichnung in LaTeX},label={lst:tikz-figure}]
\begin{figure}[h]%
  \centering%
  \tikzsetnextfilename{TikZ-Bild}%
  \resizebox{\textwidth}{!}{%   <--- optionale Skalierung
    \input{./figures-src/TikZ-Bild.tex}%
  }%                            <--- optionale Skalierung
  \caption[Kurzversion für das Abbildungsverzeichnis]{%
           Eine tolle sehr lange Abbildungsunterschrift}%
  \label{fig:my-tikz-figure}%
\end{figure}
\end{latex}

Eine Skalierung auf die volle Seitenbreite oder ein vielfaches davon im Falle von Unterabbildungen kann bei Bedarf mit Hilfe der Anweisung
\texttt{\bs resizebox\{\bs textwidth\}\{!\}\{...\}}
durchgeführt werden.

Das Kommando \verb+\tikzsetnextfilename{...}+ ist nicht unbedingt notwendig,
 aber sehr zu empfehlen, da dies als Name für das temporäre Kompilat im Ordner
\printfilepath{./figures-compiled/} genommen wird.
Dieser sollte gleich dem Namen des Quelldatei (ohne Endung) gewählt werden.
Ansonsten nimmt \texttt{pdflatex} eine hochlaufende Nummer als Dateiname,
was die Fehlersuche sehr erschwert.

Nachfolgend finden sich einige Beispiele für TikZ-Zeichnungen, nämlich
eine Übersicht über die KIT-Corporate-Identity-Farben (\cref{fig:kit-colors}),
ein kommutatives Diagramm (\cref{fig:kpca}),
ein Netzwerkkommunikationsgraph (\cref{fig:net-comm}),
einfache \index{Diagramm!Punkt-}Punktdiagramme (\cref{fig:ica})
und etwas aufwendigere Diagramme mit mehreren
\index{Achsensystem|see{Diagramm}}Achsensystemen (\cref{fig:pca}).

\begin{figure}[hp]%
	\centering%
  \tikzsetnextfilename{kit-colors}%
	%\resizebox{\textwidth}{!}{%
		\input{./figures-src/kit-colors.tex}%
	%}%
	\caption{KIT-Corporate-Identity-Farben}%
  \label{fig:kit-colors}%
\end{figure}

\begin{figure}[hp]%
	\centering%
  \tikzsetnextfilename{kpca}%
	%\resizebox{\textwidth}{!}{%
		\begin{tikzpicture}
\node (M) at (0,0) {$M$};
\node (F) [right=5em of M] {$F$};
\node (N) [right=5em of F] {$M'$};

\draw[-latex] (M)--(F) node[above,midway,font=\scriptsize] {$\varphi$};
\draw[-latex] (F)--(N) node[above,midway,font=\scriptsize] {PCA};
\draw[-latex,dotted] (M) to[bend left] node[above,midway,font=\scriptsize] {kernelized PCA} (N);

\node[below=3ex of M,font=\scriptsize] {$\dim(M) = d$};
\node[below=3ex of F,font=\scriptsize] {$\dim(F) \gg d$};
\node[below=3ex of N,font=\scriptsize] {$\dim(M') = d' < d$};
\end{tikzpicture}%
	%}%
	\caption{Kommutative Diagramm mit TikZ}%
  \label{fig:kpca}%
\end{figure}

\begin{figure}[hp]%
	\centering
  \tikzsetnextfilename{net-comm}%
	\resizebox{\textwidth}{!}{%
		\input{./figures-src/net-comm.tex}%
	}%
	\caption{Netzwerkkommunikationsgraph mit TikZ}%
  \label{fig:net-comm}%
\end{figure}

\begin{figure}[hp]%
	\centering%
	\subfloat[Ursprünglicher Merkmalsraum]{%
		\label{fig:ica-1}%
		\tikzsetnextfilename{ica-1}%
		\resizebox{0.45\textwidth}{!}{%
			\input{./figures-src/ica-1.tex}
		}%
	}%
	\hfill%
	\subfloat[Transienter Merkmalsraum (Nach Whitening, z.\,B. durch \gls{ac:PCA} inkl. Normalisierung)]{%
		\label{fig:ica-2}%
		\tikzsetnextfilename{ica-2}%
		\resizebox{0.45\textwidth}{!}{%
			\input{./figures-src/ica-2.tex}%
		}%
	}%
	\hfill%
	\subfloat[Transformierter Merkmalsraum]{%
		\label{fig:ica-3}%
		\tikzsetnextfilename{ica-3}%
		\resizebox{0.45\textwidth}{!}{%
			\input{./figures-src/ica-3.tex}%
		}%
}%
\caption{Diagramme mit TikZ direkt in LaTeX (hier: Die Schritte der \enquote{Independent component analysis})}%
\label{fig:ica}%
\end{figure}

\begin{figure}[hp]
	\centering%
	\subfloat[Ungünstige Projektion]{%
		\label{fig:pca-1}%
		\tikzsetnextfilename{pca-1}%
		\resizebox{0.48\textwidth}{!}{%
			\input{./figures-src/pca-1.tex}%
		}%
	}\hfill%
	\subfloat[Zielführende Projektion]{%
		\label{fig:pca-2}%
		\tikzsetnextfilename{pca-2}%
		\resizebox{0.48\textwidth}{!}{%
			\input{./figures-src/pca-2.tex}%
		}%
	}%
	\caption{Aufwändiges Diagramm mit TikZ (hier: Probleme der \enquote{Principal component analysis})}%
	\label{fig:pca}%
\end{figure}
%%%%%%%%%%%%%%%%%%%%%%%%%%%%%%%%%%%%%%%%%%%%%%%%%%%%%%%%%%%%
\section{Tabellen}
\label{sec:Tabellen}
%%%%%%%%%%%%%%%%%%%%%%%%%%%%%%%%%%%%%%%%%%%%%%%%%%%%%%%%%%%%

Typografisch gute Tabellen haben \emph{niemals} vertikale Trennlinien,
sondern nur wenige horizontale Linien.
Ferner haben sie eine trennende Linie ganz oben und ganz unten.
Hierfür stellt das Paket \pkg{booktabs} die Befehle
\begin{itemize*}
  \item \lstinline|\toprule|
	\item \lstinline|\midrule|
	\item \lstinline|\bottomrule|
\end{itemize*}
zur Verfügung.
Der Befehl \lstinline|\hline| ist tabu.
Für eine ausführliche Erläuterung auch über gute und schlechte Tabellen
siehe die Dokumentation des \pkg{booktabs}-Pakets \cite{Fear2005}.

Eine einfache Tabelle hat den folgenden Code:
\begin{latex}[caption={Einfache Tabelle in \LaTeX},label={lst:tabellenbeispiel}]
\begin{table}%
	\centering%
	\begin{tabularx}{\columnwidth}{l l X}%
		\toprule%
		Datei       &  Bedeutung    &  Benutzerinteraktion \\%
		\midrule%
		Diss.tex  &  Hauptdatei   &  nein     \\%
		images/   &  Bilder       &  ja       \\%
		content/  &  Kapitel      &  ja       \\%
		\bottomrule%
	\end{tabularx}%
	\caption{Dateien der Vorlage}%
	\label{tab:tabellenbeispiel}%
\end{table}
\end{latex}

Das Ergebnis sieht man in \cref{tab:tabellenbeispiel}.

\begin{table}%
	\centering%
	\begin{tabular}{l l l}%
		\toprule%
		Datei       &  Bedeutung    &  Benutzerinteraktion \\%
		\midrule%
		Diss.tex  &  Hauptdatei   &  nein     \\%
		images/   &  Bilder       &  ja       \\%
		content/  &  Kapitel      &  ja       \\%
		\bottomrule%
	\end{tabular}%
	\caption{Dateien der Vorlage}%
	\label{tab:tabellenbeispiel}%
\end{table}

Etwas komplizierter wird es, wenn man eine Tabelle mit alternierender Farbe einfügen möchte (\cref{tab:AlternierendeZeilenfarben}).

\begin{table}
\caption{Tabelle mit alternierender Zeilenfarbe}%
\label{tab:AlternierendeZeilenfarben}%
	\tablestyle%
	\tablealtcolored%
	\begin{tabular}{*{2}{v{0.45\textwidth}}}
		\toprule%
		\tableheadcolor%
		\tableheadformat Tabellenkopf &	\tableheadformat Tabellenkopf
		\tabularnewline%
		\midrule%
		%% Zwischenkopf ---------------------------------------------
		\multicolumn{2}{>{\columncolor{tablesubheadcolor}}l}{\bfseries\color{KITblue} Zwischenkopf}%
		\tabularnewline%
		%%-----------------------------------------------------------
		Inhalt  & Inhalt \tabularnewline
		Inhalt  & Inhalt \tabularnewline
		Inhalt  & Inhalt \tabularnewline
		%% Zwischenkopf ---------------------------------------------
		\multicolumn{2}{>{\columncolor{tablesubheadcolor}}l}{\bfseries\color{KITgreen} Zwischenkopf}%
		\tabularnewline
		%%-----------------------------------------------------------
		Inhalt  & Inhalt \tabularnewline
		Inhalt  & Inhalt \tabularnewline
		\bottomrule%
	\end{tabular}%
\end{table}

%% ------------------------------------------------------------
Lange Tabellen, die umbrochen werden sollen, können mit
\lstinline|\LTXtable{\textwidth}{Datei}|
eingebunden werden, wobei die Tabelle in eine Datei ausgelagert werden muss.
Ein Beispiel dafür sieht man in \cref{tab:MehrseitigeTabelle}.

\IfDefined{LTXtable}{%
	%--Einstellungen für Tabellen ----------
	\colorlet{tablerowcolor}{gray!10.0}%
	\renewcommand\tableheadcolor{\rowcolor{tableheadcolor}}%
	\renewcommand\tablehead{%
			\tableheadfontsize%
			\sffamily\bfseries%
			\slshape%
			\color{black}%
	}%
	%---------------------------------------
	{
		\tablestyle%
		%\tablealtcolored
		\rowcolors{1}{tablerowcolor}{white!100}%
		 \LTXtable{\textwidth}{tables/LongTableExample.tex}%
	}%
} % End If 
%
%
%
Sollten eine Tabelle einmal so breit sein, dass sie nicht mehr horizontal auf
eine Seite passt, so ist es natürlich möglich, diese mithilfe des Pakets
\printkeyword{rotfloat} \parencite{Sommerfeldt2004} in eine
\printkeyword{sidewaystable} statt in eine \printkeyword{table}-Umgebung zu setzen.
Also so:
\begin{latex}[caption={Gedrehte Tabelle},label={lst:rotated-table}]
\begin{sidewaystable}
  \centering%
  \begin{tabular}{...}%
    ...
  \end{tabular}%
  \caption{Bezeichnung}%
  \label{Referenzmarke}%
\end{sidewaystable}%
\end{latex}

Das Ergebnis sieht man in \cref{tab:ex-sideways}.

%% Um 90° gedrehte Tabelle
%
\begin{sidewaystable}[p]
\scriptsize%
%\tiny
\centering%
\begin{tabular}{l M M M M M}%
\toprule%
\addlinespace[0pt]%
 & \multicolumn{5}{c}{\bfseries Level} \tabularnewline
 & \multicolumn{2}{c}{\bfseries Qualitative} & \multicolumn{3}{c}{\bfseries Quantitative} \tabularnewline
 & \bfseries\centering Nominal & \bfseries\centering Ordinal & \bfseries\centering Interval & \bfseries\centering Ratio & \bfseries\centering Absolute \tabularnewline \addlinespace[0pt]\midrule\addlinespace[0pt]
Empirical relation &
\begin{tabitemize}\item[$\sim$] Equivalence\end{tabitemize} &
\begin{tabitemize}\item[$\sim$] Equivalence\item[$\prec$] Ordering\end{tabitemize} &
\begin{tabitemize}\item[$\sim$] Equivalence\item[$\prec$] Ordering\end{tabitemize} &
\begin{tabitemize}\item[$\sim$] Equivalence\item[$\prec$] Ordering\strut\end{tabitemize} &
\begin{tabitemize}\item[$\sim$] Equivalence\item[$\prec$] Ordering\strut\end{tabitemize} \tabularnewline \midrule
Empirical operation &
 &
 &
\begin{tabitemize}\item[$\oplus$] Addition\end{tabitemize} &
\begin{tabitemize}\item[$\oplus$] Addition\item[$\otimes$] Multiplication\strut\end{tabitemize} &
\begin{tabitemize}\item[$\oplus$] Addition\item[$\otimes$] Multiplication\strut\end{tabitemize} \tabularnewline \midrule
Feasable transformation &
$m' = f( m )$ for $f$ bij.\strut &
$m' = f( m )$ for $f$ mon.\strut &
$m' = am + b$ for $a>0$\strut &
$m' = am$ for $a>0$\strut &
$m' = m$\strut \tabularnewline \midrule
Examples of features &
\begin{tabitemize}\item Telephone numbers\item Postal codes\item Gender\strut\end{tabitemize} &
\begin{tabitemize}\item Grades\item Degree of hardness\item Wind intensity\strut\end{tabitemize} &
\begin{tabitemize}\item Temperatur in F\textdegree\item Calendric time\item Geographic altitude\strut\end{tabitemize} &
\begin{tabitemize}\item Temperatur in K\item Mass\item Length\item Electric current\strut\end{tabitemize} &
\begin{tabitemize}\item Quantum numbers\item Error number\strut\end{tabitemize} \tabularnewline \midrule
Range of features &
\begin{tabitemize}\item Numbers\item Names\item Symbols\strut\end{tabitemize} &
Natural numbers &
Real numbers &
Real, positive numbers &
Natural numbers \tabularnewline \midrule
Expressiveness & low & \dots & \dots & \dots & high\strut \tabularnewline \addlinespace[0pt]
\bottomrule%
\end{tabular}%
\caption{Beispiel für eine breite, gedrehte Tabelle (hier: Taxonomie der Maßskalen)}%
\label{tab:ex-sideways}%
\end{sidewaystable}
%%%%%%%%%%%%%%%%%%%%%%%%%%%%%%%%%%%%%%%%%%%%%%%%%%%%%%%%%%%%
\section{Mathematische Sätze, Lemmas, Definitionen etc.}%
\label{sec:Theoreme}
%%%%%%%%%%%%%%%%%%%%%%%%%%%%%%%%%%%%%%%%%%%%%%%%%%%%%%%%%%%%
%
Für eine mathematische Ausarbeitung gibt es LaTeX-\glspl{gls:umgebung}, um
\index{Satz|see{Theorem}}\index{Theorem}Sätze (Theoreme), \index{Lemma|see{Theorem}}Lemma,
\index{Beispiel|see{Theorem}}Beispiele etc. im üblichen Stil von
Mathematik-Büchern zu setzen und zu referenzieren. Vordefiniert sind die
\glspl{gls:umgebung}
\begin{itemize*}
  \item \texttt{theorem} für Sätze
  \item \texttt{definition} für Definitionen
  \item \texttt{lemma} für Lemma
  \item \texttt{corollary} für Korollare
  \item \texttt{proposition} für Propositionen
\end{itemize*}
Die übliche Verwendung ist
\begin{latex}[caption={Beispiel für Theorem-Umgebungen},label={lst:ntheorem}]
\begin{theorem}[Optionaler Name]\label{thm:my-theorem}
...
\end{theorem}
\end{latex}
Weitere Informationen findet man in der Dokumentation zum \texttt{ntheorem}-Paket
\parencite{May2011}. Das Ganze sieht dann beispielsweise wie folgt aus.

\begin{theorem}[Theorem von Arthur Dent]
\label{thm:arthur-dent} Die Antwort auf die Frage nach dem Leben, dem Universum und den ganzen Rest ist 42.
\end{theorem}

\begin{definition}
\LaTeX{} ist eine von Leslie Lamport 1980 entwickelter Satz von Makros zur Erweiterung von \TeX.
\end{definition}

\begin{proposition}[Zweifelhafte Folgerung]
LaTeX ist schön. Beweis folgt unmittelbar aus \cref{thm:arthur-dent}.
\end{proposition}

%%%%%%%%%%%%%%%%%%%%%%%%%%%%%%%%%%%%%%%%%%%%%%%%%%%%%%%%%%%%
\section{Quellcode-Listings}%
\index{Listing!Gestaltungsstil}%
\label{sec:Listings}
%%%%%%%%%%%%%%%%%%%%%%%%%%%%%%%%%%%%%%%%%%%%%%%%%%%%%%%%%%%%
%
Zum Einbinden und formatieren von \index{Code|see{Listing}}Quellcode"=Beispielen
-- sog. \index{Listing}Listings -- wird das Paket \pkg{listings}
\parencite{Hoffmann2014} verwendet.
Das Hervorheben von \index{Schlusselwort@Schlüsselwort}Schlüsselwörtern
wird von LaTeX automatisch erledigt,
wenn die korrekte Sprache des Listings angegeben ist.
Dies geschieht mit Hilfe der Option \printkeyword{language}
oder durch die Angabe eines entsprechend definierten Gestaltungsstils.

Im Befehl \lstinline|\lstset{...}|,
welcher in der Datei \printfilepath{preambel/preambel.tex} zu finden ist,
kann man einen globalen Stil für alle Listings vorgeben
(welcher jedoch bei Bedarf im Einzelfall überrufen werden kann).
Aktuell ist der etwas weiter oben im Code mit dem Befehl
\lstinline|\lstdefinestyle{...}| vordefinierte Stil
\printkeyword{latex} als Standardgestaltungsstil ausgewählt.

Neben \printkeyword{latex} sind in der Datei \printfilepath{preambel/preambel.tex}
auch noch \printkeyword{java} und \printkeyword{C++} als Gestaltungsstile vordefiniert.
Bei Bedarf lassen sich dort weitere Stile definieren und auswählen.

Zur Vereinfachung der Einbindung wurden zusätzlich Umgebungen
\printkeyword{latex}, \printkeyword{java} und \printkeyword{C++}
vordefiniert, die im Code mit \lstinline|\begin{<name>}...\end{<name>}|
direkt verwendet werden können (s. \cref{lst:java-listing}).
%
So bewirkt beispielsweise
%
\begin{latex}[caption={Beispiel eines Listings in Java},label={lst:java-listing}]
\begin{java}[caption={A Java Hello-World example},%
             label={lst:hello-world}]
public class HelloWorld {
  public static void main( String[] args ) {
    System.out.println( "HelloWorld" );
  }
}
\end{java}
\end{latex}
%
das folgende Ergebnis:
%
\begin{C++}[caption={A Java Hello-World example},label={lst:hello-world}]
public class HelloWorld {
  public static void main( String[] args ) {
    System.out.println( "HelloWorld" );
  }
}
\end{C++}

Man beachte, dass anders als bei Abbildungen und Tabellen
die Bezeichnung (\texttt{caption}) und die Referenzmarke (\texttt{label})
nicht als gesonderte Befehle sondern als optionale Argumente übergeben werden.
Dies liegt daran, dass ein Listing in der Regel keine Fließumgebung ist,
sondern an der Stelle im Text erscheint, an der sie im Code auch steht.
Ferner folgt ein Listing den ganz normalen Seitenumbruchsregeln.
Das heißt, überlanger Code wird einfach umgebrochen. 
Um ein Listing zu einem Fließobjekt zu machen, muss das optionale Argument
\texttt{float=<tbp>} angegeben werden.
Die \index{Platzierung}Plazierungsangabe \enquote{\texttt{h}} für \enquote{hier}
ist nicht erlaubt. Denn dies ist das Standardverhalten ohne \texttt{float}.
%%%%%%%%%%%%%%%%%%%%%%%%%%%%%%%%%%%%%%%%%%%%%%%%%%%%%%%%%%%%
\section{Querverweise}%
\label{sec:Querverweise}
%%%%%%%%%%%%%%%%%%%%%%%%%%%%%%%%%%%%%%%%%%%%%%%%%%%%%%%%%%%%
%
Querverweise sollten nicht mit dem Befehl \verb#\ref{...}# gesetzt werden,
sondern mit \verb#\cref{...}# und verwandten Befehlen aus dem Paket
\pkg{cleveref} \cite{Cubitt2013}.
Diese Befehle haben den Vorteil nicht nur die Nummer zu referenzieren,
sondern auch den Typ mit anzugeben.
Hinzu kommt eine intelligente Verwendung der Pluralform und
\index{Sortierung}Sortierung bei Mehrfachaufzählungen auch unterschiedlichen Typs.
Will man \bspw auf zwei Abbildungen und eine Tabelle mit den Marken (\enquote{Labels})
%
\begin{itemize*}
\item \texttt{fig:subfloat-example}
\item \texttt{tab:files-dirs-of-template}
\item \texttt{fig:kit-colors}
\end{itemize*}
%
verweisen, so schreibt man einfach per Komma getrennt
%
\begin{latex}[caption={Cleveres Referenzieren mit \bs cref},label={lst:cref}]
\cref{fig:subfloat-example,
      tab:ex-sideways,
      fig:kit-colors}
\end{latex}
%
und erhält als Resultat
\enquote{\cref{fig:subfloat-example,tab:ex-sideways,fig:kit-colors}}.
%%%%%%%%%%%%%%%%%%%%%%%%%%%%%%%%%%%%%%%%%%%%%%%%%%%%%%%%%%%%
\section{Mathematik}%
\label{sec:Mathe}
%%%%%%%%%%%%%%%%%%%%%%%%%%%%%%%%%%%%%%%%%%%%%%%%%%%%%%%%%%%%
%
Grundsätzlich gilt, was in \parencites{ams1999a}{ams1999b} steht. In der Datei
\printfilepath{preambel/05-math.tex} sind eine Menge Kurzkommandos definiert, um eine
einheitliche Typografie von \index{Skalare}Skalaren, \index{Vektoren}Vektoren,
\index{Matrizen}Matrizen, \index{Zufallsvariablen}Zufallsvariablen etc.
zur vereinfachen. In diese Dateien einfach mal reinschauen, welche Kurzkommandos
es gibt.

Auf zwei besondere Kommandos wird näher eingegangen, weil dies häufig falsch
gemacht wird.
\begin{itemize}
  \item Für die Matrixtransponierte gibt es das Kommando \verb#\Tr#, also
	\verb#$A^{\Tr}$# liefert $A^{\Tr}$
	
	\item Bei \index{Integral}Integralen muss das \enquote{Differential-d} gemäß
	ISO in aufrechter Schrift als Operator gesetzt sein mit einem kleinen Abstand
	zum Integranden. Hierfür gibt es das spezielle Kommando \verb#\diff#. Also
	\begin{equation}
	 \int^1_0 x^2 d x = \frac{1}{3} \qquad \text{(falsche Typografie!)}
	\end{equation}
	ist falsch, während \verb#\int^1_0 x^2 \diff x = \frac{1}{3}# das Richtige
	liefert
	\begin{equation}
	 \int^1_0 x^2 \diff x = \frac{1}{3} \qquad \text{(richtige Typografie!)}
	\end{equation}
\end{itemize}

%%%%%%%%%%%%%%%%%%%%%%%%%%%%%%%%%%%%%%%%%%%%%%%%%%%%%%%%%%%%
\section{Abkürzungsverzeichnis, Stichwortverzeichnis (Index) und Glossar}%
\label{sec:Glossare}
%%%%%%%%%%%%%%%%%%%%%%%%%%%%%%%%%%%%%%%%%%%%%%%%%%%%%%%%%%%%
%
Die Vorlage unterstützt auch ein Abkürzungsverzeichnis, ein Stichwortverzeichnis,
ein Symbolverzeichnis sowie ein allgemeines \index{Glossar}\gls{gls:Glossar},
das Definitionen von \index{Fachbegriff!Definition}Fachtermini oder
\index{Uebersetzung@Übersetzung}Übersetzungen von fremdsprachlichen Begriffen
(d.h. eine Art Lexikons) enthalten kann.


%%%%%%%%%%%%%%%%%%%%%%%%%%%%%%%%%%%%%%%%%%%%%%%%%%%%%%%%%%%%
\subsection{Abkürzungen und Abkürzungsverzeichnis}%
\label{sec:Akronyme}
%%%%%%%%%%%%%%%%%%%%%%%%%%%%%%%%%%%%%%%%%%%%%%%%%%%%%%%%%%%%
Zur Erzeugung des Abkürzungsverzeichnis und des Glossar wird intern das
\pkg{glossaries}-\gls{gls:pkg} verwendet \cite{talbot2014}.
Zudem wurden einige Makros definiert, welche die Erfassung der Begriffe erleichtern sollen.

Um ein Abkürzungsverzeichnis zu erzeugen, muss zuerst eine Liste der Abkürzungen angelegt werden.
Dies geschieht in der Datei \printfilepath{./preambel/Acronyms.tex}.
Hierfür wird das Makro \lc{newacronym} verwendet.
Dieses Makro hat drei obligatorische Argumente, nämlich das
\begin{itemize*}
\item die Marke,
\item das Akronym und
\item die Langform.
\end{itemize*}
Als Konvention wird der Marke eines Akronyms ein \printkeyword{ac:} als Präfix vorangestellt.
Optional können Pluralformen, sowie Genitiv-, Dativ- und Akkusativ-Formen 
der Abkürzung und des eigentlichen Begriffs angegeben werden,
sofern sie im Quelltext verwendet werden und sich von der Grundform unterscheiden.
Außerdem kann mit dem Schlüsselwort \printkeyword{description}
eine abweichende Version der Langform für die Verwendung im Abkürzungsverzeichnis definiert werden.

Eine Abkürzung wird folgendermaßen definiert (Angaben in eckigen Klammern und auskommentierte Zeilen sind optional):
\begin{latex}[caption={Definition einer Abkürzung},label={lst:AcronymEntry}]
\newacronym[shortgenitive={MSAs},%
            genitive={meines schönen Akronyms},%
            %shortdative={MSA},%
            dative={meinem schönen Akronym},%
            %shortaccusative={MSA},%
            %accusative={mein schönes Akronym},%
            shortplural={MSAs},
            longplural={meine schönen Akronyme},%
            %shortpluralgenitive={MSAs},%
            pluralgenitive={meiner schönen Akronyme},%
            %shortpluraldative={MSAs},%
            pluraldative={meinen schönen Akronymen},%
            %shortpluralaccusative={MSAs},%
            pluralaccusative={meine schönen Akronyme},%
            description={mein schönes Akronym, %   <- optional
                         ein Beispiel für eine Abkürzung}%
           ]{ac:MSA}{MSA}{mein schönes Akronym}
\end{latex}

Im Text des Dokumentes werden die Einträge durch den Befehl \lc{ac\{<Marke>\}} verwendet.
Bei der erstmaliger Verwendung wird die Langform gedruckt, gefolgt von der Abkürzung, welche in Klammern gesetzt wird.
Beim zweiten Vorkommen wird nur noch die Abkürzung gedruckt.
Zusätzlich definiert die Vorlage die Befehle
\lc{acgen\{...\}}, \lc{acdat\{...\}} und \lc{acacc\{...\}}, sowie
\lc{acplgen\{...\}}, \lc{acpldat\{...\}} und \lc{acplacc\{...\}},
die jeweils die Genitiv-, Dativ- und Akkusativ-Form (singular und Plural) drucken.

Im \gls{gls:pkg} \pkg{glossaries} stellt der Befehl \lc{ac\{...\}} ein Shortcut für den Befehl \lc{gls\{...\}} dar.
Mit diesem kann ein allgemeines Glossar-Eintrag im Text referenziert werden.

Mit den Befehlen
\lc{acrshort\{...\}}, \lc{acrlong\{...\}}, \lc{acrfull\{...\}}
und ihren Abwandlungen sowie den Shortcuts
\lc{acs\{...\}}, \lc{acl\{...\}}, \lc{acf\{...\}},
kann jeweils nur die Abkürzung, nur die Langform oder beides explizit angefordert werden.
Allerdings wird eine solche Verwendung ggf. nicht als \enquote{erstmalige Verwendung} zählen.

Damit resultiert der folgende Quellcode
\begin{latex}[caption={Verwendung von Abkürzungen},label={lst:AcronymUsage}]
    \Acf{ac:MSA} ist ein Beispiel für die Verwendung einer
    Abkürzung am Anfang des Satzes. Man beachte, dass
    der Aufruf der Marke mit dem Makro \lc{acf} bzw. \lc{Acf}
    nicht als die erste Erwähnung im Text zählt.
    Bei der ersten Nennung des \acgen{ac:MSA} unter Verwendung
    der Makros \lc{ac}, \lc{acgen} \oae erscheint die Langform,
    gefolgt von der Kurzform. Bei der zweiten Nennung des
    \acgen{ac:MSA} erscheint nur noch die Kurzform.
\end{latex}
%
in der Ausgabe
%
\begin{quote}
\Acf{ac:MSA} ist ein Beispiel für die Verwendung einer
Abkürzung am Anfang des Satzes. Man beachte, dass
der Aufruf der Marke mit dem Makro \lc{acf} bzw. \lc{Acf}
nicht als die erste Erwähnung im Text zählt.
Bei der ersten Nennung des \acgen{ac:MSA} unter Verwendung
der Makros \lc{ac}, \lc{acgen} \oae erscheint die Langform,
gefolgt von der Kurzform. Bei der zweiten Nennung des
\acgen{ac:MSA} erscheint nur noch die Kurzform.
\end{quote}


%%%%%%%%%%%%%%%%%%%%%%%%%%%%%%%%%%%%%%%%%%%%%%%%%%%%%%%%%%%%
\subsection{Glossar}%
\label{sec:Glossar}
%%%%%%%%%%%%%%%%%%%%%%%%%%%%%%%%%%%%%%%%%%%%%%%%%%%%%%%%%%%%

Neben einem Abkürzungsverzeichnis kann man auch ein
\index{Glossar}\gls{gls:Glossar} erstellen lassen.
In diesem können Definitionen von \index{Fachbegriff!Definition}Fachtermini oder
\index{Uebersetzung@Übersetzung}Übersetzungen von fremdsprachlichen Begriffen stehen.

Der wesentliche Unterschied zwischen einer Abkürzung und einem allgemeinen Glossar-Eintrag ist,
dass bei Abkürzungen bei erstmaliger Verwendung die Abkürzung gedruckt und die Langform in Klammer dahinter gesetzt wird.
Bei allgemeinen Glossar-Einträgen wird normalerweise nur der Name gesetzt.
Durch eine Option des \pkg{glossaries}-\glsgen{gls:pkg} kann man sicher stellen,
dass alle Glossar-Einträge auf das Glossar am Ende des Manuskripts verlinkt werden.
Standardmäßig ist diese Option jedoch deaktiviert.

Die Glossar-Einträge werden in der Datei  \printfilepath{./preambel/Glossary.tex} definiert.

Die Definition der Glossar-Einträge geschieht mit dem Makro \lc{myglossaryentry},
welches drei obligatorische Argumente hat, nämlich
\begin{itemize*}
\item die Marke,
\item den Begriff und
\item die Erklärung / Definition / Übersetzung.
\end{itemize*}
Als Konvention wird der Marke eines Glossar-Eintrages ein \enquote{gls:} als Präfix vorangestellt.
Optional kann die Plural-, sowie Genitiv-, Dativ- und Akkusativ-Form des Begriffs angegeben werden,
sofern sie im Quelltext verwendet werden und sich von der Grundform unterscheiden.

Ein Glossar-Eintrag kann beispielsweise folgendermaßen definiert werden:
\begin{latex}[caption={Definition eines Glossar-Eintrages},label={lst:GlossEntry}]
\myglossaryentry[plural={Glossare},%
                 genitive={Glossars}]%
                {gls:Glossare}{Glossar}{alphabetisch sortierte Liste von Begriffen mit Erklärung}
\end{latex}

Normalerweise werden im Glossar nur diejenigen Begriffe angezeigt,
die im Text des Dokumentes erwähnt und entsprechend referenziert worden sind.
Eine Referenzierung der Glossar-Einträge im Text geschieht normalerweise mit dem
\lc{gls\{<Marke>\}}-Befehl,
welcher den Begriff im Text druckt und für seine Aufnahme ins Glossar sorgt.
Weitere mögliche Befehle sind \lc{Gls\{...\}} und \lc{GLS\{...\}},
die den ersten bzw. alle Buchstaben in Großbuchstaben umwandeln,
\lc{glpl\{...\}}, \lc{Glspl\{...\}}, \lc{GLSpl\{...\}} für die Pluralform \usw.
Zusätzlich definiert die Vorlage die Befehle
\lc{glsgen\{...\}}, \lc{glsdat\{...\}} und \lc{glsacc\{...\}}, sowie
\lc{glsplgen\{...\}}, \lc{glspldat\{...\}} und \lc{glsplacc\{...\}},
die jeweils die Genitiv- Dativ- und Akkusativ-Form drucken.

Außerdem gibt es mit dem Befehl \lc{glsadd\{...\}} die Möglichkeit,
eine Stelle im Text mit einem Glossar-Begriff zu verlinken, ohne diesen explizit zu drucken.
Mit \lc{glsaddall} kann man alle definierte Glossar-Einträge ins Glossar aufnehmen,
ohne sie im Text des Dokumentes referenziert zu haben.

Die Verwendung des oben definierten Glossar-Eintrages im Text mit dem Befehl
\lc{glsgen\{gls:Glossar\}} mündet im Text in Erwähnung des \glsgen{gls:Glossar}.


%%%%%%%%%%%%%%%%%%%%%%%%%%%%%%%%%%%%%%%%%%%%%%%%%%%%%%%%%%%%
\subsection{Stichwortverzeichnis (Index)}%
\label{sec:Index}
%%%%%%%%%%%%%%%%%%%%%%%%%%%%%%%%%%%%%%%%%%%%%%%%%%%%%%%%%%%%
%
Ein \index{Stichwortverzeichnis}Stichwortverzeichnis (oder
\index{Index|see{Stichwortverzeichnis}}Index)
ist einfach nur eine alphabetisch sortierte Liste von Begriffen mit einer Auflistung der Fundstellen im Dokument.
Diese ist nützlich, wenn sich der Leser zu einem Begriff alle Vorkommnisse anschauen möchte.
Der Index wird erzeugt, indem im Quellcode der Befehl \verb+\index{Begriff}+
eingefügt wird. Der Begriff selbst wird dadurch nicht gedruckt und muss daher
noch einmal wiederholt werden, um auch im Text gedruckt zu werden. Dieses Verhalten
ist beabsichtigt, sodass im Index immer nur die Grundform des Wortes verwendet
wird, aber im Text natürlich die richtige Deklination.


%%%%%%%%%%%%%%%%%%%%%%%%%%%%%%%%%%%%%%%%%%%%%%%%%%%%%%%%%%%%
\subsection{Symbolverzeichnis}%
\label{sec:Symbolverz}
%%%%%%%%%%%%%%%%%%%%%%%%%%%%%%%%%%%%%%%%%%%%%%%%%%%%%%%%%%%%
%
Ein Symbolverzeichnis kann auf zwei Arten angelegt werden.
Normalerweise reicht eine manuell erstellte Übersicht über die Notation,
so wie sie in der Datei \texttt{./00-Front-Matter/Notation.tex}
mit Hilfe von Befehlen
\texttt{\bs myNotationTableEntryMath\{<Mathe-Ausdruck>\}\{<Beschreibung>\}}
und
\texttt{\bs myNotationTableEntryText\{<Text-Ausdruck>\}\{<Beschreibung>\}}
definiert wird.
Diese ist recht einfach und lässt sich bei Bedarf beliebig ergänzen.

Allerdings gibt es auch die Möglichkeit zur automatischen Erzeugung eines
Symbolverzeichnisses mit Hilfe des \pkg{glossaries}-\glsgen{gls:pkg}.
Um dieses, am Ende des Manuskriptes eingebundene Symbolverzeichnis zu erzeugen,
müssen Symbole in Form von Glossar-Einträgen angelegt und im Text des Dokumentes
zumindest einmal entsprechend mit dem Befehl \lc{gls\{<Marke>\}} referenziert werden.
Dafür müsste ein Symboleintrag folgendermaßen angelegt werden:
\begin{latex}[caption={Definition eines Symboleintrages},label={lst:SymbEntry}]
\newglossaryentry{symb:pi}{%
                  name={\ensuremath{\pi}},%
                  sort={pi},%
                  type=symbols,%
                  description={Kreiszahl, Verhältnis des Umfangs eines Kreises zu seinem Durchmesser}%
                 }
\end{latex}
Die Referenzierung des Symbols \gls{symb:pi} im Text geschieht dann mit \verb+\gls{symb:pi}+.

Die Einbindung eines automatisch erzeugten Symbolverzeichnisses ist am Ende des Manuskripts vorgesehen.
Es passiert in der Datei
\texttt{./content/Inhalt-BackMatter.tex}.
Dabei ist darauf zu achten, dass in der Hauptdatei \texttt{Diss.tex}
die Einblendung durch \lc{showif\{showListOfSymbols\}} aktiviert ist.

%%%%%%%%%%%%%%%%%%%%%%%%%%%%%%%%%%%%%%%%%%%%%%%%%%%%%%%%%%%%
\section{Randnotizen}%
\label{sec:Randnotizen}
%%%%%%%%%%%%%%%%%%%%%%%%%%%%%%%%%%%%%%%%%%%%%%%%%%%%%%%%%%%%
%
Randnotizen 
\floatmarginnote{Ich bin eine überflüssige Randnotiz}%
werden mit dem Kommando \lc{floatmarginnote} gesetzt.
Diese eignet sich zum Beispiel um im Text Stellen zu kennzeichnen,
an denen man noch arbeiten sollte.
Da die Vorgaben des \glsgen{ac:KSP} für den Seitenlayout
einen sehr kleinen Randbereich vorsehen, der zudem nicht bedruckt werden darf,
werden keine Randnotizen in der endgültigen Version des Manuskriptes akzeptiert.
Die Randnotizen lassen sich bequem in der Hauptdatei ausschalten,
indem man \texttt{\bs showif\{showMarginNotes\}}
zu \texttt{\bs hideif\{showMarginNotes\}} ändert.
%
% Alte Anleitung von Philipp Woock
%\input{content/0X-ExampleContents/0-Anleitung.tex}
%\input{content/0X-ExampleContents/0-Beispiele.tex}
%\input{content/0X-ExampleContents/1-Bilder}
%\input{content/0X-ExampleContents/1-Mathematik}
%\input{content/0X-ExampleContents/2-Experimente}
%\input{content/0X-ExampleContents/3-Ergebnisse}
%\input{content/0X-ExampleContents/4-Zusammenfassung}
%
\end{showExamples}%