%%%%%%%%%%%%%%%%%%%%%%%%%%%%%%%%%%%%%%%%%%%%%%%%%%%%%%%%%%%%
\chapter*{Notation}
%%%%%%%%%%%%%%%%%%%%%%%%%%%%%%%%%%%%%%%%%%%%%%%%%%%%%%%%%%%%
%
%% Add an entry to the table of contents
\addcontentsline{toc}{chapter}{Notation}
%
%% No need for a separate bookmark (done via toc entry above)
%\pdfbookmark[0]{Notation}{notation}
%
%Add a header title (Kolumnentitel setzen)
\markboth{Notation}{Notation}
%
This chapter introduces the notation and symbols which are used in this thesis.
%In cases where a symbol has more than one meaning, the context (or a specific statement) resolves the ambiguity.

%%%%%%%%%%%%%%%%%%%%%%%%%%%%%%%%%%%%%%%%%%%%%%%%%%%%%%%%%%%%
\section*{General notation}
%%%%%%%%%%%%%%%%%%%%%%%%%%%%%%%%%%%%%%%%%%%%%%%%%%%%%%%%%%%%
\begin{myNotationDescTable}
%
% Als dreispaltige Tabelle angelegt,
% Beim Makro myNotationDescTableEntry ist die dritte Spalte ist bereits im Mathe-Modus
\myNotationDescTableEntry{Scalars}{italic Roman and Greek lowercase letters}{x, \alpha}
\myNotationDescTableEntry{Sets}{Greek uppercase letters}{\Theta}
\myNotationDescTableEntry{Vectors}{bold Roman lowercase letters}{\vt}
\myNotationDescTableEntry{Matrices}{bold Roman uppercase letters}{\mR}
\myNotationDescTableEntry{State spaces}{bold calligraphic Roman uppercase letters}{\spX}
\myNotationDescTableEntry{Random variables}{italic Roman uppercase letters}{\rvE}
\myNotationDescTableEntry{Multi-dimensional random variables}{bold italic Roman uppercase letters}{\mdrvE}
%\myNotationDescTableEntry{Distributions}{calligraphic uppercase letters }{\mathcal{N}}
%
\end{myNotationDescTable}
\vspace{-\parskip}
In multidimensional sets of elements related to time series, the first index denotes time.
%
%%%%%%%%%%%%%%%%%%%%%%%%%%%%%%%%%%%%%%%%%%%%%%%%%%%%%%%%%%%%
%%
%% Ggf. kann ein eigenes Symbolverzeichnis ("List of Symbols") am
%% Ende des Manuskripts eingefügt werden (s. Inhalt-BackMatter.tex).
%% Um dieses zu erzeugen, müssen Symbole in Form von Glossareinträgen
%% angelegt und im Text der Arbeit entsprechend referenziert werden.
%% Normalerweise reicht jedoch eine manuell erstellte Übersicht
%% über die einzelnen Symbole wie nachfolgend angeführt.
%%
%%%%%%%%%%%%%%%%%%%%%%%%%%%%%%%%%%%%%%%%%%%%%%%%%%%%%%%%%%%%

%%%%%%%%%%%%%%%%%%%%%%%%%%%%%%%%%%%%%%%%%%%%%%%%%%%%%%%%%%%%
\section*{Distributions}
%%%%%%%%%%%%%%%%%%%%%%%%%%%%%%%%%%%%%%%%%%%%%%%%%%%%%%%%%%%%
\begin{myNotationTable}
%
% Als zweispaltige Tabelle angelegt,
% Beim Makro myNotationTableEntryMath ist die erste Spalte bereits im Mathe-Modus
\myNotationTableEntryMath{\GaussDist}{Gaussian normal distribution}
\myNotationTableEntryMath{\ChiSqDist_{n}}{n-dimensional chi-square distribution}
%
\end{myNotationTable}


%%%%%%%%%%%%%%%%%%%%%%%%%%%%%%%%%%%%%%%%%%%%%%%%%%%%%%%%%%%%
\section*{Numbers and indexing}
%%%%%%%%%%%%%%%%%%%%%%%%%%%%%%%%%%%%%%%%%%%%%%%%%%%%%%%%%%%%
\begin{myNotationTable}
%
% Als zweispaltige Tabelle angelegt,
% Beim Makro myNotationTableEntryMath ist die erste Spalte bereits im Mathe-Modus
\myNotationTableEntryMath{\NatNum}{natural numbers}
\myNotationTableEntryMath{\NatNum_0}{natural numbers including zero (non-negative integers)}
\myNotationTableEntryMath{k, t}{discrete points in time}
\myNotationTableEntryMath{i, j, \ell, q}{indexing for objects, measurements and points}
%\myNotationTableEntryMath{m, n}{number of detections/measurements, number of tracked objects}
%
\end{myNotationTable}
%
%
%
%%%%%%%%%%%%%%%%%%%%%%%%%%%%%%%%%%%%%%%%%%%%%%%%%%%%%%%%%%%%%
%\section*{Indexing}
%%%%%%%%%%%%%%%%%%%%%%%%%%%%%%%%%%%%%%%%%%%%%%%%%%%%%%%%%%%%%
%\begin{myNotationTable}
%
%% Als zweispaltige Tabelle angelegt,
%% Beim Makro myNotationTableEntryMath ist die erste Spalte bereits im Mathe-Modus
%\myNotationTableEntryText{$k$, $t$}{discrete points in time}
%\myNotationTableEntryText{$i$, $j$, $q$}{indexing for objects, measurements and points}
%\myNotationTableEntryText{$m$, $n$}{number of detections/measurements, number of tracked objects}
%
%\end{myNotationTable}


%%%%%%%%%%%%%%%%%%%%%%%%%%%%%%%%%%%%%%%%%%%%%%%%%%%%%%%%%%%%
\section*{Geometry (coordinates, vehicle, {\newline}and camera modeling)}
%%%%%%%%%%%%%%%%%%%%%%%%%%%%%%%%%%%%%%%%%%%%%%%%%%%%%%%%%%%%
\begin{myNotationTable}
%
% Als zweispaltige Tabelle angelegt,
% Beim Makro myNotationTableEntryMath ist die erste Spalte bereits im Mathe-Modus
\myNotationTableEntryMath{x, y, z}{world coordinates}
\myNotationTableEntryMath{u, v}{image coordinates}
\myNotationTableEntryMath{b}{stereo base line}
\myNotationTableEntryMath{f}{focal length}
\myNotationTableEntryMath{\Delta{u}}{displacement in the image, disparity}
\myNotationTableEntryMath{d(\cdot)}{distortion function}
\myNotationTableEntryMath{\kappa_1, \kappa_2}{radial distortion parameters}
\myNotationTableEntryMath{\rho_1, \rho_2}{tangential distortion parameters}
%
\myNotationTableEntryMath{l, w, h}{length, width, height}
\myNotationTableEntryMath{r}{radius}
%
\myNotationTableEntryMath{A}{area}
\myNotationTableEntryMath{V}{volume}
%
\myNotationTableEntryMath{v}{velocity}
\myNotationTableEntryMath{a}{acceleration}
\myNotationTableEntryMath{\alpha}{steering angle}
%\myNotationTableEntryText{}{}
\myNotationTableEntryMath{\varphi}{orientation angle}
\myNotationTableEntryMath{\dot{\varphi}}{yaw rate}
%
\myNotationTableEntryMath{\pkt}{point in 2D and 3D space}
\myNotationTableEntryMath{\pHomogen}{point in homogeneous coordinates}
%
\myNotationTableEntryMath{\RotMat}{rotation matrix}
\myNotationTableEntryMath{\transVec}{translation vector}
%
\end{myNotationTable}


%%%%%%%%%%%%%%%%%%%%%%%%%%%%%%%%%%%%%%%%%%%%%%%%%%%%%%%%%%%%
\section*{Object state modeling and probabilities}
%%%%%%%%%%%%%%%%%%%%%%%%%%%%%%%%%%%%%%%%%%%%%%%%%%%%%%%%%%%%
\begin{myNotationTable}
%
% Als zweispaltige Tabelle angelegt,
% Beim Makro myNotationTableEntryMath ist die erste Spalte bereits im Mathe-Modus
\myNotationTableEntryMath{\statespace}{state space}
\myNotationTableEntryMath{\measurementspace}{measurement space}
%
\myNotationTableEntryText{}{}
\myNotationTableEntryMath{\Fmat}{system matrix of the Kalman Filter}
\myNotationTableEntryMath{\Gmat}{control matrix of the Kalman Filter}
\myNotationTableEntryMath{\Hmat}{measurement matrix of the Kalman Filter}
\myNotationTableEntryMath{\KGk}{Kalman gain at time $k$}
%
\end{myNotationTable}
%