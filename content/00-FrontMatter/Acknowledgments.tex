%%%%%%%%%%%%%%%%%%%%%%%%%%%%%%%%%%%%%
\chapter*{\TransAcknowledgements}
%%%%%%%%%%%%%%%%%%%%%%%%%%%%%%%%%%%%%
%
%% No entry to the table of contents:
%\addcontentsline{toc}{chapter}{\TransAcknowledgements}
%
%% However, set a PDF bookmark:
\pdfbookmark[0]{\TransAcknowledgements}{acknowledgements}
%
%Add a header title (Kolumnentitel setzen)
\markboth{\TransAcknowledgements}{\TransAcknowledgements}
%
%%%%%%%%%%%%%%%%%%%%%%%%%%%%%%%%%%%%%
%
Am Lehrstuhl für Interaktive Echtzeitsysteme (IES) des Karlsruhe Instituts für Technologie (KIT) wurde von Philipp Wook die erste LaTeX-Vorlage zur Erstellung von wissenschaftlichen Arbeiten erstellt.
Diese basierte ihrerseits auf der Vorlage von Matthias Pospiech von der Leibniz Universität Hannover.
Die von Matthias Pospiech und durch Philipp Wook stark erweitere Vorlage, hatte den Anspruch die \enquote{eierlegende Wollmilchsau} zu sein und möglichst alle
Anwendungsfälle abzudecken.

Leider hatte die alte Vorlage -- die mittlerweile ziemlich in die Jahre gekommen ist -- auch zwei entscheidende Nachteile:
\begin{itemize}
  \item Es setzte auf den alten \glspl{bibtex}-Paketen auf anstatt des neueren \gls{biblatex}-Ökosystems.
				Dadurch war eine durchgängige	\gls{utf8}-Unterstützung nicht möglich und die ein oder andere Konstellation von Umlauten hat immer mal wieder \enquote{geknallt}.
		
  \item Zum Erstellen von \index{Grafik|see{Bild}}\index{Bild!Vektor-}Vektorgrafiken mit einer hohen Druck- und Typografiequalität gibt es \gls{tikz}.
				Die alte Vorlage unterstütze zwar grundlegendes \gls{tikz}, aber bei vielen \gls{tikz}-Zusatzpaketen kam es zu Inkompatibilitäten mit anderen Paketen.
				
	\item Die alte Vorlage war mit den neuen Richtlinien des KIT Scientific Publishing Verlag nicht mehr zu 100\% kompatibel, was zu vielen Iterationsschleifen bei Einreichung von Manuskripten geführt hat.
\end{itemize}

Im Jahre 2018 wurde diese Vorlage von Michael Grinberg im Zuge seiner Promotion überarbeitet und an die neuen Vorgaben des KIT Scientific Publishing Verlages angepasst.
Insbesondere wurde die Vorlage dahingehend abgeändert, dass sie nun XeLaTeX statt pdfLaTeX, \gls{biblatex} statt \gls{bibtex} verwendet und die inzwischen aufgetretene Inkompatibilität mit neuen Versionen einiger \gls{latex}-\glspl{pkg} behebt.
Außerdem wurde die Vorlage etwas \enquote{entschlankt}.

Zu beachten ist, dass einige Möglichkeiten der alten Vorlage nicht mehr unterstützt werden.
Dies betrifft alle LaTeX-Pakete die in irgendeiner Weise \gls{postscript} benötigen.
Um das volle Potential von \gls{utf8}, \gls{biblatex} und \gls{tikz} ausnutzen zu können, ist die Verarbeitungskette auf
\begin{equation*}
\text{\texttt{LaTeX}-Quellcode} \xrightarrow{\text{\texttt{xelatex}}} \text{\texttt{PDF}}
\end{equation*}
reduziert worden.
Umwege wie
\begin{equation*}
\text{\texttt{LaTeX}-Quellcode} \xrightarrow{\text{\texttt{latex}}} \text{\texttt{DVI}} \xrightarrow{\text{\texttt{dvi2ps}}} \text{\texttt{PS}} \xrightarrow{\text{\texttt{ps2pdf}}} \text{\texttt{PDF}}
\end{equation*}
sind ausgeschlossen.
Das bedeutet insbesondere, dass alle Optionen, die das bekannte Paket \index{pstricks}\texttt{pstricks} bietet nicht mehr zur Verfügung stehen.
Allerdings bietet hier \gls{tikz} immer eine Ersatzlösung an.
Das einzige, was definitiv gar nicht mehr funktioniert und auch nicht durch \gls{tikz} nachgestellt werden kann, ist das direkte Einbinden von \gls{postscript}"~ bzw. \glstext{eps}-Abbildungen.
Diese müssen nun zunächst durch externe Tools in \glstext{pdf} konvertiert werden.

Zu beachten ist auch, dass nun \texttt{biber} statt \texttt{bibtex} zur Kompilierung von Bibliografien verwendet werden soll.
