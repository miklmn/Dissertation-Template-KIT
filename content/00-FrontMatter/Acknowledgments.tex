%%%%%%%%%%%%%%%%%%%%%%%%%%%%%%%%%%%%%
% Note: the translation macros (e.g. "\TransAcknowledgements")
% are to be found in /preambel/Translations.tex
%%%%%%%%%%%%%%%%%%%%%%%%%%%%%%%%%%%%%
\chapter*{\TransAcknowledgements}
%%%%%%%%%%%%%%%%%%%%%%%%%%%%%%%%%%%%%
%
%% No entry for this chapter to the table of contents:
%% Kein Eintrag für dieses Kapitel im Imhaltsverzeichnis
%\addcontentsline{toc}{chapter}{\TransAcknowledgements}
%
%% However, set a PDF bookmark (PDF-Lesezeichen setzen):
\pdfbookmark[0]{\TransAcknowledgements}{acknowledgements}
%
%Add a header title (Kolumnentitel beidseitig setzen)
\markboth{\TransAcknowledgements}{\TransAcknowledgements}
%
%%%%%%%%%%%%%%%%%%%%%%%%%%%%%%%%%%%%%
%
\lettrine[nindent=0.2em]{H}{erzlich Willkommen} zur \LaTeX-Vorlage
%Nutzung von \glsdisp wegen der Genitiv-Form
des \glsgen{ac:IES} %{Lehrstuhls für Interalktive Echtzeitsysteme (IES)}
des \glsgen{ac:KIT}. %{Karlsruher Institus für Technilogie (KIT)}.
Sie wurde wurde ursprünglich von
\href{http://ies.anthropomatik.kit.edu/mitarbeiter.php?person=woock}{Philipp Woock}
entwickelt und basiert in ihren Grundzügen auf der \enquote{allgemeinen sehr umfassenden}
\href{http://www.matthiaspospiech.de/latex/templates/thesis/}{Vorlage} von 
\href{http://www.matthiaspospiech.de}{Matthias Pospiech} von der Leibniz Universität Hannover.
Ohne diese Basis wäre die Vorlage niemals das geworden, was sie ist. Vielen Dank!

Im Jahre 2018 wurde diese Vorlage von
\href{http://ies.anthropomatik.kit.edu/mitarbeiter.php?person=grinberg}{Michael Grinberg}
im Zuge seiner Promotion überarbeitet und an die neuen Vorgaben des \glsgen{ac:KSP} angepasst.
Insbesondere wurde die Vorlage dahingehend abgeändert, dass sie nun XeLaTeX statt pdfLaTeX, 
\gls{gls:biblatex} statt \gls{gls:bibtex} verwendet und die inzwischen aufgetretene
Inkompatibilitäten mit neuen Versionen einiger \gls{gls:latex}-\glspl{gls:pkg} behebt.
In 2019 wurde die Vorlage weiter angepasst und mit Beispielen aus einer Vorlagevon
\href{https://crypto.iti.kit.edu/index.php?id=nagel}{Mathias Nagel} angereichert.

Zu beachten ist, dass zum Kompilieren der Vorlage \texttt{xelatex} statt \texttt{pdflatex}
und zum Kompilieren der Bibliographien \texttt{biber} statt \texttt{bibtex} verwendet werden sollen.