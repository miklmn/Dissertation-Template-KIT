%%%%%%%%%%%%%%%%%%%%%%%%%%%%%%%%%%%%%
% Note: the translation macros (e.g. "\TransAcknowledgements")
% are to be found in /preambel/Translations.tex
%%%%%%%%%%%%%%%%%%%%%%%%%%%%%%%%%%%%%
\chapter*{\TransAcknowledgements}
%%%%%%%%%%%%%%%%%%%%%%%%%%%%%%%%%%%%%
%
%% No entry for this chapter to the table of contents:
%% Kein Eintrag für dieses Kapitel im Imhaltsverzeichnis
%\addcontentsline{toc}{chapter}{\TransAcknowledgements}
%
%% However, set a PDF bookmark (PDF-Lesezeichen setzen):
\pdfbookmark[0]{\TransAcknowledgements}{acknowledgements}
%
%Add a header title (Kolumnentitel beidseitig setzen)
\markboth{\TransAcknowledgements}{\TransAcknowledgements}
%
%%%%%%%%%%%%%%%%%%%%%%%%%%%%%%%%%%%%%
%
\lettrine[nindent=0.2em]{H}{erzlich Willkommen} zur \LaTeX-Vorlage des Lehrstuhls für Interaktive Echtzeitsysteme (IES) des Karlsruhe Instituts für Technologie (KIT).
Sie wurde wurde ursprünglich von Philipp Woock entwickelt und basiert in ihren Grundzügen auf der exzellenten \enquote{allgemeinen sehr umfassenden} Vorlage von Matthias Pospiech (\url{http://www.matthiaspospiech.de/latex/templates/thesis/}) von der Leibniz Universität Hannover.
Ohne diese Basis wäre die Vorlage niemals das geworden, was sie ist. Vielen Dank!

Im Jahre 2018 wurde diese Vorlage von Michael Grinberg im Zuge seiner Promotion überarbeitet und an die neuen Vorgaben des KIT Scientific Publishing Verlages angepasst.
Insbesondere wurde die Vorlage dahingehend abgeändert, dass sie nun XeLaTeX statt pdfLaTeX, \gls{biblatex} statt \gls{bibtex} verwendet und die inzwischen aufgetretene Inkompatibilitäten mit neuen Versionen einiger \gls{latex}-\glspl{pkg} behebt.
Außerdem wurde die Vorlage etwas entschlankt und umstrukturiert.

Zu beachten ist, dass zum Kompileren der Vorlage \texttt{xelatex} statt \texttt{pdflatex} und zum Kompilieren der Bibliograpghien \texttt{biber} statt \texttt{bibtex} verwendet werden sollen.